\documentclass[11pt, a4paper]{article}
\usepackage[utf8]{inputenc}
\usepackage{graphicx}
\usepackage[svgnames]{xcolor}
\usepackage[margin=1in]{geometry}
\usepackage{enumitem}
\title{Borrador Anotado - Convención Constitucional 2021-2022}
\author{Análisis de Procedencia}
\date{\today}
\begin{document}
\maketitle
\begin{enumerate}

\item \textbf{Artículo} \newline
El Estado reconoce y promueve una sociedad en la que mujeres, hombres, diversidades y disidencias sexogenéricas participen en condiciones de igualdad sustantiva, reconociendo que su representación efectiva en el conjunto del proceso democrático es un principio y condición mínima para el ejercicio pleno y sustantivo de la democracia y la ciudadanía. 
\newline {\color{gray} \textbf{1º:} 116-1-c-Iniciativa-de-la-cc-Alondra-Carrillo-Democracia-Paritaria.pdf}
\newline {\color{gray} (Emb: 0.929, TF-IDF: 0.853)}
\newline {\color{gray} \textbf{2º:} 237-1-Iniciativa-Convencional-de-la-cc-Tania-Madriaga-sobre-Estado-Plurinacional-y-Libre-Determinacion-1146-hrs.pdf}
\newline {\color{gray} (Emb: 0.624, TF-IDF: 0.384)}

Todos los órganos colegiados del Estado, los órganos autónomos constitucionales y los órganos superiores y directivos de la Administración, así como los directorios de las empresas públicas y semipúblicas, deberán tener una composición paritaria que asegure que, al menos, el cincuenta por ciento de sus integrantes sean mujeres. 
\newline {\color{gray} \textbf{1º:} 116-1-c-Iniciativa-de-la-cc-Alondra-Carrillo-Democracia-Paritaria.pdf}
\newline {\color{gray} (Emb: 0.705, TF-IDF: 0.429)}
\newline {\color{gray} \textbf{2º:} 230-2-Iniciativa-Convencional-de-la-cc-Alondra-Carrillo-sobre-Participacion-en-la-Democracia-1142-hrs.pdf}
\newline {\color{gray} (Emb: 0.658, TF-IDF: 0.391)}

Asimismo, el Estado adoptará medidas para la representación de diversidades y disidencias de género a través del mecanismo que establezca la ley. 
\newline {\color{gray} \textbf{1º:} 587-Iniciativa-Convencional-Constituyente-de-cc-Marcos-Barraza-sobre-Derecho-al-Trabajo-2351-hrs.-01-02.pdf}
\newline {\color{gray} (Emb: 0.587, TF-IDF: 0.357)}
\newline {\color{gray} \textbf{2º:} 362-2-Iniciativa-Convencional-Constituyente-de-la-cc-Janis-Meneses-sobre-Derechos-Laborales-1836-hrs-21-01.pdf}
\newline {\color{gray} (Emb: 0.587, TF-IDF: 0.343)}

El Estado promoverá la integración paritaria en sus instituciones y en todos los espacios públicos y privados. 
\newline {\color{gray} \textbf{1º:} 822-Iniciativa-Convencional-Constituyente-del-cc-Nicolas-Nunez-sobre-Comunidades-Resilientes.pdf}
\newline {\color{gray} (Emb: 0.672, TF-IDF: 0.477)}
\newline {\color{gray} \textbf{2º:} 1030-Iniciativa-Convencional-Constituyente-del-cc-Arturo-Zuniga-sobre-Derecho-a-la-Salud.pdf}
\newline {\color{gray} (Emb: 0.637, TF-IDF: 0.394)}


\item \textbf{Artículo} \newline
Corresponderá al Estado, en sus diferentes ámbitos y funciones, garantizar la participación democrática e incidencia política de todas las personas, especialmente la de los grupos históricamente excluidos y de especial protección. 
\newline {\color{gray} \textbf{1º:} 253-1-Iniciativa-Convencional-de-la-cc-Barbara-Sepulveda-sobre-Formacion-de-la-Ley-1150-hrs-1.pdf}
\newline {\color{gray} (Emb: 0.932, TF-IDF: 0.822)}
\newline {\color{gray} \textbf{2º:} 968-Iniciativa-Convencional-Constituyente-de-la-cc-Valentina-Miranda-sobre-Derecho-a-la-salud.pdf}
\newline {\color{gray} (Emb: 0.549, TF-IDF: 0.429)}

El Estado deberá garantizar la inclusión de estos grupos en las políticas públicas y en el proceso de formación de las leyes, mediante mecanismos de participación popular y deliberación política, asegurando medidas afirmativas que posibiliten su participación efectiva. 
\newline {\color{gray} \textbf{1º:} 249-4-Iniciativa-Convencional-de-la-cc-Giovanna-Grandon-sobre-Derecho-a-la-Participacion1148-hrs.pdf}
\newline {\color{gray} (Emb: 0.677, TF-IDF: 0.339)}
\newline {\color{gray} \textbf{2º:} 267-4-Iniciativa-Convencional-de-la-cc-Janis-Meneses-sobre-Derechos-Politicos-1154-hrs.pdf}
\newline {\color{gray} (Emb: 0.677, TF-IDF: 0.329)}


\item \textbf{Artículo} \newline
Los poderes públicos adoptarán las medidas necesarias para adecuar e impulsar la legislación, instituciones, marcos normativos y prestación de servicios, con el fin de alcanzar la igualdad sustantiva y la paridad. 
\newline {\color{gray} \textbf{1º:} 116-1-c-Iniciativa-de-la-cc-Alondra-Carrillo-Democracia-Paritaria.pdf}
\newline {\color{gray} (Emb: 0.991, TF-IDF: 0.899)}
\newline {\color{gray} \textbf{2º:} 667-Iniciativa-Convencional-Constituyente-de-la-cc-Tatiana-Urrutia-sobre-Servicios-Publicos-130001-02-1.pdf}
\newline {\color{gray} (Emb: 0.660, TF-IDF: 0.305)}

Con ese objetivo, el Poder Ejecutivo, el Poder Legislativo y los Sistemas de Justicia, así como los órganos de la Administración del Estado y los órganos autónomos, deberán incorporar el enfoque de género en su diseño institucional y en el ejercicio de sus funciones. 
\newline {\color{gray} \textbf{1º:} 116-1-c-Iniciativa-de-la-cc-Alondra-Carrillo-Democracia-Paritaria.pdf}
\newline {\color{gray} (Emb: 0.963, TF-IDF: 0.935)}
\newline {\color{gray} \textbf{2º:} 643-Iniciativa-Convencional-Constituyente-del-cc-Marcos-Barraza-sobre-Banco-Central-174201-02.pdf}
\newline {\color{gray} (Emb: 0.593, TF-IDF: 0.429)}

La política fiscal y el diseño de los presupuestos públicos se adecuarán al cumplimiento de un enfoque transversal de igualdad sustantiva de género en las políticas públicas. 
\newline {\color{gray} \textbf{1º:} 116-1-c-Iniciativa-de-la-cc-Alondra-Carrillo-Democracia-Paritaria.pdf}
\newline {\color{gray} (Emb: 1.000, TF-IDF: 1.000)}
\newline {\color{gray} \textbf{2º:} 436-4-Iniciativa-Convencional-de-la-cc-Elsa-Labrana-sobre-Derecho-al-Trabajo-1203-27-01.pdf}
\newline {\color{gray} (Emb: 0.616, TF-IDF: 0.441)}


\item \textbf{Artículo} \newline
La ley deberá establecer las medidas afirmativas necesarias para garantizar la participación y representación política de las personas en situación de discapacidad. 
\newline {\color{gray} \textbf{1º:} 345-4-Iniciativa-Convencional-Constituyente-de-la-cc-Alejandra-Flores-sobre-Derecho-a-la-Alimentacion.pdf}
\newline {\color{gray} (Emb: 0.563, TF-IDF: 0.381)}
\newline {\color{gray} \textbf{2º:} 507-2-Iniciativa-Convencional-Constituyente-de-la-cc-Giovanna-Grandon-sobre-Nacionalidad-1213-01-02.pdf}
\newline {\color{gray} (Emb: 0.546, TF-IDF: 0.328)}


\item \textbf{Artículo} \newline
Chile es un Estado Plurinacional e Intercultural que reconoce la coexistencia de diversas naciones y pueblos en el marco de la unidad del Estado. 
\newline {\color{gray} \textbf{1º:} 213-1-c-Iniciativa-Convencional-del-cc-Jaime-Bassa-sobre-Congreso-plurinacional-2058-hrs.pdf}
\newline {\color{gray} (Emb: 0.967, TF-IDF: 0.856)}
\newline {\color{gray} \textbf{2º:} 94-1-Iniciativa-de-la-cc-Rosa-Catrileo-Establece-el-reconocimiento-de-los-Pueblos-Indigenas-2.pdf}
\newline {\color{gray} (Emb: 0.826, TF-IDF: 0.543)}

Son pueblos y naciones indígenas preexistentes los Mapuche, Aymara, Rapa Nui, Lickanantay, Quechua, Colla, Diaguita, Chango, Kawashkar, Yaghan, Selk'nam y otros que puedan ser reconocidos en la forma que establezca la ley. 
\newline {\color{gray} \textbf{1º:} 94-1-Iniciativa-de-la-cc-Rosa-Catrileo-Establece-el-reconocimiento-de-los-Pueblos-Indigenas-2.pdf}
\newline {\color{gray} (Emb: 0.993, TF-IDF: 0.981)}
\newline {\color{gray} \textbf{2º:} 237-1-Iniciativa-Convencional-de-la-cc-Tania-Madriaga-sobre-Estado-Plurinacional-y-Libre-Determinacion-1146-hrs.pdf}
\newline {\color{gray} (Emb: 0.944, TF-IDF: 0.822)}


\item \textbf{Artículo} \newline
Junto con ello, garantizará el diálogo intercultural en el ejercicio de las funciones públicas, creando institucionalidad y promoviendo políticas públicas que favorezcan el reconocimiento y comprensión de la diversidad étnica y cultural de los pueblos y naciones indígenas preexistentes al Estado. 
\newline {\color{gray} \textbf{1º:} 94-1-Iniciativa-de-la-cc-Rosa-Catrileo-Establece-el-reconocimiento-de-los-Pueblos-Indigenas-2.pdf}
\newline {\color{gray} (Emb: 1.000, TF-IDF: 1.000)}
\newline {\color{gray} \textbf{2º:} 683-Iniciativa-Convencional-Constituyente-del-cc-Jorge-Abarca-sobre-Pueblo-Tribal-150001-02.pdf}
\newline {\color{gray} (Emb: 0.792, TF-IDF: 0.689)}

En cumplimiento de lo anterior, el Estado debe garantizar la efectiva participación de los pueblos indígenas en el ejercicio y distribución del poder, incorporando su representación en la estructura del Estado, sus órganos e instituciones, así como su representación política en órganos de elección popular a nivel local, regional y nacional. 
\newline {\color{gray} \textbf{1º:} 94-1-Iniciativa-de-la-cc-Rosa-Catrileo-Establece-el-reconocimiento-de-los-Pueblos-Indigenas-2.pdf}
\newline {\color{gray} (Emb: 0.937, TF-IDF: 0.929)}
\newline {\color{gray} \textbf{2º:} 683-Iniciativa-Convencional-Constituyente-del-cc-Jorge-Abarca-sobre-Pueblo-Tribal-150001-02.pdf}
\newline {\color{gray} (Emb: 0.745, TF-IDF: 0.691)}

Es deber del Estado Plurinacional, respetar, garantizar y promover con participación de los pueblos y naciones indígenas, el ejercicio de la libre determinación y de los derechos colectivos e individuales de que son titulares. 
\newline {\color{gray} \textbf{1º:} 94-1-Iniciativa-de-la-cc-Rosa-Catrileo-Establece-el-reconocimiento-de-los-Pueblos-Indigenas-2.pdf}
\newline {\color{gray} (Emb: 1.000, TF-IDF: 1.000)}
\newline {\color{gray} \textbf{2º:} 94-1-Iniciativa-de-la-cc-Rosa-Catrileo-Establece-el-reconocimiento-de-los-Pueblos-Indigenas-2.pdf}
\newline {\color{gray} (Emb: 0.795, TF-IDF: 0.573)}

En especial, tienen derecho a la autonomía y al autogobierno, a su propia cultura, a la identidad y cosmovisión, al patrimonio y la lengua, al reconocimiento de sus tierras, territorios, la protección del territorio marítimo, de la naturaleza en su dimensión material e inmaterial y al especial vínculo que mantienen con estos, a la cooperación e integración, al reconocimiento de sus instituciones, jurisdicciones y autoridades propias o tradicionales y a participar plenamente, si así lo desean, en la vida política, económica, social y cultural del Estado. 
\newline {\color{gray} \textbf{1º:} 94-1-Iniciativa-de-la-cc-Rosa-Catrileo-Establece-el-reconocimiento-de-los-Pueblos-Indigenas-2.pdf}
\newline {\color{gray} (Emb: 0.900, TF-IDF: 0.911)}
\newline {\color{gray} \textbf{2º:} 670-Iniciativa-Convencional-Constituyente-del-cc-Alvin-Saldana-sobre-Pueblos-Originarios-121101-02.pdf}
\newline {\color{gray} (Emb: 0.710, TF-IDF: 0.343)}

Los pueblos y naciones indígenas preexistentes y sus miembros, en virtud de su libre determinación, tienen derecho al pleno ejercicio de sus derechos colectivos e individuales. 
\newline {\color{gray} \textbf{1º:} 94-1-Iniciativa-de-la-cc-Rosa-Catrileo-Establece-el-reconocimiento-de-los-Pueblos-Indigenas-2.pdf}
\newline {\color{gray} (Emb: 1.000, TF-IDF: 1.000)}
\newline {\color{gray} \textbf{2º:} 94-1-Iniciativa-de-la-cc-Rosa-Catrileo-Establece-el-reconocimiento-de-los-Pueblos-Indigenas-2.pdf}
\newline {\color{gray} (Emb: 0.795, TF-IDF: 0.550)}


\item \textbf{Artículo} \newline
El poder legislativo se compone del Congreso de Diputadas y Diputados y la Cámara de las Regiones. 
\newline {\color{gray} \textbf{1º:} 240-1-Iniciativa-Convencional-de-la-cc-Tania-Madriaga-sobre-Poder-Legislativo-1146-hrs.pdf}
\newline {\color{gray} (Emb: 0.832, TF-IDF: 0.533)}
\newline {\color{gray} \textbf{2º:} 120-3-c-Iniciativa-de-la-cc-Tammy-Pustilnick-atribuciones-exclusivas-de-la-Asamblea-Regional.pdf}
\newline {\color{gray} (Emb: 0.700, TF-IDF: 0.522)}


\item \textbf{Artículo} \newline
El Congreso de Diputadas y Diputados es un órgano deliberativo, paritario y plurinacional que representa al pueblo. 
\newline {\color{gray} \textbf{1º:} 82-1-Iniciativa-Convencional-Constituyente-del-cc-Martin-Arrau-y-otros.pdf}
\newline {\color{gray} (Emb: 0.684, TF-IDF: 0.290)}
\newline {\color{gray} \textbf{2º:} 240-1-Iniciativa-Convencional-de-la-cc-Tania-Madriaga-sobre-Poder-Legislativo-1146-hrs.pdf}
\newline {\color{gray} (Emb: 0.684, TF-IDF: 0.255)}

Concurre a la formación de las leyes y ejerce las demás facultades encomendadas por la Constitución. 
\newline {\color{gray} \textbf{1º:} 353-1-Iniciativa-Convencional-Constituyente-del-cc-Jaime-Bassa-sobre-Sistema-de-Gobierno-1158-21-01.pdf}
\newline {\color{gray} (Emb: 0.875, TF-IDF: 0.413)}
\newline {\color{gray} \textbf{2º:} 151-3-c-Iniciativa-de-la-cc-Angelica-Tepper-Competencias-de-los-Gobiernos-Regionales.pdf}
\newline {\color{gray} (Emb: 0.854, TF-IDF: 0.413)}

El Congreso está integrado por un número no inferior a 155 miembros electos en votación directa por distritos electorales. 
\newline {\color{gray} \textbf{1º:} 240-1-Iniciativa-Convencional-de-la-cc-Tania-Madriaga-sobre-Poder-Legislativo-1146-hrs.pdf}
\newline {\color{gray} (Emb: 0.864, TF-IDF: 0.601)}
\newline {\color{gray} \textbf{2º:} 400-1-Iniciativa-Convencional-Constituyente-de-la-cc-Constanza-Hube-sobre-Servicio-y-Registro-Electoral-1905-24-01.pdf}
\newline {\color{gray} (Emb: 0.744, TF-IDF: 0.597)}

Una ley de acuerdo regional determinará el número de integrantes, los distritos electorales y la forma de su elección, atendiendo el criterio de proporcionalidad. 
\newline {\color{gray} \textbf{1º:} 400-1-Iniciativa-Convencional-Constituyente-de-la-cc-Constanza-Hube-sobre-Servicio-y-Registro-Electoral-1905-24-01.pdf}
\newline {\color{gray} (Emb: 0.815, TF-IDF: 0.430)}
\newline {\color{gray} \textbf{2º:} 400-1-Iniciativa-Convencional-Constituyente-de-la-cc-Constanza-Hube-sobre-Servicio-y-Registro-Electoral-1905-24-01.pdf}
\newline {\color{gray} (Emb: 0.771, TF-IDF: 0.337)}


\item \textbf{Artículo} \newline
Son atribuciones exclusivas del Congreso de Diputadas y Diputados: a) Fiscalizar los actos del Gobierno. 
\newline {\color{gray} \textbf{1º:} 82-1-Iniciativa-Convencional-Constituyente-del-cc-Martin-Arrau-y-otros.pdf}
\newline {\color{gray} (Emb: 0.576, TF-IDF: 0.603)}
\newline {\color{gray} \textbf{2º:} 269-3-Iniciativa-Convencional-del-cc-Felipe-Mena-sobre-Organizacion-Territorial-del-Estado-17-01-1154-hrs.pdf}
\newline {\color{gray} (Emb: 0.565, TF-IDF: 0.599)}

El Congreso tendrá la facultad de solicitar información relativa al contenido y fundamentos de los actos de gobierno; c) Declarar, cuando la Presidenta o Presidente presente la renuncia a su cargo, si los motivos que la originan son o no fundados y, en consecuencia, admitirla o desecharla; d) Otorgar su acuerdo para que la Presidenta o Presidente de la República pueda ausentarse del país por más de treinta días o a contar desde el tercer domingo de noviembre del año anterior a aquel en que deba cesar en el cargo el que esté en funciones, y e) Las otras que establezca la Constitución. 
\newline {\color{gray} \textbf{1º:} 798-Iniciativa-Convencional-Constituyente-de-la-cc-Patricia-Labra-sonbre-Contraloria-General.pdf}
\newline {\color{gray} (Emb: 0.663, TF-IDF: 0.320)}
\newline {\color{gray} \textbf{2º:} 807-Iniciativa-Convencional-Constituyente-del-cc-Jaime-Bassa-sobre-Formacion-de-la-Ley.pdf}
\newline {\color{gray} (Emb: 0.663, TF-IDF: 0.320)}


\item \textbf{Artículo} \newline
No obstante, una misma comisión investigadora no podrá citar más de tres veces a la misma persona, sin previo acuerdo de la mayoría de sus integrantes. 
\newline {\color{gray} \textbf{1º:} 240-1-Iniciativa-Convencional-de-la-cc-Tania-Madriaga-sobre-Poder-Legislativo-1146-hrs.pdf}
\newline {\color{gray} (Emb: 0.770, TF-IDF: 0.652)}
\newline {\color{gray} \textbf{2º:} 240-1-Iniciativa-Convencional-de-la-cc-Tania-Madriaga-sobre-Poder-Legislativo-1146-hrs.pdf}
\newline {\color{gray} (Emb: 0.749, TF-IDF: 0.623)}

Toda persona que sea citada por estas comisiones estará obligada a comparecer y a suministrar los antecedentes y las informaciones que se le soliciten. 
\newline {\color{gray} \textbf{1º:} 721-Iniciativa-Convencional-Constituyente-de-la-cc-Maria-Trinidad-Castillo-sobre-Mineria-01-02.pdf}
\newline {\color{gray} (Emb: 0.689, TF-IDF: 0.534)}
\newline {\color{gray} \textbf{2º:} 954-5-Iniciativa-Convencional-Constituyente-de-la-cc-Carolina-Vilches-sobre-Estatuto-del-Agua.pdf}
\newline {\color{gray} (Emb: 0.564, TF-IDF: 0.534)}

Las comisiones investigadoras, a petición de un tercio de sus miembros, podrán despachar citaciones y solicitar antecedentes. 
\newline {\color{gray} \textbf{1º:} 240-1-Iniciativa-Convencional-de-la-cc-Tania-Madriaga-sobre-Poder-Legislativo-1146-hrs.pdf}
\newline {\color{gray} (Emb: 0.508, TF-IDF: 0.309)}
\newline {\color{gray} \textbf{2º:} 560-Iniciativa-Convencional-Constituyente-de-cc-Andres-Cruz-sobre-Ministerio-Publico-2016-hrs.-01-02.pdf}
\newline {\color{gray} (Emb: 0.508, TF-IDF: 0.302)}

c) Crear comisiones especiales investigadoras a petición de a lo menos dos quintos de las diputadas y diputados en ejercicio, con el objeto de reunir informaciones relativas a determinados actos del Gobierno. 
\newline {\color{gray} \textbf{1º:} 240-1-Iniciativa-Convencional-de-la-cc-Tania-Madriaga-sobre-Poder-Legislativo-1146-hrs.pdf}
\newline {\color{gray} (Emb: 0.870, TF-IDF: 0.747)}
\newline {\color{gray} \textbf{2º:} 240-1-Iniciativa-Convencional-de-la-cc-Tania-Madriaga-sobre-Poder-Legislativo-1146-hrs.pdf}
\newline {\color{gray} (Emb: 0.749, TF-IDF: 0.668)}

En ningún caso estos actos afectarán la responsabilidad política de las y los Ministros de Estado. 
\newline {\color{gray} \textbf{1º:} 240-1-Iniciativa-Convencional-de-la-cc-Tania-Madriaga-sobre-Poder-Legislativo-1146-hrs.pdf}
\newline {\color{gray} (Emb: 0.556, TF-IDF: 0.330)}
\newline {\color{gray} \textbf{2º:} 216-1-c-Iniciativa-Convencional-de-la-cc-Rosa-Catrileo-sobre-Sistema-de-Gobierno-2126-hrs.pdf}
\newline {\color{gray} (Emb: 0.555, TF-IDF: 0.303)}

Dentro de los treinta días contados desde su comunicación, la o el Presidente deberá dar respuesta fundada por medio de la o el Ministro de Estado que corresponda. 
\newline {\color{gray} \textbf{1º:} 184-6-c-Iniciativa-Convencional-del-cc-Rodrigo-Álvarez-que-crea-la-Corte-Constitucional-1044hrs.pdf}
\newline {\color{gray} (Emb: 0.695, TF-IDF: 0.333)}
\newline {\color{gray} \textbf{2º:} 184-6-c-Iniciativa-Convencional-del-cc-Rodrigo-Álvarez-que-crea-la-Corte-Constitucional-1044hrs.pdf}
\newline {\color{gray} (Emb: 0.609, TF-IDF: 0.328)}

b) Solicitar antecedentes a la o el Presidente de la República, con el patrocinio de un cuarto de sus miembros. 
\newline {\color{gray} \textbf{1º:} 95-6-Iniciativa-Convencional-Constituyente-de-Cc-Mauricio-Daza-y-otros-2.pdf}
\newline {\color{gray} (Emb: 0.679, TF-IDF: 0.296)}
\newline {\color{gray} \textbf{2º:} 353-1-Iniciativa-Convencional-Constituyente-del-cc-Jaime-Bassa-sobre-Sistema-de-Gobierno-1158-21-01.pdf}
\newline {\color{gray} (Emb: 0.654, TF-IDF: 0.269)}

Para ejercer esta atribución puede: a) Adoptar acuerdos o sugerir observaciones, los que se transmitirán por escrito a la o el Presidente de la República. 
\newline {\color{gray} \textbf{1º:} 240-1-Iniciativa-Convencional-de-la-cc-Tania-Madriaga-sobre-Poder-Legislativo-1146-hrs.pdf}
\newline {\color{gray} (Emb: 0.634, TF-IDF: 0.482)}
\newline {\color{gray} \textbf{2º:} 224-1-Iniciativa-Convencional-del-cc-Jaime-Bassa-sobre-Formacion-de-la-Ley-1139-17-01.pdf}
\newline {\color{gray} (Emb: 0.632, TF-IDF: 0.310)}

El Congreso de Diputadas y Diputados tendrá por función fiscalizar los actos del Gobierno. 
\newline {\color{gray} \textbf{1º:} 154-3-c-Iniciativa-del-cc-Felipe-Mena-sobre-Organizacion-Territorial-del-Estado.pdf}
\newline {\color{gray} (Emb: 0.654, TF-IDF: 0.674)}
\newline {\color{gray} \textbf{2º:} 269-3-Iniciativa-Convencional-del-cc-Felipe-Mena-sobre-Organizacion-Territorial-del-Estado-17-01-1154-hrs.pdf}
\newline {\color{gray} (Emb: 0.654, TF-IDF: 0.412)}

La o el Presidente deberá contestar fundadamente por medio del Ministro o Ministra de Estado que corresponda dentro de los tres días desde su comunicación. 
\newline {\color{gray} \textbf{1º:} 353-1-Iniciativa-Convencional-Constituyente-del-cc-Jaime-Bassa-sobre-Sistema-de-Gobierno-1158-21-01.pdf}
\newline {\color{gray} (Emb: 0.563, TF-IDF: 0.323)}
\newline {\color{gray} \textbf{2º:} 234-1-Iniciativa-Convencional-del-cc-Jaime-Bassa-sobre-Justicia-Complementaria-1144-hrs.pdf}
\newline {\color{gray} (Emb: 0.563, TF-IDF: 0.323)}


\item \textbf{Artículo} \newline
La Cámara de las Regiones es un órgano deliberativo, paritario y plurinacional de representación regional encargado de concurrir a la formación de las leyes de acuerdo regional y de ejercer las demás facultades encomendadas por esta Constitución. 
\newline {\color{gray} \textbf{1º:} 120-3-c-Iniciativa-de-la-cc-Tammy-Pustilnick-atribuciones-exclusivas-de-la-Asamblea-Regional.pdf}
\newline {\color{gray} (Emb: 0.685, TF-IDF: 0.357)}
\newline {\color{gray} \textbf{2º:} 120-3-c-Iniciativa-de-la-cc-Tammy-Pustilnick-atribuciones-exclusivas-de-la-Asamblea-Regional.pdf}
\newline {\color{gray} (Emb: 0.674, TF-IDF: 0.301)}

Sus integrantes se denominarán representantes regionales. 
\newline {\color{gray} \textbf{1º:} 675-Iniciativa-Convencional-Constituyente-del-cc-Felipe-Harboe-sobre-Congreso-Nacional-121101-02.pdf}
\newline {\color{gray} (Emb: 0.645, TF-IDF: 0.366)}
\newline {\color{gray} \textbf{2º:} 377-2-Iniciativa-Convencional-Constituyente-del-cc-Alvin-Saldana-sobre-Mecanismos-de-Democracia-1045-hrs-24-01.pdf}
\newline {\color{gray} (Emb: 0.607, TF-IDF: 0.323)}


\item \textbf{Artículo} \newline
La ley determinará el número de representantes regionales a ser elegidas y elegidos por región, el que deberá ser el mismo para cada una y en ningún caso inferior a tres, asegurando que la integración final del órgano respete el principio de paridad. 
\newline {\color{gray} \textbf{1º:} 384-3-Iniciativa-Convencional-Constituyente-del-cc-Felipe-Mena-sobre-Gobiernos-Regionales-1159-24-01.pdf}
\newline {\color{gray} (Emb: 0.636, TF-IDF: 0.240)}
\newline {\color{gray} \textbf{2º:} 400-1-Iniciativa-Convencional-Constituyente-de-la-cc-Constanza-Hube-sobre-Servicio-y-Registro-Electoral-1905-24-01.pdf}
\newline {\color{gray} (Emb: 0.623, TF-IDF: 0.238)}

Las y los miembros de la Cámara de las Regiones se elegirán en votación popular conjuntamente con las autoridades comunales y regionales, tres años después de la elección presidencial y del Congreso. 
\newline {\color{gray} \textbf{1º:} 675-Iniciativa-Convencional-Constituyente-del-cc-Felipe-Harboe-sobre-Congreso-Nacional-121101-02.pdf}
\newline {\color{gray} (Emb: 0.654, TF-IDF: 0.293)}
\newline {\color{gray} \textbf{2º:} 234-1-Iniciativa-Convencional-del-cc-Jaime-Bassa-sobre-Justicia-Complementaria-1144-hrs.pdf}
\newline {\color{gray} (Emb: 0.652, TF-IDF: 0.287)}

La ley especificará sus derechos y obligaciones especiales, las que en todo caso deberán incluir la obligación de rendir cuenta periódicamente ante la Asamblea Regional que representa. 
\newline {\color{gray} \textbf{1º:} 333-3-Iniciativa-Convencional-Constituyente-del-cc-Jorge-Arancibia-sobre-Autonomia-Territorial-y-Pueblos-Originarios.pdf}
\newline {\color{gray} (Emb: 0.597, TF-IDF: 0.244)}
\newline {\color{gray} \textbf{2º:} 369-4-Iniciativa-Convencional-Constituyente-de-la-cc-Loreto-Vallejos-sobre-Derecho-a-la-Educacion-0900-hrs-24-01.pdf}
\newline {\color{gray} (Emb: 0.581, TF-IDF: 0.237)}

También podrán ser especialmente convocadas y convocados al efecto. 
\newline {\color{gray} \textbf{1º:} 267-4-Iniciativa-Convencional-de-la-cc-Janis-Meneses-sobre-Derechos-Politicos-1154-hrs.pdf}
\newline {\color{gray} (Emb: 0.500, TF-IDF: 0.389)}
\newline {\color{gray} \textbf{2º:} 753-Iniciativa-Convencional-Constituyente-del-cc-Raul-Celis-sobre-Gobiernos-Locales.pdf}
\newline {\color{gray} (Emb: 0.495, TF-IDF: 0.372)}

La Cámara de las Regiones no podrá fiscalizar los actos del Gobierno ni de las entidades que de él dependan. 
\newline {\color{gray} \textbf{1º:} 384-3-Iniciativa-Convencional-Constituyente-del-cc-Felipe-Mena-sobre-Gobiernos-Regionales-1159-24-01.pdf}
\newline {\color{gray} (Emb: 0.643, TF-IDF: 0.520)}
\newline {\color{gray} \textbf{2º:} 191-2-c-Iniciativa-Convencional-del-cc-Martín-Arrau-sonre-Principio-de-función-eficiente-del-Estado-1302-hrs.pdf}
\newline {\color{gray} (Emb: 0.636, TF-IDF: 0.357)}


\item \textbf{Artículo} \newline
La suspensión cesará si la Cámara de las Regiones desestimare la acusación o si no se pronunciare dentro de los treinta días siguientes. 
\newline {\color{gray} \textbf{1º:} 184-6-c-Iniciativa-Convencional-del-cc-Rodrigo-Álvarez-que-crea-la-Corte-Constitucional-1044hrs.pdf}
\newline {\color{gray} (Emb: 0.647, TF-IDF: 0.374)}
\newline {\color{gray} \textbf{2º:} 602-Iniciativa-Convencional-Constituyente-de-cc-Jorge-Baradit-sobre-Mecanismos-de-Demoracia-Directa-y-Semidirecta.pdf}
\newline {\color{gray} (Emb: 0.592, TF-IDF: 0.369)}

En los demás casos se requerirá el de la mayoría de los diputados y diputadas presentes y la o el acusado quedará suspendido en sus funciones desde el momento en que el Congreso de Diputadas y Diputados declare que ha lugar la acusación. 
\newline {\color{gray} \textbf{1º:} 801-Iniciativa-Convencional-Constituyente-del-cc-Christian-Viera-sobre-Consejo-de-Contiendas-de-Trabajo.pdf}
\newline {\color{gray} (Emb: 0.613, TF-IDF: 0.337)}
\newline {\color{gray} \textbf{2º:} 240-1-Iniciativa-Convencional-de-la-cc-Tania-Madriaga-sobre-Poder-Legislativo-1146-hrs.pdf}
\newline {\color{gray} (Emb: 0.556, TF-IDF: 0.279)}

Para declarar que ha lugar la acusación en contra del Presidente o Presidenta de la República o de un gobernador regional se necesitará el voto de la mayoría de los diputados y diputadas en ejercicio. 
\newline {\color{gray} \textbf{1º:} 240-1-Iniciativa-Convencional-de-la-cc-Tania-Madriaga-sobre-Poder-Legislativo-1146-hrs.pdf}
\newline {\color{gray} (Emb: 0.715, TF-IDF: 0.482)}
\newline {\color{gray} \textbf{2º:} 151-3-c-Iniciativa-de-la-cc-Angelica-Tepper-Competencias-de-los-Gobiernos-Regionales.pdf}
\newline {\color{gray} (Emb: 0.676, TF-IDF: 0.390)}

Interpuesta la acusación, la o el afectado no podrá ausentarse del país sin permiso del Congreso de Diputadas y Diputados y no podrá hacerlo en caso alguno si la acusación ya estuviere aprobada por éste. 
\newline {\color{gray} \textbf{1º:} 181-6-c-Iniciativa-Convencional-del-cc-Rodrigo-Álvarez-sobre-recurso-protección-y-acción-del-legislador-1044-hrs.pdf}
\newline {\color{gray} (Emb: 0.586, TF-IDF: 0.603)}
\newline {\color{gray} \textbf{2º:} 240-1-Iniciativa-Convencional-de-la-cc-Tania-Madriaga-sobre-Poder-Legislativo-1146-hrs.pdf}
\newline {\color{gray} (Emb: 0.571, TF-IDF: 0.343)}

Es atribución exclusiva del Congreso de las Diputadas y Diputados declarar si han o no lugar las acusaciones que no menos de diez ni más de veinte de sus integrantes formulen en contra de: a) La Presidenta o Presidente de la República, por actos de su administración que hayan comprometido gravemente el honor o la seguridad de la Nación, o infringido abiertamente la Constitución o las leyes. 
\newline {\color{gray} \textbf{1º:} 240-1-Iniciativa-Convencional-de-la-cc-Tania-Madriaga-sobre-Poder-Legislativo-1146-hrs.pdf}
\newline {\color{gray} (Emb: 0.784, TF-IDF: 0.878)}
\newline {\color{gray} \textbf{2º:} 240-1-Iniciativa-Convencional-de-la-cc-Tania-Madriaga-sobre-Poder-Legislativo-1146-hrs.pdf}
\newline {\color{gray} (Emb: 0.745, TF-IDF: 0.714)}

La acusación se tramitará en conformidad a la Ley de Organización, Funcionamiento y Procedimientos del Poder Legislativo. 
\newline {\color{gray} \textbf{1º:} 88-6-Iniciativa-Convencional-Constituyente-del-cc-Christian-Viera-y-otros.pdf}
\newline {\color{gray} (Emb: 0.821, TF-IDF: 0.345)}
\newline {\color{gray} \textbf{2º:} 648-Iniciativa-Convencional-Constituyente-de-la-cc-Carolina-Sepulveda-sobre-Funcion-Publica-121101-02.pdf}
\newline {\color{gray} (Emb: 0.739, TF-IDF: 0.323)}

Durante este último tiempo no podrá ausentarse de la República sin acuerdo del Congreso de las Diputadas y Diputados; b) Las Ministras y Ministros de Estado, por haber comprometido gravemente el honor o la seguridad de la Nación, por infringir la Constitución o las leyes o haber dejado éstas sin ejecución, y por los delitos de traición, concusión, malversación de fondos públicos y soborno; c) Las juezas y jueces de las Cortes de Apelaciones y la Corte Suprema, y de la o el Contralor General de la República, por notable abandono de sus deberes; d) Las y los generales o almirantes de las instituciones pertenecientes a las Fuerzas Armadas, del General Director de Carabineros de Chile y del Director General de la Policía de Investigaciones de Chile, por haber comprometido gravemente el honor o la seguridad de la Nación; e) Las y los gobernadores regionales y de la autoridad en los territorios especiales e indígenas, por infracción de la Constitución y por los delitos de traición, sedición, malversación de fondos públicos y concusión. 
\newline {\color{gray} \textbf{1º:} 240-1-Iniciativa-Convencional-de-la-cc-Tania-Madriaga-sobre-Poder-Legislativo-1146-hrs.pdf}
\newline {\color{gray} (Emb: 0.738, TF-IDF: 0.735)}
\newline {\color{gray} \textbf{2º:} 325-6-Iniciativa-Convencional-del-cc-Tomas-Laibe-sobre-Jurisdiccion-Constitucional.pdf}
\newline {\color{gray} (Emb: 0.692, TF-IDF: 0.347)}

Esta acusación podrá interponerse mientras la Presidenta o Presidente esté en funciones y en los seis meses siguientes a su expiración en el cargo. 
\newline {\color{gray} \textbf{1º:} 240-1-Iniciativa-Convencional-de-la-cc-Tania-Madriaga-sobre-Poder-Legislativo-1146-hrs.pdf}
\newline {\color{gray} (Emb: 0.964, TF-IDF: 0.969)}
\newline {\color{gray} \textbf{2º:} 240-1-Iniciativa-Convencional-de-la-cc-Tania-Madriaga-sobre-Poder-Legislativo-1146-hrs.pdf}
\newline {\color{gray} (Emb: 0.848, TF-IDF: 0.916)}

Las acusaciones referidas en las letras b), c), d) y e) podrán interponerse mientras la o el afectado esté en funciones o en los tres meses siguientes a la expiración en su cargo. 
\newline {\color{gray} \textbf{1º:} 240-1-Iniciativa-Convencional-de-la-cc-Tania-Madriaga-sobre-Poder-Legislativo-1146-hrs.pdf}
\newline {\color{gray} (Emb: 0.676, TF-IDF: 0.649)}
\newline {\color{gray} \textbf{2º:} 801-Iniciativa-Convencional-Constituyente-del-cc-Christian-Viera-sobre-Consejo-de-Contiendas-de-Trabajo.pdf}
\newline {\color{gray} (Emb: 0.545, TF-IDF: 0.581)}


\item \textbf{Artículo} \newline
La persona destituida no podrá desempeñar ningún otro cargo de exclusiva confianza de la o el Presidente durante el tiempo que reste de su mandato o presentarse al cargo de elección popular del cual fue destituido en el período siguiente, según corresponda. 
\newline {\color{gray} \textbf{1º:} 234-1-Iniciativa-Convencional-del-cc-Jaime-Bassa-sobre-Justicia-Complementaria-1144-hrs.pdf}
\newline {\color{gray} (Emb: 0.712, TF-IDF: 0.397)}
\newline {\color{gray} \textbf{2º:} 90-6-Iniciativa-Convencional-Constituyente-del-cc-Tomas-Laibe-y-otros.pdf}
\newline {\color{gray} (Emb: 0.705, TF-IDF: 0.359)}

La o el funcionario declarado culpable será juzgado de acuerdo a las leyes por el tribunal competente, tanto para la aplicación de la pena señalada al delito, si lo hubiere, como para hacer efectiva la responsabilidad civil por los daños y perjuicios causados al Estado o a particulares. 
\newline {\color{gray} \textbf{1º:} 172-6-c-Iniciativa-Convencional-del-cc-Rodrigo-Álvarez-sobre-Banco-central1044-hrs.pdf}
\newline {\color{gray} (Emb: 0.791, TF-IDF: 0.256)}
\newline {\color{gray} \textbf{2º:} 184-6-c-Iniciativa-Convencional-del-cc-Rodrigo-Álvarez-que-crea-la-Corte-Constitucional-1044hrs.pdf}
\newline {\color{gray} (Emb: 0.771, TF-IDF: 0.256)}

Por la declaración de culpabilidad queda la o el acusado destituido de su cargo. 
\newline {\color{gray} \textbf{1º:} 319-6-Iniciativa-Convencional-del-cc-Mauricio-Daza-sobre-el-Sistema-Nacional-de-Justicia17-09-hrs.pdf}
\newline {\color{gray} (Emb: 0.667, TF-IDF: 0.270)}
\newline {\color{gray} \textbf{2º:} 319-6-Iniciativa-Convencional-del-cc-Mauricio-Daza-sobre-el-Sistema-Nacional-de-Justicia17-09-hrs.pdf}
\newline {\color{gray} (Emb: 0.667, TF-IDF: 0.250)}

La declaración de culpabilidad deberá ser pronunciada por los dos tercios de las y los representantes regionales en ejercicio cuando se trate de una acusación en contra de la Presidenta o Presidente de la República o de un gobernador regional, y por la mayoría de los representantes regionales en ejercicio en los demás casos. 
\newline {\color{gray} \textbf{1º:} 240-1-Iniciativa-Convencional-de-la-cc-Tania-Madriaga-sobre-Poder-Legislativo-1146-hrs.pdf}
\newline {\color{gray} (Emb: 0.670, TF-IDF: 0.410)}
\newline {\color{gray} \textbf{2º:} 631-Iniciativa-Convencional-Constituyente-de-cc-Ingrid-Villena-sobre-Contraloria-General-de-la-Republica.pdf}
\newline {\color{gray} (Emb: 0.626, TF-IDF: 0.316)}

La Cámara de las Regiones resolverá como jurado y se limitará a declarar si la o el acusado es o no culpable. 
\newline {\color{gray} \textbf{1º:} 90-6-Iniciativa-Convencional-Constituyente-del-cc-Tomas-Laibe-y-otros.pdf}
\newline {\color{gray} (Emb: 0.610, TF-IDF: 0.230)}
\newline {\color{gray} \textbf{2º:} 181-6-c-Iniciativa-Convencional-del-cc-Rodrigo-Álvarez-sobre-recurso-protección-y-acción-del-legislador-1044-hrs.pdf}
\newline {\color{gray} (Emb: 0.607, TF-IDF: 0.209)}

Es atribución exclusiva de la Cámara de las Regiones conocer de las acusaciones que el Congreso de Diputadas y Diputados entable con arreglo a lo establecido en el artículo 11 bis. 
\newline {\color{gray} \textbf{1º:} 400-1-Iniciativa-Convencional-Constituyente-de-la-cc-Constanza-Hube-sobre-Servicio-y-Registro-Electoral-1905-24-01.pdf}
\newline {\color{gray} (Emb: 0.539, TF-IDF: 0.248)}
\newline {\color{gray} \textbf{2º:} 184-6-c-Iniciativa-Convencional-del-cc-Rodrigo-Álvarez-que-crea-la-Corte-Constitucional-1044hrs.pdf}
\newline {\color{gray} (Emb: 0.529, TF-IDF: 0.239)}


\item \textbf{Artículo} \newline
El Congreso de Diputadas y Diputados y la Cámara de las Regiones se reunirán en sesión conjunta para tomar el juramento o promesa de la Presidenta o Presidente de la República al momento de asumir el cargo, para recibir la cuenta pública anual y para inaugurar el año legislativo. 
\newline {\color{gray} \textbf{1º:} 467-6-Iniciativa-Convencional-Constituyente-del-cc-Daniel-Bravo-sobre-Reforma-y-Reemplazo-1945-31-01.pdf}
\newline {\color{gray} (Emb: 0.699, TF-IDF: 0.339)}
\newline {\color{gray} \textbf{2º:} 807-Iniciativa-Convencional-Constituyente-del-cc-Jaime-Bassa-sobre-Formacion-de-la-Ley.pdf}
\newline {\color{gray} (Emb: 0.694, TF-IDF: 0.339)}


\item \textbf{Artículo} \newline
El Congreso de Diputadas y Diputados y la Cámara de las Regiones se reunirán en sesión conjunta para decidir los nombramientos que conforme a esta Constitución corresponda, garantizando un estricto escrutinio de la idoneidad de las y los candidatos para el cargo correspondiente. 
\newline {\color{gray} \textbf{1º:} 791-Iniciativa-Convencional-Constituyente-de-la-cc-Cristina-Dorador-sobre-Proteccion-de-la-Salud.pdf}
\newline {\color{gray} (Emb: 0.617, TF-IDF: 0.274)}
\newline {\color{gray} \textbf{2º:} 120-3-c-Iniciativa-de-la-cc-Tammy-Pustilnick-atribuciones-exclusivas-de-la-Asamblea-Regional.pdf}
\newline {\color{gray} (Emb: 0.607, TF-IDF: 0.219)}


\item \textbf{Artículo} \newline
Para ser elegida diputada, diputado o representante regional se requiere ser ciudadana o ciudadano con derecho a sufragio, haber cumplido dieciocho años de edad al día de la elección y tener avecindamiento en el territorio correspondiente durante un plazo no inferior a dos años, en el caso de las diputadas o diputados, y de cuatro años en el caso de las y los representantes regionales, contados hacia atrás desde el día de la elección. 
\newline {\color{gray} \textbf{1º:} 321-1-Iniciativa-Convencional-Constituyente-de-la-cc-Barbara-Sepulveda-sobre-Estatuto-de-Diputados-y-Diputadas.pdf}
\newline {\color{gray} (Emb: 0.858, TF-IDF: 0.779)}
\newline {\color{gray} \textbf{2º:} 240-1-Iniciativa-Convencional-de-la-cc-Tania-Madriaga-sobre-Poder-Legislativo-1146-hrs.pdf}
\newline {\color{gray} (Emb: 0.811, TF-IDF: 0.571)}

Se entenderá que una diputada, diputado o representante regional tiene su residencia en el territorio correspondiente mientras ejerza su cargo. 
\newline {\color{gray} \textbf{1º:} 234-1-Iniciativa-Convencional-del-cc-Jaime-Bassa-sobre-Justicia-Complementaria-1144-hrs.pdf}
\newline {\color{gray} (Emb: 0.921, TF-IDF: 0.683)}
\newline {\color{gray} \textbf{2º:} 400-1-Iniciativa-Convencional-Constituyente-de-la-cc-Constanza-Hube-sobre-Servicio-y-Registro-Electoral-1905-24-01.pdf}
\newline {\color{gray} (Emb: 0.705, TF-IDF: 0.464)}


\item \textbf{Artículo} \newline
Las inhabilidades establecidas en este artículo serán aplicables a quienes hubieren tenido las calidades o cargos antes mencionados dentro del año inmediatamente anterior a la elección, excepto respecto de las personas mencionadas en el número 11, las que no deberán reunir esas condiciones al momento de inscribir su candidatura y de las indicadas en el número 9, 10 y 12, respecto de las cuales el plazo de la inhabilidad será de los dos años inmediatamente anteriores a la elección. 
\newline {\color{gray} \textbf{1º:} 641-Iniciativa-Convencional-Constituyente-del-cc-Mauricio-Daza-sobre-Contraloria-General-1730-01-02.pdf}
\newline {\color{gray} (Emb: 0.535, TF-IDF: 0.299)}
\newline {\color{gray} \textbf{2º:} 377-2-Iniciativa-Convencional-Constituyente-del-cc-Alvin-Saldana-sobre-Mecanismos-de-Democracia-1045-hrs-24-01.pdf}
\newline {\color{gray} (Emb: 0.532, TF-IDF: 0.237)}

Las y los militares en servicio activo. 
\newline {\color{gray} \textbf{1º:} 321-1-Iniciativa-Convencional-Constituyente-de-la-cc-Barbara-Sepulveda-sobre-Estatuto-de-Diputados-y-Diputadas.pdf}
\newline {\color{gray} (Emb: 0.802, TF-IDF: 1.000)}
\newline {\color{gray} \textbf{2º:} 230-2-Iniciativa-Convencional-de-la-cc-Alondra-Carrillo-sobre-Participacion-en-la-Democracia-1142-hrs.pdf}
\newline {\color{gray} (Emb: 0.675, TF-IDF: 0.350)}

Las personas naturales o administradores de personas jurídicas que celebren o caucionen contratos con el Estado, y 12. 
\newline {\color{gray} \textbf{1º:} 230-2-Iniciativa-Convencional-de-la-cc-Alondra-Carrillo-sobre-Participacion-en-la-Democracia-1142-hrs.pdf}
\newline {\color{gray} (Emb: 0.804, TF-IDF: 0.829)}
\newline {\color{gray} \textbf{2º:} 321-1-Iniciativa-Convencional-Constituyente-de-la-cc-Barbara-Sepulveda-sobre-Estatuto-de-Diputados-y-Diputadas.pdf}
\newline {\color{gray} (Emb: 0.723, TF-IDF: 0.803)}

Los funcionarios o funcionarias en servicio activo de las policías; 11. 
\newline {\color{gray} \textbf{1º:} 321-1-Iniciativa-Convencional-Constituyente-de-la-cc-Barbara-Sepulveda-sobre-Estatuto-de-Diputados-y-Diputadas.pdf}
\newline {\color{gray} (Emb: 0.584, TF-IDF: 0.494)}
\newline {\color{gray} \textbf{2º:} 321-1-Iniciativa-Convencional-Constituyente-de-la-cc-Barbara-Sepulveda-sobre-Estatuto-de-Diputados-y-Diputadas.pdf}
\newline {\color{gray} (Emb: 0.488, TF-IDF: 0.340)}

La o el Fiscal Nacional, fiscales regionales o fiscales adjuntos del Ministerio Público; 10. 
\newline {\color{gray} \textbf{1º:} 230-2-Iniciativa-Convencional-de-la-cc-Alondra-Carrillo-sobre-Participacion-en-la-Democracia-1142-hrs.pdf}
\newline {\color{gray} (Emb: 0.829, TF-IDF: 0.949)}
\newline {\color{gray} \textbf{2º:} 909-Iniciativa-Convencional-Constituyente-del-cc-Hugo-Gutierrez-Sobre-Ministerio-Publico.pdf}
\newline {\color{gray} (Emb: 0.670, TF-IDF: 0.697)}

La o el Contralor General de la República; 9. 
\newline {\color{gray} \textbf{1º:} 230-2-Iniciativa-Convencional-de-la-cc-Alondra-Carrillo-sobre-Participacion-en-la-Democracia-1142-hrs.pdf}
\newline {\color{gray} (Emb: 0.809, TF-IDF: 0.607)}
\newline {\color{gray} \textbf{2º:} 580-Iniciativa-Convencional-Constituyente-de-cc-Daniel-Stingo-sobre-Contraloria-General-de-la-Republica-2318-hrs.-01-02.pdf}
\newline {\color{gray} (Emb: 0.702, TF-IDF: 0.550)}

Las y los miembros del Tribunal Calificador de Elecciones y de los tribunales electorales; 8. 
\newline {\color{gray} \textbf{1º:} 90-6-Iniciativa-Convencional-Constituyente-del-cc-Tomas-Laibe-y-otros.pdf}
\newline {\color{gray} (Emb: 0.694, TF-IDF: 0.559)}
\newline {\color{gray} \textbf{2º:} 321-1-Iniciativa-Convencional-Constituyente-de-la-cc-Barbara-Sepulveda-sobre-Estatuto-de-Diputados-y-Diputadas.pdf}
\newline {\color{gray} (Emb: 0.675, TF-IDF: 0.539)}

Las y los que ejerzan jurisdicción en los Sistemas de Justicia; 7. 
\newline {\color{gray} \textbf{1º:} 321-1-Iniciativa-Convencional-Constituyente-de-la-cc-Barbara-Sepulveda-sobre-Estatuto-de-Diputados-y-Diputadas.pdf}
\newline {\color{gray} (Emb: 0.613, TF-IDF: 0.377)}
\newline {\color{gray} \textbf{2º:} 472-6-Iniciativa-Convencional-Constituyente-del-cc-Daniel-Bravo-sobre-Corte-Constitucional-2003-31-01.pdf}
\newline {\color{gray} (Emb: 0.604, TF-IDF: 0.373)}

Las y los directivos de los órganos autónomos; 6. 
\newline {\color{gray} \textbf{1º:} 783-Iniciativa-Convencional-Constituyente-de-la-cc-Ivanna-Olivares-sobre-Gestion-Publica.pdf}
\newline {\color{gray} (Emb: 0.637, TF-IDF: 0.407)}
\newline {\color{gray} \textbf{2º:} 321-1-Iniciativa-Convencional-Constituyente-de-la-cc-Barbara-Sepulveda-sobre-Estatuto-de-Diputados-y-Diputadas.pdf}
\newline {\color{gray} (Emb: 0.537, TF-IDF: 0.407)}

Las y los Consejeros del Banco Central y del Consejo Electoral; 5. 
\newline {\color{gray} \textbf{1º:} 230-2-Iniciativa-Convencional-de-la-cc-Alondra-Carrillo-sobre-Participacion-en-la-Democracia-1142-hrs.pdf}
\newline {\color{gray} (Emb: 0.704, TF-IDF: 0.699)}
\newline {\color{gray} \textbf{2º:} 349-6-Iniciativa-Convencional-Constituyente-del-cc-Luis-Mayol-sobre-Banco-Central-1826-20-01.pdf}
\newline {\color{gray} (Emb: 0.630, TF-IDF: 0.461)}

Las autoridades regionales y comunales de elección popular; 4. 
\newline {\color{gray} \textbf{1º:} 321-1-Iniciativa-Convencional-Constituyente-de-la-cc-Barbara-Sepulveda-sobre-Estatuto-de-Diputados-y-Diputadas.pdf}
\newline {\color{gray} (Emb: 0.793, TF-IDF: 0.498)}
\newline {\color{gray} \textbf{2º:} 485-2-Iniciativa-Convencional-Constituyente-del-cc-Jorge-Baradit-sobre-Nacionalidad-2113-31-01.pdf}
\newline {\color{gray} (Emb: 0.677, TF-IDF: 0.477)}

Las y los Ministros de Estado y las y los Subsecretarios; 3. 
\newline {\color{gray} \textbf{1º:} 321-1-Iniciativa-Convencional-Constituyente-de-la-cc-Barbara-Sepulveda-sobre-Estatuto-de-Diputados-y-Diputadas.pdf}
\newline {\color{gray} (Emb: 0.788, TF-IDF: 0.629)}
\newline {\color{gray} \textbf{2º:} 239-1-Iniciativa-Convencional-de-la-cc-Tania-Madriaga-sobre-Poder-Ejecutivo-1146-hrs.pdf}
\newline {\color{gray} (Emb: 0.760, TF-IDF: 0.485)}

La Presidenta o Presidente de la República o quien lo sustituya en el ejercicio de la Presidencia al tiempo de la elección; 2. 
\newline {\color{gray} \textbf{1º:} 321-1-Iniciativa-Convencional-Constituyente-de-la-cc-Barbara-Sepulveda-sobre-Estatuto-de-Diputados-y-Diputadas.pdf}
\newline {\color{gray} (Emb: 1.000, TF-IDF: 1.000)}
\newline {\color{gray} \textbf{2º:} 937-IniciativaConvencional-Constituyente-del-cc-Tomas-Laibe-sobre-Banco-Central.pdf}
\newline {\color{gray} (Emb: 0.747, TF-IDF: 0.390)}

No pueden ser candidatos a diputadas o diputados ni a representante regional: 1. 
\newline {\color{gray} \textbf{1º:} 321-1-Iniciativa-Convencional-Constituyente-de-la-cc-Barbara-Sepulveda-sobre-Estatuto-de-Diputados-y-Diputadas.pdf}
\newline {\color{gray} (Emb: 0.809, TF-IDF: 0.326)}
\newline {\color{gray} \textbf{2º:} 230-2-Iniciativa-Convencional-de-la-cc-Alondra-Carrillo-sobre-Participacion-en-la-Democracia-1142-hrs.pdf}
\newline {\color{gray} (Emb: 0.699, TF-IDF: 0.324)}


\item \textbf{Artículo} \newline
Los cargos de diputadas o diputados y de representante regional son incompatibles entre sí y con otros cargos de representación y con todo empleo, función, comisión o cargo de carácter público o privado. 
\newline {\color{gray} \textbf{1º:} 240-1-Iniciativa-Convencional-de-la-cc-Tania-Madriaga-sobre-Poder-Legislativo-1146-hrs.pdf}
\newline {\color{gray} (Emb: 0.640, TF-IDF: 0.321)}
\newline {\color{gray} \textbf{2º:} 321-1-Iniciativa-Convencional-Constituyente-de-la-cc-Barbara-Sepulveda-sobre-Estatuto-de-Diputados-y-Diputadas.pdf}
\newline {\color{gray} (Emb: 0.607, TF-IDF: 0.306)}

Son también incompatibles con las funciones de directores o consejeros, aun cuando sean ad honorem, de entidades fiscales autónomas, semifiscales, y de empresas estatales o en las que el Estado tenga participación por aporte de capital. 
\newline {\color{gray} \textbf{1º:} 240-1-Iniciativa-Convencional-de-la-cc-Tania-Madriaga-sobre-Poder-Legislativo-1146-hrs.pdf}
\newline {\color{gray} (Emb: 0.974, TF-IDF: 0.848)}
\newline {\color{gray} \textbf{2º:} 240-1-Iniciativa-Convencional-de-la-cc-Tania-Madriaga-sobre-Poder-Legislativo-1146-hrs.pdf}
\newline {\color{gray} (Emb: 0.974, TF-IDF: 0.848)}

Por el solo hecho de su proclamación por el Tribunal Calificador de Elecciones, la diputada o diputado o representante regional cesará en el otro cargo, empleo, función o comisión incompatible que desempeñe. 
\newline {\color{gray} \textbf{1º:} 321-1-Iniciativa-Convencional-Constituyente-de-la-cc-Barbara-Sepulveda-sobre-Estatuto-de-Diputados-y-Diputadas.pdf}
\newline {\color{gray} (Emb: 0.854, TF-IDF: 0.825)}
\newline {\color{gray} \textbf{2º:} 711-Iniciativa-Convencional-Constituyente-de-la-cc-Ingrid-Villena-sobre-Tricel.pdf}
\newline {\color{gray} (Emb: 0.619, TF-IDF: 0.571)}


\item \textbf{Artículo} \newline
Las diputadas y diputados y las y los representantes regionales podrán ser reelegidos sucesivamente en el cargo hasta por un período. 
\newline {\color{gray} \textbf{1º:} 400-1-Iniciativa-Convencional-Constituyente-de-la-cc-Constanza-Hube-sobre-Servicio-y-Registro-Electoral-1905-24-01.pdf}
\newline {\color{gray} (Emb: 0.724, TF-IDF: 0.719)}
\newline {\color{gray} \textbf{2º:} 321-1-Iniciativa-Convencional-Constituyente-de-la-cc-Barbara-Sepulveda-sobre-Estatuto-de-Diputados-y-Diputadas.pdf}
\newline {\color{gray} (Emb: 0.691, TF-IDF: 0.592)}

Para estos efectos se entenderá que han ejercido su cargo durante un período cuando han cumplido más de la mitad de su mandato. 
\newline {\color{gray} \textbf{1º:} 400-1-Iniciativa-Convencional-Constituyente-de-la-cc-Constanza-Hube-sobre-Servicio-y-Registro-Electoral-1905-24-01.pdf}
\newline {\color{gray} (Emb: 0.827, TF-IDF: 0.929)}
\newline {\color{gray} \textbf{2º:} 184-6-c-Iniciativa-Convencional-del-cc-Rodrigo-Álvarez-que-crea-la-Corte-Constitucional-1044hrs.pdf}
\newline {\color{gray} (Emb: 0.454, TF-IDF: 0.265)}


\item \textbf{Artículo} \newline
El Congreso de Diputadas y Diputados y la Cámara de las Regiones tomarán sus decisiones por la mayoría de sus miembros presentes, salvo que esta Constitución disponga un quorum diferente. 
\newline {\color{gray} \textbf{1º:} 240-1-Iniciativa-Convencional-de-la-cc-Tania-Madriaga-sobre-Poder-Legislativo-1146-hrs.pdf}
\newline {\color{gray} (Emb: 0.651, TF-IDF: 0.339)}
\newline {\color{gray} \textbf{2º:} 118-3-c-Iniciativa-del-cc-Cristobal-Andrade-Regiones-Autonomas.pdf}
\newline {\color{gray} (Emb: 0.638, TF-IDF: 0.327)}

El Congreso de Diputadas y Diputados y la Cámara de las Regiones se renovarán en su totalidad cada cuatro años. 
\newline {\color{gray} \textbf{1º:} 400-1-Iniciativa-Convencional-Constituyente-de-la-cc-Constanza-Hube-sobre-Servicio-y-Registro-Electoral-1905-24-01.pdf}
\newline {\color{gray} (Emb: 0.903, TF-IDF: 0.705)}
\newline {\color{gray} \textbf{2º:} 81-1-Iniciativa-Convencional-Constituyente-del-cc-Martin-Arrau-y-otros.pdf}
\newline {\color{gray} (Emb: 0.840, TF-IDF: 0.639)}

La ley establecerá sus reglas de organización, funcionamiento y tramitación, la que podrá ser complementada con los reglamentos de funcionamiento que estos órganos dicten. 
\newline {\color{gray} \textbf{1º:} 751-Iniciativa-Convencional-Constituyente-del-cc-Raul-Celis-sobre-Fuerzas-Armadas-01-02.pdf}
\newline {\color{gray} (Emb: 0.757, TF-IDF: 0.388)}
\newline {\color{gray} \textbf{2º:} 752-Iniciativa-Convencional-Constituyente-del-cc-Raul-Celis-sobre-Fuerzas-Policiales-01-02.pdf}
\newline {\color{gray} (Emb: 0.757, TF-IDF: 0.320)}


\item \textbf{Artículo} \newline
El Congreso de Diputadas y Diputados no podrá entrar en sesión ni adoptar acuerdos sin la concurrencia de la tercera parte de sus miembros en ejercicio. 
\newline {\color{gray} \textbf{1º:} 240-1-Iniciativa-Convencional-de-la-cc-Tania-Madriaga-sobre-Poder-Legislativo-1146-hrs.pdf}
\newline {\color{gray} (Emb: 0.884, TF-IDF: 0.877)}
\newline {\color{gray} \textbf{2º:} 240-1-Iniciativa-Convencional-de-la-cc-Tania-Madriaga-sobre-Poder-Legislativo-1146-hrs.pdf}
\newline {\color{gray} (Emb: 0.833, TF-IDF: 0.765)}


\item \textbf{Artículo} \newline
La o el reemplazante deberá reunir los requisitos establecidos por esta Constitución para ser elegido en el cargo respectivo y le alcanzarán las inhabilidades establecidas en el artículo 14 y las incompatibilidades del artículo 15. 
\newline {\color{gray} \textbf{1º:} 325-6-Iniciativa-Convencional-del-cc-Tomas-Laibe-sobre-Jurisdiccion-Constitucional.pdf}
\newline {\color{gray} (Emb: 0.507, TF-IDF: 0.499)}
\newline {\color{gray} \textbf{2º:} 400-1-Iniciativa-Convencional-Constituyente-de-la-cc-Constanza-Hube-sobre-Servicio-y-Registro-Electoral-1905-24-01.pdf}
\newline {\color{gray} (Emb: 0.492, TF-IDF: 0.487)}

Se asegurará a todo evento la composición paritaria del órgano. 
\newline {\color{gray} \textbf{1º:} 220-6-c-Iniciativa-Convencional-del-cc-Daniel-Bravo-sobre-organizacion-de-tribunales-2315-hrs.pdf}
\newline {\color{gray} (Emb: 0.513, TF-IDF: 0.363)}
\newline {\color{gray} \textbf{2º:} 988-Iniciativa-Convencional-Constituyente-de-la-cc-Elisa-Loncon-sobre-Preambulo.pdf}
\newline {\color{gray} (Emb: 0.482, TF-IDF: 0.342)}


\item \textbf{Artículo} \newline
La Corte procederá conforme a lo dispuesto en el inciso anterior. 
\newline {\color{gray} \textbf{1º:} 440-Iniciativa-Convencional-Constituyente-del-cc-Felipe-Harboe-sobre-Principios-del-Debido-Proceso-1401-28-01.pdf}
\newline {\color{gray} (Emb: 0.701, TF-IDF: 0.449)}
\newline {\color{gray} \textbf{2º:} 514-4-Iniciativa-Convencional-Constituyente-del-cc-Felipe-Harboe-sobre-Derecho-a-la-Privacidad-1245-01-02.pdf}
\newline {\color{gray} (Emb: 0.701, TF-IDF: 0.446)}

Las diputadas, diputados y representantes regionales son inviolables por las opiniones que manifiesten y los votos que emitan en el desempeño de sus cargos. 
\newline {\color{gray} \textbf{1º:} 99-3-c-Iniciativa-de-la-cc-Tammy-Pustilnick-Disposiciones-del-Estado-Regional.pdf}
\newline {\color{gray} (Emb: 0.658, TF-IDF: 0.298)}
\newline {\color{gray} \textbf{2º:} 240-1-Iniciativa-Convencional-de-la-cc-Tania-Madriaga-sobre-Poder-Legislativo-1146-hrs.pdf}
\newline {\color{gray} (Emb: 0.630, TF-IDF: 0.286)}

Desde el día de su elección o investidura, ningún diputado, diputada o representante regional puede ser acusado o privado de libertad, salvo el caso de delito flagrante, si la Corte de Apelaciones de la jurisdicción respectiva, en pleno, no declara previamente haber lugar a la formación de causa. 
\newline {\color{gray} \textbf{1º:} 940-Iniciativa-Convencional-Constituyente-del-cc-Christian-Viera-sobre-Proteccion-de-Derechos-Fundamentales.pdf}
\newline {\color{gray} (Emb: 0.599, TF-IDF: 0.249)}
\newline {\color{gray} \textbf{2º:} 880-Iniciativa-Convencional-Constituyente-de-la-cc-Ingrid-Villena-sobre-Acciones-Constitucionales.pdf}
\newline {\color{gray} (Emb: 0.565, TF-IDF: 0.247)}

En contra de las resoluciones que al respecto dictaren estas Cortes podrá apelarse ante la Corte Suprema. 
\newline {\color{gray} \textbf{1º:} 143-4-c-Iniciativa-de-la-cc-Rocio-Cantuarias-Incorpora-Libertad-de-Trabajo-y-Sindical.pdf}
\newline {\color{gray} (Emb: 0.864, TF-IDF: 0.663)}
\newline {\color{gray} \textbf{2º:} 579-Iniciativa-Convencional-Constituyente-de-cc-Christian-Viera-sobre-Sistema-Electoral-y-Justicia-Electoral-2330-hrs.pdf}
\newline {\color{gray} (Emb: 0.723, TF-IDF: 0.346)}

En caso de que un diputado, diputada o representante regional sea detenido por delito flagrante, será puesto inmediatamente a disposición de la Corte de Apelaciones respectiva, con la información sumaria correspondiente. 
\newline {\color{gray} \textbf{1º:} 271-4-Iniciativa-Convencional-de-la-cc-Natalia-Henriquez-sobre-Libertad-Personal-17-01-1154-hrs.pdf}
\newline {\color{gray} (Emb: 0.646, TF-IDF: 0.347)}
\newline {\color{gray} \textbf{2º:} 172-6-c-Iniciativa-Convencional-del-cc-Rodrigo-Álvarez-sobre-Banco-central1044-hrs.pdf}
\newline {\color{gray} (Emb: 0.595, TF-IDF: 0.347)}

Desde el momento en que se declare, por resolución firme, haber lugar a formación de causa, el diputado, diputada o representante regional quedará suspendido de su cargo y sujeto al juez competente. 
\newline {\color{gray} \textbf{1º:} 98-6-Iniciativa-del-cc-Ruggero-Cozzi-Funcion-y-Principios-de-la-Jurisdiccion.pdf}
\newline {\color{gray} (Emb: 0.639, TF-IDF: 0.309)}
\newline {\color{gray} \textbf{2º:} 90-6-Iniciativa-Convencional-Constituyente-del-cc-Tomas-Laibe-y-otros.pdf}
\newline {\color{gray} (Emb: 0.635, TF-IDF: 0.261)}


\item \textbf{Artículo} \newline
Las diputadas, diputados y representantes regionales podrán renunciar a sus cargos cuando les afecte una enfermedad grave, debidamente acreditada, que les impida desempeñarlos, y así lo califique el tribunal que realice el control de constitucionalidad. 
\newline {\color{gray} \textbf{1º:} 560-Iniciativa-Convencional-Constituyente-de-cc-Andres-Cruz-sobre-Ministerio-Publico-2016-hrs.-01-02.pdf}
\newline {\color{gray} (Emb: 0.575, TF-IDF: 0.220)}
\newline {\color{gray} \textbf{2º:} 608-Iniciativa-Convencional-Constituyente-de-cc-Miguel-Angel-Botto-sobre-Ministerio-Publico-2119-hrs.-01-02.pdf}
\newline {\color{gray} (Emb: 0.550, TF-IDF: 0.192)}

f) Que, durante su ejercicio, pierda algún requisito general de elegibilidad, o incurra en una inhabilidad de las establecidas en el artículo 14. 
\newline {\color{gray} \textbf{1º:} 170-4-c-Iniciativa-Constituyente-de-la-cc-Marcela-Cubillos-sobre-Buen-Gobierno-839-hrs-1.pdf}
\newline {\color{gray} (Emb: 0.421, TF-IDF: 0.243)}
\newline {\color{gray} \textbf{2º:} 184-6-c-Iniciativa-Convencional-del-cc-Rodrigo-Álvarez-que-crea-la-Corte-Constitucional-1044hrs.pdf}
\newline {\color{gray} (Emb: 0.398, TF-IDF: 0.240)}

Esta inhabilidad tendrá lugar sea que la diputada, diputado o representante regional actúe por sí o por interpósita persona, natural o jurídica; d) Que, durante su ejercicio, actúe como abogada o abogado o mandataria o mandatario en cualquier clase de juicio, que ejercite cualquier influencia ante las autoridades administrativas o judiciales en favor o representación del empleador o de las y los trabajadores en negociaciones o conflictos laborales, sean del sector público o privado, o que intervenga en ellos ante cualquiera de las partes; e) Que haya infringido gravemente las normas sobre transparencia, límites y control del gasto electoral, desde la fecha que lo declare por sentencia firme el Tribunal Calificador de Elecciones, a requerimiento del Consejo Directivo del Servicio Electoral. 
\newline {\color{gray} \textbf{1º:} 216-1-c-Iniciativa-Convencional-de-la-cc-Rosa-Catrileo-sobre-Sistema-de-Gobierno-2126-hrs.pdf}
\newline {\color{gray} (Emb: 0.617, TF-IDF: 0.231)}
\newline {\color{gray} \textbf{2º:} 652-Iniciativa-Convencional-Constituyente-de-la-cc-Ericka-Portilla-sobre-Trabajo-Decente-151101-02.pdf}
\newline {\color{gray} (Emb: 0.601, TF-IDF: 0.230)}

Cesará en el cargo la diputada, diputado o representante regional: b) Que se ausentare del país por más de treinta días sin permiso de la corporación respectiva o, en receso de ésta, de su Mesa Directiva; c) Que, durante su ejercicio, celebrare o caucionare contratos con el Estado, o actuare como procuradora o procurador o agente en gestiones particulares de carácter administrativo, en la provisión de empleos públicos, consejerías, funciones o comisiones de similar naturaleza. 
\newline {\color{gray} \textbf{1º:} 239-1-Iniciativa-Convencional-de-la-cc-Tania-Madriaga-sobre-Poder-Ejecutivo-1146-hrs.pdf}
\newline {\color{gray} (Emb: 0.583, TF-IDF: 0.321)}
\newline {\color{gray} \textbf{2º:} 216-1-c-Iniciativa-Convencional-de-la-cc-Rosa-Catrileo-sobre-Sistema-de-Gobierno-2126-hrs.pdf}
\newline {\color{gray} (Emb: 0.572, TF-IDF: 0.288)}

Una ley señalará los casos en que existe una infracción grave. 
\newline {\color{gray} \textbf{1º:} 784-niciativa-Convencional-Constituyente-de-la-cc-Damaris-Abarca-sobre-Derecho-a-vivir-en-un-ambiente-sano.pdf}
\newline {\color{gray} (Emb: 0.765, TF-IDF: 0.252)}
\newline {\color{gray} \textbf{2º:} 131-4-c-Iniciativa-de-la-cc-Rocio-Cantuarias-Establece-la-Libertad-Personal-y-la-Seguridad-Individual.pdf}
\newline {\color{gray} (Emb: 0.689, TF-IDF: 0.221)}


\item \textbf{Artículo} \newline
Fijar las bases de los procedimientos que rigen los actos de la administración pública; n. 
\newline {\color{gray} \textbf{1º:} 807-Iniciativa-Convencional-Constituyente-del-cc-Jaime-Bassa-sobre-Formacion-de-la-Ley.pdf}
\newline {\color{gray} (Emb: 0.951, TF-IDF: 0.846)}
\newline {\color{gray} \textbf{2º:} 224-1-Iniciativa-Convencional-del-cc-Jaime-Bassa-sobre-Formacion-de-la-Ley-1139-17-01.pdf}
\newline {\color{gray} (Emb: 0.951, TF-IDF: 0.846)}

Singularizar la ciudad en que debe residir la Presidenta o el Presidente de la República, celebrar sus sesiones el Congreso de Diputadas y Diputados y la Cámara de las Regiones y funcionar la Corte Suprema; l. 
\newline {\color{gray} \textbf{1º:} 807-Iniciativa-Convencional-Constituyente-del-cc-Jaime-Bassa-sobre-Formacion-de-la-Ley.pdf}
\newline {\color{gray} (Emb: 0.901, TF-IDF: 0.851)}
\newline {\color{gray} \textbf{2º:} 224-1-Iniciativa-Convencional-del-cc-Jaime-Bassa-sobre-Formacion-de-la-Ley-1139-17-01.pdf}
\newline {\color{gray} (Emb: 0.901, TF-IDF: 0.851)}

Establecer la creación y modificación de servicios públicos y empleos públicos, sean fiscales, semifiscales, autónomos o de las empresas del Estado, y determinar sus funciones y atribuciones; ñ. 
\newline {\color{gray} \textbf{1º:} 240-1-Iniciativa-Convencional-de-la-cc-Tania-Madriaga-sobre-Poder-Legislativo-1146-hrs.pdf}
\newline {\color{gray} (Emb: 0.841, TF-IDF: 0.656)}
\newline {\color{gray} \textbf{2º:} 671-Iniciativa-Convencional-Constituyente-del-cc-Bernardo-Fontaine-Formacion-de-la-Ley-121101-02.pdf}
\newline {\color{gray} (Emb: 0.807, TF-IDF: 0.656)}

Establecer el sistema de determinación de las remuneraciones de la Presidenta o Presidente de la República y las Ministras o Ministros de Estado, de las diputadas y diputados, las gobernadoras y gobernadores y de las y los representantes regionales; k. 
\newline {\color{gray} \textbf{1º:} 224-1-Iniciativa-Convencional-del-cc-Jaime-Bassa-sobre-Formacion-de-la-Ley-1139-17-01.pdf}
\newline {\color{gray} (Emb: 0.813, TF-IDF: 0.753)}
\newline {\color{gray} \textbf{2º:} 807-Iniciativa-Convencional-Constituyente-del-cc-Jaime-Bassa-sobre-Formacion-de-la-Ley.pdf}
\newline {\color{gray} (Emb: 0.813, TF-IDF: 0.753)}

Crear loterías y apuestas; p. 
\newline {\color{gray} \textbf{1º:} 240-1-Iniciativa-Convencional-de-la-cc-Tania-Madriaga-sobre-Poder-Legislativo-1146-hrs.pdf}
\newline {\color{gray} (Emb: 0.570, TF-IDF: 0.404)}
\newline {\color{gray} \textbf{2º:} 179-6-c-Iniciativa-Convencional-DEL-CC-Rodrigo-Álvarez-Ministerio-Público-1044-hrs.pdf}
\newline {\color{gray} (Emb: 0.430, TF-IDF: 0.393)}

Regular aquellas materias que la Constitución señale como leyes de concurrencia presidencial necesaria, y q. 
\newline {\color{gray} \textbf{1º:} 807-Iniciativa-Convencional-Constituyente-del-cc-Jaime-Bassa-sobre-Formacion-de-la-Ley.pdf}
\newline {\color{gray} (Emb: 0.685, TF-IDF: 0.697)}
\newline {\color{gray} \textbf{2º:} 472-6-Iniciativa-Convencional-Constituyente-del-cc-Daniel-Bravo-sobre-Corte-Constitucional-2003-31-01.pdf}
\newline {\color{gray} (Emb: 0.657, TF-IDF: 0.478)}

Regular las demás materias que la Constitución exija que sean establecidas por una ley. 
\newline {\color{gray} \textbf{1º:} 151-3-c-Iniciativa-de-la-cc-Angelica-Tepper-Competencias-de-los-Gobiernos-Regionales.pdf}
\newline {\color{gray} (Emb: 0.811, TF-IDF: 0.642)}
\newline {\color{gray} \textbf{2º:} 151-3-c-Iniciativa-de-la-cc-Angelica-Tepper-Competencias-de-los-Gobiernos-Regionales.pdf}
\newline {\color{gray} (Emb: 0.811, TF-IDF: 0.425)}

Establecer el régimen jurídico aplicable en materia laboral, sindical, de la huelga y la negociación colectiva en sus diversas manifestaciones, previsional y de seguridad social; o. 
\newline {\color{gray} \textbf{1º:} 807-Iniciativa-Convencional-Constituyente-del-cc-Jaime-Bassa-sobre-Formacion-de-la-Ley.pdf}
\newline {\color{gray} (Emb: 0.938, TF-IDF: 0.876)}
\newline {\color{gray} \textbf{2º:} 240-1-Iniciativa-Convencional-de-la-cc-Tania-Madriaga-sobre-Poder-Legislativo-1146-hrs.pdf}
\newline {\color{gray} (Emb: 0.675, TF-IDF: 0.581)}

Conceder indultos generales y amnistías, salvo en crímenes de lesa humanidad; i. 
\newline {\color{gray} \textbf{1º:} 224-1-Iniciativa-Convencional-del-cc-Jaime-Bassa-sobre-Formacion-de-la-Ley-1139-17-01.pdf}
\newline {\color{gray} (Emb: 0.805, TF-IDF: 0.769)}
\newline {\color{gray} \textbf{2º:} 807-Iniciativa-Convencional-Constituyente-del-cc-Jaime-Bassa-sobre-Formacion-de-la-Ley.pdf}
\newline {\color{gray} (Emb: 0.805, TF-IDF: 0.769)}

Autorizar la declaración de guerra, a propuesta de la Presidenta o Presidente de la República; m. 
\newline {\color{gray} \textbf{1º:} 807-Iniciativa-Convencional-Constituyente-del-cc-Jaime-Bassa-sobre-Formacion-de-la-Ley.pdf}
\newline {\color{gray} (Emb: 1.000, TF-IDF: 1.000)}
\newline {\color{gray} \textbf{2º:} 224-1-Iniciativa-Convencional-del-cc-Jaime-Bassa-sobre-Formacion-de-la-Ley-1139-17-01.pdf}
\newline {\color{gray} (Emb: 1.000, TF-IDF: 1.000)}

Establecer o modificar la división político o administrativa del país; g. 
\newline {\color{gray} \textbf{1º:} 224-1-Iniciativa-Convencional-del-cc-Jaime-Bassa-sobre-Formacion-de-la-Ley-1139-17-01.pdf}
\newline {\color{gray} (Emb: 0.999, TF-IDF: 0.899)}
\newline {\color{gray} \textbf{2º:} 807-Iniciativa-Convencional-Constituyente-del-cc-Jaime-Bassa-sobre-Formacion-de-la-Ley.pdf}
\newline {\color{gray} (Emb: 0.999, TF-IDF: 0.899)}

Sólo en virtud de una ley se puede: a. 
\newline {\color{gray} \textbf{1º:} 807-Iniciativa-Convencional-Constituyente-del-cc-Jaime-Bassa-sobre-Formacion-de-la-Ley.pdf}
\newline {\color{gray} (Emb: 1.000, TF-IDF: 1.000)}
\newline {\color{gray} \textbf{2º:} 224-1-Iniciativa-Convencional-del-cc-Jaime-Bassa-sobre-Formacion-de-la-Ley-1139-17-01.pdf}
\newline {\color{gray} (Emb: 0.948, TF-IDF: 0.405)}

Crear, modificar y suprimir tributos de cualquier clase o naturaleza y los beneficios tributarios aplicables a éstos, determinar su progresión, exenciones y proporcionalidad, sin perjuicio de las excepciones que establezca esta Constitución; b. 
\newline {\color{gray} \textbf{1º:} 224-1-Iniciativa-Convencional-del-cc-Jaime-Bassa-sobre-Formacion-de-la-Ley-1139-17-01.pdf}
\newline {\color{gray} (Emb: 0.721, TF-IDF: 0.606)}
\newline {\color{gray} \textbf{2º:} 119-3-c-Iniciativa-de-la-cc-Tammy-Pustilnick-competencias-de-las-Regiones-Autonomas.pdf}
\newline {\color{gray} (Emb: 0.681, TF-IDF: 0.512)}

Autorizar la contratación de empréstitos y otras operaciones que puedan comprometer el crédito y la responsabilidad financiera del Estado, sus organismos y municipalidades, sin perjuicio de lo consagrado respecto de las entidades territoriales y de lo establecido en la letra siguiente. 
\newline {\color{gray} \textbf{1º:} 807-Iniciativa-Convencional-Constituyente-del-cc-Jaime-Bassa-sobre-Formacion-de-la-Ley.pdf}
\newline {\color{gray} (Emb: 0.929, TF-IDF: 0.887)}
\newline {\color{gray} \textbf{2º:} 224-1-Iniciativa-Convencional-del-cc-Jaime-Bassa-sobre-Formacion-de-la-Ley-1139-17-01.pdf}
\newline {\color{gray} (Emb: 0.860, TF-IDF: 0.750)}

Esta disposición no se aplicará al Banco Central; c. 
\newline {\color{gray} \textbf{1º:} 240-1-Iniciativa-Convencional-de-la-cc-Tania-Madriaga-sobre-Poder-Legislativo-1146-hrs.pdf}
\newline {\color{gray} (Emb: 0.713, TF-IDF: 0.380)}
\newline {\color{gray} \textbf{2º:} 643-Iniciativa-Convencional-Constituyente-del-cc-Marcos-Barraza-sobre-Banco-Central-174201-02.pdf}
\newline {\color{gray} (Emb: 0.533, TF-IDF: 0.348)}

Señalar el valor, tipo y denominación de las monedas, y el sistema de pesos y medidas; h. 
\newline {\color{gray} \textbf{1º:} 224-1-Iniciativa-Convencional-del-cc-Jaime-Bassa-sobre-Formacion-de-la-Ley-1139-17-01.pdf}
\newline {\color{gray} (Emb: 1.000, TF-IDF: 1.000)}
\newline {\color{gray} \textbf{2º:} 807-Iniciativa-Convencional-Constituyente-del-cc-Jaime-Bassa-sobre-Formacion-de-la-Ley.pdf}
\newline {\color{gray} (Emb: 1.000, TF-IDF: 1.000)}

Instituir las normas sobre enajenación de bienes del Estado, los gobiernos regionales o de las municipalidades y sobre su arrendamiento, títulos habilitantes para su uso o explotación, y concesión; e. 
\newline {\color{gray} \textbf{1º:} 224-1-Iniciativa-Convencional-del-cc-Jaime-Bassa-sobre-Formacion-de-la-Ley-1139-17-01.pdf}
\newline {\color{gray} (Emb: 0.928, TF-IDF: 0.770)}
\newline {\color{gray} \textbf{2º:} 807-Iniciativa-Convencional-Constituyente-del-cc-Jaime-Bassa-sobre-Formacion-de-la-Ley.pdf}
\newline {\color{gray} (Emb: 0.928, TF-IDF: 0.770)}

Establecer las condiciones y reglas conforme a las cuales las universidades y las empresas del Estado y aquellas en que éste tenga participación puedan contratar empréstitos, los que en ningún caso podrán efectuarse con el Estado, sus organismos y empresas; d. 
\newline {\color{gray} \textbf{1º:} 807-Iniciativa-Convencional-Constituyente-del-cc-Jaime-Bassa-sobre-Formacion-de-la-Ley.pdf}
\newline {\color{gray} (Emb: 0.998, TF-IDF: 0.999)}
\newline {\color{gray} \textbf{2º:} 224-1-Iniciativa-Convencional-del-cc-Jaime-Bassa-sobre-Formacion-de-la-Ley-1139-17-01.pdf}
\newline {\color{gray} (Emb: 0.861, TF-IDF: 0.921)}

Disponer, organizar y distribuir las Fuerzas Armadas para su desarrollo y empleo conjunto, así como permitir la entrada de tropas extranjeras en el territorio de la República, como, asimismo, la salida de tropas nacionales fuera de él; f. 
\newline {\color{gray} \textbf{1º:} 807-Iniciativa-Convencional-Constituyente-del-cc-Jaime-Bassa-sobre-Formacion-de-la-Ley.pdf}
\newline {\color{gray} (Emb: 0.646, TF-IDF: 0.680)}
\newline {\color{gray} \textbf{2º:} 224-1-Iniciativa-Convencional-del-cc-Jaime-Bassa-sobre-Formacion-de-la-Ley-1139-17-01.pdf}
\newline {\color{gray} (Emb: 0.646, TF-IDF: 0.643)}


\item \textbf{Artículo} \newline
La Presidenta o Presidente de la República tendrá la potestad de dictar aquellos reglamentos, decretos e instrucciones que crea necesarios para la ejecución de las leyes. 
\newline {\color{gray} \textbf{1º:} 807-Iniciativa-Convencional-Constituyente-del-cc-Jaime-Bassa-sobre-Formacion-de-la-Ley.pdf}
\newline {\color{gray} (Emb: 0.970, TF-IDF: 0.881)}
\newline {\color{gray} \textbf{2º:} 239-1-Iniciativa-Convencional-de-la-cc-Tania-Madriaga-sobre-Poder-Ejecutivo-1146-hrs.pdf}
\newline {\color{gray} (Emb: 0.801, TF-IDF: 0.622)}


\item \textbf{Artículo} \newline
La Presidenta o Presidente de la República podrá ejercer la potestad reglamentaria en todas aquellas materias que no estén comprendidas en el artículo 22. 
\newline {\color{gray} \textbf{1º:} 807-Iniciativa-Convencional-Constituyente-del-cc-Jaime-Bassa-sobre-Formacion-de-la-Ley.pdf}
\newline {\color{gray} (Emb: 0.611, TF-IDF: 0.557)}
\newline {\color{gray} \textbf{2º:} 349-6-Iniciativa-Convencional-Constituyente-del-cc-Luis-Mayol-sobre-Banco-Central-1826-20-01.pdf}
\newline {\color{gray} (Emb: 0.606, TF-IDF: 0.557)}

Cuando sobre una materia no comprendida en los literales del artículo 22 sean aplicables reglas de rango legal y reglamentario, primará la ley. 
\newline {\color{gray} \textbf{1º:} 89-6-Iniciativa-Convencional-Constituyente-del-cc-Christian-Viera-y-otros.pdf}
\newline {\color{gray} (Emb: 0.590, TF-IDF: 0.298)}
\newline {\color{gray} \textbf{2º:} 703-Iniciativa-Convencional-Constituyente-de-la-cc-Carolina-Videla-sobre-Personas-Privadas-de-Libertad.pdf}
\newline {\color{gray} (Emb: 0.529, TF-IDF: 0.233)}

La Presidenta o Presidente deberá informar mensualmente al Congreso sobre los reglamentos, decretos e instrucciones que se hayan dictado en virtud de este artículo. 
\newline {\color{gray} \textbf{1º:} 807-Iniciativa-Convencional-Constituyente-del-cc-Jaime-Bassa-sobre-Formacion-de-la-Ley.pdf}
\newline {\color{gray} (Emb: 0.628, TF-IDF: 0.670)}
\newline {\color{gray} \textbf{2º:} 239-1-Iniciativa-Convencional-de-la-cc-Tania-Madriaga-sobre-Poder-Ejecutivo-1146-hrs.pdf}
\newline {\color{gray} (Emb: 0.626, TF-IDF: 0.290)}


\item \textbf{Artículo} \newline
La ley delegatoria de potestades que corresponda a leyes de acuerdo regional es ley de acuerdo regional. 
\newline {\color{gray} \textbf{1º:} 325-6-Iniciativa-Convencional-del-cc-Tomas-Laibe-sobre-Jurisdiccion-Constitucional.pdf}
\newline {\color{gray} (Emb: 0.801, TF-IDF: 0.341)}
\newline {\color{gray} \textbf{2º:} 120-3-c-Iniciativa-de-la-cc-Tammy-Pustilnick-atribuciones-exclusivas-de-la-Asamblea-Regional.pdf}
\newline {\color{gray} (Emb: 0.778, TF-IDF: 0.263)}

Los decretos con fuerza de ley estarán sometidos en cuanto a su publicación, vigencia y efectos, a las mismas normas que rigen para la ley. 
\newline {\color{gray} \textbf{1º:} 224-1-Iniciativa-Convencional-del-cc-Jaime-Bassa-sobre-Formacion-de-la-Ley-1139-17-01.pdf}
\newline {\color{gray} (Emb: 1.000, TF-IDF: 1.000)}
\newline {\color{gray} \textbf{2º:} 807-Iniciativa-Convencional-Constituyente-del-cc-Jaime-Bassa-sobre-Formacion-de-la-Ley.pdf}
\newline {\color{gray} (Emb: 1.000, TF-IDF: 1.000)}

A la Contraloría General de la República corresponderá tomar razón de estos decretos con fuerza de ley, debiendo rechazarlos cuando ellos excedan o contravengan la autorización referida. 
\newline {\color{gray} \textbf{1º:} 224-1-Iniciativa-Convencional-del-cc-Jaime-Bassa-sobre-Formacion-de-la-Ley-1139-17-01.pdf}
\newline {\color{gray} (Emb: 1.000, TF-IDF: 1.000)}
\newline {\color{gray} \textbf{2º:} 807-Iniciativa-Convencional-Constituyente-del-cc-Jaime-Bassa-sobre-Formacion-de-la-Ley.pdf}
\newline {\color{gray} (Emb: 0.966, TF-IDF: 0.978)}

En ejercicio de esta facultad, podrá introducirle los cambios de forma que sean indispensables, sin alterar, en caso alguno, su verdadero sentido y alcance. 
\newline {\color{gray} \textbf{1º:} 807-Iniciativa-Convencional-Constituyente-del-cc-Jaime-Bassa-sobre-Formacion-de-la-Ley.pdf}
\newline {\color{gray} (Emb: 0.999, TF-IDF: 0.929)}
\newline {\color{gray} \textbf{2º:} 224-1-Iniciativa-Convencional-del-cc-Jaime-Bassa-sobre-Formacion-de-la-Ley-1139-17-01.pdf}
\newline {\color{gray} (Emb: 0.999, TF-IDF: 0.929)}

La autorización nunca podrá comprender facultades que afecten a la organización, atribuciones y régimen de los funcionarios del Sistema de Justicia, del Congreso de Diputadas y Diputados, de la Cámara de las Regiones, de la Corte Constitucional, ni de la Contraloría General de la República. 
\newline {\color{gray} \textbf{1º:} 750-Iniciativa-Convencional-Constituyente-del-cc-Raul-Celis-sobre-Estados-de-Excepcion-01-02.pdf}
\newline {\color{gray} (Emb: 0.623, TF-IDF: 0.366)}
\newline {\color{gray} \textbf{2º:} 180-6-c-Iniciativa-Convencional-del-cc-Rodrigo-Álvarez-que-regula-el-Poder-Judicial-1044-hrs.pdf}
\newline {\color{gray} (Emb: 0.622, TF-IDF: 0.363)}

La ley que otorgue la referida autorización señalará las materias precisas sobre las que recaerá la delegación y podrá establecer o determinar las limitaciones, restricciones y formalidades que se estimen convenientes. 
\newline {\color{gray} \textbf{1º:} 224-1-Iniciativa-Convencional-del-cc-Jaime-Bassa-sobre-Formacion-de-la-Ley-1139-17-01.pdf}
\newline {\color{gray} (Emb: 1.000, TF-IDF: 1.000)}
\newline {\color{gray} \textbf{2º:} 807-Iniciativa-Convencional-Constituyente-del-cc-Jaime-Bassa-sobre-Formacion-de-la-Ley.pdf}
\newline {\color{gray} (Emb: 0.991, TF-IDF: 0.944)}

Esta autorización no podrá extenderse a derechos fundamentales, nacionalidad, ciudadanía, elecciones y plebiscitos. 
\newline {\color{gray} \textbf{1º:} 807-Iniciativa-Convencional-Constituyente-del-cc-Jaime-Bassa-sobre-Formacion-de-la-Ley.pdf}
\newline {\color{gray} (Emb: 0.910, TF-IDF: 0.870)}
\newline {\color{gray} \textbf{2º:} 224-1-Iniciativa-Convencional-del-cc-Jaime-Bassa-sobre-Formacion-de-la-Ley-1139-17-01.pdf}
\newline {\color{gray} (Emb: 0.719, TF-IDF: 0.490)}

La Presidenta o Presidente de la República podrá solicitar autorización al Congreso de Diputadas y Diputados para dictar decretos con fuerza de ley durante un plazo no superior a un año sobre materias que correspondan al dominio de la ley. 
\newline {\color{gray} \textbf{1º:} 807-Iniciativa-Convencional-Constituyente-del-cc-Jaime-Bassa-sobre-Formacion-de-la-Ley.pdf}
\newline {\color{gray} (Emb: 0.834, TF-IDF: 0.794)}
\newline {\color{gray} \textbf{2º:} 807-Iniciativa-Convencional-Constituyente-del-cc-Jaime-Bassa-sobre-Formacion-de-la-Ley.pdf}
\newline {\color{gray} (Emb: 0.712, TF-IDF: 0.782)}

Sin perjuicio de lo dispuesto en los incisos anteriores, la Presidenta o Presidente de la República queda autorizado para fijar el texto refundido, coordinado y sistematizado de las leyes cuando sea conveniente para su mejor ejecución. 
\newline {\color{gray} \textbf{1º:} 224-1-Iniciativa-Convencional-del-cc-Jaime-Bassa-sobre-Formacion-de-la-Ley-1139-17-01.pdf}
\newline {\color{gray} (Emb: 1.000, TF-IDF: 1.000)}
\newline {\color{gray} \textbf{2º:} 807-Iniciativa-Convencional-Constituyente-del-cc-Jaime-Bassa-sobre-Formacion-de-la-Ley.pdf}
\newline {\color{gray} (Emb: 0.971, TF-IDF: 0.856)}


\item \textbf{Artículo} \newline
Las que impongan, supriman, reduzcan o condonen tributos de cualquier clase o naturaleza, establezcan exenciones o modifiquen las existentes, y determinen su forma, proporcionalidad o progresión; e. 
\newline {\color{gray} \textbf{1º:} 807-Iniciativa-Convencional-Constituyente-del-cc-Jaime-Bassa-sobre-Formacion-de-la-Ley.pdf}
\newline {\color{gray} (Emb: 0.998, TF-IDF: 0.965)}
\newline {\color{gray} \textbf{2º:} 240-1-Iniciativa-Convencional-de-la-cc-Tania-Madriaga-sobre-Poder-Legislativo-1146-hrs.pdf}
\newline {\color{gray} (Emb: 0.860, TF-IDF: 0.583)}

Las que dispongan, organicen y distribuyan las Fuerzas Armadas para su desarrollo y empleo conjunto. 
\newline {\color{gray} \textbf{1º:} 71-2-Iniciativa-Convencional-Constitutente-de-Maria-Jose-Oyarzun-y-otros.pdf}
\newline {\color{gray} (Emb: 0.584, TF-IDF: 0.440)}
\newline {\color{gray} \textbf{2º:} 720-Iniciativa-Convencional-Constituyente-de-la-cc-Maria-Magdalena-Rivera-sobre-Poderes-del-Estado.pdf}
\newline {\color{gray} (Emb: 0.531, TF-IDF: 0.355)}

Las que alteren la división política o administrativa del país; d. 
\newline {\color{gray} \textbf{1º:} 807-Iniciativa-Convencional-Constituyente-del-cc-Jaime-Bassa-sobre-Formacion-de-la-Ley.pdf}
\newline {\color{gray} (Emb: 0.937, TF-IDF: 0.853)}
\newline {\color{gray} \textbf{2º:} 807-Iniciativa-Convencional-Constituyente-del-cc-Jaime-Bassa-sobre-Formacion-de-la-Ley.pdf}
\newline {\color{gray} (Emb: 0.901, TF-IDF: 0.531)}

Las que contraten o autoricen a contratar empréstitos o celebrar cualquier otra clase de operaciones que puedan comprometer la responsabilidad patrimonial del Estado, de las entidades semifiscales, autónomas y condonar, reducir o modificar obligaciones, intereses u otras cargas financieras de cualquier naturaleza establecidas en favor del Fisco o de los organismos o entidades referidos, sin perjuicio de lo dispuesto en el artículo 22, letra c, y f. 
\newline {\color{gray} \textbf{1º:} 807-Iniciativa-Convencional-Constituyente-del-cc-Jaime-Bassa-sobre-Formacion-de-la-Ley.pdf}
\newline {\color{gray} (Emb: 0.922, TF-IDF: 0.926)}
\newline {\color{gray} \textbf{2º:} 671-Iniciativa-Convencional-Constituyente-del-cc-Bernardo-Fontaine-Formacion-de-la-Ley-121101-02.pdf}
\newline {\color{gray} (Emb: 0.750, TF-IDF: 0.826)}

Las que irroguen directamente gastos al Estado; b. 
\newline {\color{gray} \textbf{1º:} 807-Iniciativa-Convencional-Constituyente-del-cc-Jaime-Bassa-sobre-Formacion-de-la-Ley.pdf}
\newline {\color{gray} (Emb: 0.888, TF-IDF: 0.922)}
\newline {\color{gray} \textbf{2º:} 633-3-Iniciativa-Convencional-Constituyente-de-la-Yarela-Gomez-sobre-Regimen-Tributario-1739-01-02.pdf}
\newline {\color{gray} (Emb: 0.502, TF-IDF: 0.423)}

Las leyes relacionadas con la administración presupuestaria del Estado, incluyendo las modificaciones de la Ley de Presupuestos; c. 
\newline {\color{gray} \textbf{1º:} 240-1-Iniciativa-Convencional-de-la-cc-Tania-Madriaga-sobre-Poder-Legislativo-1146-hrs.pdf}
\newline {\color{gray} (Emb: 0.868, TF-IDF: 0.623)}
\newline {\color{gray} \textbf{2º:} 671-Iniciativa-Convencional-Constituyente-del-cc-Bernardo-Fontaine-Formacion-de-la-Ley-121101-02.pdf}
\newline {\color{gray} (Emb: 0.622, TF-IDF: 0.487)}

Son leyes de concurrencia presidencial necesaria: a. 
\newline {\color{gray} \textbf{1º:} 807-Iniciativa-Convencional-Constituyente-del-cc-Jaime-Bassa-sobre-Formacion-de-la-Ley.pdf}
\newline {\color{gray} (Emb: 1.000, TF-IDF: 1.000)}
\newline {\color{gray} \textbf{2º:} 807-Iniciativa-Convencional-Constituyente-del-cc-Jaime-Bassa-sobre-Formacion-de-la-Ley.pdf}
\newline {\color{gray} (Emb: 0.672, TF-IDF: 0.622)}


\item \textbf{Artículo} \newline
Las mociones parlamentarias que correspondan a materias de concurrencia presidencial necesaria deberán presentarse con una estimación de gastos y origen del financiamiento. 
\newline {\color{gray} \textbf{1º:} 322-1-Iniciativa-Convencional-Constituyente-de-la-cc-Barbara-Sepulveda-sobre-Formacion-de-la-Ley.pdf}
\newline {\color{gray} (Emb: 0.565, TF-IDF: 0.434)}
\newline {\color{gray} \textbf{2º:} 134-4-c-Iniciativa-de-la-cc-Rocio-Cantuarias-Establece-la-libertad-de-ejercer-actividades-economicas-1.pdf}
\newline {\color{gray} (Emb: 0.558, TF-IDF: 0.378)}

En dicho caso, la tramitación del proyecto no podrá continuar. 
\newline {\color{gray} \textbf{1º:} 788-Iniciativa-Convencional-Constituyente-de-la-cc-Camila-Zarate-sobre-Democracia-Ecologica.pdf}
\newline {\color{gray} (Emb: 0.731, TF-IDF: 0.408)}
\newline {\color{gray} \textbf{2º:} 788-Iniciativa-Convencional-Constituyente-de-la-cc-Camila-Zarate-sobre-Democracia-Ecologica.pdf}
\newline {\color{gray} (Emb: 0.577, TF-IDF: 0.336)}

La Presidenta o Presidente de la República siempre podrá retirar su patrocinio. 
\newline {\color{gray} \textbf{1º:} 377-2-Iniciativa-Convencional-Constituyente-del-cc-Alvin-Saldana-sobre-Mecanismos-de-Democracia-1045-hrs-24-01.pdf}
\newline {\color{gray} (Emb: 0.770, TF-IDF: 0.495)}
\newline {\color{gray} \textbf{2º:} 353-1-Iniciativa-Convencional-Constituyente-del-cc-Jaime-Bassa-sobre-Sistema-de-Gobierno-1158-21-01.pdf}
\newline {\color{gray} (Emb: 0.706, TF-IDF: 0.495)}

Transcurrido ese plazo sin el patrocinio correspondiente, el proyecto se entenderá desechado y no se podrá insistir en su tramitación. 
\newline {\color{gray} \textbf{1º:} 574-Iniciativa-Convencional-Constituyente-de-cc-Vanessa-Hoppe-sobre-Defensoria-de-los-Pueblos-2351-hrs.-01-02.pdf}
\newline {\color{gray} (Emb: 0.564, TF-IDF: 0.335)}
\newline {\color{gray} \textbf{2º:} 788-Iniciativa-Convencional-Constituyente-de-la-cc-Camila-Zarate-sobre-Democracia-Ecologica.pdf}
\newline {\color{gray} (Emb: 0.533, TF-IDF: 0.316)}

La moción parlamentaria deberá ser patrocinada por no menos de un cuarto y no más de un tercio de las diputadas y diputados o, en su caso, de los representantes regionales en ejercicio, y deberá declarar que se trata de un proyecto de ley de concurrencia necesaria de la Presidencia. 
\newline {\color{gray} \textbf{1º:} 807-Iniciativa-Convencional-Constituyente-del-cc-Jaime-Bassa-sobre-Formacion-de-la-Ley.pdf}
\newline {\color{gray} (Emb: 0.904, TF-IDF: 0.957)}
\newline {\color{gray} \textbf{2º:} 120-3-c-Iniciativa-de-la-cc-Tammy-Pustilnick-atribuciones-exclusivas-de-la-Asamblea-Regional.pdf}
\newline {\color{gray} (Emb: 0.658, TF-IDF: 0.352)}

Las leyes de concurrencia presidencial necesaria sólo podrán ser aprobadas si la Presidenta o Presidente de la República entrega su patrocinio durante la tramitación del proyecto. 
\newline {\color{gray} \textbf{1º:} 807-Iniciativa-Convencional-Constituyente-del-cc-Jaime-Bassa-sobre-Formacion-de-la-Ley.pdf}
\newline {\color{gray} (Emb: 1.000, TF-IDF: 1.000)}
\newline {\color{gray} \textbf{2º:} 322-1-Iniciativa-Convencional-Constituyente-de-la-cc-Barbara-Sepulveda-sobre-Formacion-de-la-Ley.pdf}
\newline {\color{gray} (Emb: 0.792, TF-IDF: 0.500)}

Las mociones de concurrencia presidencial necesaria deberán presentarse acompañadas de un informe técnico financiero de la Secretaría de Presupuestos. 
\newline {\color{gray} \textbf{1º:} 970-Iniciativa-Convencional-Consttuyente-del-cc-Nicolas-Nunez-sobre-Presupuestos.pdf}
\newline {\color{gray} (Emb: 0.539, TF-IDF: 0.427)}
\newline {\color{gray} \textbf{2º:} 176-2-c-Iniciativa-Convencional-del-cc-Rodrigo-Álvarez-sobre-Responsabilidad-Fiscal-1044-hrs.pdf}
\newline {\color{gray} (Emb: 0.538, TF-IDF: 0.311)}

La Presidenta o Presidente de la República podrá patrocinar el proyecto de ley en cualquier momento hasta transcurridos quince días desde que haya sido despachado por la Comisión respectiva. 
\newline {\color{gray} \textbf{1º:} 322-1-Iniciativa-Convencional-Constituyente-de-la-cc-Barbara-Sepulveda-sobre-Formacion-de-la-Ley.pdf}
\newline {\color{gray} (Emb: 0.848, TF-IDF: 0.348)}
\newline {\color{gray} \textbf{2º:} 239-1-Iniciativa-Convencional-de-la-cc-Tania-Madriaga-sobre-Poder-Ejecutivo-1146-hrs.pdf}
\newline {\color{gray} (Emb: 0.688, TF-IDF: 0.280)}

Las leyes de concurrencia presidencial necesaria pueden tener su origen en un mensaje presidencial o en una moción parlamentaria. 
\newline {\color{gray} \textbf{1º:} 807-Iniciativa-Convencional-Constituyente-del-cc-Jaime-Bassa-sobre-Formacion-de-la-Ley.pdf}
\newline {\color{gray} (Emb: 0.999, TF-IDF: 0.995)}
\newline {\color{gray} \textbf{2º:} 425-6-Iniciativa-Convencional-del-cc-Christian-Viera-sobre-Reforma-de-la-Constitucion-1554-26-01.pdf}
\newline {\color{gray} (Emb: 0.709, TF-IDF: 0.619)}


\item \textbf{Artículo} \newline
Si se generare un conflicto de competencia entre la Cámara de las Regiones y el Congreso de Diputadas y Diputados con relación a si una o más materias dispuestas en este artículo deben ser revisadas por la Cámara de las Regiones, esta aprobará su competencia por mayoría simple de sus miembros y el Congreso lo ratificará por mayoría simple. 
\newline {\color{gray} \textbf{1º:} 871-Iniciativa-Convencional-Constituyente-de-la-cc-Amaya-Alvez-sobre-Asambleas-Regionales.pdf}
\newline {\color{gray} (Emb: 0.510, TF-IDF: 0.408)}
\newline {\color{gray} \textbf{2º:} 686-Iniciativa-Convencional-Constituyente-del-cc-Mario-Vrgas-sobre-Democracia-Digital-181101-02.pdf}
\newline {\color{gray} (Emb: 0.506, TF-IDF: 0.291)}

En caso que el Congreso rechace la revisión aprobada por la Cámara de las Regiones, ésta podrá recurrir a la Corte Constitucional por acuerdo de mayoría simple. 
\newline {\color{gray} \textbf{1º:} 602-Iniciativa-Convencional-Constituyente-de-cc-Jorge-Baradit-sobre-Mecanismos-de-Demoracia-Directa-y-Semidirecta.pdf}
\newline {\color{gray} (Emb: 0.620, TF-IDF: 0.399)}
\newline {\color{gray} \textbf{2º:} 89-6-Iniciativa-Convencional-Constituyente-del-cc-Christian-Viera-y-otros.pdf}
\newline {\color{gray} (Emb: 0.607, TF-IDF: 0.305)}

Sólo son leyes de acuerdo regional las que reformen la Constitución; las que regulen la organización, atribuciones y funcionamiento de los Sistemas de Justicia, del Poder Legislativo y de los órganos autónomos constitucionales; las que regulen los estados de excepción constitucional; las que creen, modifiquen o supriman tributos o exenciones y determinen su progresión y proporcionalidad; las que directamente irroguen al Estado gastos cuya ejecución corresponda a las entidades territoriales; las que implementen el derecho a la salud, derecho a la educación y derecho a la vivienda; la de Presupuestos; las que aprueben el Estatuto Regional; las que regulen la elección, designación, competencias, atribuciones y procedimientos de los órganos y autoridades de las entidades territoriales; las que establezcan o alteren la división político-administrativa del país; las que establezcan los mecanismos de distribución fiscal y presupuestaria, y otros mecanismos de compensación económica entre las distintas entidades territoriales; las que autoricen la celebración de operaciones que comprometan la responsabilidad patrimonial de las entidades territoriales; las que autoricen a las entidades territoriales la creación de empresas públicas; las que deleguen potestades legislativas en conformidad al artículo 31 Nº12 de esta Constitución; las que regulen la planificación territorial y urbanística y su ejecución; las que regulen la protección del medio ambiente; las que regulen las votaciones populares y escrutinios; las que regulen las organizaciones políticas, y las demás que esta Constitución califique como de acuerdo regional. 
\newline {\color{gray} \textbf{1º:} 394-3-Iniciativa-Convencional-Constituyente-de-la-cc-Ramona-Reyes-sobre-Comuna-Autonoma-1525-24-01.pdf}
\newline {\color{gray} (Emb: 0.550, TF-IDF: 0.508)}
\newline {\color{gray} \textbf{2º:} 1008-Iniciativa-Convencional-Constituyente-de-la-cc-Claudia-Castro-sobre-Propiedad.pdf}
\newline {\color{gray} (Emb: 0.543, TF-IDF: 0.348)}


\item \textbf{Artículo} \newline
La Cámara de las Regiones conocerá de los estatutos regionales aprobados por una Asamblea Regional, de las propuestas de creación de empresas regionales efectuadas por una o más Asambleas Regionales de conformidad con lo dispuesto en el artículo 31 número 7 de esta Constitución y de las solicitudes de delegación de potestades legislativas realizadas por éstas. 
\newline {\color{gray} \textbf{1º:} 122-3-c-Iniciativa-de-la-cc-Jennifer-Mella-Forma-del-Estado.pdf}
\newline {\color{gray} (Emb: 0.560, TF-IDF: 0.367)}
\newline {\color{gray} \textbf{2º:} 159-3-c-Iniciativa-de-la-cc-Jennifer-Mella-.pdf}
\newline {\color{gray} (Emb: 0.560, TF-IDF: 0.361)}

Para el conocimiento del Estatuto Regional, el Congreso y la Cámara contarán con un plazo de seis meses. 
\newline {\color{gray} \textbf{1º:} 240-1-Iniciativa-Convencional-de-la-cc-Tania-Madriaga-sobre-Poder-Legislativo-1146-hrs.pdf}
\newline {\color{gray} (Emb: 0.619, TF-IDF: 0.330)}
\newline {\color{gray} \textbf{2º:} 120-3-c-Iniciativa-de-la-cc-Tammy-Pustilnick-atribuciones-exclusivas-de-la-Asamblea-Regional.pdf}
\newline {\color{gray} (Emb: 0.600, TF-IDF: 0.302)}

Recibida una propuesta, la Cámara podrá aprobar el proyecto o efectuar las enmiendas que estime necesarias. 
\newline {\color{gray} \textbf{1º:} 807-Iniciativa-Convencional-Constituyente-del-cc-Jaime-Bassa-sobre-Formacion-de-la-Ley.pdf}
\newline {\color{gray} (Emb: 0.635, TF-IDF: 0.292)}
\newline {\color{gray} \textbf{2º:} 425-6-Iniciativa-Convencional-del-cc-Christian-Viera-sobre-Reforma-de-la-Constitucion-1554-26-01.pdf}
\newline {\color{gray} (Emb: 0.615, TF-IDF: 0.274)}

Tratándose de las delegaciones, estas no podrán extenderse a ámbitos de concurrencia presidencial necesaria, a la nacionalidad, la ciudadanía y las elecciones, a los ámbitos que sean objeto de codificación general, ni a la organización, atribuciones y régimen de los órganos nacionales o de los Sistemas de Justicia. 
\newline {\color{gray} \textbf{1º:} 224-1-Iniciativa-Convencional-del-cc-Jaime-Bassa-sobre-Formacion-de-la-Ley-1139-17-01.pdf}
\newline {\color{gray} (Emb: 0.622, TF-IDF: 0.370)}
\newline {\color{gray} \textbf{2º:} 289-1-Iniciativa-Convencional-de-la-cc-Rosa-Catrileo-sobre-Representacion-Indigena-Electos-1522-hrs.pdf}
\newline {\color{gray} (Emb: 0.570, TF-IDF: 0.289)}

La ley que delegue potestades señalará las materias precisas sobre las que recaerá la delegación y podrá establecer o determinar las limitaciones, restricciones y formalidades que se estimen convenientes. 
\newline {\color{gray} \textbf{1º:} 224-1-Iniciativa-Convencional-del-cc-Jaime-Bassa-sobre-Formacion-de-la-Ley-1139-17-01.pdf}
\newline {\color{gray} (Emb: 0.985, TF-IDF: 0.875)}
\newline {\color{gray} \textbf{2º:} 807-Iniciativa-Convencional-Constituyente-del-cc-Jaime-Bassa-sobre-Formacion-de-la-Ley.pdf}
\newline {\color{gray} (Emb: 0.978, TF-IDF: 0.812)}

La Contraloría General de la República deberá tomar razón de las leyes regionales dictadas de conformidad con este artículo, debiendo rechazarlas cuando ellas excedan o contravengan la autorización referida. 
\newline {\color{gray} \textbf{1º:} 224-1-Iniciativa-Convencional-del-cc-Jaime-Bassa-sobre-Formacion-de-la-Ley-1139-17-01.pdf}
\newline {\color{gray} (Emb: 0.827, TF-IDF: 0.741)}
\newline {\color{gray} \textbf{2º:} 807-Iniciativa-Convencional-Constituyente-del-cc-Jaime-Bassa-sobre-Formacion-de-la-Ley.pdf}
\newline {\color{gray} (Emb: 0.789, TF-IDF: 0.713)}

De aceptarse las enmiendas por la Asamblea respectiva, el proyecto quedará en estado de ser despachado al Congreso de Diputadas y Diputados para su tramitación como ley de acuerdo regional. 
\newline {\color{gray} \textbf{1º:} 394-3-Iniciativa-Convencional-Constituyente-de-la-cc-Ramona-Reyes-sobre-Comuna-Autonoma-1525-24-01.pdf}
\newline {\color{gray} (Emb: 0.648, TF-IDF: 0.348)}
\newline {\color{gray} \textbf{2º:} 120-3-c-Iniciativa-de-la-cc-Tammy-Pustilnick-atribuciones-exclusivas-de-la-Asamblea-Regional.pdf}
\newline {\color{gray} (Emb: 0.639, TF-IDF: 0.282)}


\item \textbf{Artículo} \newline
Todo proyecto puede ser objeto de adiciones o correcciones en los trámites que corresponda, tanto en el Congreso de Diputadas y Diputados como en la Cámara de las Regiones si ésta interviene en conformidad con lo establecido en esta Constitución, pero en ningún caso se admitirán las que no tengan relación directa con las ideas matrices o fundamentales del proyecto. 
\newline {\color{gray} \textbf{1º:} 322-1-Iniciativa-Convencional-Constituyente-de-la-cc-Barbara-Sepulveda-sobre-Formacion-de-la-Ley.pdf}
\newline {\color{gray} (Emb: 0.782, TF-IDF: 0.871)}
\newline {\color{gray} \textbf{2º:} 240-1-Iniciativa-Convencional-de-la-cc-Tania-Madriaga-sobre-Poder-Legislativo-1146-hrs.pdf}
\newline {\color{gray} (Emb: 0.647, TF-IDF: 0.607)}

Las leyes pueden iniciarse por mensaje de la Presidenta o Presidente de la República o por moción de no menos del diez por ciento ni más del quince por ciento de las diputadas y diputados o representantes regionales. 
\newline {\color{gray} \textbf{1º:} 240-1-Iniciativa-Convencional-de-la-cc-Tania-Madriaga-sobre-Poder-Legislativo-1146-hrs.pdf}
\newline {\color{gray} (Emb: 0.675, TF-IDF: 0.638)}
\newline {\color{gray} \textbf{2º:} 322-1-Iniciativa-Convencional-Constituyente-de-la-cc-Barbara-Sepulveda-sobre-Formacion-de-la-Ley.pdf}
\newline {\color{gray} (Emb: 0.641, TF-IDF: 0.638)}

Adicionalmente, podrán tener su origen en iniciativa popular o iniciativa indígena de ley. 
\newline {\color{gray} \textbf{1º:} 899-Iniciativa-Convencional-Constituyente-del-cc-Cesar-Uribe-Sobre-Participacion-Ciudadana.pdf}
\newline {\color{gray} (Emb: 0.647, TF-IDF: 0.631)}
\newline {\color{gray} \textbf{2º:} 399-2-Iniciativa-Convencional-Constituyente-de-la-cc-Constanza-San-Juan-sobre-Democracia-Directa-1850-24-01.pdf}
\newline {\color{gray} (Emb: 0.623, TF-IDF: 0.475)}

Una o más Asambleas Regionales podrán presentar iniciativas a la Cámara de las Regiones en materias de interés regional. 
\newline {\color{gray} \textbf{1º:} 907-Iniciativa-Convencional-Constituyente-del-cc-Hugo-Gutierrez-Sobre-Forma-del-Estado.pdf}
\newline {\color{gray} (Emb: 0.586, TF-IDF: 0.354)}
\newline {\color{gray} \textbf{2º:} 240-1-Iniciativa-Convencional-de-la-cc-Tania-Madriaga-sobre-Poder-Legislativo-1146-hrs.pdf}
\newline {\color{gray} (Emb: 0.545, TF-IDF: 0.328)}

Si ésta las patrocina, serán ingresadas como moción parlamentaria ordinaria en el Congreso. 
\newline {\color{gray} \textbf{1º:} 240-1-Iniciativa-Convencional-de-la-cc-Tania-Madriaga-sobre-Poder-Legislativo-1146-hrs.pdf}
\newline {\color{gray} (Emb: 0.683, TF-IDF: 0.396)}
\newline {\color{gray} \textbf{2º:} 807-Iniciativa-Convencional-Constituyente-del-cc-Jaime-Bassa-sobre-Formacion-de-la-Ley.pdf}
\newline {\color{gray} (Emb: 0.653, TF-IDF: 0.343)}

Todos los proyectos de ley, cualquiera sea la forma de su iniciativa, comenzarán su tramitación en el Congreso de Diputadas y Diputados. 
\newline {\color{gray} \textbf{1º:} 234-1-Iniciativa-Convencional-del-cc-Jaime-Bassa-sobre-Justicia-Complementaria-1144-hrs.pdf}
\newline {\color{gray} (Emb: 0.680, TF-IDF: 0.424)}
\newline {\color{gray} \textbf{2º:} 240-1-Iniciativa-Convencional-de-la-cc-Tania-Madriaga-sobre-Poder-Legislativo-1146-hrs.pdf}
\newline {\color{gray} (Emb: 0.668, TF-IDF: 0.342)}


\item \textbf{Artículo} \newline
Terminada la tramitación del proyecto en el Congreso de Diputadas y Diputados, será despachado a la Presidenta o Presidente de la República para efectos de lo establecido en el artículo 32. 
\newline {\color{gray} \textbf{1º:} 807-Iniciativa-Convencional-Constituyente-del-cc-Jaime-Bassa-sobre-Formacion-de-la-Ley.pdf}
\newline {\color{gray} (Emb: 0.705, TF-IDF: 0.398)}
\newline {\color{gray} \textbf{2º:} 807-Iniciativa-Convencional-Constituyente-del-cc-Jaime-Bassa-sobre-Formacion-de-la-Ley.pdf}
\newline {\color{gray} (Emb: 0.661, TF-IDF: 0.338)}

En caso de tratarse de una ley de acuerdo regional, la Presidencia del Congreso enviará el proyecto aprobado a la Cámara de las Regiones para continuar con su tramitación. 
\newline {\color{gray} \textbf{1º:} 240-1-Iniciativa-Convencional-de-la-cc-Tania-Madriaga-sobre-Poder-Legislativo-1146-hrs.pdf}
\newline {\color{gray} (Emb: 0.676, TF-IDF: 0.328)}
\newline {\color{gray} \textbf{2º:} 871-Iniciativa-Convencional-Constituyente-de-la-cc-Amaya-Alvez-sobre-Asambleas-Regionales.pdf}
\newline {\color{gray} (Emb: 0.651, TF-IDF: 0.326)}

Las leyes deberán ser aprobadas, modificadas o derogadas por la mayoría de los miembros presentes en el Congreso de Diputadas y Diputados al momento de su votación. 
\newline {\color{gray} \textbf{1º:} 807-Iniciativa-Convencional-Constituyente-del-cc-Jaime-Bassa-sobre-Formacion-de-la-Ley.pdf}
\newline {\color{gray} (Emb: 0.961, TF-IDF: 0.933)}
\newline {\color{gray} \textbf{2º:} 322-1-Iniciativa-Convencional-Constituyente-de-la-cc-Barbara-Sepulveda-sobre-Formacion-de-la-Ley.pdf}
\newline {\color{gray} (Emb: 0.758, TF-IDF: 0.523)}


\item \textbf{Artículo} \newline
Las leyes referidas a la organización, funcionamiento y procedimientos del Poder Legislativo y de los Sistemas de Justicia; a los procesos electorales y plebiscitarios; a la regulación de los estados de excepción constitucional, y a la regulación de las organizaciones políticas, deberán ser aprobadas por el voto favorable de la mayoría en ejercicio de los miembros del Congreso de Diputadas y Diputados y de la Cámara de las Regiones. 
\newline {\color{gray} \textbf{1º:} 120-3-c-Iniciativa-de-la-cc-Tammy-Pustilnick-atribuciones-exclusivas-de-la-Asamblea-Regional.pdf}
\newline {\color{gray} (Emb: 0.686, TF-IDF: 0.344)}
\newline {\color{gray} \textbf{2º:} 46-1-Iniciativa-Convencional-Constituyente-del-cc-Martín-Arrau-y-otros.pdf}
\newline {\color{gray} (Emb: 0.633, TF-IDF: 0.329)}


\item \textbf{Artículo} \newline
Recibido por la Cámara de las Regiones un proyecto de ley de acuerdo regional aprobado por el Congreso de Diputadas y Diputados, la Cámara de las Regiones se pronunciará, aprobándolo o rechazándolo. 
\newline {\color{gray} \textbf{1º:} 120-3-c-Iniciativa-de-la-cc-Tammy-Pustilnick-atribuciones-exclusivas-de-la-Asamblea-Regional.pdf}
\newline {\color{gray} (Emb: 0.668, TF-IDF: 0.363)}
\newline {\color{gray} \textbf{2º:} 907-Iniciativa-Convencional-Constituyente-del-cc-Hugo-Gutierrez-Sobre-Forma-del-Estado.pdf}
\newline {\color{gray} (Emb: 0.664, TF-IDF: 0.360)}

Si lo aprobare, el proyecto será enviado al Congreso para que lo despache a la Presidenta o Presidente de la República para su promulgación como ley. 
\newline {\color{gray} \textbf{1º:} 240-1-Iniciativa-Convencional-de-la-cc-Tania-Madriaga-sobre-Poder-Legislativo-1146-hrs.pdf}
\newline {\color{gray} (Emb: 0.901, TF-IDF: 0.647)}
\newline {\color{gray} \textbf{2º:} 322-1-Iniciativa-Convencional-Constituyente-de-la-cc-Barbara-Sepulveda-sobre-Formacion-de-la-Ley.pdf}
\newline {\color{gray} (Emb: 0.889, TF-IDF: 0.537)}

Si lo rechazare, lo tramitará y propondrá al Congreso las enmiendas que considere pertinentes. 
\newline {\color{gray} \textbf{1º:} 721-Iniciativa-Convencional-Constituyente-de-la-cc-Maria-Trinidad-Castillo-sobre-Mineria-01-02.pdf}
\newline {\color{gray} (Emb: 0.673, TF-IDF: 0.274)}
\newline {\color{gray} \textbf{2º:} 807-Iniciativa-Convencional-Constituyente-del-cc-Jaime-Bassa-sobre-Formacion-de-la-Ley.pdf}
\newline {\color{gray} (Emb: 0.646, TF-IDF: 0.249)}

Si el Congreso rechazare una o más de esas enmiendas u observaciones, se convocará a una comisión mixta que propondrá nuevas enmiendas para resolver la discrepancia. 
\newline {\color{gray} \textbf{1º:} 807-Iniciativa-Convencional-Constituyente-del-cc-Jaime-Bassa-sobre-Formacion-de-la-Ley.pdf}
\newline {\color{gray} (Emb: 0.618, TF-IDF: 0.224)}
\newline {\color{gray} \textbf{2º:} 240-1-Iniciativa-Convencional-de-la-cc-Tania-Madriaga-sobre-Poder-Legislativo-1146-hrs.pdf}
\newline {\color{gray} (Emb: 0.576, TF-IDF: 0.215)}

Estas enmiendas serán votadas por la Cámara y luego por el Congreso. 
\newline {\color{gray} \textbf{1º:} 918-Iniciativa-Convencional-Constituyente-del-cc-Mauricio-Daza-sobre-Disposiciones-Transitorias.pdf}
\newline {\color{gray} (Emb: 0.620, TF-IDF: 0.237)}
\newline {\color{gray} \textbf{2º:} 45-1-Iniciativa-Convencional-Constituyente-del-cc-Martín-Arrau-y-otros.pdf}
\newline {\color{gray} (Emb: 0.616, TF-IDF: 0.225)}

Si todas ellas fueren aprobadas, el proyecto será despachado para su promulgación. 
\newline {\color{gray} \textbf{1º:} 788-Iniciativa-Convencional-Constituyente-de-la-cc-Camila-Zarate-sobre-Democracia-Ecologica.pdf}
\newline {\color{gray} (Emb: 0.627, TF-IDF: 0.298)}
\newline {\color{gray} \textbf{2º:} 788-Iniciativa-Convencional-Constituyente-de-la-cc-Camila-Zarate-sobre-Democracia-Ecologica.pdf}
\newline {\color{gray} (Emb: 0.567, TF-IDF: 0.295)}

La comisión mixta estará conformada por igual número de diputadas y diputados y de representantes regionales. 
\newline {\color{gray} \textbf{1º:} 234-1-Iniciativa-Convencional-del-cc-Jaime-Bassa-sobre-Justicia-Complementaria-1144-hrs.pdf}
\newline {\color{gray} (Emb: 0.647, TF-IDF: 0.413)}
\newline {\color{gray} \textbf{2º:} 400-1-Iniciativa-Convencional-Constituyente-de-la-cc-Constanza-Hube-sobre-Servicio-y-Registro-Electoral-1905-24-01.pdf}
\newline {\color{gray} (Emb: 0.634, TF-IDF: 0.244)}

La ley fijará el mecanismo para designar a los integrantes de la comisión y establecerá el plazo en que deberá informar. 
\newline {\color{gray} \textbf{1º:} 899-Iniciativa-Convencional-Constituyente-del-cc-Cesar-Uribe-Sobre-Participacion-Ciudadana.pdf}
\newline {\color{gray} (Emb: 0.642, TF-IDF: 0.265)}
\newline {\color{gray} \textbf{2º:} 399-2-Iniciativa-Convencional-Constituyente-de-la-cc-Constanza-San-Juan-sobre-Democracia-Directa-1850-24-01.pdf}
\newline {\color{gray} (Emb: 0.631, TF-IDF: 0.244)}

De no evacuar su informe dentro de plazo, se entenderá que la comisión mixta mantiene las observaciones originalmente formuladas por la Cámara y rechazadas por el Congreso y se aplicará lo dispuesto en el inciso anterior. 
\newline {\color{gray} \textbf{1º:} 324-6-Iniciativa-Convencional-Constituyente-de-la-cc-Manuela-Royo-sobre-Justicia-Feminista.pdf}
\newline {\color{gray} (Emb: 0.511, TF-IDF: 0.390)}
\newline {\color{gray} \textbf{2º:} 320-2-Iniciativa-Convencional-de-la-cc-Valentina-Miranda-sobre-Democracia-Participativa-10-07-hrs.pdf}
\newline {\color{gray} (Emb: 0.433, TF-IDF: 0.277)}


\item \textbf{Artículo} \newline
Si dentro del plazo señalado la Cámara no evacúa su informe, el proyecto quedará en condiciones de ser despachado por el Congreso. 
\newline {\color{gray} \textbf{1º:} 239-1-Iniciativa-Convencional-de-la-cc-Tania-Madriaga-sobre-Poder-Ejecutivo-1146-hrs.pdf}
\newline {\color{gray} (Emb: 0.611, TF-IDF: 0.276)}
\newline {\color{gray} \textbf{2º:} 240-1-Iniciativa-Convencional-de-la-cc-Tania-Madriaga-sobre-Poder-Legislativo-1146-hrs.pdf}
\newline {\color{gray} (Emb: 0.552, TF-IDF: 0.267)}

Éste podrá aprobarlas o insistir en el proyecto original con el voto favorable de la mayoría. 
\newline {\color{gray} \textbf{1º:} 425-6-Iniciativa-Convencional-del-cc-Christian-Viera-sobre-Reforma-de-la-Constitucion-1554-26-01.pdf}
\newline {\color{gray} (Emb: 0.628, TF-IDF: 0.492)}
\newline {\color{gray} \textbf{2º:} 425-6-Iniciativa-Convencional-del-cc-Christian-Viera-sobre-Reforma-de-la-Constitucion-1554-26-01.pdf}
\newline {\color{gray} (Emb: 0.608, TF-IDF: 0.405)}

En la sesión siguiente a su despacho por el Congreso de Diputadas y Diputados y con el voto favorable de la mayoría, la Cámara de las Regiones podrá requerir conocer de un proyecto de ley que no sea de acuerdo regional. 
\newline {\color{gray} \textbf{1º:} 120-3-c-Iniciativa-de-la-cc-Tammy-Pustilnick-atribuciones-exclusivas-de-la-Asamblea-Regional.pdf}
\newline {\color{gray} (Emb: 0.696, TF-IDF: 0.421)}
\newline {\color{gray} \textbf{2º:} 871-Iniciativa-Convencional-Constituyente-de-la-cc-Amaya-Alvez-sobre-Asambleas-Regionales.pdf}
\newline {\color{gray} (Emb: 0.645, TF-IDF: 0.314)}

La Cámara contará con sesenta días desde que recibe el proyecto para formularle enmiendas y remitirlas al Congreso. 
\newline {\color{gray} \textbf{1º:} 750-Iniciativa-Convencional-Constituyente-del-cc-Raul-Celis-sobre-Estados-de-Excepcion-01-02.pdf}
\newline {\color{gray} (Emb: 0.656, TF-IDF: 0.253)}
\newline {\color{gray} \textbf{2º:} 353-1-Iniciativa-Convencional-Constituyente-del-cc-Jaime-Bassa-sobre-Sistema-de-Gobierno-1158-21-01.pdf}
\newline {\color{gray} (Emb: 0.598, TF-IDF: 0.238)}


\item \textbf{Artículo} \newline
Si la Presidenta o Presidente de la República aprobare el proyecto despachado por el Congreso de Diputadas y Diputados, dispondrá su promulgación como ley. 
\newline {\color{gray} \textbf{1º:} 240-1-Iniciativa-Convencional-de-la-cc-Tania-Madriaga-sobre-Poder-Legislativo-1146-hrs.pdf}
\newline {\color{gray} (Emb: 0.916, TF-IDF: 0.618)}
\newline {\color{gray} \textbf{2º:} 322-1-Iniciativa-Convencional-Constituyente-de-la-cc-Barbara-Sepulveda-sobre-Formacion-de-la-Ley.pdf}
\newline {\color{gray} (Emb: 0.894, TF-IDF: 0.540)}

En caso contrario, lo devolverá dentro de treinta días con las observaciones que estime pertinentes o comunicando su rechazo total al proyecto. 
\newline {\color{gray} \textbf{1º:} 240-1-Iniciativa-Convencional-de-la-cc-Tania-Madriaga-sobre-Poder-Legislativo-1146-hrs.pdf}
\newline {\color{gray} (Emb: 0.647, TF-IDF: 0.430)}
\newline {\color{gray} \textbf{2º:} 240-1-Iniciativa-Convencional-de-la-cc-Tania-Madriaga-sobre-Poder-Legislativo-1146-hrs.pdf}
\newline {\color{gray} (Emb: 0.620, TF-IDF: 0.360)}

En ningún caso se admitirán las observaciones que no tengan relación directa con las ideas matrices o fundamentales del proyecto, a menos que hubieran sido consideradas en el mensaje respectivo. 
\newline {\color{gray} \textbf{1º:} 240-1-Iniciativa-Convencional-de-la-cc-Tania-Madriaga-sobre-Poder-Legislativo-1146-hrs.pdf}
\newline {\color{gray} (Emb: 0.877, TF-IDF: 0.686)}
\newline {\color{gray} \textbf{2º:} 322-1-Iniciativa-Convencional-Constituyente-de-la-cc-Barbara-Sepulveda-sobre-Formacion-de-la-Ley.pdf}
\newline {\color{gray} (Emb: 0.738, TF-IDF: 0.589)}

Las observaciones parciales podrán ser aprobadas por mayoría. 
\newline {\color{gray} \textbf{1º:} 602-Iniciativa-Convencional-Constituyente-de-cc-Jorge-Baradit-sobre-Mecanismos-de-Demoracia-Directa-y-Semidirecta.pdf}
\newline {\color{gray} (Emb: 0.605, TF-IDF: 0.302)}
\newline {\color{gray} \textbf{2º:} 173-6-c-Iniciativa-Convencional-del-cc-Rodrigo-Álvarez-Contraloría-1044-hrs.pdf}
\newline {\color{gray} (Emb: 0.514, TF-IDF: 0.298)}

Con el mismo quórum, el Congreso podrá insistir en el proyecto original. 
\newline {\color{gray} \textbf{1º:} 425-6-Iniciativa-Convencional-del-cc-Christian-Viera-sobre-Reforma-de-la-Constitucion-1554-26-01.pdf}
\newline {\color{gray} (Emb: 0.618, TF-IDF: 0.376)}
\newline {\color{gray} \textbf{2º:} 322-1-Iniciativa-Convencional-Constituyente-de-la-cc-Barbara-Sepulveda-sobre-Formacion-de-la-Ley.pdf}
\newline {\color{gray} (Emb: 0.558, TF-IDF: 0.353)}

Si el Presidente hubiere rechazado totalmente el proyecto, el Congreso deberá desecharlo, salvo que insista por tres quintos de sus integrantes. 
\newline {\color{gray} \textbf{1º:} 240-1-Iniciativa-Convencional-de-la-cc-Tania-Madriaga-sobre-Poder-Legislativo-1146-hrs.pdf}
\newline {\color{gray} (Emb: 0.791, TF-IDF: 0.324)}
\newline {\color{gray} \textbf{2º:} 240-1-Iniciativa-Convencional-de-la-cc-Tania-Madriaga-sobre-Poder-Legislativo-1146-hrs.pdf}
\newline {\color{gray} (Emb: 0.773, TF-IDF: 0.312)}

Si la Presidenta o Presidente de la República no devolviere el proyecto dentro de treinta días, contados desde la fecha de su remisión, se entenderá́ que lo aprueba y se promulgará como ley. 
\newline {\color{gray} \textbf{1º:} 240-1-Iniciativa-Convencional-de-la-cc-Tania-Madriaga-sobre-Poder-Legislativo-1146-hrs.pdf}
\newline {\color{gray} (Emb: 0.985, TF-IDF: 0.948)}
\newline {\color{gray} \textbf{2º:} 807-Iniciativa-Convencional-Constituyente-del-cc-Jaime-Bassa-sobre-Formacion-de-la-Ley.pdf}
\newline {\color{gray} (Emb: 0.928, TF-IDF: 0.859)}

La promulgación deberá́ hacerse siempre dentro del plazo de diez días, contados desde que ella sea procedente. 
\newline {\color{gray} \textbf{1º:} 240-1-Iniciativa-Convencional-de-la-cc-Tania-Madriaga-sobre-Poder-Legislativo-1146-hrs.pdf}
\newline {\color{gray} (Emb: 0.988, TF-IDF: 0.989)}
\newline {\color{gray} \textbf{2º:} 807-Iniciativa-Convencional-Constituyente-del-cc-Jaime-Bassa-sobre-Formacion-de-la-Ley.pdf}
\newline {\color{gray} (Emb: 0.988, TF-IDF: 0.989)}

La publicación se hará́ dentro de los cinco días hábiles siguientes a la fecha en que quede totalmente tramitado el decreto promulgatorio. 
\newline {\color{gray} \textbf{1º:} 807-Iniciativa-Convencional-Constituyente-del-cc-Jaime-Bassa-sobre-Formacion-de-la-Ley.pdf}
\newline {\color{gray} (Emb: 0.982, TF-IDF: 0.961)}
\newline {\color{gray} \textbf{2º:} 240-1-Iniciativa-Convencional-de-la-cc-Tania-Madriaga-sobre-Poder-Legislativo-1146-hrs.pdf}
\newline {\color{gray} (Emb: 0.982, TF-IDF: 0.961)}


\item \textbf{Artículo} \newline
El proyecto que fuere desechado en general por el Congreso de Diputadas y Diputados, no podrá renovarse sino después de un año. 
\newline {\color{gray} \textbf{1º:} 322-1-Iniciativa-Convencional-Constituyente-de-la-cc-Barbara-Sepulveda-sobre-Formacion-de-la-Ley.pdf}
\newline {\color{gray} (Emb: 0.940, TF-IDF: 0.921)}
\newline {\color{gray} \textbf{2º:} 374-2-Iniciativa-Convencional-Constituyente-de-la-cc-Loreto-Vallejos-sobre-Democracia-Directa-0900-hrs-24-01.pdf}
\newline {\color{gray} (Emb: 0.663, TF-IDF: 0.362)}


\item \textbf{Artículo} \newline
Sólo la Presidenta o Presidente contará con la facultad de determinar la discusión inmediata de un proyecto de ley. 
\newline {\color{gray} \textbf{1º:} 807-Iniciativa-Convencional-Constituyente-del-cc-Jaime-Bassa-sobre-Formacion-de-la-Ley.pdf}
\newline {\color{gray} (Emb: 0.780, TF-IDF: 0.904)}
\newline {\color{gray} \textbf{2º:} 353-1-Iniciativa-Convencional-Constituyente-del-cc-Jaime-Bassa-sobre-Sistema-de-Gobierno-1158-21-01.pdf}
\newline {\color{gray} (Emb: 0.638, TF-IDF: 0.468)}

La ley especificará los casos y condiciones de la urgencia popular. 
\newline {\color{gray} \textbf{1º:} 368-7-Iniciativa-Convencional-Constituyente-del-cc-Francisco-Caamano-sobre-los-conocimientos-0900-hrs-24-01.pdf}
\newline {\color{gray} (Emb: 0.729, TF-IDF: 0.610)}
\newline {\color{gray} \textbf{2º:} 420-7-Iniciativa-Convencional-del-cc-Francisco-Caamano-sobre-Derechos-de-Autor-1200-26-01.pdf}
\newline {\color{gray} (Emb: 0.691, TF-IDF: 0.350)}

La ley que regule el funcionamiento del Congreso de Diputadas y Diputados deberá establecer los mecanismos para determinar el orden en que se conocerán los proyectos de ley, debiendo distinguir entre urgencia simple, suma urgencia y discusión inmediata. 
\newline {\color{gray} \textbf{1º:} 807-Iniciativa-Convencional-Constituyente-del-cc-Jaime-Bassa-sobre-Formacion-de-la-Ley.pdf}
\newline {\color{gray} (Emb: 0.956, TF-IDF: 0.963)}
\newline {\color{gray} \textbf{2º:} 322-1-Iniciativa-Convencional-Constituyente-de-la-cc-Barbara-Sepulveda-sobre-Formacion-de-la-Ley.pdf}
\newline {\color{gray} (Emb: 0.945, TF-IDF: 0.949)}

La ley especificará los casos en que la urgencia será fijada por la Presidenta o Presidente de la República y por el Congreso. 
\newline {\color{gray} \textbf{1º:} 807-Iniciativa-Convencional-Constituyente-del-cc-Jaime-Bassa-sobre-Formacion-de-la-Ley.pdf}
\newline {\color{gray} (Emb: 0.907, TF-IDF: 0.865)}
\newline {\color{gray} \textbf{2º:} 807-Iniciativa-Convencional-Constituyente-del-cc-Jaime-Bassa-sobre-Formacion-de-la-Ley.pdf}
\newline {\color{gray} (Emb: 0.750, TF-IDF: 0.403)}


\item \textbf{Artículo} \newline
El proyecto de Ley de Presupuestos deberá ser presentado por la Presidenta o Presidente de la República a lo menos con tres meses de anterioridad a la fecha en que debe empezar a regir. 
\newline {\color{gray} \textbf{1º:} 970-Iniciativa-Convencional-Consttuyente-del-cc-Nicolas-Nunez-sobre-Presupuestos.pdf}
\newline {\color{gray} (Emb: 0.838, TF-IDF: 0.767)}
\newline {\color{gray} \textbf{2º:} 240-1-Iniciativa-Convencional-de-la-cc-Tania-Madriaga-sobre-Poder-Legislativo-1146-hrs.pdf}
\newline {\color{gray} (Emb: 0.720, TF-IDF: 0.724)}

Si el proyecto no fuera despachado dentro de los 90 días de presentado, regirá el proyecto inicialmente enviado por la o el Presidente. 
\newline {\color{gray} \textbf{1º:} 807-Iniciativa-Convencional-Constituyente-del-cc-Jaime-Bassa-sobre-Formacion-de-la-Ley.pdf}
\newline {\color{gray} (Emb: 0.628, TF-IDF: 0.508)}
\newline {\color{gray} \textbf{2º:} 239-1-Iniciativa-Convencional-de-la-cc-Tania-Madriaga-sobre-Poder-Ejecutivo-1146-hrs.pdf}
\newline {\color{gray} (Emb: 0.599, TF-IDF: 0.498)}

El proyecto de ley comenzará su tramitación en una comisión especial de presupuestos compuesta por igual número de diputados y representantes regionales. 
\newline {\color{gray} \textbf{1º:} 907-Iniciativa-Convencional-Constituyente-del-cc-Hugo-Gutierrez-Sobre-Forma-del-Estado.pdf}
\newline {\color{gray} (Emb: 0.599, TF-IDF: 0.316)}
\newline {\color{gray} \textbf{2º:} 1014-Iniciativa-Convencional-Constituyente-cc-Adriana-Ampuero-Haciendas-territoriales-y-autonomia-financiera.pdf}
\newline {\color{gray} (Emb: 0.593, TF-IDF: 0.253)}

La comisión especial no podrá aumentar ni disminuir la estimación de los ingresos, pero podrá reducir los gastos contenidos en el proyecto de Ley de Presupuestos, salvo los que estén establecidos por ley permanente. 
\newline {\color{gray} \textbf{1º:} 807-Iniciativa-Convencional-Constituyente-del-cc-Jaime-Bassa-sobre-Formacion-de-la-Ley.pdf}
\newline {\color{gray} (Emb: 0.821, TF-IDF: 0.940)}
\newline {\color{gray} \textbf{2º:} 240-1-Iniciativa-Convencional-de-la-cc-Tania-Madriaga-sobre-Poder-Legislativo-1146-hrs.pdf}
\newline {\color{gray} (Emb: 0.773, TF-IDF: 0.897)}

Aprobado el proyecto por la comisión especial de presupuestos, será enviado al Congreso de Diputadas y Diputados para su tramitación como ley de acuerdo regional. 
\newline {\color{gray} \textbf{1º:} 120-3-c-Iniciativa-de-la-cc-Tammy-Pustilnick-atribuciones-exclusivas-de-la-Asamblea-Regional.pdf}
\newline {\color{gray} (Emb: 0.648, TF-IDF: 0.373)}
\newline {\color{gray} \textbf{2º:} 402-3-Iniciativa-Convencional-Constituyente-de-la-cc-Elisa-Giustinianovich-sobre-Territorios-Especiales-1906-24-01.pdf}
\newline {\color{gray} (Emb: 0.626, TF-IDF: 0.368)}

La estimación del rendimiento de los recursos que consulta la Ley de Presupuestos y de los nuevos que establezca cualquiera otra iniciativa de ley, corresponderá a la Presidenta o Presidente de la República, previo informe de los organismos técnicos respectivos, sin perjuicio de lo señalado en el artículo 38. 
\newline {\color{gray} \textbf{1º:} 807-Iniciativa-Convencional-Constituyente-del-cc-Jaime-Bassa-sobre-Formacion-de-la-Ley.pdf}
\newline {\color{gray} (Emb: 0.841, TF-IDF: 0.885)}
\newline {\color{gray} \textbf{2º:} 240-1-Iniciativa-Convencional-de-la-cc-Tania-Madriaga-sobre-Poder-Legislativo-1146-hrs.pdf}
\newline {\color{gray} (Emb: 0.818, TF-IDF: 0.885)}

No se podrá aprobar ningún nuevo gasto con cargo al erario público sin que se indiquen, al mismo tiempo, las fuentes de recursos necesarios para atender dicho gasto. 
\newline {\color{gray} \textbf{1º:} 807-Iniciativa-Convencional-Constituyente-del-cc-Jaime-Bassa-sobre-Formacion-de-la-Ley.pdf}
\newline {\color{gray} (Emb: 0.926, TF-IDF: 0.974)}
\newline {\color{gray} \textbf{2º:} 240-1-Iniciativa-Convencional-de-la-cc-Tania-Madriaga-sobre-Poder-Legislativo-1146-hrs.pdf}
\newline {\color{gray} (Emb: 0.863, TF-IDF: 0.884)}

Si la fuente de recursos otorgada por el Congreso de Diputadas y Diputados fuere insuficiente para financiar cualquier nuevo gasto que se apruebe, la Presidenta o Presidente de la República, al promulgar la ley, previo informe favorable del servicio o institución a través del cual se recaude el nuevo ingreso, refrendado por la Contraloría General de la República, deberá reducir proporcionalmente todos los gastos, cualquiera que sea su naturaleza. 
\newline {\color{gray} \textbf{1º:} 176-2-c-Iniciativa-Convencional-del-cc-Rodrigo-Álvarez-sobre-Responsabilidad-Fiscal-1044-hrs.pdf}
\newline {\color{gray} (Emb: 0.981, TF-IDF: 0.976)}
\newline {\color{gray} \textbf{2º:} 807-Iniciativa-Convencional-Constituyente-del-cc-Jaime-Bassa-sobre-Formacion-de-la-Ley.pdf}
\newline {\color{gray} (Emb: 0.964, TF-IDF: 0.943)}


\item \textbf{Artículo} \newline
Deberá también rendir cuentas y fiscalizar la ejecución del presupuesto nacional, haciendo público asimismo la información sobre el desempeño de los programas ejecutados en base a éste. 
\newline {\color{gray} \textbf{1º:} 807-Iniciativa-Convencional-Constituyente-del-cc-Jaime-Bassa-sobre-Formacion-de-la-Ley.pdf}
\newline {\color{gray} (Emb: 0.998, TF-IDF: 0.958)}
\newline {\color{gray} \textbf{2º:} 176-2-c-Iniciativa-Convencional-del-cc-Rodrigo-Álvarez-sobre-Responsabilidad-Fiscal-1044-hrs.pdf}
\newline {\color{gray} (Emb: 0.697, TF-IDF: 0.297)}

El Gobierno deberá dar acceso al Congreso de Diputadas y Diputados a toda la información disponible para la toma de decisiones presupuestarias. 
\newline {\color{gray} \textbf{1º:} 807-Iniciativa-Convencional-Constituyente-del-cc-Jaime-Bassa-sobre-Formacion-de-la-Ley.pdf}
\newline {\color{gray} (Emb: 0.916, TF-IDF: 0.907)}
\newline {\color{gray} \textbf{2º:} 322-1-Iniciativa-Convencional-Constituyente-de-la-cc-Barbara-Sepulveda-sobre-Formacion-de-la-Ley.pdf}
\newline {\color{gray} (Emb: 0.635, TF-IDF: 0.277)}


\item \textbf{Artículo} \newline
En la tramitación de la Ley de Presupuestos, así como respecto de los presupuestos regionales y comunales, se deberán garantizar espacios de participación popular. 
\newline {\color{gray} \textbf{1º:} 1014-Iniciativa-Convencional-Constituyente-cc-Adriana-Ampuero-Haciendas-territoriales-y-autonomia-financiera.pdf}
\newline {\color{gray} (Emb: 0.654, TF-IDF: 0.377)}
\newline {\color{gray} \textbf{2º:} 931-Iniciativa-Convencional-Constituyente-del-cc-Felipe-Mena-sobre-Descentralizacion-Fiscal.pdf}
\newline {\color{gray} (Emb: 0.653, TF-IDF: 0.361)}


\item \textbf{Artículo} \newline
El Congreso de Diputadas y Diputados y la Cámara de las Regiones contarán con una Unidad Técnica dependiente administrativamente del Congreso. 
\newline {\color{gray} \textbf{1º:} 240-1-Iniciativa-Convencional-de-la-cc-Tania-Madriaga-sobre-Poder-Legislativo-1146-hrs.pdf}
\newline {\color{gray} (Emb: 0.608, TF-IDF: 0.278)}
\newline {\color{gray} \textbf{2º:} 234-1-Iniciativa-Convencional-del-cc-Jaime-Bassa-sobre-Justicia-Complementaria-1144-hrs.pdf}
\newline {\color{gray} (Emb: 0.586, TF-IDF: 0.268)}

Su Secretaría Legislativa estará encargada de asesorar en los aspectos jurídicos de las leyes que tramiten. 
\newline {\color{gray} \textbf{1º:} 470-3-Iniciativa-Convencional-Constituyente-del-cc-Jaime-Bassa-sobre-Reconocimiento-del-P.-Fernandeciano-1956-31-01.pdf}
\newline {\color{gray} (Emb: 0.661, TF-IDF: 0.242)}
\newline {\color{gray} \textbf{2º:} 579-Iniciativa-Convencional-Constituyente-de-cc-Christian-Viera-sobre-Sistema-Electoral-y-Justicia-Electoral-2330-hrs.pdf}
\newline {\color{gray} (Emb: 0.626, TF-IDF: 0.227)}

Podrá asimismo emitir informes sobre ámbitos de la legislación que hayan caído en desuso o que presenten problemas técnicos. 
\newline {\color{gray} \textbf{1º:} 89-6-Iniciativa-Convencional-Constituyente-del-cc-Christian-Viera-y-otros.pdf}
\newline {\color{gray} (Emb: 0.468, TF-IDF: 0.299)}
\newline {\color{gray} \textbf{2º:} 908-Iniciativa-Convencional-Constituyente-del-cc-Hugo-Gutierrez-sobre-Formacion-de-la-Ley.pdf}
\newline {\color{gray} (Emb: 0.455, TF-IDF: 0.244)}

Su Secretaría de Presupuestos estará encargada de estudiar el efecto presupuestario y fiscal de los proyectos de ley y de asesorar a las diputadas, diputados y representantes regionales durante la tramitación de la Ley de Presupuestos. 
\newline {\color{gray} \textbf{1º:} 384-3-Iniciativa-Convencional-Constituyente-del-cc-Felipe-Mena-sobre-Gobiernos-Regionales-1159-24-01.pdf}
\newline {\color{gray} (Emb: 0.614, TF-IDF: 0.286)}
\newline {\color{gray} \textbf{2º:} 384-3-Iniciativa-Convencional-Constituyente-del-cc-Felipe-Mena-sobre-Gobiernos-Regionales-1159-24-01.pdf}
\newline {\color{gray} (Emb: 0.614, TF-IDF: 0.260)}


\item \textbf{Artículo} \newline
El gobierno y la administración del Estado corresponden a la Presidenta o Presidente de la República, quien ejerce la jefatura de Estado y la jefatura de Gobierno. 
\newline {\color{gray} \textbf{1º:} 227-1-Iniciativa-Convencional-de-la-cc-Alondra-Carrillo-sobre-Poder-Ejecutivo-1142-hrs.pdf}
\newline {\color{gray} (Emb: 0.959, TF-IDF: 0.849)}
\newline {\color{gray} \textbf{2º:} 286-1-Iniciativa-Convencional-de-la-cc-Barbara-Sepulveda-sobre-Poder-Ejecutivo-1210-hrs.pdf}
\newline {\color{gray} (Emb: 0.919, TF-IDF: 0.849)}

El 5 de julio de cada año, la Presidenta o el Presidente dará cuenta al país del estado administrativo y político de la República ante el Congreso de Diputadas y Diputados y la Cámara de las Regiones, en sesión conjunta. 
\newline {\color{gray} \textbf{1º:} 353-1-Iniciativa-Convencional-Constituyente-del-cc-Jaime-Bassa-sobre-Sistema-de-Gobierno-1158-21-01.pdf}
\newline {\color{gray} (Emb: 0.755, TF-IDF: 0.722)}
\newline {\color{gray} \textbf{2º:} 631-Iniciativa-Convencional-Constituyente-de-cc-Ingrid-Villena-sobre-Contraloria-General-de-la-Republica.pdf}
\newline {\color{gray} (Emb: 0.590, TF-IDF: 0.427)}


\item \textbf{Artículo} \newline
Para ser elegida Presidenta o Presidente de la República se requiere tener nacionalidad chilena, ser ciudadana o ciudadano con derecho a sufragio y haber cumplido treinta años de edad. 
\newline {\color{gray} \textbf{1º:} 353-1-Iniciativa-Convencional-Constituyente-del-cc-Jaime-Bassa-sobre-Sistema-de-Gobierno-1158-21-01.pdf}
\newline {\color{gray} (Emb: 0.924, TF-IDF: 0.737)}
\newline {\color{gray} \textbf{2º:} 216-1-c-Iniciativa-Convencional-de-la-cc-Rosa-Catrileo-sobre-Sistema-de-Gobierno-2126-hrs.pdf}
\newline {\color{gray} (Emb: 0.813, TF-IDF: 0.581)}

Asimismo, deberá tener residencia efectiva en el territorio nacional los cuatro años anteriores a la elección. 
\newline {\color{gray} \textbf{1º:} 239-1-Iniciativa-Convencional-de-la-cc-Tania-Madriaga-sobre-Poder-Ejecutivo-1146-hrs.pdf}
\newline {\color{gray} (Emb: 0.654, TF-IDF: 0.387)}
\newline {\color{gray} \textbf{2º:} 909-Iniciativa-Convencional-Constituyente-del-cc-Hugo-Gutierrez-Sobre-Ministerio-Publico.pdf}
\newline {\color{gray} (Emb: 0.619, TF-IDF: 0.387)}

No se exigirá este requisito cuando la ausencia del país se deba a que ella o él, su cónyuge o su conviviente civil cumplan misión diplomática, trabajen en organismos internacionales o existan otras circunstancias que la justifiquen fundadamente. 
\newline {\color{gray} \textbf{1º:} 236-1-Iniciativa-Convencional-de-la-cc-Barbara-Sepulveda-sobre-Poder-Ejecutivo-1146-hrs.pdf}
\newline {\color{gray} (Emb: 0.519, TF-IDF: 0.193)}
\newline {\color{gray} \textbf{2º:} 286-1-Iniciativa-Convencional-de-la-cc-Barbara-Sepulveda-sobre-Poder-Ejecutivo-1210-hrs.pdf}
\newline {\color{gray} (Emb: 0.519, TF-IDF: 0.167)}

Tales circunstancias deberán ser calificadas por los tribunales electorales. 
\newline {\color{gray} \textbf{1º:} 472-6-Iniciativa-Convencional-Constituyente-del-cc-Daniel-Bravo-sobre-Corte-Constitucional-2003-31-01.pdf}
\newline {\color{gray} (Emb: 0.664, TF-IDF: 0.265)}
\newline {\color{gray} \textbf{2º:} 400-1-Iniciativa-Convencional-Constituyente-de-la-cc-Constanza-Hube-sobre-Servicio-y-Registro-Electoral-1905-24-01.pdf}
\newline {\color{gray} (Emb: 0.609, TF-IDF: 0.253)}


\item \textbf{Artículo} \newline
La Presidenta o Presidente se elegirá mediante sufragio universal, directo, libre y secreto. 
\newline {\color{gray} \textbf{1º:} 236-1-Iniciativa-Convencional-de-la-cc-Barbara-Sepulveda-sobre-Poder-Ejecutivo-1146-hrs.pdf}
\newline {\color{gray} (Emb: 0.919, TF-IDF: 0.668)}
\newline {\color{gray} \textbf{2º:} 286-1-Iniciativa-Convencional-de-la-cc-Barbara-Sepulveda-sobre-Poder-Ejecutivo-1210-hrs.pdf}
\newline {\color{gray} (Emb: 0.919, TF-IDF: 0.668)}


\item \textbf{Artículo} \newline
La elección se celebrará noventa días después de la convocatoria si ese día correspondiere a un domingo. 
\newline {\color{gray} \textbf{1º:} 353-1-Iniciativa-Convencional-Constituyente-del-cc-Jaime-Bassa-sobre-Sistema-de-Gobierno-1158-21-01.pdf}
\newline {\color{gray} (Emb: 0.814, TF-IDF: 0.887)}
\newline {\color{gray} \textbf{2º:} 216-1-c-Iniciativa-Convencional-de-la-cc-Rosa-Catrileo-sobre-Sistema-de-Gobierno-2126-hrs.pdf}
\newline {\color{gray} (Emb: 0.775, TF-IDF: 0.653)}

En caso de muerte de uno o de ambos candidatos o candidatas presidenciales a que se refiere el inciso segundo, la o el Presidente de la República convocará a una nueva elección dentro del plazo de diez días, contado desde la fecha del deceso. 
\newline {\color{gray} \textbf{1º:} 216-1-c-Iniciativa-Convencional-de-la-cc-Rosa-Catrileo-sobre-Sistema-de-Gobierno-2126-hrs.pdf}
\newline {\color{gray} (Emb: 0.877, TF-IDF: 0.736)}
\newline {\color{gray} \textbf{2º:} 236-1-Iniciativa-Convencional-de-la-cc-Barbara-Sepulveda-sobre-Poder-Ejecutivo-1146-hrs.pdf}
\newline {\color{gray} (Emb: 0.712, TF-IDF: 0.321)}

En caso contrario, se realizará el domingo siguiente. 
\newline {\color{gray} \textbf{1º:} 216-1-c-Iniciativa-Convencional-de-la-cc-Rosa-Catrileo-sobre-Sistema-de-Gobierno-2126-hrs.pdf}
\newline {\color{gray} (Emb: 0.882, TF-IDF: 0.613)}
\newline {\color{gray} \textbf{2º:} 353-1-Iniciativa-Convencional-Constituyente-del-cc-Jaime-Bassa-sobre-Sistema-de-Gobierno-1158-21-01.pdf}
\newline {\color{gray} (Emb: 0.816, TF-IDF: 0.579)}

El día de la elección presidencial será feriado irrenunciable. 
\newline {\color{gray} \textbf{1º:} 234-1-Iniciativa-Convencional-del-cc-Jaime-Bassa-sobre-Justicia-Complementaria-1144-hrs.pdf}
\newline {\color{gray} (Emb: 0.679, TF-IDF: 0.287)}
\newline {\color{gray} \textbf{2º:} 675-Iniciativa-Convencional-Constituyente-del-cc-Felipe-Harboe-sobre-Congreso-Nacional-121101-02.pdf}
\newline {\color{gray} (Emb: 0.644, TF-IDF: 0.271)}

En el caso de proceder la segunda votación, las candidatas y candidatos podrán efectuar modificaciones a su programa hasta una semana antes de ella. 
\newline {\color{gray} \textbf{1º:} 353-1-Iniciativa-Convencional-Constituyente-del-cc-Jaime-Bassa-sobre-Sistema-de-Gobierno-1158-21-01.pdf}
\newline {\color{gray} (Emb: 0.589, TF-IDF: 0.349)}
\newline {\color{gray} \textbf{2º:} 353-1-Iniciativa-Convencional-Constituyente-del-cc-Jaime-Bassa-sobre-Sistema-de-Gobierno-1158-21-01.pdf}
\newline {\color{gray} (Emb: 0.570, TF-IDF: 0.295)}

Será electa la candidatura que obtenga la mayoría de los sufragios válidamente emitidos. 
\newline {\color{gray} \textbf{1º:} 353-1-Iniciativa-Convencional-Constituyente-del-cc-Jaime-Bassa-sobre-Sistema-de-Gobierno-1158-21-01.pdf}
\newline {\color{gray} (Emb: 0.792, TF-IDF: 0.647)}
\newline {\color{gray} \textbf{2º:} 353-1-Iniciativa-Convencional-Constituyente-del-cc-Jaime-Bassa-sobre-Sistema-de-Gobierno-1158-21-01.pdf}
\newline {\color{gray} (Emb: 0.751, TF-IDF: 0.609)}

Esta votación se realizará el cuarto domingo después de la primera. 
\newline {\color{gray} \textbf{1º:} 216-1-c-Iniciativa-Convencional-de-la-cc-Rosa-Catrileo-sobre-Sistema-de-Gobierno-2126-hrs.pdf}
\newline {\color{gray} (Emb: 0.630, TF-IDF: 0.384)}
\newline {\color{gray} \textbf{2º:} 216-1-c-Iniciativa-Convencional-de-la-cc-Rosa-Catrileo-sobre-Sistema-de-Gobierno-2126-hrs.pdf}
\newline {\color{gray} (Emb: 0.577, TF-IDF: 0.362)}

Si a la elección se presentaren más de dos candidaturas y ninguna de ellas obtuviere más de la mitad de los sufragios válidamente emitidos, se procederá a una segunda votación entre las candidaturas que hubieren obtenido las dos más altas mayorías. 
\newline {\color{gray} \textbf{1º:} 353-1-Iniciativa-Convencional-Constituyente-del-cc-Jaime-Bassa-sobre-Sistema-de-Gobierno-1158-21-01.pdf}
\newline {\color{gray} (Emb: 0.965, TF-IDF: 0.781)}
\newline {\color{gray} \textbf{2º:} 239-1-Iniciativa-Convencional-de-la-cc-Tania-Madriaga-sobre-Poder-Ejecutivo-1146-hrs.pdf}
\newline {\color{gray} (Emb: 0.917, TF-IDF: 0.582)}

La elección se efectuará el tercer domingo de noviembre del año anterior a aquel en que deba cesar en el cargo el que esté en funciones. 
\newline {\color{gray} \textbf{1º:} 472-6-Iniciativa-Convencional-Constituyente-del-cc-Daniel-Bravo-sobre-Corte-Constitucional-2003-31-01.pdf}
\newline {\color{gray} (Emb: 0.630, TF-IDF: 0.275)}
\newline {\color{gray} \textbf{2º:} 675-Iniciativa-Convencional-Constituyente-del-cc-Felipe-Harboe-sobre-Congreso-Nacional-121101-02.pdf}
\newline {\color{gray} (Emb: 0.577, TF-IDF: 0.245)}

La Presidenta o Presidente será elegido por la mayoría absoluta de los votos válidamente emitidos. 
\newline {\color{gray} \textbf{1º:} 353-1-Iniciativa-Convencional-Constituyente-del-cc-Jaime-Bassa-sobre-Sistema-de-Gobierno-1158-21-01.pdf}
\newline {\color{gray} (Emb: 0.927, TF-IDF: 0.565)}
\newline {\color{gray} \textbf{2º:} 353-1-Iniciativa-Convencional-Constituyente-del-cc-Jaime-Bassa-sobre-Sistema-de-Gobierno-1158-21-01.pdf}
\newline {\color{gray} (Emb: 0.912, TF-IDF: 0.511)}


\item \textbf{Artículo} \newline
El proceso de calificación de la elección de la o el Presidente deberá́ quedar concluido dentro de los quince días siguientes a la primera votación y dentro de los treinta siguientes a la segunda. 
\newline {\color{gray} \textbf{1º:} 353-1-Iniciativa-Convencional-Constituyente-del-cc-Jaime-Bassa-sobre-Sistema-de-Gobierno-1158-21-01.pdf}
\newline {\color{gray} (Emb: 0.934, TF-IDF: 0.846)}
\newline {\color{gray} \textbf{2º:} 216-1-c-Iniciativa-Convencional-de-la-cc-Rosa-Catrileo-sobre-Sistema-de-Gobierno-2126-hrs.pdf}
\newline {\color{gray} (Emb: 0.932, TF-IDF: 0.836)}

El Tribunal Calificador de Elecciones comunicará de inmediato al Congreso de Diputadas y Diputados y a la Cámara de las Regiones la proclamación de la Presidenta o Presidente electo. 
\newline {\color{gray} \textbf{1º:} 216-1-c-Iniciativa-Convencional-de-la-cc-Rosa-Catrileo-sobre-Sistema-de-Gobierno-2126-hrs.pdf}
\newline {\color{gray} (Emb: 0.854, TF-IDF: 0.875)}
\newline {\color{gray} \textbf{2º:} 353-1-Iniciativa-Convencional-Constituyente-del-cc-Jaime-Bassa-sobre-Sistema-de-Gobierno-1158-21-01.pdf}
\newline {\color{gray} (Emb: 0.849, TF-IDF: 0.717)}

El Congreso de Diputadas y Diputados y la Cámara de las Regiones, reunidos en sesión conjunta el día en que deba cesar en su cargo el o la Presidenta en funciones, y con las y los miembros que asistan, tomará conocimiento de esa resolución del Tribunal Calificador de Elecciones y proclamará a el o la electa. 
\newline {\color{gray} \textbf{1º:} 353-1-Iniciativa-Convencional-Constituyente-del-cc-Jaime-Bassa-sobre-Sistema-de-Gobierno-1158-21-01.pdf}
\newline {\color{gray} (Emb: 0.891, TF-IDF: 0.653)}
\newline {\color{gray} \textbf{2º:} 216-1-c-Iniciativa-Convencional-de-la-cc-Rosa-Catrileo-sobre-Sistema-de-Gobierno-2126-hrs.pdf}
\newline {\color{gray} (Emb: 0.887, TF-IDF: 0.610)}

En este mismo acto, la Presidenta o Presidente prestará promesa o juramento de desempeñar fielmente su cargo, conservar la independencia de la República, guardar y hacer guardar la Constitución y las leyes, y de inmediato asumirá́sus funciones. 
\newline {\color{gray} \textbf{1º:} 353-1-Iniciativa-Convencional-Constituyente-del-cc-Jaime-Bassa-sobre-Sistema-de-Gobierno-1158-21-01.pdf}
\newline {\color{gray} (Emb: 0.905, TF-IDF: 0.883)}
\newline {\color{gray} \textbf{2º:} 405-1-Iniciativa-Convencional-Constituyente-del-cc-Martin-Arrau-sobre-Posesion-Presidencial-1946-24-01.pdf}
\newline {\color{gray} (Emb: 0.857, TF-IDF: 0.802)}


\item \textbf{Artículo} \newline
La o el Presidente así elegido asumirá sus funciones en la oportunidad que señale la ley y durará en ellas el resto del período ya iniciado. 
\newline {\color{gray} \textbf{1º:} 216-1-c-Iniciativa-Convencional-de-la-cc-Rosa-Catrileo-sobre-Sistema-de-Gobierno-2126-hrs.pdf}
\newline {\color{gray} (Emb: 0.769, TF-IDF: 0.401)}
\newline {\color{gray} \textbf{2º:} 216-1-c-Iniciativa-Convencional-de-la-cc-Rosa-Catrileo-sobre-Sistema-de-Gobierno-2126-hrs.pdf}
\newline {\color{gray} (Emb: 0.725, TF-IDF: 0.253)}

Si la o el Presidente electo se hallare impedido para tomar posesión del cargo, asumirá, provisoriamente y con el título de Vicepresidente o Vicepresidenta de la República, la o el Presidente del Congreso de Diputadas y Diputados, de la Cámara de las Regiones o de la Corte Suprema, en ese orden. 
\newline {\color{gray} \textbf{1º:} 216-1-c-Iniciativa-Convencional-de-la-cc-Rosa-Catrileo-sobre-Sistema-de-Gobierno-2126-hrs.pdf}
\newline {\color{gray} (Emb: 0.791, TF-IDF: 0.713)}
\newline {\color{gray} \textbf{2º:} 353-1-Iniciativa-Convencional-Constituyente-del-cc-Jaime-Bassa-sobre-Sistema-de-Gobierno-1158-21-01.pdf}
\newline {\color{gray} (Emb: 0.757, TF-IDF: 0.340)}

Si el impedimento fuese absoluto o durase indefinidamente, la Vicepresidenta o Vicepresidente, en los diez días siguientes al acuerdo del Congreso de Diputadas y Diputados, convocará a una nueva elección presidencial que se celebrará noventa días después si ese día correspondiere a un domingo, o el domingo inmediatamente siguiente. 
\newline {\color{gray} \textbf{1º:} 216-1-c-Iniciativa-Convencional-de-la-cc-Rosa-Catrileo-sobre-Sistema-de-Gobierno-2126-hrs.pdf}
\newline {\color{gray} (Emb: 0.848, TF-IDF: 0.725)}
\newline {\color{gray} \textbf{2º:} 239-1-Iniciativa-Convencional-de-la-cc-Tania-Madriaga-sobre-Poder-Ejecutivo-1146-hrs.pdf}
\newline {\color{gray} (Emb: 0.771, TF-IDF: 0.593)}


\item \textbf{Artículo} \newline
La o el Presidente durará cuatro años en el ejercicio de sus funciones, tras los cuales podrá ser reelegido, de forma inmediata o posterior, solo una vez. 
\newline {\color{gray} \textbf{1º:} 236-1-Iniciativa-Convencional-de-la-cc-Barbara-Sepulveda-sobre-Poder-Ejecutivo-1146-hrs.pdf}
\newline {\color{gray} (Emb: 0.941, TF-IDF: 0.807)}
\newline {\color{gray} \textbf{2º:} 286-1-Iniciativa-Convencional-de-la-cc-Barbara-Sepulveda-sobre-Poder-Ejecutivo-1210-hrs.pdf}
\newline {\color{gray} (Emb: 0.941, TF-IDF: 0.807)}


\item \textbf{Artículo} \newline
En el caso que la Presidenta o Presidente postulare a la reelección inmediata, y desde el día de la inscripción de su candidatura, no podrá ejecutar gasto que no sea de mera administración ni realizar actividades públicas que conlleven propaganda a su campaña para la reelección. 
\newline {\color{gray} \textbf{1º:} 239-1-Iniciativa-Convencional-de-la-cc-Tania-Madriaga-sobre-Poder-Ejecutivo-1146-hrs.pdf}
\newline {\color{gray} (Emb: 0.643, TF-IDF: 0.252)}
\newline {\color{gray} \textbf{2º:} 818-Iniciativa-Convencional-Constituyente-del-cc-Mauricio-Daza-sobre-Financiamiento-de-la-Politica.pdf}
\newline {\color{gray} (Emb: 0.594, TF-IDF: 0.228)}

La Contraloría General de la República deberá dictar un instructivo que regule las situaciones descritas en este artículo. 
\newline {\color{gray} \textbf{1º:} 641-Iniciativa-Convencional-Constituyente-del-cc-Mauricio-Daza-sobre-Contraloria-General-1730-01-02.pdf}
\newline {\color{gray} (Emb: 0.671, TF-IDF: 0.310)}
\newline {\color{gray} \textbf{2º:} 320-2-Iniciativa-Convencional-de-la-cc-Valentina-Miranda-sobre-Democracia-Participativa-10-07-hrs.pdf}
\newline {\color{gray} (Emb: 0.593, TF-IDF: 0.299)}


\item \textbf{Artículo} \newline
Cuando por enfermedad, ausencia del territorio de la República u otro grave motivo, la Presidenta o Presidente de la República no pudiere ejercer su cargo, le subrogará, con el título de Vicepresidenta o Vicepresidente de la República, la o el Ministro de Estado que corresponda, según el orden de precedencia que señale la ley. 
\newline {\color{gray} \textbf{1º:} 216-1-c-Iniciativa-Convencional-de-la-cc-Rosa-Catrileo-sobre-Sistema-de-Gobierno-2126-hrs.pdf}
\newline {\color{gray} (Emb: 0.715, TF-IDF: 0.721)}
\newline {\color{gray} \textbf{2º:} 239-1-Iniciativa-Convencional-de-la-cc-Tania-Madriaga-sobre-Poder-Ejecutivo-1146-hrs.pdf}
\newline {\color{gray} (Emb: 0.673, TF-IDF: 0.522)}


\item \textbf{Artículo} \newline
La o el Vicepresidente que subrogue y la o el Presidente nombrado conforme a lo dispuesto en el inciso anterior, tendrán todas las atribuciones que esta Constitución confiere al Presidente o Presidenta de la República. 
\newline {\color{gray} \textbf{1º:} 239-1-Iniciativa-Convencional-de-la-cc-Tania-Madriaga-sobre-Poder-Ejecutivo-1146-hrs.pdf}
\newline {\color{gray} (Emb: 0.764, TF-IDF: 0.446)}
\newline {\color{gray} \textbf{2º:} 353-1-Iniciativa-Convencional-Constituyente-del-cc-Jaime-Bassa-sobre-Sistema-de-Gobierno-1158-21-01.pdf}
\newline {\color{gray} (Emb: 0.755, TF-IDF: 0.446)}

La Presidenta o Presidente que resulte elegido asumirá su cargo el décimo día después de su proclamación, y hasta completar el período que restaba a quien se reemplaza. 
\newline {\color{gray} \textbf{1º:} 234-1-Iniciativa-Convencional-del-cc-Jaime-Bassa-sobre-Justicia-Complementaria-1144-hrs.pdf}
\newline {\color{gray} (Emb: 0.859, TF-IDF: 0.755)}
\newline {\color{gray} \textbf{2º:} 216-1-c-Iniciativa-Convencional-de-la-cc-Rosa-Catrileo-sobre-Sistema-de-Gobierno-2126-hrs.pdf}
\newline {\color{gray} (Emb: 0.783, TF-IDF: 0.389)}

Si la vacancia se produjere faltando dos años o más para la siguiente elección presidencial, el Vicepresidente o Vicepresidenta, dentro de los diez primeros días de su subrogancia, convocará a una elección presidencial para ciento veinte días después de la convocatoria, si ese día correspondiere a un domingo, o el domingo siguiente. 
\newline {\color{gray} \textbf{1º:} 234-1-Iniciativa-Convencional-del-cc-Jaime-Bassa-sobre-Justicia-Complementaria-1144-hrs.pdf}
\newline {\color{gray} (Emb: 0.929, TF-IDF: 0.822)}
\newline {\color{gray} \textbf{2º:} 216-1-c-Iniciativa-Convencional-de-la-cc-Rosa-Catrileo-sobre-Sistema-de-Gobierno-2126-hrs.pdf}
\newline {\color{gray} (Emb: 0.667, TF-IDF: 0.545)}

Para los efectos del artículo 45, este período presidencial se considerará como uno completo. 
\newline {\color{gray} \textbf{1º:} 240-1-Iniciativa-Convencional-de-la-cc-Tania-Madriaga-sobre-Poder-Legislativo-1146-hrs.pdf}
\newline {\color{gray} (Emb: 0.475, TF-IDF: 0.272)}
\newline {\color{gray} \textbf{2º:} 239-1-Iniciativa-Convencional-de-la-cc-Tania-Madriaga-sobre-Poder-Ejecutivo-1146-hrs.pdf}
\newline {\color{gray} (Emb: 0.464, TF-IDF: 0.256)}

Si la vacancia se produjere faltando menos de dos años para la próxima elección presidencial, la Presidenta o Presidente será nombrado en sesión conjunta del Congreso de Diputadas y Diputados y de la Cámara de las Regiones. 
\newline {\color{gray} \textbf{1º:} 234-1-Iniciativa-Convencional-del-cc-Jaime-Bassa-sobre-Justicia-Complementaria-1144-hrs.pdf}
\newline {\color{gray} (Emb: 0.862, TF-IDF: 0.719)}
\newline {\color{gray} \textbf{2º:} 316-1-Iniciativa-Convencional-del-cc-Roberto-Vega-sobre-Domicilio-Electoral-13-00-hrs.pdf}
\newline {\color{gray} (Emb: 0.703, TF-IDF: 0.464)}

En caso de impedimento definitivo, asumirá como subrogante la o el Ministro de Estado indicado en el artículo anterior y se procederá conforme a los incisos siguientes. 
\newline {\color{gray} \textbf{1º:} 353-1-Iniciativa-Convencional-Constituyente-del-cc-Jaime-Bassa-sobre-Sistema-de-Gobierno-1158-21-01.pdf}
\newline {\color{gray} (Emb: 0.559, TF-IDF: 0.398)}
\newline {\color{gray} \textbf{2º:} 353-1-Iniciativa-Convencional-Constituyente-del-cc-Jaime-Bassa-sobre-Sistema-de-Gobierno-1158-21-01.pdf}
\newline {\color{gray} (Emb: 0.530, TF-IDF: 0.228)}

Son impedimentos definitivos para el ejercicio del cargo de Presidenta o Presidente de la República y causan su vacancia: la muerte, la dimisión debidamente aceptada por el Congreso de Diputadas y Diputados y la condena por acusación constitucional, conforme a las reglas establecidas en esta Constitución. 
\newline {\color{gray} \textbf{1º:} 405-1-Iniciativa-Convencional-Constituyente-del-cc-Martin-Arrau-sobre-Posesion-Presidencial-1946-24-01.pdf}
\newline {\color{gray} (Emb: 0.668, TF-IDF: 0.338)}
\newline {\color{gray} \textbf{2º:} 239-1-Iniciativa-Convencional-de-la-cc-Tania-Madriaga-sobre-Poder-Ejecutivo-1146-hrs.pdf}
\newline {\color{gray} (Emb: 0.664, TF-IDF: 0.338)}

El nombramiento se realizará dentro de los diez días siguientes a la fecha de la vacancia y la o el nombrado asumirá su cargo dentro de los treinta días siguientes. 
\newline {\color{gray} \textbf{1º:} 234-1-Iniciativa-Convencional-del-cc-Jaime-Bassa-sobre-Justicia-Complementaria-1144-hrs.pdf}
\newline {\color{gray} (Emb: 0.823, TF-IDF: 0.826)}
\newline {\color{gray} \textbf{2º:} 234-1-Iniciativa-Convencional-del-cc-Jaime-Bassa-sobre-Justicia-Complementaria-1144-hrs.pdf}
\newline {\color{gray} (Emb: 0.631, TF-IDF: 0.419)}


\item \textbf{Artículo} \newline
Participar en los nombramientos de las demás autoridades en conformidad con lo establecido en esta Constitución; 15. 
\newline {\color{gray} \textbf{1º:} 198-6-c-Iniciativa-Convencional-de-la-cc-Ingrid-Villena-que-crea-el-Consejo-de-la-Justicia-1644-hrs.pdf}
\newline {\color{gray} (Emb: 0.628, TF-IDF: 0.385)}
\newline {\color{gray} \textbf{2º:} 216-1-c-Iniciativa-Convencional-de-la-cc-Rosa-Catrileo-sobre-Sistema-de-Gobierno-2126-hrs.pdf}
\newline {\color{gray} (Emb: 0.581, TF-IDF: 0.378)}

En tal caso, la sesión deberá celebrarse a la brevedad posible. 
\newline {\color{gray} \textbf{1º:} 239-1-Iniciativa-Convencional-de-la-cc-Tania-Madriaga-sobre-Poder-Ejecutivo-1146-hrs.pdf}
\newline {\color{gray} (Emb: 0.865, TF-IDF: 1.000)}
\newline {\color{gray} \textbf{2º:} 240-1-Iniciativa-Convencional-de-la-cc-Tania-Madriaga-sobre-Poder-Legislativo-1146-hrs.pdf}
\newline {\color{gray} (Emb: 0.506, TF-IDF: 0.295)}

Pedir, indicando los motivos, que se cite a sesión especial al Congreso de Diputadas y Diputados y a la Cámara de las Regiones. 
\newline {\color{gray} \textbf{1º:} 402-3-Iniciativa-Convencional-Constituyente-de-la-cc-Elisa-Giustinianovich-sobre-Territorios-Especiales-1906-24-01.pdf}
\newline {\color{gray} (Emb: 0.616, TF-IDF: 0.530)}
\newline {\color{gray} \textbf{2º:} 239-1-Iniciativa-Convencional-de-la-cc-Tania-Madriaga-sobre-Poder-Ejecutivo-1146-hrs.pdf}
\newline {\color{gray} (Emb: 0.512, TF-IDF: 0.473)}

Presentar anualmente al Congreso de Diputadas y Diputados el proyecto de ley de presupuestos, y 20. 
\newline {\color{gray} \textbf{1º:} 353-1-Iniciativa-Convencional-Constituyente-del-cc-Jaime-Bassa-sobre-Sistema-de-Gobierno-1158-21-01.pdf}
\newline {\color{gray} (Emb: 0.737, TF-IDF: 0.618)}
\newline {\color{gray} \textbf{2º:} 160-4-c-Iniciativa-de-la-cc-Valentina-Miranda-sobre-contenido-de-los-Derechos-Fundamentales.pdf}
\newline {\color{gray} (Emb: 0.665, TF-IDF: 0.423)}

Convocar referendos, plebiscitos y consultas en los casos previstos en esta Constitución; 19. 
\newline {\color{gray} \textbf{1º:} 353-1-Iniciativa-Convencional-Constituyente-del-cc-Jaime-Bassa-sobre-Sistema-de-Gobierno-1158-21-01.pdf}
\newline {\color{gray} (Emb: 0.612, TF-IDF: 0.487)}
\newline {\color{gray} \textbf{2º:} 239-1-Iniciativa-Convencional-de-la-cc-Tania-Madriaga-sobre-Poder-Ejecutivo-1146-hrs.pdf}
\newline {\color{gray} (Emb: 0.515, TF-IDF: 0.423)}

Las y los Ministros de Estado o funcionarios que autoricen o den curso a gastos que contravengan lo dispuesto en este numeral serán responsables, solidaria y personalmente de su reintegro, y culpables del delito de malversación de caudales públicos; 18. 
\newline {\color{gray} \textbf{1º:} 216-1-c-Iniciativa-Convencional-de-la-cc-Rosa-Catrileo-sobre-Sistema-de-Gobierno-2126-hrs.pdf}
\newline {\color{gray} (Emb: 0.929, TF-IDF: 0.942)}
\newline {\color{gray} \textbf{2º:} 216-1-c-Iniciativa-Convencional-de-la-cc-Rosa-Catrileo-sobre-Sistema-de-Gobierno-2126-hrs.pdf}
\newline {\color{gray} (Emb: 0.591, TF-IDF: 0.199)}

Se podrá contratar empleados con cargo a esta misma ley, pero sin que el ítem respectivo pueda ser incrementado ni disminuido mediante traspasos. 
\newline {\color{gray} \textbf{1º:} 216-1-c-Iniciativa-Convencional-de-la-cc-Rosa-Catrileo-sobre-Sistema-de-Gobierno-2126-hrs.pdf}
\newline {\color{gray} (Emb: 1.000, TF-IDF: 1.000)}
\newline {\color{gray} \textbf{2º:} 968-Iniciativa-Convencional-Constituyente-de-la-cc-Valentina-Miranda-sobre-Derecho-a-la-salud.pdf}
\newline {\color{gray} (Emb: 0.592, TF-IDF: 0.230)}

El total de los giros que se hagan con estos objetos no podrá exceder anualmente del dos por ciento (2\textbackslash{}%) del monto de los gastos que autorice la Ley de Presupuestos. 
\newline {\color{gray} \textbf{1º:} 216-1-c-Iniciativa-Convencional-de-la-cc-Rosa-Catrileo-sobre-Sistema-de-Gobierno-2126-hrs.pdf}
\newline {\color{gray} (Emb: 1.000, TF-IDF: 1.000)}
\newline {\color{gray} \textbf{2º:} 1014-Iniciativa-Convencional-Constituyente-cc-Adriana-Ampuero-Haciendas-territoriales-y-autonomia-financiera.pdf}
\newline {\color{gray} (Emb: 0.626, TF-IDF: 0.254)}

La Presidenta o Presidente de la República, con la firma de todas las y los Ministros de Estado, podrá decretar pagos no autorizados por ley, para atender necesidades impostergables derivadas de calamidades públicas, agresión exterior, conmoción interior, grave daño o peligro para la seguridad del país o el agotamiento de los recursos destinados a mantener servicios que no puedan paralizarse sin serio perjuicio para el país. 
\newline {\color{gray} \textbf{1º:} 216-1-c-Iniciativa-Convencional-de-la-cc-Rosa-Catrileo-sobre-Sistema-de-Gobierno-2126-hrs.pdf}
\newline {\color{gray} (Emb: 0.806, TF-IDF: 0.797)}
\newline {\color{gray} \textbf{2º:} 750-Iniciativa-Convencional-Constituyente-del-cc-Raul-Celis-sobre-Estados-de-Excepcion-01-02.pdf}
\newline {\color{gray} (Emb: 0.553, TF-IDF: 0.387)}

Velar por la recaudación de las rentas públicas y decretar su inversión con arreglo a la ley. 
\newline {\color{gray} \textbf{1º:} 353-1-Iniciativa-Convencional-Constituyente-del-cc-Jaime-Bassa-sobre-Sistema-de-Gobierno-1158-21-01.pdf}
\newline {\color{gray} (Emb: 0.987, TF-IDF: 0.863)}
\newline {\color{gray} \textbf{2º:} 239-1-Iniciativa-Convencional-de-la-cc-Tania-Madriaga-sobre-Poder-Ejecutivo-1146-hrs.pdf}
\newline {\color{gray} (Emb: 0.987, TF-IDF: 0.863)}

Conceder indultos particulares, salvo en crímenes de guerra y de lesa humanidad; 17. 
\newline {\color{gray} \textbf{1º:} 216-1-c-Iniciativa-Convencional-de-la-cc-Rosa-Catrileo-sobre-Sistema-de-Gobierno-2126-hrs.pdf}
\newline {\color{gray} (Emb: 0.661, TF-IDF: 0.443)}
\newline {\color{gray} \textbf{2º:} 231-6-Iniciativa-Convencional-del-cc-Ruggero-Cozzi-sobre-Justicia-Militar-1143-hrs.pdf}
\newline {\color{gray} (Emb: 0.519, TF-IDF: 0.443)}

Designar y remover funcionarias y funcionarios de su exclusiva confianza, de conformidad con lo que establece la ley; 16. 
\newline {\color{gray} \textbf{1º:} 353-1-Iniciativa-Convencional-Constituyente-del-cc-Jaime-Bassa-sobre-Sistema-de-Gobierno-1158-21-01.pdf}
\newline {\color{gray} (Emb: 0.713, TF-IDF: 0.546)}
\newline {\color{gray} \textbf{2º:} 239-1-Iniciativa-Convencional-de-la-cc-Tania-Madriaga-sobre-Poder-Ejecutivo-1146-hrs.pdf}
\newline {\color{gray} (Emb: 0.702, TF-IDF: 0.453)}

Nombrar a la Contralora o Contralor General conforme a lo dispuesto en esta Constitución; 14. 
\newline {\color{gray} \textbf{1º:} 184-6-c-Iniciativa-Convencional-del-cc-Rodrigo-Álvarez-que-crea-la-Corte-Constitucional-1044hrs.pdf}
\newline {\color{gray} (Emb: 0.545, TF-IDF: 0.556)}
\newline {\color{gray} \textbf{2º:} 915-Iniciativa-Convencional-Constituyente-del-cc-Luis-Jimenez-sobre-Plurinacionalidad.pdf}
\newline {\color{gray} (Emb: 0.542, TF-IDF: 0.526)}

Nombrar y remover a las Ministras y Ministros de Estado, a las Subsecretarias y Subsecretarios y a las demás funcionarias y funcionarios que corresponda, de acuerdo con esta Constitución y la ley. 
\newline {\color{gray} \textbf{1º:} 236-1-Iniciativa-Convencional-de-la-cc-Barbara-Sepulveda-sobre-Poder-Ejecutivo-1146-hrs.pdf}
\newline {\color{gray} (Emb: 0.936, TF-IDF: 0.919)}
\newline {\color{gray} \textbf{2º:} 286-1-Iniciativa-Convencional-de-la-cc-Barbara-Sepulveda-sobre-Poder-Ejecutivo-1210-hrs.pdf}
\newline {\color{gray} (Emb: 0.936, TF-IDF: 0.919)}

Remover al Jefe del Estado Mayor Conjunto y a los Comandantes en Jefe de las Fuerzas Armadas; 12. 
\newline {\color{gray} \textbf{1º:} 239-1-Iniciativa-Convencional-de-la-cc-Tania-Madriaga-sobre-Poder-Ejecutivo-1146-hrs.pdf}
\newline {\color{gray} (Emb: 0.670, TF-IDF: 0.519)}
\newline {\color{gray} \textbf{2º:} 239-1-Iniciativa-Convencional-de-la-cc-Tania-Madriaga-sobre-Poder-Ejecutivo-1146-hrs.pdf}
\newline {\color{gray} (Emb: 0.583, TF-IDF: 0.507)}

Serán atribuciones de la Presidenta o Presidente de la República: 1. 
\newline {\color{gray} \textbf{1º:} 286-1-Iniciativa-Convencional-de-la-cc-Barbara-Sepulveda-sobre-Poder-Ejecutivo-1210-hrs.pdf}
\newline {\color{gray} (Emb: 0.984, TF-IDF: 0.976)}
\newline {\color{gray} \textbf{2º:} 236-1-Iniciativa-Convencional-de-la-cc-Barbara-Sepulveda-sobre-Poder-Ejecutivo-1146-hrs.pdf}
\newline {\color{gray} (Emb: 0.984, TF-IDF: 0.976)}

Cumplir y hacer cumplir esta Constitución, las leyes y los tratados internacionales, de acuerdo con sus competencias y atribuciones; 2. 
\newline {\color{gray} \textbf{1º:} 236-1-Iniciativa-Convencional-de-la-cc-Barbara-Sepulveda-sobre-Poder-Ejecutivo-1146-hrs.pdf}
\newline {\color{gray} (Emb: 0.936, TF-IDF: 1.000)}
\newline {\color{gray} \textbf{2º:} 286-1-Iniciativa-Convencional-de-la-cc-Barbara-Sepulveda-sobre-Poder-Ejecutivo-1210-hrs.pdf}
\newline {\color{gray} (Emb: 0.936, TF-IDF: 1.000)}

Dirigir la administración del Estado; 3. 
\newline {\color{gray} \textbf{1º:} 286-1-Iniciativa-Convencional-de-la-cc-Barbara-Sepulveda-sobre-Poder-Ejecutivo-1210-hrs.pdf}
\newline {\color{gray} (Emb: 0.782, TF-IDF: 1.000)}
\newline {\color{gray} \textbf{2º:} 236-1-Iniciativa-Convencional-de-la-cc-Barbara-Sepulveda-sobre-Poder-Ejecutivo-1146-hrs.pdf}
\newline {\color{gray} (Emb: 0.782, TF-IDF: 1.000)}

Ejercer la jefatura máxima de las fuerzas de seguridad pública y designar y remover a los integrantes del alto mando policial; 13. 
\newline {\color{gray} \textbf{1º:} 239-1-Iniciativa-Convencional-de-la-cc-Tania-Madriaga-sobre-Poder-Ejecutivo-1146-hrs.pdf}
\newline {\color{gray} (Emb: 0.629, TF-IDF: 0.289)}
\newline {\color{gray} \textbf{2º:} 587-Iniciativa-Convencional-Constituyente-de-cc-Marcos-Barraza-sobre-Derecho-al-Trabajo-2351-hrs.-01-02.pdf}
\newline {\color{gray} (Emb: 0.615, TF-IDF: 0.284)}

Conducir las relaciones exteriores, suscribir y ratificar los tratados, convenios o acuerdos internacionales, nombrar y remover a Embajadoras y Embajadores y jefas y jefes de misiones diplomáticas; 5. 
\newline {\color{gray} \textbf{1º:} 286-1-Iniciativa-Convencional-de-la-cc-Barbara-Sepulveda-sobre-Poder-Ejecutivo-1210-hrs.pdf}
\newline {\color{gray} (Emb: 0.920, TF-IDF: 0.919)}
\newline {\color{gray} \textbf{2º:} 236-1-Iniciativa-Convencional-de-la-cc-Barbara-Sepulveda-sobre-Poder-Ejecutivo-1146-hrs.pdf}
\newline {\color{gray} (Emb: 0.920, TF-IDF: 0.919)}

Estos funcionarios serán de exclusiva confianza del Presidente de la República y se mantendrán en sus puestos mientras cuenten con ella; 4. 
\newline {\color{gray} \textbf{1º:} 949-Iniciativa-convencional-constituyente-de-cc-Fuad-Chahin-Asuntos-internacionales-en-la-Constitucion.pdf}
\newline {\color{gray} (Emb: 0.921, TF-IDF: 0.999)}
\newline {\color{gray} \textbf{2º:} 353-1-Iniciativa-Convencional-Constituyente-del-cc-Jaime-Bassa-sobre-Sistema-de-Gobierno-1158-21-01.pdf}
\newline {\color{gray} (Emb: 0.647, TF-IDF: 0.703)}

Concurrir a la formación de las leyes, conforme a lo que establece esta Constitución, y promulgarlas; 7. 
\newline {\color{gray} \textbf{1º:} 319-6-Iniciativa-Convencional-del-cc-Mauricio-Daza-sobre-el-Sistema-Nacional-de-Justicia17-09-hrs.pdf}
\newline {\color{gray} (Emb: 0.702, TF-IDF: 0.641)}
\newline {\color{gray} \textbf{2º:} 319-6-Iniciativa-Convencional-del-cc-Mauricio-Daza-sobre-el-Sistema-Nacional-de-Justicia17-09-hrs.pdf}
\newline {\color{gray} (Emb: 0.702, TF-IDF: 0.555)}

Dictar decretos con fuerza de ley, previa delegación del Congreso de Diputadas y Diputados, conforme a lo que se establece en esta Constitución; 8. 
\newline {\color{gray} \textbf{1º:} 216-1-c-Iniciativa-Convencional-de-la-cc-Rosa-Catrileo-sobre-Sistema-de-Gobierno-2126-hrs.pdf}
\newline {\color{gray} (Emb: 0.614, TF-IDF: 0.575)}
\newline {\color{gray} \textbf{2º:} 168-1-c-Iniciativa-Convencional-del-cc-Harry-Jurgensen-que-Crea-el-Consejo-Nacional-de-Políticas-de-Estado.pdf}
\newline {\color{gray} (Emb: 0.612, TF-IDF: 0.276)}

Ejercer la potestad reglamentaria de conformidad con esta Constitución y la ley; 9. 
\newline {\color{gray} \textbf{1º:} 924-Iniciativa-Convencional-Constituyente-de-la-cc-Constanza-Schonhaut-sobre-Administracion-del-Estado.pdf}
\newline {\color{gray} (Emb: 0.680, TF-IDF: 0.695)}
\newline {\color{gray} \textbf{2º:} 95-6-Iniciativa-Convencional-Constituyente-de-Cc-Mauricio-Daza-y-otros-2.pdf}
\newline {\color{gray} (Emb: 0.674, TF-IDF: 0.593)}

Ejercer permanentemente la jefatura suprema de las Fuerzas Armadas, disponerlas, organizarlas y distribuirlas para su desarrollo y empleo conjunto; 10. 
\newline {\color{gray} \textbf{1º:} 71-2-Iniciativa-Convencional-Constitutente-de-Maria-Jose-Oyarzun-y-otros.pdf}
\newline {\color{gray} (Emb: 0.516, TF-IDF: 0.726)}
\newline {\color{gray} \textbf{2º:} 720-Iniciativa-Convencional-Constituyente-de-la-cc-Maria-Magdalena-Rivera-sobre-Poderes-del-Estado.pdf}
\newline {\color{gray} (Emb: 0.487, TF-IDF: 0.427)}

Designar al Jefe del Estado Mayor Conjunto, a los Comandantes en Jefe de las Fuerzas Armadas y disponer los nombramientos, ascensos y retiros de los oficiales de las Fuerzas Armadas; 11. 
\newline {\color{gray} \textbf{1º:} 239-1-Iniciativa-Convencional-de-la-cc-Tania-Madriaga-sobre-Poder-Ejecutivo-1146-hrs.pdf}
\newline {\color{gray} (Emb: 0.675, TF-IDF: 0.498)}
\newline {\color{gray} \textbf{2º:} 410-1-Iniciativa-Convencional-Constituyente-del-cc-Eduardo-Castillo-sobre-FFAA-1129-25-01.pdf}
\newline {\color{gray} (Emb: 0.639, TF-IDF: 0.498)}

Declarar los estados de excepción constitucional en los casos y formas que se señalan en esta Constitución y la ley; 6. 
\newline {\color{gray} \textbf{1º:} 216-1-c-Iniciativa-Convencional-de-la-cc-Rosa-Catrileo-sobre-Sistema-de-Gobierno-2126-hrs.pdf}
\newline {\color{gray} (Emb: 0.791, TF-IDF: 0.899)}
\newline {\color{gray} \textbf{2º:} 227-1-Iniciativa-Convencional-de-la-cc-Alondra-Carrillo-sobre-Poder-Ejecutivo-1142-hrs.pdf}
\newline {\color{gray} (Emb: 0.731, TF-IDF: 0.643)}


\item \textbf{Artículo} \newline
La Presidenta o Presidente de la República podrá encomendar a uno o más Ministros la coordinación de la labor que corresponde a los secretarios de Estado y las relaciones del Gobierno con el Congreso de Diputadas y Diputados y la Cámara de las Regiones. 
\newline {\color{gray} \textbf{1º:} 216-1-c-Iniciativa-Convencional-de-la-cc-Rosa-Catrileo-sobre-Sistema-de-Gobierno-2126-hrs.pdf}
\newline {\color{gray} (Emb: 0.809, TF-IDF: 0.641)}
\newline {\color{gray} \textbf{2º:} 286-1-Iniciativa-Convencional-de-la-cc-Barbara-Sepulveda-sobre-Poder-Ejecutivo-1210-hrs.pdf}
\newline {\color{gray} (Emb: 0.788, TF-IDF: 0.598)}

Las y los Ministros de Estado son los colaboradores directos e inmediatos de la Presidenta o Presidente de la República en el gobierno y administración del Estado. 
\newline {\color{gray} \textbf{1º:} 239-1-Iniciativa-Convencional-de-la-cc-Tania-Madriaga-sobre-Poder-Ejecutivo-1146-hrs.pdf}
\newline {\color{gray} (Emb: 0.945, TF-IDF: 0.796)}
\newline {\color{gray} \textbf{2º:} 216-1-c-Iniciativa-Convencional-de-la-cc-Rosa-Catrileo-sobre-Sistema-de-Gobierno-2126-hrs.pdf}
\newline {\color{gray} (Emb: 0.829, TF-IDF: 0.706)}

La ley determinará el número y organización de los ministerios, así como el orden de precedencia de los Ministros titulares. 
\newline {\color{gray} \textbf{1º:} 216-1-c-Iniciativa-Convencional-de-la-cc-Rosa-Catrileo-sobre-Sistema-de-Gobierno-2126-hrs.pdf}
\newline {\color{gray} (Emb: 0.991, TF-IDF: 0.965)}
\newline {\color{gray} \textbf{2º:} 159-3-c-Iniciativa-de-la-cc-Jennifer-Mella-.pdf}
\newline {\color{gray} (Emb: 0.845, TF-IDF: 0.582)}


\item \textbf{Artículo} \newline
Los Ministros y Ministras de Estado se reemplazarán en caso de ausencia, impedimento, renuncia o cuando por otra causa se produzca la vacancia del cargo, de acuerdo a lo que establece la ley. 
\newline {\color{gray} \textbf{1º:} 216-1-c-Iniciativa-Convencional-de-la-cc-Rosa-Catrileo-sobre-Sistema-de-Gobierno-2126-hrs.pdf}
\newline {\color{gray} (Emb: 0.923, TF-IDF: 0.707)}
\newline {\color{gray} \textbf{2º:} 353-1-Iniciativa-Convencional-Constituyente-del-cc-Jaime-Bassa-sobre-Sistema-de-Gobierno-1158-21-01.pdf}
\newline {\color{gray} (Emb: 0.916, TF-IDF: 0.705)}

Para ser nombrada Ministra o Ministro de Estado se requiere ser ciudadana o ciudadano con derecho a sufragio y cumplir con los requisitos generales para el ingreso a la Administración Pública. 
\newline {\color{gray} \textbf{1º:} 353-1-Iniciativa-Convencional-Constituyente-del-cc-Jaime-Bassa-sobre-Sistema-de-Gobierno-1158-21-01.pdf}
\newline {\color{gray} (Emb: 0.986, TF-IDF: 0.929)}
\newline {\color{gray} \textbf{2º:} 216-1-c-Iniciativa-Convencional-de-la-cc-Rosa-Catrileo-sobre-Sistema-de-Gobierno-2126-hrs.pdf}
\newline {\color{gray} (Emb: 0.686, TF-IDF: 0.494)}


\item \textbf{Artículo} \newline
Los reglamentos y decretos de la Presidenta o Presidente de la República deberán firmarse por la Ministra o el Ministro de Estado respectivo y no serán obedecidos sin este esencial requisito. 
\newline {\color{gray} \textbf{1º:} 216-1-c-Iniciativa-Convencional-de-la-cc-Rosa-Catrileo-sobre-Sistema-de-Gobierno-2126-hrs.pdf}
\newline {\color{gray} (Emb: 0.924, TF-IDF: 0.909)}
\newline {\color{gray} \textbf{2º:} 239-1-Iniciativa-Convencional-de-la-cc-Tania-Madriaga-sobre-Poder-Ejecutivo-1146-hrs.pdf}
\newline {\color{gray} (Emb: 0.904, TF-IDF: 0.767)}

Los decretos e instrucciones podrán expedirse con la sola firma de la Ministra o Ministro de Estado respectivo, por orden de la Presidenta o Presidente de la República, en conformidad con las normas que establezca la ley. 
\newline {\color{gray} \textbf{1º:} 216-1-c-Iniciativa-Convencional-de-la-cc-Rosa-Catrileo-sobre-Sistema-de-Gobierno-2126-hrs.pdf}
\newline {\color{gray} (Emb: 0.909, TF-IDF: 0.912)}
\newline {\color{gray} \textbf{2º:} 239-1-Iniciativa-Convencional-de-la-cc-Tania-Madriaga-sobre-Poder-Ejecutivo-1146-hrs.pdf}
\newline {\color{gray} (Emb: 0.852, TF-IDF: 0.880)}


\item \textbf{Artículo} \newline
Las Ministras y Ministros de Estado son responsables directamente de la conducción de sus carteras respectivas, de los actos que firmen y solidariamente de los que suscriban o acuerden con otras y otros Ministros. 
\newline {\color{gray} \textbf{1º:} 216-1-c-Iniciativa-Convencional-de-la-cc-Rosa-Catrileo-sobre-Sistema-de-Gobierno-2126-hrs.pdf}
\newline {\color{gray} (Emb: 0.832, TF-IDF: 0.775)}
\newline {\color{gray} \textbf{2º:} 234-1-Iniciativa-Convencional-del-cc-Jaime-Bassa-sobre-Justicia-Complementaria-1144-hrs.pdf}
\newline {\color{gray} (Emb: 0.759, TF-IDF: 0.499)}


\item \textbf{Artículo} \newline
Las Ministras y Ministros podrán asistir a las sesiones del Congreso de Diputadas y Diputados y de la Cámara de las Regiones y tomar parte en sus debates, con preferencia para hacer uso de la palabra. 
\newline {\color{gray} \textbf{1º:} 239-1-Iniciativa-Convencional-de-la-cc-Tania-Madriaga-sobre-Poder-Ejecutivo-1146-hrs.pdf}
\newline {\color{gray} (Emb: 0.878, TF-IDF: 0.867)}
\newline {\color{gray} \textbf{2º:} 234-1-Iniciativa-Convencional-del-cc-Jaime-Bassa-sobre-Justicia-Complementaria-1144-hrs.pdf}
\newline {\color{gray} (Emb: 0.818, TF-IDF: 0.745)}

Sin perjuicio de lo anterior, las Ministras y Ministros de Estado deberán concurrir personalmente a las sesiones especiales que convoque el Congreso o la Cámara para informarse sobre asuntos que, perteneciendo al ámbito de atribuciones de las correspondientes secretarías de Estado, acuerden tratar. 
\newline {\color{gray} \textbf{1º:} 234-1-Iniciativa-Convencional-del-cc-Jaime-Bassa-sobre-Justicia-Complementaria-1144-hrs.pdf}
\newline {\color{gray} (Emb: 0.920, TF-IDF: 0.985)}
\newline {\color{gray} \textbf{2º:} 216-1-c-Iniciativa-Convencional-de-la-cc-Rosa-Catrileo-sobre-Sistema-de-Gobierno-2126-hrs.pdf}
\newline {\color{gray} (Emb: 0.827, TF-IDF: 0.864)}


\item \textbf{Artículo} \newline
Para las elecciones populares, la ley creará un sistema electoral conforme a los principios de igualdad sustantiva, paridad, alternabilidad de género y los demás contemplados en esta Constitución y las leyes. 
\newline {\color{gray} \textbf{1º:} 116-1-c-Iniciativa-de-la-cc-Alondra-Carrillo-Democracia-Paritaria.pdf}
\newline {\color{gray} (Emb: 0.993, TF-IDF: 0.984)}
\newline {\color{gray} \textbf{2º:} 230-2-Iniciativa-Convencional-de-la-cc-Alondra-Carrillo-sobre-Participacion-en-la-Democracia-1142-hrs.pdf}
\newline {\color{gray} (Emb: 0.879, TF-IDF: 0.681)}

Dicho sistema deberá garantizar que los órganos colegiados tengan una composición paritaria y promoverá la paridad en las candidaturas a cargos unipersonales. 
\newline {\color{gray} \textbf{1º:} 116-1-c-Iniciativa-de-la-cc-Alondra-Carrillo-Democracia-Paritaria.pdf}
\newline {\color{gray} (Emb: 0.933, TF-IDF: 0.850)}
\newline {\color{gray} \textbf{2º:} 807-Iniciativa-Convencional-Constituyente-del-cc-Jaime-Bassa-sobre-Formacion-de-la-Ley.pdf}
\newline {\color{gray} (Emb: 0.567, TF-IDF: 0.314)}

Asimismo, asegurará que las listas electorales sean encabezadas siempre por una mujer. 
\newline {\color{gray} \textbf{1º:} 116-1-c-Iniciativa-de-la-cc-Alondra-Carrillo-Democracia-Paritaria.pdf}
\newline {\color{gray} (Emb: 1.000, TF-IDF: 1.000)}
\newline {\color{gray} \textbf{2º:} 239-1-Iniciativa-Convencional-de-la-cc-Tania-Madriaga-sobre-Poder-Ejecutivo-1146-hrs.pdf}
\newline {\color{gray} (Emb: 0.574, TF-IDF: 0.232)}


\item \textbf{Artículo} \newline
Las elecciones comunales, regionales y de representantes regionales se realizarán tres años después de la elección presidencial y del Congreso de Diputadas y Diputados. 
\newline {\color{gray} \textbf{1º:} 675-Iniciativa-Convencional-Constituyente-del-cc-Felipe-Harboe-sobre-Congreso-Nacional-121101-02.pdf}
\newline {\color{gray} (Emb: 0.706, TF-IDF: 0.281)}
\newline {\color{gray} \textbf{2º:} 234-1-Iniciativa-Convencional-del-cc-Jaime-Bassa-sobre-Justicia-Complementaria-1144-hrs.pdf}
\newline {\color{gray} (Emb: 0.653, TF-IDF: 0.277)}

Estas autoridades sólo podrán ser electas de manera consecutiva por un período. 
\newline {\color{gray} \textbf{1º:} 374-2-Iniciativa-Convencional-Constituyente-de-la-cc-Loreto-Vallejos-sobre-Democracia-Directa-0900-hrs-24-01.pdf}
\newline {\color{gray} (Emb: 0.832, TF-IDF: 0.375)}
\newline {\color{gray} \textbf{2º:} 321-1-Iniciativa-Convencional-Constituyente-de-la-cc-Barbara-Sepulveda-sobre-Estatuto-de-Diputados-y-Diputadas.pdf}
\newline {\color{gray} (Emb: 0.682, TF-IDF: 0.317)}


\item \textbf{Artículo} \newline
El resguardo de la seguridad pública durante las votaciones populares y plebiscitarias corresponderá a las instituciones que indique la ley. 
\newline {\color{gray} \textbf{1º:} 400-1-Iniciativa-Convencional-Constituyente-de-la-cc-Constanza-Hube-sobre-Servicio-y-Registro-Electoral-1905-24-01.pdf}
\newline {\color{gray} (Emb: 0.649, TF-IDF: 0.405)}
\newline {\color{gray} \textbf{2º:} 712-Iniciativa-Convencional-Constituyente-de-la-cc-Lisette-Vergara-sobre-Administracion-Comunal.pdf}
\newline {\color{gray} (Emb: 0.631, TF-IDF: 0.370)}

La ley establecerá las condiciones para asegurar el ejercicio de este derecho. 
\newline {\color{gray} \textbf{1º:} 1003-Iniciativa-Convencional-Constituyente-del-cc-Luis-Mayol-sobre-Derecho-de-las-Personas-Privadas-de-Libertad.pdf}
\newline {\color{gray} (Emb: 0.937, TF-IDF: 0.617)}
\newline {\color{gray} \textbf{2º:} 408-4-Iniciativa-Convencional-Constituyente-de-la-cc-Tatiana-Urrutia-sobre-Sufragio-0041-17-01.pdf}
\newline {\color{gray} (Emb: 0.811, TF-IDF: 0.608)}

Para esto se constituirá un distrito especial exterior. 
\newline {\color{gray} \textbf{1º:} 929-Iniciativa-Convencional-Constituyente-de-la-cc-Maria-Magdalena-Rivera-sobre-Region-Exterior.pdf}
\newline {\color{gray} (Emb: 0.615, TF-IDF: 0.339)}
\newline {\color{gray} \textbf{2º:} 154-3-c-Iniciativa-del-cc-Felipe-Mena-sobre-Organizacion-Territorial-del-Estado.pdf}
\newline {\color{gray} (Emb: 0.571, TF-IDF: 0.334)}

Su ejercicio constituye un derecho y un deber cívico. 
\newline {\color{gray} \textbf{1º:} 246-1-Iniciativa-Convencional-de-la-cc-Barbara-Sepulveda-sobre-Sufragio-Obligatorio-1148-hrs.pdf}
\newline {\color{gray} (Emb: 0.752, TF-IDF: 0.968)}
\newline {\color{gray} \textbf{2º:} 257-4-Iniciativa-Convencional-de-la-cc-Maria-Magdalena-Rivera-sobre-Derecho-a-Huelga-Solidaria-17-01-1151-hrs.pdf}
\newline {\color{gray} (Emb: 0.650, TF-IDF: 0.454)}

El sufragio será facultativo para las personas de dieciséis y diecisiete años de edad. 
\newline {\color{gray} \textbf{1º:} 246-1-Iniciativa-Convencional-de-la-cc-Barbara-Sepulveda-sobre-Sufragio-Obligatorio-1148-hrs.pdf}
\newline {\color{gray} (Emb: 0.670, TF-IDF: 0.726)}
\newline {\color{gray} \textbf{2º:} 881-Iniciativa-Convencional-Constituyente-de-la-cc-Ingrid-Villena-sobre-Garantias-Procesales-para-NNA.pdf}
\newline {\color{gray} (Emb: 0.610, TF-IDF: 0.453)}

En las votaciones populares, el sufragio será universal, igualitario, libre, directo, secreto y obligatorio para las personas que hayan cumplido dieciocho años. 
\newline {\color{gray} \textbf{1º:} 246-1-Iniciativa-Convencional-de-la-cc-Barbara-Sepulveda-sobre-Sufragio-Obligatorio-1148-hrs.pdf}
\newline {\color{gray} (Emb: 0.980, TF-IDF: 0.891)}
\newline {\color{gray} \textbf{2º:} 730-Iniciativa-Convencional-Constituyente-del-cc-Adolfo-Millabur-sobre-Gobiernos-Locales.pdf}
\newline {\color{gray} (Emb: 0.724, TF-IDF: 0.556)}

Las chilenas y chilenos en el exterior podrán sufragar en los plebiscitos nacionales y elecciones presidenciales y de diputadas y diputados. 
\newline {\color{gray} \textbf{1º:} 246-1-Iniciativa-Convencional-de-la-cc-Barbara-Sepulveda-sobre-Sufragio-Obligatorio-1148-hrs.pdf}
\newline {\color{gray} (Emb: 0.891, TF-IDF: 0.631)}
\newline {\color{gray} \textbf{2º:} 536-Iniciativa-Convencional-Constituyente-del-cc-Jorge-Arancibia-sobre-Chilenos-en-el-exterior-1707-01-02-1.pdf}
\newline {\color{gray} (Emb: 0.849, TF-IDF: 0.565)}


\item \textbf{Artículo} \newline
Habrá un registro electoral público al que se incorporarán, por el solo ministerio de la ley, quienes cumplan los requisitos establecidos por esta Constitución. 
\newline {\color{gray} \textbf{1º:} 400-1-Iniciativa-Convencional-Constituyente-de-la-cc-Constanza-Hube-sobre-Servicio-y-Registro-Electoral-1905-24-01.pdf}
\newline {\color{gray} (Emb: 0.842, TF-IDF: 0.744)}
\newline {\color{gray} \textbf{2º:} 46-1-Iniciativa-Convencional-Constituyente-del-cc-Martín-Arrau-y-otros.pdf}
\newline {\color{gray} (Emb: 0.824, TF-IDF: 0.722)}

La ley determinará su organización y funcionamiento. 
\newline {\color{gray} \textbf{1º:} 349-6-Iniciativa-Convencional-Constituyente-del-cc-Luis-Mayol-sobre-Banco-Central-1826-20-01.pdf}
\newline {\color{gray} (Emb: 0.853, TF-IDF: 0.668)}
\newline {\color{gray} \textbf{2º:} 805-Iniciativa-Convencional-Constituyente-del-cc-Hugo-Gutierrez-sobre-Defensoria-Penal-Publica.pdf}
\newline {\color{gray} (Emb: 0.846, TF-IDF: 0.585)}


\item \textbf{Artículo} \newline
Las personas extranjeras avecindadas en Chile por, al menos cinco años, podrán ejercer el derecho a sufragio en los casos y formas que determine la Constitución y la ley. 
\newline {\color{gray} \textbf{1º:} 830-Iniciativa-Convencional-Constituyente-de-la-cc-Barbara-Rebolledo-sobre-Sufragio.pdf}
\newline {\color{gray} (Emb: 0.876, TF-IDF: 0.672)}
\newline {\color{gray} \textbf{2º:} 367-2-Iniciativa-Convencional-Constituyente-del-cc-Eduardo-Castillo-sobre-Ciudadania-y-Migracion-0900-hrs-24-01.pdf}
\newline {\color{gray} (Emb: 0.850, TF-IDF: 0.521)}


\item \textbf{Artículo} \newline
Se establecerán escaños reservados para los pueblos y naciones indígenas en los órganos colegiados de representación popular a nivel nacional, regional y local, cuando corresponda y en proporción a la población indígena dentro del territorio electoral respectivo, aplicando criterios de paridad en sus resultados. 
\newline {\color{gray} \textbf{1º:} 99-3-c-Iniciativa-de-la-cc-Tammy-Pustilnick-Disposiciones-del-Estado-Regional.pdf}
\newline {\color{gray} (Emb: 0.701, TF-IDF: 0.378)}
\newline {\color{gray} \textbf{2º:} 219-1-c-Iniciativa-Convencional-de-la-cc-Rosa-Catrileo-sobre-Parlamento-Bicameral-2215-hrs.pdf}
\newline {\color{gray} (Emb: 0.683, TF-IDF: 0.349)}

Una ley determinará los requisitos, forma de postulación y número para cada caso, estableciendo mecanismos que aseguren su actualización. 
\newline {\color{gray} \textbf{1º:} 237-1-Iniciativa-Convencional-de-la-cc-Tania-Madriaga-sobre-Estado-Plurinacional-y-Libre-Determinacion-1146-hrs.pdf}
\newline {\color{gray} (Emb: 0.611, TF-IDF: 0.400)}
\newline {\color{gray} \textbf{2º:} 916-Iniciativa-Convencional-Constituyente-del-cc-Luis-Jimenez-sobre-Pueblo-Tribal-Afrodescendiente.pdf}
\newline {\color{gray} (Emb: 0.609, TF-IDF: 0.374)}


\item \textbf{Artículo} \newline
La ley regulará los requisitos, procedimientos y distribución de los escaños reservados. 
\newline {\color{gray} \textbf{1º:} 320-2-Iniciativa-Convencional-de-la-cc-Valentina-Miranda-sobre-Democracia-Participativa-10-07-hrs.pdf}
\newline {\color{gray} (Emb: 0.777, TF-IDF: 0.394)}
\newline {\color{gray} \textbf{2º:} 320-2-Iniciativa-Convencional-de-la-cc-Valentina-Miranda-sobre-Democracia-Participativa-10-07-hrs.pdf}
\newline {\color{gray} (Emb: 0.777, TF-IDF: 0.392)}

Se deberán adicionar al número total de integrantes del Congreso. 
\newline {\color{gray} \textbf{1º:} 234-1-Iniciativa-Convencional-del-cc-Jaime-Bassa-sobre-Justicia-Complementaria-1144-hrs.pdf}
\newline {\color{gray} (Emb: 0.561, TF-IDF: 0.383)}
\newline {\color{gray} \textbf{2º:} 807-Iniciativa-Convencional-Constituyente-del-cc-Jaime-Bassa-sobre-Formacion-de-la-Ley.pdf}
\newline {\color{gray} (Emb: 0.555, TF-IDF: 0.305)}

La integración de los escaños reservados en la Cámara de las Regiones será determinada por ley. 
\newline {\color{gray} \textbf{1º:} 631-Iniciativa-Convencional-Constituyente-de-cc-Ingrid-Villena-sobre-Contraloria-General-de-la-Republica.pdf}
\newline {\color{gray} (Emb: 0.701, TF-IDF: 0.448)}
\newline {\color{gray} \textbf{2º:} 325-6-Iniciativa-Convencional-del-cc-Tomas-Laibe-sobre-Jurisdiccion-Constitucional.pdf}
\newline {\color{gray} (Emb: 0.696, TF-IDF: 0.344)}

Los escaños reservados en el Congreso de Diputadas y Diputados para los pueblos y naciones indígenas serán elegidos en un distrito único nacional. 
\newline {\color{gray} \textbf{1º:} 219-1-c-Iniciativa-Convencional-de-la-cc-Rosa-Catrileo-sobre-Parlamento-Bicameral-2215-hrs.pdf}
\newline {\color{gray} (Emb: 0.744, TF-IDF: 0.458)}
\newline {\color{gray} \textbf{2º:} 217-1-c-Iniciativa-Convencional-de-la-cc-Rosa-Catrileo-sobre-Escanos-Reservados-Parlamento-Unicameral-2144-hrs.pdf}
\newline {\color{gray} (Emb: 0.705, TF-IDF: 0.458)}

Su número se definirá en forma proporcional a la población indígena en relación a la población total del país. 
\newline {\color{gray} \textbf{1º:} 219-1-c-Iniciativa-Convencional-de-la-cc-Rosa-Catrileo-sobre-Parlamento-Bicameral-2215-hrs.pdf}
\newline {\color{gray} (Emb: 0.768, TF-IDF: 0.673)}
\newline {\color{gray} \textbf{2º:} 400-1-Iniciativa-Convencional-Constituyente-de-la-cc-Constanza-Hube-sobre-Servicio-y-Registro-Electoral-1905-24-01.pdf}
\newline {\color{gray} (Emb: 0.657, TF-IDF: 0.617)}


\item \textbf{Artículo} \newline
Podrán votar por los escaños reservados para pueblos y naciones indígenas sólo los ciudadanos y ciudadanas que pertenezcan a dichos pueblos y naciones y que formen parte de un registro especial denominado Registro Electoral Indígena, que administrará el Servicio Electoral. 
\newline {\color{gray} \textbf{1º:} 215-1-c-Iniciativa-Convencional-de-la-cc-Rosa-Catrileo-sobre-Registro-Electoral-Indigena-2116-hrs.pdf}
\newline {\color{gray} (Emb: 0.930, TF-IDF: 0.947)}
\newline {\color{gray} \textbf{2º:} 215-1-c-Iniciativa-Convencional-de-la-cc-Rosa-Catrileo-sobre-Registro-Electoral-Indigena-2116-hrs.pdf}
\newline {\color{gray} (Emb: 0.843, TF-IDF: 0.560)}

Dicho registro será construido por el Servicio Electoral sobre la base de los archivos que administren los órganos estatales, los que posean los pueblos y naciones indígenas sobre sus miembros y de las solicitudes de ciudadanos y ciudadanas que se autoidentifiquen como tales, en los términos que indique la ley. 
\newline {\color{gray} \textbf{1º:} 215-1-c-Iniciativa-Convencional-de-la-cc-Rosa-Catrileo-sobre-Registro-Electoral-Indigena-2116-hrs.pdf}
\newline {\color{gray} (Emb: 0.958, TF-IDF: 0.906)}
\newline {\color{gray} \textbf{2º:} 215-1-c-Iniciativa-Convencional-de-la-cc-Rosa-Catrileo-sobre-Registro-Electoral-Indigena-2116-hrs.pdf}
\newline {\color{gray} (Emb: 0.706, TF-IDF: 0.338)}

Se creará un registro del pueblo tribal afrodescendiente chileno bajo las mismas reglas del presente artículo. 
\newline {\color{gray} \textbf{1º:} 683-Iniciativa-Convencional-Constituyente-del-cc-Jorge-Abarca-sobre-Pueblo-Tribal-150001-02.pdf}
\newline {\color{gray} (Emb: 0.641, TF-IDF: 0.599)}
\newline {\color{gray} \textbf{2º:} 59-2-Iniciativa-Convencional-Constituyente-de-la-cc-Ericka-Portilla-y-otros.pdf}
\newline {\color{gray} (Emb: 0.640, TF-IDF: 0.531)}


\item \textbf{Artículo} \newline
Las organizaciones políticas reconocidas legalmente implementarán la paridad de género en sus espacios de dirección, asegurando la igualdad sustantiva en sus dimensiones organizativa y electoral, y promoviendo la plena participación política de las mujeres. 
\newline {\color{gray} \textbf{1º:} 116-1-c-Iniciativa-de-la-cc-Alondra-Carrillo-Democracia-Paritaria.pdf}
\newline {\color{gray} (Emb: 0.855, TF-IDF: 0.784)}
\newline {\color{gray} \textbf{2º:} 116-1-c-Iniciativa-de-la-cc-Alondra-Carrillo-Democracia-Paritaria.pdf}
\newline {\color{gray} (Emb: 0.693, TF-IDF: 0.339)}

A su vez, deberán destinar un financiamiento electoral proporcional al número de dichas candidaturas. 
\newline {\color{gray} \textbf{1º:} 377-2-Iniciativa-Convencional-Constituyente-del-cc-Alvin-Saldana-sobre-Mecanismos-de-Democracia-1045-hrs-24-01.pdf}
\newline {\color{gray} (Emb: 0.547, TF-IDF: 0.409)}
\newline {\color{gray} \textbf{2º:} 377-2-Iniciativa-Convencional-Constituyente-del-cc-Alvin-Saldana-sobre-Mecanismos-de-Democracia-1045-hrs-24-01.pdf}
\newline {\color{gray} (Emb: 0.537, TF-IDF: 0.248)}

El Estado y las organizaciones políticas deberán tomar las medidas necesarias para erradicar la violencia de género con el fin de asegurar que todas las personas ejerzan plenamente sus derechos políticos. 
\newline {\color{gray} \textbf{1º:} 774-Iniciativa-Convencional-Constituyente-de-la-cc-Barbara-Rebolledo-sobre-Derechos-de-las-Mujeres.pdf}
\newline {\color{gray} (Emb: 0.688, TF-IDF: 0.334)}
\newline {\color{gray} \textbf{2º:} 655-Iniciativa-Convencional-Constituyente-de-la-cc-Jeniffer-Mella-sobre-Trabajo-y-Seguridad-Social-121101-02.pdf}
\newline {\color{gray} (Emb: 0.653, TF-IDF: 0.293)}

La ley arbitrará los medios para incentivar la participación de las personas de las diversidades y disidencias sexuales y de género en los procesos electorales. 
\newline {\color{gray} \textbf{1º:} 116-1-c-Iniciativa-de-la-cc-Alondra-Carrillo-Democracia-Paritaria.pdf}
\newline {\color{gray} (Emb: 0.667, TF-IDF: 0.333)}
\newline {\color{gray} \textbf{2º:} 99-3-c-Iniciativa-de-la-cc-Tammy-Pustilnick-Disposiciones-del-Estado-Regional.pdf}
\newline {\color{gray} (Emb: 0.651, TF-IDF: 0.318)}


\item \textbf{Artículo} \newline
Con la finalidad de garantizar la integridad pública y erradicar la corrupción en todas sus formas, los órganos competentes en la materia deberán coordinar su actuar a través de la instancia o mecanismos que correspondan para el cumplimiento de estos fines, en la forma que determine la ley. 
\newline {\color{gray} \textbf{1º:} 11-4-Iniciativa-Convencional-Constituyente-de-la-cc-María-Elisa-Quinteros-y-otras.pdf}
\newline {\color{gray} (Emb: 0.591, TF-IDF: 0.628)}
\newline {\color{gray} \textbf{2º:} 528-5-Iniciativa-Convencional-Constituyente-de-cc-Margarita-Vargas-sobre-Responsavilidad-Empresarial-1317-hrs.-01-02.pdf}
\newline {\color{gray} (Emb: 0.591, TF-IDF: 0.623)}


\item \textbf{Artículo} \newline
El principio de probidad consiste en observar una conducta intachable y un desempeño honesto y leal de la función o cargo, con preeminencia del interés general sobre el particular. 
\newline {\color{gray} \textbf{1º:} 1015-Iniciativa-Convencional-Constituyente-de-la-cc-Paulina-Valenzuela-sobre-Probidad.pdf}
\newline {\color{gray} (Emb: 1.000, TF-IDF: 1.000)}
\newline {\color{gray} \textbf{2º:} 193-1-c-Iniciativa-Convencional-del-cc-Ricardo-Montero-sobre-Buen-Gobierno-y-Probidad-1328-hrs.pdf}
\newline {\color{gray} (Emb: 1.000, TF-IDF: 1.000)}


\item \textbf{Artículo} \newline
Es pública la información elaborada con presupuesto público y toda otra información que obre en poder del Estado, cualquiera sea su formato, soporte, fecha de creación, origen, clasificación o procesamiento, salvo cuando la publicidad afectare el debido cumplimiento de las funciones de dichos órganos, la protección de datos personales, los derechos de las personas, la seguridad del Estado o el interés nacional, conforme lo establezca la ley. 
\newline {\color{gray} \textbf{1º:} 193-1-c-Iniciativa-Convencional-del-cc-Ricardo-Montero-sobre-Buen-Gobierno-y-Probidad-1328-hrs.pdf}
\newline {\color{gray} (Emb: 0.949, TF-IDF: 0.960)}
\newline {\color{gray} \textbf{2º:} 599-Iniciativa-Convencional-Constituyente-de-cc-Francisco-Caamano-sobre-Derechos-a-la-Informacion-en-la-era-Digital.pdf}
\newline {\color{gray} (Emb: 0.711, TF-IDF: 0.372)}

El principio de transparencia exige a los órganos del Estado que la información pública sea puesta a disposición de toda persona que la requiera, independiente del uso que se le dé, facilitando su acceso y procurando su oportuna entrega y accesibilidad. 
\newline {\color{gray} \textbf{1º:} 193-1-c-Iniciativa-Convencional-del-cc-Ricardo-Montero-sobre-Buen-Gobierno-y-Probidad-1328-hrs.pdf}
\newline {\color{gray} (Emb: 1.000, TF-IDF: 1.000)}
\newline {\color{gray} \textbf{2º:} 266-4-Iniciativa-Convencional-de-la-cc-Janis-Meneses-sobre-Derecho-de-las-personas-frente-al-Estado-1154-hrs.pdf}
\newline {\color{gray} (Emb: 0.637, TF-IDF: 0.265)}

Toda institución que desarrolle una función pública, o que administre recursos públicos, deberá dar estricto cumplimiento al principio de transparencia. 
\newline {\color{gray} \textbf{1º:} 70-2-Iniciativa-Convencional-Constituyente-de-la-cc-Paulina-Veloso-y-otros-2.pdf}
\newline {\color{gray} (Emb: 0.710, TF-IDF: 0.509)}
\newline {\color{gray} \textbf{2º:} 40-2-Iniciativa-Convencional-Constituyente-de-la-cc-Bernardo-de-la-Maza-y-otros.pdf}
\newline {\color{gray} (Emb: 0.684, TF-IDF: 0.468)}


\item \textbf{Artículo} \newline
El Estado promoverá la participación activa de las personas y la sociedad civil en la fiscalización del cumplimiento de este principio. 
\newline {\color{gray} \textbf{1º:} 193-1-c-Iniciativa-Convencional-del-cc-Ricardo-Montero-sobre-Buen-Gobierno-y-Probidad-1328-hrs.pdf}
\newline {\color{gray} (Emb: 1.000, TF-IDF: 1.000)}
\newline {\color{gray} \textbf{2º:} 783-Iniciativa-Convencional-Constituyente-de-la-cc-Ivanna-Olivares-sobre-Gestion-Publica.pdf}
\newline {\color{gray} (Emb: 0.605, TF-IDF: 0.339)}

Los órganos del Estado y quienes ejerzan una función pública deberán rendir cuenta en la forma y condiciones que establezca la ley. 
\newline {\color{gray} \textbf{1º:} 193-1-c-Iniciativa-Convencional-del-cc-Ricardo-Montero-sobre-Buen-Gobierno-y-Probidad-1328-hrs.pdf}
\newline {\color{gray} (Emb: 0.828, TF-IDF: 0.413)}
\newline {\color{gray} \textbf{2º:} 191-2-c-Iniciativa-Convencional-del-cc-Martín-Arrau-sonre-Principio-de-función-eficiente-del-Estado-1302-hrs.pdf}
\newline {\color{gray} (Emb: 0.763, TF-IDF: 0.378)}

El principio de rendición de cuentas implica el deber de asumir la responsabilidad en el ejercicio de su cargo. 
\newline {\color{gray} \textbf{1º:} 193-1-c-Iniciativa-Convencional-del-cc-Ricardo-Montero-sobre-Buen-Gobierno-y-Probidad-1328-hrs.pdf}
\newline {\color{gray} (Emb: 0.668, TF-IDF: 0.830)}
\newline {\color{gray} \textbf{2º:} 390-5-Iniciativa-Convencional-Constituyente-de-la-cc-Isabel-Godoy-sobre-Estatuto-Constitucional-del-Agua-1431-24-01.pdf}
\newline {\color{gray} (Emb: 0.577, TF-IDF: 0.366)}


\item \textbf{Artículo} \newline
Todas las personas tendrán el derecho a buscar, solicitar, recibir y difundir información pública de cualquier órgano del Estado o de entidades que presten servicios de utilidad pública, en la forma y condiciones que establezca la ley. 
\newline {\color{gray} \textbf{1º:} 193-1-c-Iniciativa-Convencional-del-cc-Ricardo-Montero-sobre-Buen-Gobierno-y-Probidad-1328-hrs.pdf}
\newline {\color{gray} (Emb: 1.000, TF-IDF: 1.000)}
\newline {\color{gray} \textbf{2º:} 784-niciativa-Convencional-Constituyente-de-la-cc-Damaris-Abarca-sobre-Derecho-a-vivir-en-un-ambiente-sano.pdf}
\newline {\color{gray} (Emb: 0.728, TF-IDF: 0.372)}

El derecho de acceso a la información pública reconoce los principios establecidos en esta Constitución y las leyes. 
\newline {\color{gray} \textbf{1º:} 137-4-c-Iniciativa-de-la-cc-Rocio-Cantuarias-Establece-el-Derecho-al-Acceso-a-la-Informacion-Publica.pdf}
\newline {\color{gray} (Emb: 0.738, TF-IDF: 0.484)}
\newline {\color{gray} \textbf{2º:} 33-2-Iniciativa-Convencional-Constituyente-del-cc-Martín-Arrau-y-otros.pdf}
\newline {\color{gray} (Emb: 0.734, TF-IDF: 0.443)}


\item \textbf{Artículo} \newline
El Consejo para la Transparencia es un órgano autónomo, especializado e imparcial con personalidad jurídica y patrimonio propio, que tiene por función promover la transparencia de la función pública, fiscalizar el cumplimiento de las normas sobre transparencia y publicidad de la información de los órganos del Estado y garantizar el derecho de acceso a la información pública. 
\newline {\color{gray} \textbf{1º:} 60-2-Iniciativa-Convencional-Constituyente-del-cc-Jorge-Baradit-y-otros.pdf}
\newline {\color{gray} (Emb: 0.774, TF-IDF: 0.509)}
\newline {\color{gray} \textbf{2º:} 33-2-Iniciativa-Convencional-Constituyente-del-cc-Martín-Arrau-y-otros.pdf}
\newline {\color{gray} (Emb: 0.708, TF-IDF: 0.452)}

La composición, organización, el funcionamiento y las atribuciones del Consejo para la Transparencia serán materias de ley. 
\newline {\color{gray} \textbf{1º:} 349-6-Iniciativa-Convencional-Constituyente-del-cc-Luis-Mayol-sobre-Banco-Central-1826-20-01.pdf}
\newline {\color{gray} (Emb: 0.697, TF-IDF: 0.529)}
\newline {\color{gray} \textbf{2º:} 111-1-c-Iniciativa-del-cc-Jorge-Arancibia-Fuerzas-Armadas-de-Orden-y-Seguridad.pdf}
\newline {\color{gray} (Emb: 0.654, TF-IDF: 0.503)}


\item \textbf{Artículo} \newline
La corrupción es contraria al bien común y atenta contra el sistema democrático. 
\newline {\color{gray} \textbf{1º:} 194-1-c-Iniciativa-Convenciona-de-la-cc-Francisca-Arauna-sobre-Pp.-Buen-Gobierno-y-Probidad-1402-hrs.pdf}
\newline {\color{gray} (Emb: 0.571, TF-IDF: 0.438)}
\newline {\color{gray} \textbf{2º:} 1015-Iniciativa-Convencional-Constituyente-de-la-cc-Paulina-Valenzuela-sobre-Probidad.pdf}
\newline {\color{gray} (Emb: 0.531, TF-IDF: 0.348)}

El Estado tomará las medidas necesarias para su estudio, prevención, investigación, persecución y sanción. 
\newline {\color{gray} \textbf{1º:} 519-4-Iniciativa-Convencional-Constituyente-de-cc-Carolina-Videla-sobre-DDHH-y-Garantias-de-no-repeticion-1257-hrs.-01-02.pdf}
\newline {\color{gray} (Emb: 0.732, TF-IDF: 0.440)}
\newline {\color{gray} \textbf{2º:} 451-4-Iniciativa-Convencional-Constituyente-de-la-cc-Carolina-Videla-sobre-Tortura-y-desaparicion-1409-31-01.pdf}
\newline {\color{gray} (Emb: 0.732, TF-IDF: 0.440)}


\item \textbf{Artículo} \newline
El Estado asegura a todas las personas la debida protección, confidencialidad e indemnidad al denunciar infracciones en el ejercicio de la función pública, especialmente faltas a la probidad, transparencia y hechos de corrupción. 
\newline {\color{gray} \textbf{1º:} 241-1-Iniciativa-Convencional-de-la-cc-Alejandra-Flores-sobre-Buen-Gobierno-1146-hrs.pdf}
\newline {\color{gray} (Emb: 0.782, TF-IDF: 0.685)}
\newline {\color{gray} \textbf{2º:} 194-1-c-Iniciativa-Convenciona-de-la-cc-Francisca-Arauna-sobre-Pp.-Buen-Gobierno-y-Probidad-1402-hrs.pdf}
\newline {\color{gray} (Emb: 0.778, TF-IDF: 0.589)}


\item \textbf{Artículo} \newline
El ejercicio de funciones públicas se regirá por los principios de probidad, eficiencia, eficacia, responsabilidad, transparencia, publicidad, rendición de cuentas, buena fe, interculturalidad, enfoque de género, inclusión, no discriminación y sustentabilidad. 
\newline {\color{gray} \textbf{1º:} 241-1-Iniciativa-Convencional-de-la-cc-Alejandra-Flores-sobre-Buen-Gobierno-1146-hrs.pdf}
\newline {\color{gray} (Emb: 0.921, TF-IDF: 0.773)}
\newline {\color{gray} \textbf{2º:} 921-Iniciativa-Convencional-Constituyente-de-la-cc-Yarela-Gomez-sobre-Modernizacion-del-Estado.pdf}
\newline {\color{gray} (Emb: 0.825, TF-IDF: 0.517)}

Las autoridades electas popularmente, y las demás autoridades y funcionarios que determine la ley, deberán declarar sus intereses y patrimonio en forma pública. 
\newline {\color{gray} \textbf{1º:} 33-2-Iniciativa-Convencional-Constituyente-del-cc-Martín-Arrau-y-otros.pdf}
\newline {\color{gray} (Emb: 0.904, TF-IDF: 0.892)}
\newline {\color{gray} \textbf{2º:} 400-1-Iniciativa-Convencional-Constituyente-de-la-cc-Constanza-Hube-sobre-Servicio-y-Registro-Electoral-1905-24-01.pdf}
\newline {\color{gray} (Emb: 0.665, TF-IDF: 0.325)}


\item \textbf{Artículo} \newline
Respecto de las altas autoridades del Estado, la ley establecerá mayores exigencias y estándares de responsabilidad para el cumplimiento de los principios de probidad, transparencia y rendición de cuentas. 
\newline {\color{gray} \textbf{1º:} 469-3-Iniciativa-Convencional-Constituyente-del-cc-Felipe-Mena-sobre-Modernizacion-del-Estado-1952-31-01.pdf}
\newline {\color{gray} (Emb: 0.765, TF-IDF: 0.573)}
\newline {\color{gray} \textbf{2º:} 40-2-Iniciativa-Convencional-Constituyente-de-la-cc-Bernardo-de-la-Maza-y-otros.pdf}
\newline {\color{gray} (Emb: 0.715, TF-IDF: 0.450)}


\item \textbf{Artículo} \newline
Una ley establecerá la integración, funcionamiento y atribuciones de esta comisión. 
\newline {\color{gray} \textbf{1º:} 806-Iniciativa-Convencional-Constituyente-del-cc-Hugo-Gutierrez-sobre-Gobiernos-Regionales.pdf}
\newline {\color{gray} (Emb: 0.777, TF-IDF: 0.549)}
\newline {\color{gray} \textbf{2º:} 924-Iniciativa-Convencional-Constituyente-de-la-cc-Constanza-Schonhaut-sobre-Administracion-del-Estado.pdf}
\newline {\color{gray} (Emb: 0.697, TF-IDF: 0.523)}

Las remuneraciones serán fijadas cada cuatro años, con al menos dieciocho meses de anterioridad al término de un periodo presidencial. 
\newline {\color{gray} \textbf{1º:} 216-1-c-Iniciativa-Convencional-de-la-cc-Rosa-Catrileo-sobre-Sistema-de-Gobierno-2126-hrs.pdf}
\newline {\color{gray} (Emb: 0.680, TF-IDF: 0.347)}
\newline {\color{gray} \textbf{2º:} 353-1-Iniciativa-Convencional-Constituyente-del-cc-Jaime-Bassa-sobre-Sistema-de-Gobierno-1158-21-01.pdf}
\newline {\color{gray} (Emb: 0.650, TF-IDF: 0.250)}

Una comisión fijará las remuneraciones de las autoridades de elección popular, así como de quienes sirvan de confianza exclusiva de ellas. 
\newline {\color{gray} \textbf{1º:} 45-1-Iniciativa-Convencional-Constituyente-del-cc-Martín-Arrau-y-otros.pdf}
\newline {\color{gray} (Emb: 0.576, TF-IDF: 0.300)}
\newline {\color{gray} \textbf{2º:} 88-6-Iniciativa-Convencional-Constituyente-del-cc-Christian-Viera-y-otros.pdf}
\newline {\color{gray} (Emb: 0.569, TF-IDF: 0.296)}

Los acuerdos de la comisión serán públicos, se fundarán en antecedentes técnicos y deberán garantizar una retribución adecuada a la responsabilidad del cargo. 
\newline {\color{gray} \textbf{1º:} 370-4-Iniciativa-Convencional-Constituyente-de-la-cc-Constanza-San-Juan-sobre-Justicia-Transicional-0900-hrs-24-01.pdf}
\newline {\color{gray} (Emb: 0.501, TF-IDF: 0.263)}
\newline {\color{gray} \textbf{2º:} 152-4-c-Iniciativa-del-cc-Bernardo-Fontaine-sobre-Derecho-de-propiedad.pdf}
\newline {\color{gray} (Emb: 0.491, TF-IDF: 0.253)}


\item \textbf{Artículo} \newline
Los colegios profesionales son corporaciones de derecho público, nacionales y autónomas, que colaboran con los propósitos y las responsabilidades del Estado. 
\newline {\color{gray} \textbf{1º:} 423-1-Iniciativa-Convencional-Constituyente-del-cc-Marcos-Barraza-sobre-Colegios-Profesionales-0937-01-02.pdf}
\newline {\color{gray} (Emb: 0.968, TF-IDF: 0.903)}
\newline {\color{gray} \textbf{2º:} 1004-Iniciativa-Convencional-Constituyente-del-cc-Marcos-Barraza-sobre-Colegios-Profesionales.pdf}
\newline {\color{gray} (Emb: 0.968, TF-IDF: 0.903)}

Sus labores consisten en velar por el ejercicio ético de sus miembros, promover la credibilidad de la disciplina que profesan sus afiliados, representar oficialmente a la profesión ante el Estado y las demás que establezca la ley. 
\newline {\color{gray} \textbf{1º:} 423-1-Iniciativa-Convencional-Constituyente-del-cc-Marcos-Barraza-sobre-Colegios-Profesionales-0937-01-02.pdf}
\newline {\color{gray} (Emb: 0.924, TF-IDF: 0.880)}
\newline {\color{gray} \textbf{2º:} 1004-Iniciativa-Convencional-Constituyente-del-cc-Marcos-Barraza-sobre-Colegios-Profesionales.pdf}
\newline {\color{gray} (Emb: 0.924, TF-IDF: 0.880)}


\item \textbf{Artículo} \newline
No podrán optar a cargos públicos ni de elección popular las personas condenadas por crímenes de lesa humanidad, delitos sexuales y de violencia intrafamiliar, aquellos vinculados a corrupción como fraude al fisco, lavado de activos, soborno, cohecho, malversación de caudales públicos y los demás que así establezca la ley. 
\newline {\color{gray} \textbf{1º:} 323-1-Iniciativa-Convencional-Constituyente-de-la-cc-Barbara-Sepulveda-sobre-Buen-Gobierno.pdf}
\newline {\color{gray} (Emb: 0.837, TF-IDF: 0.778)}
\newline {\color{gray} \textbf{2º:} 230-2-Iniciativa-Convencional-de-la-cc-Alondra-Carrillo-sobre-Participacion-en-la-Democracia-1142-hrs.pdf}
\newline {\color{gray} (Emb: 0.705, TF-IDF: 0.569)}

Los términos y plazos de estas inhabilidades se determinarán por ley. 
\newline {\color{gray} \textbf{1º:} 100-7-c-Iniciativa-del-cc-Francisco-Caamano-Derecho-al-Acceso-y-a-la-Conectividad-Digital.pdf}
\newline {\color{gray} (Emb: 0.750, TF-IDF: 0.364)}
\newline {\color{gray} \textbf{2º:} 368-7-Iniciativa-Convencional-Constituyente-del-cc-Francisco-Caamano-sobre-los-conocimientos-0900-hrs-24-01.pdf}
\newline {\color{gray} (Emb: 0.706, TF-IDF: 0.333)}


\item \textbf{Artículo} \newline
El Estado tiene el monopolio indelegable del uso legítimo de la fuerza, la que ejerce a través de las instituciones competentes, conforme a esta Constitución, las leyes y con pleno respeto a los derechos humanos. 
\newline {\color{gray} \textbf{1º:} 933-Iniciativa-Convencional-Constituyente-del-cc-Ricardo-Montero-sobre-Fuerzas-Armadas-y-de-Orden.pdf}
\newline {\color{gray} (Emb: 0.799, TF-IDF: 0.694)}
\newline {\color{gray} \textbf{2º:} 933-Iniciativa-Convencional-Constituyente-del-cc-Ricardo-Montero-sobre-Fuerzas-Armadas-y-de-Orden.pdf}
\newline {\color{gray} (Emb: 0.638, TF-IDF: 0.482)}

La ley regulará el uso de la fuerza y el armamento que pueda ser utilizado en el ejercicio de las funciones de las instituciones autorizadas por esta Constitución. 
\newline {\color{gray} \textbf{1º:} 933-Iniciativa-Convencional-Constituyente-del-cc-Ricardo-Montero-sobre-Fuerzas-Armadas-y-de-Orden.pdf}
\newline {\color{gray} (Emb: 1.000, TF-IDF: 1.000)}
\newline {\color{gray} \textbf{2º:} 280-4-Iniciativa-Convencional-de-la-cc-Ivanna-Olivares-sobre-Derecho-de-Prensa-1200-hrs.pdf}
\newline {\color{gray} (Emb: 0.720, TF-IDF: 0.267)}

Ninguna persona, grupo u organización podrá poseer, tener o portar armas u otros elementos similares, salvo en los casos que señale la ley, la que fijará los requisitos, autorizaciones y controles del uso, porte y tenencia de armas. 
\newline {\color{gray} \textbf{1º:} 933-Iniciativa-Convencional-Constituyente-del-cc-Ricardo-Montero-sobre-Fuerzas-Armadas-y-de-Orden.pdf}
\newline {\color{gray} (Emb: 0.997, TF-IDF: 0.955)}
\newline {\color{gray} \textbf{2º:} 111-1-c-Iniciativa-del-cc-Jorge-Arancibia-Fuerzas-Armadas-de-Orden-y-Seguridad.pdf}
\newline {\color{gray} (Emb: 0.886, TF-IDF: 0.631)}


\item \textbf{Artículo} \newline
A la o el Presidente de la República le corresponde la conducción de la defensa nacional y es el jefe supremo de las Fuerzas Armadas. 
\newline {\color{gray} \textbf{1º:} 751-Iniciativa-Convencional-Constituyente-del-cc-Raul-Celis-sobre-Fuerzas-Armadas-01-02.pdf}
\newline {\color{gray} (Emb: 0.857, TF-IDF: 0.742)}
\newline {\color{gray} \textbf{2º:} 933-Iniciativa-Convencional-Constituyente-del-cc-Ricardo-Montero-sobre-Fuerzas-Armadas-y-de-Orden.pdf}
\newline {\color{gray} (Emb: 0.854, TF-IDF: 0.675)}

Ejercerá el mando a través del ministerio a cargo de la defensa nacional. 
\newline {\color{gray} \textbf{1º:} 42-1-Iniciativa-Convencional-Constituyente-de-la-cc-Pollyana-Rivera-y-otros-1.pdf}
\newline {\color{gray} (Emb: 0.865, TF-IDF: 0.602)}
\newline {\color{gray} \textbf{2º:} 933-Iniciativa-Convencional-Constituyente-del-cc-Ricardo-Montero-sobre-Fuerzas-Armadas-y-de-Orden.pdf}
\newline {\color{gray} (Emb: 0.819, TF-IDF: 0.383)}

La disposición, organización y criterios de distribución de las Fuerzas Armadas se establecerán en la Política de Defensa Nacional y la Política Militar. 
\newline {\color{gray} \textbf{1º:} 933-Iniciativa-Convencional-Constituyente-del-cc-Ricardo-Montero-sobre-Fuerzas-Armadas-y-de-Orden.pdf}
\newline {\color{gray} (Emb: 0.981, TF-IDF: 0.917)}
\newline {\color{gray} \textbf{2º:} 933-Iniciativa-Convencional-Constituyente-del-cc-Ricardo-Montero-sobre-Fuerzas-Armadas-y-de-Orden.pdf}
\newline {\color{gray} (Emb: 0.759, TF-IDF: 0.551)}

La ley regulará la vigencia, alcances y mecanismos de elaboración y aprobación de dichas políticas, las que deberán comprender los principios de cooperación internacional, de igualdad de género y de interculturalidad, y el pleno respeto al derecho internacional y los derechos fundamentales. 
\newline {\color{gray} \textbf{1º:} 933-Iniciativa-Convencional-Constituyente-del-cc-Ricardo-Montero-sobre-Fuerzas-Armadas-y-de-Orden.pdf}
\newline {\color{gray} (Emb: 0.946, TF-IDF: 0.907)}
\newline {\color{gray} \textbf{2º:} 933-Iniciativa-Convencional-Constituyente-del-cc-Ricardo-Montero-sobre-Fuerzas-Armadas-y-de-Orden.pdf}
\newline {\color{gray} (Emb: 0.842, TF-IDF: 0.794)}


\item \textbf{Artículo} \newline
La ley regulará la organización de la defensa, su institucionalidad, su estructura y empleo conjunto, sus jefaturas, mando y la carrera militar. 
\newline {\color{gray} \textbf{1º:} 933-Iniciativa-Convencional-Constituyente-del-cc-Ricardo-Montero-sobre-Fuerzas-Armadas-y-de-Orden.pdf}
\newline {\color{gray} (Emb: 0.829, TF-IDF: 0.727)}
\newline {\color{gray} \textbf{2º:} 466-6-Iniciativa-Convencional-Constituyente-de-la-cc-Adriana-Cancino-sobre-Defensoria-de-los-DDHH-1933-31-01.pdf}
\newline {\color{gray} (Emb: 0.725, TF-IDF: 0.384)}

La educación militar se funda en el respeto irrestricto a los derechos humanos. 
\newline {\color{gray} \textbf{1º:} 409-6-Iniciativa-Convencional-Constituyente-de-la-cc-Ingrid-Villena-sobre-Defensoria-del-Pueblo-2239-24-01.pdf}
\newline {\color{gray} (Emb: 0.707, TF-IDF: 0.401)}
\newline {\color{gray} \textbf{2º:} 629-Iniciativa-Convencional-Constituyente-de-cc-Valentina-Miranda-sobre-Derecho-a-la-educacion-1632-hrs.-01-02.pdf}
\newline {\color{gray} (Emb: 0.690, TF-IDF: 0.315)}

Sus integrantes no podrán pertenecer a partidos políticos, asociarse en organizaciones políticas, gremiales o sindicales, ejercer el derecho a huelga, ni postularse a cargos de elección popular. 
\newline {\color{gray} \textbf{1º:} 933-Iniciativa-Convencional-Constituyente-del-cc-Ricardo-Montero-sobre-Fuerzas-Armadas-y-de-Orden.pdf}
\newline {\color{gray} (Emb: 0.999, TF-IDF: 0.964)}
\newline {\color{gray} \textbf{2º:} 933-Iniciativa-Convencional-Constituyente-del-cc-Ricardo-Montero-sobre-Fuerzas-Armadas-y-de-Orden.pdf}
\newline {\color{gray} (Emb: 0.999, TF-IDF: 0.964)}

Las instituciones militares y sus miembros estarán sujetos a controles en materia de probidad y transparencia. 
\newline {\color{gray} \textbf{1º:} 111-1-c-Iniciativa-del-cc-Jorge-Arancibia-Fuerzas-Armadas-de-Orden-y-Seguridad.pdf}
\newline {\color{gray} (Emb: 0.724, TF-IDF: 0.444)}
\newline {\color{gray} \textbf{2º:} 174-1-c-Iniciativa-convencional-del-cc-Rodrigo-Álvarez-sobre-Fuerzas-Armadas1044-hrs.pdf}
\newline {\color{gray} (Emb: 0.641, TF-IDF: 0.418)}

El ingreso y la formación en las Fuerzas Armadas será gratuito y no discriminatorio, en el modo que establezca la ley. 
\newline {\color{gray} \textbf{1º:} 959-1-Iniciativa-Convencional-Constituyente-de-la-cc-Rosa-Catrileo-sobre-Defensa-Plurinacional-1.pdf}
\newline {\color{gray} (Emb: 0.664, TF-IDF: 0.525)}
\newline {\color{gray} \textbf{2º:} 366-4-Iniciativa-Convencional-Constituyente-del-cc-Marco-Arellano-sobre-Derechos-Linguisticos-2232-hrs-21-01.pdf}
\newline {\color{gray} (Emb: 0.616, TF-IDF: 0.355)}

Las Fuerzas Armadas deberán incorporar la perspectiva de género en el desempeño de sus funciones, promover la paridad en espacios de toma de decisión y actuar con pleno respeto al derecho internacional y los derechos fundamentales garantizados en esta Constitución. 
\newline {\color{gray} \textbf{1º:} 933-Iniciativa-Convencional-Constituyente-del-cc-Ricardo-Montero-sobre-Fuerzas-Armadas-y-de-Orden.pdf}
\newline {\color{gray} (Emb: 0.821, TF-IDF: 0.755)}
\newline {\color{gray} \textbf{2º:} 933-Iniciativa-Convencional-Constituyente-del-cc-Ricardo-Montero-sobre-Fuerzas-Armadas-y-de-Orden.pdf}
\newline {\color{gray} (Emb: 0.712, TF-IDF: 0.634)}

Colaboran con la paz y seguridad internacional, conforme a la Política de Defensa Nacional. 
\newline {\color{gray} \textbf{1º:} 933-Iniciativa-Convencional-Constituyente-del-cc-Ricardo-Montero-sobre-Fuerzas-Armadas-y-de-Orden.pdf}
\newline {\color{gray} (Emb: 1.000, TF-IDF: 1.000)}
\newline {\color{gray} \textbf{2º:} 60-2-Iniciativa-Convencional-Constituyente-del-cc-Jorge-Baradit-y-otros.pdf}
\newline {\color{gray} (Emb: 0.732, TF-IDF: 0.420)}

Son instituciones profesionales, jerarquizadas, disciplinadas y por esencia obedientes y no deliberantes. 
\newline {\color{gray} \textbf{1º:} 933-Iniciativa-Convencional-Constituyente-del-cc-Ricardo-Montero-sobre-Fuerzas-Armadas-y-de-Orden.pdf}
\newline {\color{gray} (Emb: 1.000, TF-IDF: 1.000)}
\newline {\color{gray} \textbf{2º:} 933-Iniciativa-Convencional-Constituyente-del-cc-Ricardo-Montero-sobre-Fuerzas-Armadas-y-de-Orden.pdf}
\newline {\color{gray} (Emb: 0.989, TF-IDF: 0.929)}

Dependen del ministerio a cargo de la defensa nacional y son instituciones destinadas para el resguardo de la soberanía, independencia e integridad territorial de la República, ante agresiones de carácter externo, según lo establecido en la Carta de Naciones Unidas. 
\newline {\color{gray} \textbf{1º:} 933-Iniciativa-Convencional-Constituyente-del-cc-Ricardo-Montero-sobre-Fuerzas-Armadas-y-de-Orden.pdf}
\newline {\color{gray} (Emb: 0.966, TF-IDF: 0.966)}
\newline {\color{gray} \textbf{2º:} 431-6-Iniciativa-Convencional-de-la-cc-Bessy-Gallardo-sobre-Defensoria-Penal-Publica-1145-27-01.pdf}
\newline {\color{gray} (Emb: 0.600, TF-IDF: 0.392)}

Las Fuerzas Armadas están integradas única y exclusivamente por el Ejército, la Armada y la Fuerza Aérea. 
\newline {\color{gray} \textbf{1º:} 751-Iniciativa-Convencional-Constituyente-del-cc-Raul-Celis-sobre-Fuerzas-Armadas-01-02.pdf}
\newline {\color{gray} (Emb: 0.796, TF-IDF: 0.757)}
\newline {\color{gray} \textbf{2º:} 42-1-Iniciativa-Convencional-Constituyente-de-la-cc-Pollyana-Rivera-y-otros-1.pdf}
\newline {\color{gray} (Emb: 0.673, TF-IDF: 0.660)}


\item \textbf{Artículo} \newline
El Congreso supervisará periódicamente la ejecución del presupuesto asignado a defensa, así como la implementación de la política de defensa nacional y la política militar. 
\newline {\color{gray} \textbf{1º:} 933-Iniciativa-Convencional-Constituyente-del-cc-Ricardo-Montero-sobre-Fuerzas-Armadas-y-de-Orden.pdf}
\newline {\color{gray} (Emb: 0.578, TF-IDF: 0.310)}
\newline {\color{gray} \textbf{2º:} 933-Iniciativa-Convencional-Constituyente-del-cc-Ricardo-Montero-sobre-Fuerzas-Armadas-y-de-Orden.pdf}
\newline {\color{gray} (Emb: 0.539, TF-IDF: 0.308)}


\item \textbf{Artículo} \newline
A la Presidenta o Presidente de la República le corresponde la conducción de la seguridad pública a través del ministerio correspondiente. 
\newline {\color{gray} \textbf{1º:} 933-Iniciativa-Convencional-Constituyente-del-cc-Ricardo-Montero-sobre-Fuerzas-Armadas-y-de-Orden.pdf}
\newline {\color{gray} (Emb: 0.764, TF-IDF: 0.496)}
\newline {\color{gray} \textbf{2º:} 752-Iniciativa-Convencional-Constituyente-del-cc-Raul-Celis-sobre-Fuerzas-Policiales-01-02.pdf}
\newline {\color{gray} (Emb: 0.758, TF-IDF: 0.404)}

La disposición, organización y criterios de distribución de las policías se establecerá en la Política Nacional de Seguridad Pública. 
\newline {\color{gray} \textbf{1º:} 933-Iniciativa-Convencional-Constituyente-del-cc-Ricardo-Montero-sobre-Fuerzas-Armadas-y-de-Orden.pdf}
\newline {\color{gray} (Emb: 1.000, TF-IDF: 1.000)}
\newline {\color{gray} \textbf{2º:} 933-Iniciativa-Convencional-Constituyente-del-cc-Ricardo-Montero-sobre-Fuerzas-Armadas-y-de-Orden.pdf}
\newline {\color{gray} (Emb: 0.796, TF-IDF: 0.569)}

La ley regulará la vigencia, alcances y mecanismos de elaboración y aprobación de dicha política, la que deberá comprender la perspectiva de género y de interculturalidad, y el pleno respeto al derecho internacional y los derechos fundamentales. 
\newline {\color{gray} \textbf{1º:} 933-Iniciativa-Convencional-Constituyente-del-cc-Ricardo-Montero-sobre-Fuerzas-Armadas-y-de-Orden.pdf}
\newline {\color{gray} (Emb: 0.909, TF-IDF: 0.943)}
\newline {\color{gray} \textbf{2º:} 933-Iniciativa-Convencional-Constituyente-del-cc-Ricardo-Montero-sobre-Fuerzas-Armadas-y-de-Orden.pdf}
\newline {\color{gray} (Emb: 0.890, TF-IDF: 0.851)}


\item \textbf{Artículo} \newline
La educación y formación policial se funda en el respeto irrestricto a los Derechos Humanos. 
\newline {\color{gray} \textbf{1º:} 409-6-Iniciativa-Convencional-Constituyente-de-la-cc-Ingrid-Villena-sobre-Defensoria-del-Pueblo-2239-24-01.pdf}
\newline {\color{gray} (Emb: 0.670, TF-IDF: 0.368)}
\newline {\color{gray} \textbf{2º:} 574-Iniciativa-Convencional-Constituyente-de-cc-Vanessa-Hoppe-sobre-Defensoria-de-los-Pueblos-2351-hrs.-01-02.pdf}
\newline {\color{gray} (Emb: 0.639, TF-IDF: 0.353)}

El ingreso y la formación en las policías será gratuito y no discriminatorio, del modo que establezca la ley. 
\newline {\color{gray} \textbf{1º:} 933-Iniciativa-Convencional-Constituyente-del-cc-Ricardo-Montero-sobre-Fuerzas-Armadas-y-de-Orden.pdf}
\newline {\color{gray} (Emb: 0.647, TF-IDF: 0.456)}
\newline {\color{gray} \textbf{2º:} 366-4-Iniciativa-Convencional-Constituyente-del-cc-Marco-Arellano-sobre-Derechos-Linguisticos-2232-hrs-21-01.pdf}
\newline {\color{gray} (Emb: 0.606, TF-IDF: 0.357)}

Las policías y sus miembros estarán sujetos a controles en materia de probidad y transparencia en la forma y condiciones que determine la Constitución y la ley. 
\newline {\color{gray} \textbf{1º:} 865-Iniciativa-Convencional-Constituyente-del-cc-Marcos-Barraza-sobre-Seguridad-Publica.pdf}
\newline {\color{gray} (Emb: 0.678, TF-IDF: 0.431)}
\newline {\color{gray} \textbf{2º:} 863-Iniciativa-Convencional-Constituyente-del-cc-Marcos-Barraza-sobre-Fuerzas-Armadas.pdf}
\newline {\color{gray} (Emb: 0.666, TF-IDF: 0.409)}

Sus integrantes no podrán pertenecer a partidos políticos, asociarse en organizaciones políticas, gremiales o sindicales, ejercer el derecho a huelga, ni postularse a cargos de elección popular. 
\newline {\color{gray} \textbf{1º:} 933-Iniciativa-Convencional-Constituyente-del-cc-Ricardo-Montero-sobre-Fuerzas-Armadas-y-de-Orden.pdf}
\newline {\color{gray} (Emb: 0.999, TF-IDF: 0.964)}
\newline {\color{gray} \textbf{2º:} 933-Iniciativa-Convencional-Constituyente-del-cc-Ricardo-Montero-sobre-Fuerzas-Armadas-y-de-Orden.pdf}
\newline {\color{gray} (Emb: 0.999, TF-IDF: 0.964)}

Las policías deberán incorporar la perspectiva de género en el desempeño de sus funciones y promover la paridad en espacios de toma de decisión. 
\newline {\color{gray} \textbf{1º:} 863-Iniciativa-Convencional-Constituyente-del-cc-Marcos-Barraza-sobre-Fuerzas-Armadas.pdf}
\newline {\color{gray} (Emb: 0.683, TF-IDF: 0.532)}
\newline {\color{gray} \textbf{2º:} 232-6-Iniciativa-Convencional-del-cc-Marco-Arellano-que-Crea-el-Consejo-Nacional-de-Justicia-1144-hrs.pdf}
\newline {\color{gray} (Emb: 0.626, TF-IDF: 0.375)}

Las policías dependen del ministerio a cargo de la seguridad pública y son instituciones policiales, no militares, de carácter centralizado, con competencia en todo el territorio de Chile, y están destinadas para garantizar la seguridad pública, dar eficacia al derecho y resguardar los derechos fundamentales, en el marco de sus competencias. 
\newline {\color{gray} \textbf{1º:} 933-Iniciativa-Convencional-Constituyente-del-cc-Ricardo-Montero-sobre-Fuerzas-Armadas-y-de-Orden.pdf}
\newline {\color{gray} (Emb: 0.944, TF-IDF: 0.919)}
\newline {\color{gray} \textbf{2º:} 877-Iniciativa-Convencional-Constituyente-de-la-cc-Elsa-Labrana-sobre-Fuerzas-de-Orden.pdf}
\newline {\color{gray} (Emb: 0.664, TF-IDF: 0.399)}

Son instituciones profesionales, jerarquizadas, disciplinadas, obedientes y no deliberantes. 
\newline {\color{gray} \textbf{1º:} 933-Iniciativa-Convencional-Constituyente-del-cc-Ricardo-Montero-sobre-Fuerzas-Armadas-y-de-Orden.pdf}
\newline {\color{gray} (Emb: 0.978, TF-IDF: 0.921)}
\newline {\color{gray} \textbf{2º:} 933-Iniciativa-Convencional-Constituyente-del-cc-Ricardo-Montero-sobre-Fuerzas-Armadas-y-de-Orden.pdf}
\newline {\color{gray} (Emb: 0.961, TF-IDF: 0.856)}

Deberán actuar respetando los principios de necesidad y proporcionalidad en el uso de la fuerza, con pleno respeto al derecho internacional y los derechos fundamentales garantizados en esta Constitución. 
\newline {\color{gray} \textbf{1º:} 933-Iniciativa-Convencional-Constituyente-del-cc-Ricardo-Montero-sobre-Fuerzas-Armadas-y-de-Orden.pdf}
\newline {\color{gray} (Emb: 0.797, TF-IDF: 0.601)}
\newline {\color{gray} \textbf{2º:} 933-Iniciativa-Convencional-Constituyente-del-cc-Ricardo-Montero-sobre-Fuerzas-Armadas-y-de-Orden.pdf}
\newline {\color{gray} (Emb: 0.759, TF-IDF: 0.597)}


\item \textbf{Artículo} \newline
Las relaciones internacionales de Chile, como expresión de su soberanía, se fundan en el respeto al derecho internacional, los principios de autodeterminación de los pueblos, no intervención en asuntos que son de la jurisdicción interna de los Estados, multilateralismo, solidaridad, cooperación, autonomía política e igualdad jurídica entre los Estados. 
\newline {\color{gray} \textbf{1º:} 754-Iniciativa-Convencional-Constituyente-del-cc-Ricardo-Montero-sobre-RREE.pdf}
\newline {\color{gray} (Emb: 0.759, TF-IDF: 0.515)}
\newline {\color{gray} \textbf{2º:} 870-Iniciativa-Convencional-Constituyente-de-la-cc-Alondra-Carrillo-Sobre-Relaciones-Internacionales.pdf}
\newline {\color{gray} (Emb: 0.740, TF-IDF: 0.508)}

De igual forma, se compromete con la promoción y respeto de la democracia, el reconocimiento y protección de los Derechos Humanos, la inclusión e igualdad de género, la justicia social, el respeto a la naturaleza, la paz, convivencia y solución pacífica de los conflictos, y con el reconocimiento, respeto y promoción de los derechos de los pueblos y naciones indígenas y tribales conforme al derecho internacional de los Derechos Humanos. 
\newline {\color{gray} \textbf{1º:} 870-Iniciativa-Convencional-Constituyente-de-la-cc-Alondra-Carrillo-Sobre-Relaciones-Internacionales.pdf}
\newline {\color{gray} (Emb: 0.733, TF-IDF: 0.618)}
\newline {\color{gray} \textbf{2º:} 754-Iniciativa-Convencional-Constituyente-del-cc-Ricardo-Montero-sobre-RREE.pdf}
\newline {\color{gray} (Emb: 0.730, TF-IDF: 0.573)}

Chile declara a América Latina y el Caribe como zona prioritaria en sus relaciones internacionales. 
\newline {\color{gray} \textbf{1º:} 870-Iniciativa-Convencional-Constituyente-de-la-cc-Alondra-Carrillo-Sobre-Relaciones-Internacionales.pdf}
\newline {\color{gray} (Emb: 0.918, TF-IDF: 0.726)}
\newline {\color{gray} \textbf{2º:} 864-Iniciativa-Convencional-Constituyente-del-cc-Marcos-Barraza-sobre-RREE.pdf}
\newline {\color{gray} (Emb: 0.637, TF-IDF: 0.536)}

Se compromete con el mantenimiento de la región como una zona de paz y libre de violencia, impulsa la integración regional, política, social, cultural, económica y productiva entre los Estados, y facilita el contacto y la cooperación transfronteriza entre pueblos indígenas. 
\newline {\color{gray} \textbf{1º:} 925-Iniciativa-Convencional-Constituyente-de-la-cc-Ivanna-Olivares-sobre-Tratados-Internacionales.pdf}
\newline {\color{gray} (Emb: 0.670, TF-IDF: 0.442)}
\newline {\color{gray} \textbf{2º:} 794-Iniciativa-Convencional-Constituyente-de-la-cc-Ivanna-Olivares-sobre-Comercio-Regional-y-Bienes-Naturales.pdf}
\newline {\color{gray} (Emb: 0.670, TF-IDF: 0.428)}


\item \textbf{Artículo} \newline
La ley fijará el plazo para su pronunciamiento. 
\newline {\color{gray} \textbf{1º:} 322-1-Iniciativa-Convencional-Constituyente-de-la-cc-Barbara-Sepulveda-sobre-Formacion-de-la-Ley.pdf}
\newline {\color{gray} (Emb: 0.720, TF-IDF: 0.392)}
\newline {\color{gray} \textbf{2º:} 344-3-Iniciativa-Convencional-Constituyente-del-cc-Hernan-Larrain-sobre-Reforma-Administrativa-y-Modernizacion-del-Estado.pdf}
\newline {\color{gray} (Emb: 0.688, TF-IDF: 0.376)}

Será necesario el acuerdo del Poder Legislativo para el retiro o denuncia de un tratado que haya aprobado y para el retiro de una reserva que haya considerado al aprobarlo. 
\newline {\color{gray} \textbf{1º:} 108-4-c-Iniciativa-de-la-cc-Giovanna-Grandon-Derecho-a-la-Sindicalizacion.pdf}
\newline {\color{gray} (Emb: 0.647, TF-IDF: 0.528)}
\newline {\color{gray} \textbf{2º:} 399-2-Iniciativa-Convencional-Constituyente-de-la-cc-Constanza-San-Juan-sobre-Democracia-Directa-1850-24-01.pdf}
\newline {\color{gray} (Emb: 0.634, TF-IDF: 0.342)}

Las y los habitantes del territorio que hayan cumplido los dieciséis años de edad, en el porcentaje, y de acuerdo a los demás requisitos que defina la ley, tendrán iniciativa para solicitar al Presidente o Presidenta de la República la suscripción de tratados internacionales de derechos humanos. 
\newline {\color{gray} \textbf{1º:} 870-Iniciativa-Convencional-Constituyente-de-la-cc-Alondra-Carrillo-Sobre-Relaciones-Internacionales.pdf}
\newline {\color{gray} (Emb: 0.661, TF-IDF: 0.501)}
\newline {\color{gray} \textbf{2º:} 216-1-c-Iniciativa-Convencional-de-la-cc-Rosa-Catrileo-sobre-Sistema-de-Gobierno-2126-hrs.pdf}
\newline {\color{gray} (Emb: 0.591, TF-IDF: 0.385)}

Asimismo, la ley definirá el plazo dentro del cual la o el Presidente deberá dar respuesta a la referida solicitud. 
\newline {\color{gray} \textbf{1º:} 291-4-Iniciativa-Convencional-de-la-cc-Tatiana-Urrutia-sobre-Derecho-de-Reunion-Peticion-y-Asociacion-1605-hrs.pdf}
\newline {\color{gray} (Emb: 0.629, TF-IDF: 0.310)}
\newline {\color{gray} \textbf{2º:} 807-Iniciativa-Convencional-Constituyente-del-cc-Jaime-Bassa-sobre-Formacion-de-la-Ley.pdf}
\newline {\color{gray} (Emb: 0.599, TF-IDF: 0.302)}

El acuerdo aprobatorio de un tratado podrá autorizar a la Presidenta o Presidente de la República a fin de que, durante la vigencia del tratado, dicte las disposiciones con fuerza de ley que estime necesarias para su cabal cumplimiento, sujeto a las limitaciones previstas en el inciso segundo del artículo 25. 
\newline {\color{gray} \textbf{1º:} 807-Iniciativa-Convencional-Constituyente-del-cc-Jaime-Bassa-sobre-Formacion-de-la-Ley.pdf}
\newline {\color{gray} (Emb: 0.588, TF-IDF: 0.268)}
\newline {\color{gray} \textbf{2º:} 240-1-Iniciativa-Convencional-de-la-cc-Tania-Madriaga-sobre-Poder-Legislativo-1146-hrs.pdf}
\newline {\color{gray} (Emb: 0.556, TF-IDF: 0.254)}

Serán públicos, conforme a las reglas generales, los hechos que digan relación con el tratado internacional, incluidas las negociaciones del mismo, su entrada en vigor, la formulación y retiro de reservas, las declaraciones interpretativas, las objeciones a una reserva y su retiro, la denuncia o retiro del tratado, la suspensión, la terminación y la nulidad del mismo. 
\newline {\color{gray} \textbf{1º:} 240-1-Iniciativa-Convencional-de-la-cc-Tania-Madriaga-sobre-Poder-Legislativo-1146-hrs.pdf}
\newline {\color{gray} (Emb: 0.725, TF-IDF: 0.423)}
\newline {\color{gray} \textbf{2º:} 772-Iniciativa-Convencional-Constituyente-de-la-cc-Angelica-Tepper-sobre-Fuentes-del-Derecho-Internacional.pdf}
\newline {\color{gray} (Emb: 0.591, TF-IDF: 0.394)}

Las medidas que el Ejecutivo adopte o los acuerdos que celebre para el cumplimiento de un tratado en vigor, no requerirán de nueva aprobación del Poder Legislativo, a menos que se trate de materias de ley. 
\newline {\color{gray} \textbf{1º:} 108-4-c-Iniciativa-de-la-cc-Giovanna-Grandon-Derecho-a-la-Sindicalizacion.pdf}
\newline {\color{gray} (Emb: 0.567, TF-IDF: 0.294)}
\newline {\color{gray} \textbf{2º:} 880-Iniciativa-Convencional-Constituyente-de-la-cc-Ingrid-Villena-sobre-Acciones-Constitucionales.pdf}
\newline {\color{gray} (Emb: 0.553, TF-IDF: 0.280)}

Al negociar los tratados o instrumentos internacionales de inversión o similares, la o el Presidente de la República procurará que las instancias de resolución de controversias sean, preferentemente, permanentes, imparciales e independientes. 
\newline {\color{gray} \textbf{1º:} 820-Iniciativa-Convencional-Constituyente-del-cc-Mauricio-Daza-sobre-TTII.pdf}
\newline {\color{gray} (Emb: 0.601, TF-IDF: 0.305)}
\newline {\color{gray} \textbf{2º:} 820-Iniciativa-Convencional-Constituyente-del-cc-Mauricio-Daza-sobre-TTII.pdf}
\newline {\color{gray} (Emb: 0.597, TF-IDF: 0.285)}

Una vez recibido, el Congreso de Diputadas y Diputados podrá sugerir la formulación de reservas y declaraciones interpretativas a un tratado internacional, en el curso del trámite de su aprobación, siempre que ellas procedan de conformidad a lo previsto en el propio tratado o en las normas generales de derecho internacional. 
\newline {\color{gray} \textbf{1º:} 240-1-Iniciativa-Convencional-de-la-cc-Tania-Madriaga-sobre-Poder-Legislativo-1146-hrs.pdf}
\newline {\color{gray} (Emb: 0.957, TF-IDF: 0.962)}
\newline {\color{gray} \textbf{2º:} 573-Iniciativa-Convencional-Constituyente-de-cc-Vanessa-Hoppe-sobre-Defensoria-de-la-Naturaleza-2351-hrs.-01-02.pdf}
\newline {\color{gray} (Emb: 0.648, TF-IDF: 0.356)}

La Presidenta o Presidente de la República enviará el proyecto al Congreso de Diputadas y Diputados e informará sobre el proceso de negociación, el contenido y el alcance del tratado, así como de las reservas que pretenda confirmar o formular. 
\newline {\color{gray} \textbf{1º:} 240-1-Iniciativa-Convencional-de-la-cc-Tania-Madriaga-sobre-Poder-Legislativo-1146-hrs.pdf}
\newline {\color{gray} (Emb: 0.879, TF-IDF: 0.725)}
\newline {\color{gray} \textbf{2º:} 807-Iniciativa-Convencional-Constituyente-del-cc-Jaime-Bassa-sobre-Formacion-de-la-Ley.pdf}
\newline {\color{gray} (Emb: 0.663, TF-IDF: 0.401)}

El proceso de aprobación de un tratado internacional se someterá, en lo pertinente, a los trámites de una ley de acuerdo regional. 
\newline {\color{gray} \textbf{1º:} 99-3-c-Iniciativa-de-la-cc-Tammy-Pustilnick-Disposiciones-del-Estado-Regional.pdf}
\newline {\color{gray} (Emb: 0.739, TF-IDF: 0.323)}
\newline {\color{gray} \textbf{2º:} 983-Iniciativa-Convencional-Constituyente-de-la-cc-Carolina-Vilches-sobre-Territorio-suelos-y-Agua.pdf}
\newline {\color{gray} (Emb: 0.739, TF-IDF: 0.322)}

Se informará al Poder Legislativo de la celebración de los tratados internacionales que no requieran de su aprobación. 
\newline {\color{gray} \textbf{1º:} 399-2-Iniciativa-Convencional-Constituyente-de-la-cc-Constanza-San-Juan-sobre-Democracia-Directa-1850-24-01.pdf}
\newline {\color{gray} (Emb: 0.749, TF-IDF: 0.308)}
\newline {\color{gray} \textbf{2º:} 302-4-Iniciativa-Convencional-del-cc-Fuad-Chahin-sobre-Derecho-al-trabajo-18-01.-1141-hrs.pdf}
\newline {\color{gray} (Emb: 0.705, TF-IDF: 0.302)}

No requerirán esta aprobación los celebrados en cumplimiento de una ley. 
\newline {\color{gray} \textbf{1º:} 413-4-Iniciativa-Convencional-Constituyente-del-cc-Pollyana-Rivera-sobre-Derechos-de-NNA-1255-25-01.pdf}
\newline {\color{gray} (Emb: 0.668, TF-IDF: 0.340)}
\newline {\color{gray} \textbf{2º:} 515-4-Iniciativa-Convencional-Constituyente-de-la-cc-Giovanna-Grandon-sobre-Derecho-a-Migrar-1245-01-02.pdf}
\newline {\color{gray} (Emb: 0.652, TF-IDF: 0.319)}

En aquellos casos en que los tratados internacionales se refieran a materias de ley, ellos deberán ser aprobados por el Poder Legislativo. 
\newline {\color{gray} \textbf{1º:} 236-1-Iniciativa-Convencional-de-la-cc-Barbara-Sepulveda-sobre-Poder-Ejecutivo-1146-hrs.pdf}
\newline {\color{gray} (Emb: 0.676, TF-IDF: 0.301)}
\newline {\color{gray} \textbf{2º:} 286-1-Iniciativa-Convencional-de-la-cc-Barbara-Sepulveda-sobre-Poder-Ejecutivo-1210-hrs.pdf}
\newline {\color{gray} (Emb: 0.676, TF-IDF: 0.293)}

Corresponde a la Presidenta o Presidente de la República la atribución de negociar, concluir, firmar y ratificar los tratados internacionales. 
\newline {\color{gray} \textbf{1º:} 240-1-Iniciativa-Convencional-de-la-cc-Tania-Madriaga-sobre-Poder-Legislativo-1146-hrs.pdf}
\newline {\color{gray} (Emb: 0.793, TF-IDF: 0.559)}
\newline {\color{gray} \textbf{2º:} 236-1-Iniciativa-Convencional-de-la-cc-Barbara-Sepulveda-sobre-Poder-Ejecutivo-1146-hrs.pdf}
\newline {\color{gray} (Emb: 0.675, TF-IDF: 0.484)}

Aprobado el tratado por el Congreso de Diputadas y Diputados, éste será remitido a la Cámara de las Regiones para su tramitación. 
\newline {\color{gray} \textbf{1º:} 234-1-Iniciativa-Convencional-del-cc-Jaime-Bassa-sobre-Justicia-Complementaria-1144-hrs.pdf}
\newline {\color{gray} (Emb: 0.619, TF-IDF: 0.455)}
\newline {\color{gray} \textbf{2º:} 402-3-Iniciativa-Convencional-Constituyente-de-la-cc-Elisa-Giustinianovich-sobre-Territorios-Especiales-1906-24-01.pdf}
\newline {\color{gray} (Emb: 0.589, TF-IDF: 0.432)}


\item \textbf{Artículo} \newline
Sólo se podrá suspender o limitar el ejercicio de los derechos y garantías que la Constitución asegura a todas las personas bajo las siguientes situaciones de excepción: conflicto armado internacional, conflicto armado interno según establece el derecho internacional o calamidad pública. 
\newline {\color{gray} \textbf{1º:} 885-Iniciativa-Convencional-Constituyente-de-la-cc-Lidia-Gonzalez-Sobre-Estados-de-Excepcion-Constitucional.pdf}
\newline {\color{gray} (Emb: 0.858, TF-IDF: 0.532)}
\newline {\color{gray} \textbf{2º:} 941-Iniciativa-Convencional-Constituyente-de-la-cc-Dayyana-Gonzalez-sobre-Campamentos.pdf}
\newline {\color{gray} (Emb: 0.653, TF-IDF: 0.358)}

No podrán restringirse o suspenderse sino los derechos y garantías expresamente señalados en esta Constitución. 
\newline {\color{gray} \textbf{1º:} 73-4-Iniciativa-Convencional-Constituyente-del-cc-Manuel-Jose-Ossandon-y-otro.pdf}
\newline {\color{gray} (Emb: 0.719, TF-IDF: 0.747)}
\newline {\color{gray} \textbf{2º:} 885-Iniciativa-Convencional-Constituyente-de-la-cc-Lidia-Gonzalez-Sobre-Estados-de-Excepcion-Constitucional.pdf}
\newline {\color{gray} (Emb: 0.701, TF-IDF: 0.238)}

La declaración y renovación de los estados de excepción constitucional respetará los principios de proporcionalidad y necesidad, y se limitarán, tanto respecto de su duración, extensión y medios empleados, a lo que sea estrictamente necesario para la más pronta restauración de la normalidad constitucional. 
\newline {\color{gray} \textbf{1º:} 239-1-Iniciativa-Convencional-de-la-cc-Tania-Madriaga-sobre-Poder-Ejecutivo-1146-hrs.pdf}
\newline {\color{gray} (Emb: 0.995, TF-IDF: 0.974)}
\newline {\color{gray} \textbf{2º:} 239-1-Iniciativa-Convencional-de-la-cc-Tania-Madriaga-sobre-Poder-Ejecutivo-1146-hrs.pdf}
\newline {\color{gray} (Emb: 0.626, TF-IDF: 0.347)}


\item \textbf{Artículo} \newline
El estado de asamblea mantendrá su vigencia por el tiempo que se extienda la situación de conflicto armado internacional, salvo que la Presidenta o Presidente de la República disponga su término con anterioridad o el Congreso de Diputadas y Diputados y la Cámara de las Regiones retiren su autorización. 
\newline {\color{gray} \textbf{1º:} 885-Iniciativa-Convencional-Constituyente-de-la-cc-Lidia-Gonzalez-Sobre-Estados-de-Excepcion-Constitucional.pdf}
\newline {\color{gray} (Emb: 0.684, TF-IDF: 0.540)}
\newline {\color{gray} \textbf{2º:} 405-1-Iniciativa-Convencional-Constituyente-del-cc-Martin-Arrau-sobre-Posesion-Presidencial-1946-24-01.pdf}
\newline {\color{gray} (Emb: 0.626, TF-IDF: 0.274)}

La declaración de estado de sitio sólo podrá extenderse por un plazo de quince días, sin perjuicio de que la Presidenta o Presidente de la República solicite su prórroga, para lo cual requerirá el pronunciamiento conforme de cuatro séptimos de las diputadas, los diputados y representantes regionales en ejercicio para la primera prórroga, de tres quintos para la segunda y de dos tercios para la tercera y siguientes. 
\newline {\color{gray} \textbf{1º:} 344-3-Iniciativa-Convencional-Constituyente-del-cc-Hernan-Larrain-sobre-Reforma-Administrativa-y-Modernizacion-del-Estado.pdf}
\newline {\color{gray} (Emb: 0.641, TF-IDF: 0.610)}
\newline {\color{gray} \textbf{2º:} 239-1-Iniciativa-Convencional-de-la-cc-Tania-Madriaga-sobre-Poder-Ejecutivo-1146-hrs.pdf}
\newline {\color{gray} (Emb: 0.621, TF-IDF: 0.272)}

En este caso, sólo podrá restringir el ejercicio del derecho de reunión. 
\newline {\color{gray} \textbf{1º:} 885-Iniciativa-Convencional-Constituyente-de-la-cc-Lidia-Gonzalez-Sobre-Estados-de-Excepcion-Constitucional.pdf}
\newline {\color{gray} (Emb: 0.879, TF-IDF: 0.825)}
\newline {\color{gray} \textbf{2º:} 239-1-Iniciativa-Convencional-de-la-cc-Tania-Madriaga-sobre-Poder-Ejecutivo-1146-hrs.pdf}
\newline {\color{gray} (Emb: 0.691, TF-IDF: 0.656)}

Sin embargo, la Presidenta o Presidente de la República, en circunstancias de necesidad impostergable, y sólo con la firma de todas sus Ministras y Ministros, podrá aplicar de inmediato el estado de asamblea o de sitio, mientras el Congreso de Diputadas y Diputados y la Cámara de las Regiones se pronuncien sobre la declaración. 
\newline {\color{gray} \textbf{1º:} 885-Iniciativa-Convencional-Constituyente-de-la-cc-Lidia-Gonzalez-Sobre-Estados-de-Excepcion-Constitucional.pdf}
\newline {\color{gray} (Emb: 0.945, TF-IDF: 0.832)}
\newline {\color{gray} \textbf{2º:} 467-6-Iniciativa-Convencional-Constituyente-del-cc-Daniel-Bravo-sobre-Reforma-y-Reemplazo-1945-31-01.pdf}
\newline {\color{gray} (Emb: 0.681, TF-IDF: 0.349)}

En su solicitud y posterior declaración, se deberán especificar los fundamentos que justifiquen la extrema necesidad de la declaración, pudiendo el Congreso y la Cámara solamente introducir modificaciones respecto de su extensión territorial. 
\newline {\color{gray} \textbf{1º:} 885-Iniciativa-Convencional-Constituyente-de-la-cc-Lidia-Gonzalez-Sobre-Estados-de-Excepcion-Constitucional.pdf}
\newline {\color{gray} (Emb: 0.941, TF-IDF: 0.964)}
\newline {\color{gray} \textbf{2º:} 425-6-Iniciativa-Convencional-del-cc-Christian-Viera-sobre-Reforma-de-la-Constitucion-1554-26-01.pdf}
\newline {\color{gray} (Emb: 0.486, TF-IDF: 0.257)}

El Congreso de Diputadas y Diputados y la Cámara de las Regiones, en sesión conjunta, dentro del plazo de veinticuatro horas contadas desde el momento en que la Presidenta o Presidente de la República someta la declaración de estado de asamblea o de sitio a su consideración, deberá pronunciarse por la mayoría de sus miembros aceptando o rechazando la proposición. 
\newline {\color{gray} \textbf{1º:} 885-Iniciativa-Convencional-Constituyente-de-la-cc-Lidia-Gonzalez-Sobre-Estados-de-Excepcion-Constitucional.pdf}
\newline {\color{gray} (Emb: 0.908, TF-IDF: 0.847)}
\newline {\color{gray} \textbf{2º:} 544-Iniciativa-Convencional-Constituyente-del-cc-Andres-Cruz-sobre-reforma-constitucional-1623-01-02.pdf}
\newline {\color{gray} (Emb: 0.701, TF-IDF: 0.363)}

La declaración deberá determinar las zonas afectadas por el estado de excepción correspondiente. 
\newline {\color{gray} \textbf{1º:} 885-Iniciativa-Convencional-Constituyente-de-la-cc-Lidia-Gonzalez-Sobre-Estados-de-Excepcion-Constitucional.pdf}
\newline {\color{gray} (Emb: 1.000, TF-IDF: 1.000)}
\newline {\color{gray} \textbf{2º:} 239-1-Iniciativa-Convencional-de-la-cc-Tania-Madriaga-sobre-Poder-Ejecutivo-1146-hrs.pdf}
\newline {\color{gray} (Emb: 1.000, TF-IDF: 1.000)}

El estado de asamblea, en caso de conflicto armado internacional, y el estado de sitio, en caso de conflicto armado interno, serán declarados por la Presidenta o Presidente de la República con la autorización del Congreso de Diputadas y Diputados y la Cámara de las Regiones, en sesión conjunta. 
\newline {\color{gray} \textbf{1º:} 885-Iniciativa-Convencional-Constituyente-de-la-cc-Lidia-Gonzalez-Sobre-Estados-de-Excepcion-Constitucional.pdf}
\newline {\color{gray} (Emb: 0.730, TF-IDF: 0.380)}
\newline {\color{gray} \textbf{2º:} 631-Iniciativa-Convencional-Constituyente-de-cc-Ingrid-Villena-sobre-Contraloria-General-de-la-Republica.pdf}
\newline {\color{gray} (Emb: 0.576, TF-IDF: 0.342)}

Si el Congreso y la Cámara no se pronunciaran dentro de dicho plazo, serán citado por el sólo ministerio de la Constitución a sesiones especiales diarias, hasta que se pronuncien sobre la declaración. 
\newline {\color{gray} \textbf{1º:} 885-Iniciativa-Convencional-Constituyente-de-la-cc-Lidia-Gonzalez-Sobre-Estados-de-Excepcion-Constitucional.pdf}
\newline {\color{gray} (Emb: 0.913, TF-IDF: 0.662)}
\newline {\color{gray} \textbf{2º:} 234-1-Iniciativa-Convencional-del-cc-Jaime-Bassa-sobre-Justicia-Complementaria-1144-hrs.pdf}
\newline {\color{gray} (Emb: 0.581, TF-IDF: 0.286)}


\item \textbf{Artículo} \newline
El estado de catástrofe, en caso de calamidad pública, lo declarará la Presidenta o Presidente de la República. 
\newline {\color{gray} \textbf{1º:} 750-Iniciativa-Convencional-Constituyente-del-cc-Raul-Celis-sobre-Estados-de-Excepcion-01-02.pdf}
\newline {\color{gray} (Emb: 0.854, TF-IDF: 0.698)}
\newline {\color{gray} \textbf{2º:} 239-1-Iniciativa-Convencional-de-la-cc-Tania-Madriaga-sobre-Poder-Ejecutivo-1146-hrs.pdf}
\newline {\color{gray} (Emb: 0.671, TF-IDF: 0.431)}

La declaración deberá establecer el ámbito de aplicación y el plazo de duración, el que no podrá ser mayor a treinta días. 
\newline {\color{gray} \textbf{1º:} 184-6-c-Iniciativa-Convencional-del-cc-Rodrigo-Álvarez-que-crea-la-Corte-Constitucional-1044hrs.pdf}
\newline {\color{gray} (Emb: 0.716, TF-IDF: 0.270)}
\newline {\color{gray} \textbf{2º:} 223-6-Iniciativa-Convencional-del-cc-Cristobal-Andrade-Reforma-Constitucional-1020-hrs-17-01.pdf}
\newline {\color{gray} (Emb: 0.631, TF-IDF: 0.270)}

La Presidenta o Presidente de la República estará obligado a informar al Congreso de Diputadas y Diputados de las medidas adoptadas en virtud del estado de catástrofe. 
\newline {\color{gray} \textbf{1º:} 750-Iniciativa-Convencional-Constituyente-del-cc-Raul-Celis-sobre-Estados-de-Excepcion-01-02.pdf}
\newline {\color{gray} (Emb: 0.775, TF-IDF: 0.486)}
\newline {\color{gray} \textbf{2º:} 750-Iniciativa-Convencional-Constituyente-del-cc-Raul-Celis-sobre-Estados-de-Excepcion-01-02.pdf}
\newline {\color{gray} (Emb: 0.753, TF-IDF: 0.339)}

La Presidenta o Presidente de la República sólo podrá declarar el estado de catástrofe por un período superior a treinta días con acuerdo del Congreso de Diputadas y Diputados. 
\newline {\color{gray} \textbf{1º:} 750-Iniciativa-Convencional-Constituyente-del-cc-Raul-Celis-sobre-Estados-de-Excepcion-01-02.pdf}
\newline {\color{gray} (Emb: 0.727, TF-IDF: 0.632)}
\newline {\color{gray} \textbf{2º:} 172-6-c-Iniciativa-Convencional-del-cc-Rodrigo-Álvarez-sobre-Banco-central1044-hrs.pdf}
\newline {\color{gray} (Emb: 0.715, TF-IDF: 0.364)}

El referido acuerdo se tramitará en la forma establecida en el inciso segundo del artículo 23. 
\newline {\color{gray} \textbf{1º:} 635-4-Iniciativa-Convencional-Constituyente-del-cc-Cristobal-Andrade-sobre-Libertad-de-Conciencia-1730-01-02.pdf}
\newline {\color{gray} (Emb: 0.468, TF-IDF: 0.295)}
\newline {\color{gray} \textbf{2º:} 108-4-c-Iniciativa-de-la-cc-Giovanna-Grandon-Derecho-a-la-Sindicalizacion.pdf}
\newline {\color{gray} (Emb: 0.411, TF-IDF: 0.287)}

Declarado el estado de catástrofe, las zonas respectivas quedarán bajo la dependencia inmediata de la Jefa o Jefe de Estado de Excepción, quien deberá ser una autoridad civil designada por la Presidenta o Presidente de la República. 
\newline {\color{gray} \textbf{1º:} 239-1-Iniciativa-Convencional-de-la-cc-Tania-Madriaga-sobre-Poder-Ejecutivo-1146-hrs.pdf}
\newline {\color{gray} (Emb: 0.722, TF-IDF: 0.595)}
\newline {\color{gray} \textbf{2º:} 239-1-Iniciativa-Convencional-de-la-cc-Tania-Madriaga-sobre-Poder-Ejecutivo-1146-hrs.pdf}
\newline {\color{gray} (Emb: 0.709, TF-IDF: 0.532)}

Ésta asumirá la dirección y supervigilancia de su jurisdicción con las atribuciones y deberes que la ley señale. 
\newline {\color{gray} \textbf{1º:} 239-1-Iniciativa-Convencional-de-la-cc-Tania-Madriaga-sobre-Poder-Ejecutivo-1146-hrs.pdf}
\newline {\color{gray} (Emb: 0.999, TF-IDF: 0.919)}
\newline {\color{gray} \textbf{2º:} 239-1-Iniciativa-Convencional-de-la-cc-Tania-Madriaga-sobre-Poder-Ejecutivo-1146-hrs.pdf}
\newline {\color{gray} (Emb: 0.999, TF-IDF: 0.919)}


\item \textbf{Artículo} \newline
La Presidenta o Presidente de la República podrá solicitar la prórroga del estado de catástrofe, para lo cual requerirá la aprobación de la mayoría de los integrantes del Congreso de Diputadas y Diputados y la Cámara de las Regiones, quienes resolverán en sesión conjunta. 
\newline {\color{gray} \textbf{1º:} 750-Iniciativa-Convencional-Constituyente-del-cc-Raul-Celis-sobre-Estados-de-Excepcion-01-02.pdf}
\newline {\color{gray} (Emb: 0.712, TF-IDF: 0.382)}
\newline {\color{gray} \textbf{2º:} 425-6-Iniciativa-Convencional-del-cc-Christian-Viera-sobre-Reforma-de-la-Constitucion-1554-26-01.pdf}
\newline {\color{gray} (Emb: 0.705, TF-IDF: 0.348)}


\item \textbf{Artículo} \newline
Podrá, asimismo, disponer requisiciones de bienes, establecer limitaciones al ejercicio del derecho de propiedad y adoptar todas las medidas extraordinarias de carácter legal y administrativo que sean necesarias para el pronto restablecimiento de la normalidad en la zona afectada. 
\newline {\color{gray} \textbf{1º:} 239-1-Iniciativa-Convencional-de-la-cc-Tania-Madriaga-sobre-Poder-Ejecutivo-1146-hrs.pdf}
\newline {\color{gray} (Emb: 0.983, TF-IDF: 0.982)}
\newline {\color{gray} \textbf{2º:} 750-Iniciativa-Convencional-Constituyente-del-cc-Raul-Celis-sobre-Estados-de-Excepcion-01-02.pdf}
\newline {\color{gray} (Emb: 0.648, TF-IDF: 0.399)}

Por la declaración del estado de catástrofe, la Presidenta o Presidente de la República podrá restringir las libertades de locomoción y de reunión. 
\newline {\color{gray} \textbf{1º:} 239-1-Iniciativa-Convencional-de-la-cc-Tania-Madriaga-sobre-Poder-Ejecutivo-1146-hrs.pdf}
\newline {\color{gray} (Emb: 0.966, TF-IDF: 0.938)}
\newline {\color{gray} \textbf{2º:} 239-1-Iniciativa-Convencional-de-la-cc-Tania-Madriaga-sobre-Poder-Ejecutivo-1146-hrs.pdf}
\newline {\color{gray} (Emb: 0.899, TF-IDF: 0.739)}

Por la declaración del estado de asamblea, la Presidenta o el Presidente de la República estará facultado para restringir la libertad personal, el derecho de reunión, la libertad de trabajo, el ejercicio del derecho de asociación y para interceptar, abrir o registrar documentos y toda clase de comunicaciones, disponer requisiciones de bienes y establecer limitaciones al ejercicio del derecho de propiedad. 
\newline {\color{gray} \textbf{1º:} 239-1-Iniciativa-Convencional-de-la-cc-Tania-Madriaga-sobre-Poder-Ejecutivo-1146-hrs.pdf}
\newline {\color{gray} (Emb: 0.754, TF-IDF: 0.776)}
\newline {\color{gray} \textbf{2º:} 750-Iniciativa-Convencional-Constituyente-del-cc-Raul-Celis-sobre-Estados-de-Excepcion-01-02.pdf}
\newline {\color{gray} (Emb: 0.742, TF-IDF: 0.573)}

Por la declaración del estado de sitio, la Presidenta o Presidente de la República podrá restringir la libertad de movimiento y la libertad de asociación. 
\newline {\color{gray} \textbf{1º:} 239-1-Iniciativa-Convencional-de-la-cc-Tania-Madriaga-sobre-Poder-Ejecutivo-1146-hrs.pdf}
\newline {\color{gray} (Emb: 0.873, TF-IDF: 0.642)}
\newline {\color{gray} \textbf{2º:} 239-1-Iniciativa-Convencional-de-la-cc-Tania-Madriaga-sobre-Poder-Ejecutivo-1146-hrs.pdf}
\newline {\color{gray} (Emb: 0.820, TF-IDF: 0.445)}

Podrá, además, suspender o restringir el ejercicio del derecho de reunión. 
\newline {\color{gray} \textbf{1º:} 239-1-Iniciativa-Convencional-de-la-cc-Tania-Madriaga-sobre-Poder-Ejecutivo-1146-hrs.pdf}
\newline {\color{gray} (Emb: 1.000, TF-IDF: 1.000)}
\newline {\color{gray} \textbf{2º:} 408-4-Iniciativa-Convencional-Constituyente-de-la-cc-Tatiana-Urrutia-sobre-Sufragio-0041-17-01.pdf}
\newline {\color{gray} (Emb: 0.655, TF-IDF: 0.513)}


\item \textbf{Artículo} \newline
Los actos de la Presidenta o Presidente de la República o la Jefa o Jefe de Estado de Excepción, que tengan por fundamento la declaración del estado de excepción constitucional, deberán señalar expresamente los derechos constitucionales que suspendan o restrinjan. 
\newline {\color{gray} \textbf{1º:} 174-1-c-Iniciativa-convencional-del-cc-Rodrigo-Álvarez-sobre-Fuerzas-Armadas1044-hrs.pdf}
\newline {\color{gray} (Emb: 0.709, TF-IDF: 0.364)}
\newline {\color{gray} \textbf{2º:} 325-6-Iniciativa-Convencional-del-cc-Tomas-Laibe-sobre-Jurisdiccion-Constitucional.pdf}
\newline {\color{gray} (Emb: 0.696, TF-IDF: 0.333)}

El decreto de declaración deberá indicar específicamente las medidas a adoptarse en razón de la excepción, las que deberán ser proporcionales a los fines establecidos en la declaración de excepción, y no limitar excesivamente o impedir de manera total el legítimo ejercicio de cualquier derecho establecido en esta Constitución. 
\newline {\color{gray} \textbf{1º:} 145-4-c-Iniciativa-del-cc-Manuel-Jose-Ossandon-libertad-de-Conciencia-Expresion-y-de-Ensenanza.pdf}
\newline {\color{gray} (Emb: 0.579, TF-IDF: 0.310)}
\newline {\color{gray} \textbf{2º:} 232-6-Iniciativa-Convencional-del-cc-Marco-Arellano-que-Crea-el-Consejo-Nacional-de-Justicia-1144-hrs.pdf}
\newline {\color{gray} (Emb: 0.577, TF-IDF: 0.308)}

Las medidas que se adopten durante los estados de excepción no podrán, bajo ninguna circunstancia, prolongarse más allá de la vigencia de los mismos. 
\newline {\color{gray} \textbf{1º:} 239-1-Iniciativa-Convencional-de-la-cc-Tania-Madriaga-sobre-Poder-Ejecutivo-1146-hrs.pdf}
\newline {\color{gray} (Emb: 1.000, TF-IDF: 1.000)}
\newline {\color{gray} \textbf{2º:} 126-4-c-Iniciativa-de-la-cc-Rocio-Cantuarias-asegura-la-libertad-de-eleccion-en-prestaciones-de-salud.pdf}
\newline {\color{gray} (Emb: 0.575, TF-IDF: 0.351)}

Los estados de excepción constitucional permitirán a la Presidenta o Presidente de la República el ejercicio de potestades y competencias que ordinariamente estarían reservadas al nivel regional o comunal cuando el restablecimiento de la normalidad así lo requiera. 
\newline {\color{gray} \textbf{1º:} 151-3-c-Iniciativa-de-la-cc-Angelica-Tepper-Competencias-de-los-Gobiernos-Regionales.pdf}
\newline {\color{gray} (Emb: 0.675, TF-IDF: 0.347)}
\newline {\color{gray} \textbf{2º:} 750-Iniciativa-Convencional-Constituyente-del-cc-Raul-Celis-sobre-Estados-de-Excepcion-01-02.pdf}
\newline {\color{gray} (Emb: 0.669, TF-IDF: 0.291)}


\item \textbf{Artículo} \newline
Una ley regulará los estados de excepción, así como su declaración y la aplicación de las medidas legales y administrativas que procediera adoptar bajo aquéllos, en todo lo no regulado por esta Constitución. 
\newline {\color{gray} \textbf{1º:} 239-1-Iniciativa-Convencional-de-la-cc-Tania-Madriaga-sobre-Poder-Ejecutivo-1146-hrs.pdf}
\newline {\color{gray} (Emb: 0.916, TF-IDF: 0.898)}
\newline {\color{gray} \textbf{2º:} 201-6-c-Iniciativa-Convencional-del-cc-Felipe-Harboe-sobre-la-Contraloria-1658-hrs.pdf}
\newline {\color{gray} (Emb: 0.667, TF-IDF: 0.366)}

Dicha ley no podrá afectar las competencias y el funcionamiento de los órganos constitucionales, ni los derechos e inmunidades de sus respectivos titulares. 
\newline {\color{gray} \textbf{1º:} 239-1-Iniciativa-Convencional-de-la-cc-Tania-Madriaga-sobre-Poder-Ejecutivo-1146-hrs.pdf}
\newline {\color{gray} (Emb: 0.784, TF-IDF: 0.702)}
\newline {\color{gray} \textbf{2º:} 107-4-c-Iniciativa-de-la-cc-Giovanna-Grandon-Derecho-al-Trabajo.pdf}
\newline {\color{gray} (Emb: 0.718, TF-IDF: 0.263)}

Asimismo, esta ley regulará el modo en el que la Presidenta o Presidente de la República y las autoridades que éste encomendare rendirán cuenta detallada, veraz y oportuna al Congreso de Diputadas y Diputados de las medidas extraordinarias adoptadas y de los planes para la superación de la situación de excepción, así como de los hechos de gravedad que hubieran surgido con ocasión del estado de excepción constitucional. 
\newline {\color{gray} \textbf{1º:} 174-1-c-Iniciativa-convencional-del-cc-Rodrigo-Álvarez-sobre-Fuerzas-Armadas1044-hrs.pdf}
\newline {\color{gray} (Emb: 0.702, TF-IDF: 0.307)}
\newline {\color{gray} \textbf{2º:} 239-1-Iniciativa-Convencional-de-la-cc-Tania-Madriaga-sobre-Poder-Ejecutivo-1146-hrs.pdf}
\newline {\color{gray} (Emb: 0.696, TF-IDF: 0.259)}

La omisión de este deber de rendición de cuentas se considerará una infracción a la Constitución. 
\newline {\color{gray} \textbf{1º:} 1014-Iniciativa-Convencional-Constituyente-cc-Adriana-Ampuero-Haciendas-territoriales-y-autonomia-financiera.pdf}
\newline {\color{gray} (Emb: 0.762, TF-IDF: 0.363)}
\newline {\color{gray} \textbf{2º:} 232-6-Iniciativa-Convencional-del-cc-Marco-Arellano-que-Crea-el-Consejo-Nacional-de-Justicia-1144-hrs.pdf}
\newline {\color{gray} (Emb: 0.762, TF-IDF: 0.357)}


\item \textbf{Artículo} \newline
La ley regulará la integración y funcionamiento de la Comisión de Fiscalización. 
\newline {\color{gray} \textbf{1º:} 806-Iniciativa-Convencional-Constituyente-del-cc-Hugo-Gutierrez-sobre-Gobiernos-Regionales.pdf}
\newline {\color{gray} (Emb: 0.671, TF-IDF: 0.574)}
\newline {\color{gray} \textbf{2º:} 924-Iniciativa-Convencional-Constituyente-de-la-cc-Constanza-Schonhaut-sobre-Administracion-del-Estado.pdf}
\newline {\color{gray} (Emb: 0.670, TF-IDF: 0.417)}

En caso de que tome conocimiento de vulneraciones a lo dispuesto en esta Constitución o la ley, la Comisión de Fiscalización deberá efectuar las denuncias pertinentes, las cuales serán remitidas y conocidas por los órganos competentes. 
\newline {\color{gray} \textbf{1º:} 472-6-Iniciativa-Convencional-Constituyente-del-cc-Daniel-Bravo-sobre-Corte-Constitucional-2003-31-01.pdf}
\newline {\color{gray} (Emb: 0.553, TF-IDF: 0.231)}
\newline {\color{gray} \textbf{2º:} 608-Iniciativa-Convencional-Constituyente-de-cc-Miguel-Angel-Botto-sobre-Ministerio-Publico-2119-hrs.-01-02.pdf}
\newline {\color{gray} (Emb: 0.538, TF-IDF: 0.212)}

Los órganos del Estado deberán colaborar y aportar todos los antecedentes requeridos por la comisión para el desempeño de sus funciones. 
\newline {\color{gray} \textbf{1º:} 954-5-Iniciativa-Convencional-Constituyente-de-la-cc-Carolina-Vilches-sobre-Estatuto-del-Agua.pdf}
\newline {\color{gray} (Emb: 0.675, TF-IDF: 0.305)}
\newline {\color{gray} \textbf{2º:} 344-3-Iniciativa-Convencional-Constituyente-del-cc-Hernan-Larrain-sobre-Reforma-Administrativa-y-Modernizacion-del-Estado.pdf}
\newline {\color{gray} (Emb: 0.657, TF-IDF: 0.234)}

Una vez declarado el estado de excepción, se constituirá una Comisión de Fiscalización dependiente del Congreso de Diputadas y Diputados, de composición paritaria y plurinacional, integrada por diputadas y diputados, por representantes regionales y por representantes de la Defensoría de los Pueblos, en la forma que establezca la ley. 
\newline {\color{gray} \textbf{1º:} 219-1-c-Iniciativa-Convencional-de-la-cc-Rosa-Catrileo-sobre-Parlamento-Bicameral-2215-hrs.pdf}
\newline {\color{gray} (Emb: 0.681, TF-IDF: 0.306)}
\newline {\color{gray} \textbf{2º:} 608-Iniciativa-Convencional-Constituyente-de-cc-Miguel-Angel-Botto-sobre-Ministerio-Publico-2119-hrs.-01-02.pdf}
\newline {\color{gray} (Emb: 0.645, TF-IDF: 0.306)}

Dicho órgano deberá fiscalizar las medidas adoptadas bajo el estado de excepción, para lo cual emitirá informes periódicos que contengan un análisis de ellas, su proporcionalidad y la observancia de los derechos humanos y tendrá las demás atribuciones que le encomiende la ley. 
\newline {\color{gray} \textbf{1º:} 526-2-Iniciativa-Convencional-Constituyente-de-cc-Marcos-Barraza-sobre-recepcion-de-TTII-en-DDHH-1140-hrs.-01-02.pdf}
\newline {\color{gray} (Emb: 0.593, TF-IDF: 0.333)}
\newline {\color{gray} \textbf{2º:} 243-4-Iniciativa-Convencional-de-la-cc-Elsa-Labrana-sobre-Derecho-a-la-Rebelion-1147-hrs.pdf}
\newline {\color{gray} (Emb: 0.592, TF-IDF: 0.241)}


\item \textbf{Artículo} \newline
Las medidas adoptadas en ejercicio de las facultades conferidas en los estados de excepción constitucional, podrán ser objeto de revisión por los tribunales de justicia tanto en su mérito como en su forma. 
\newline {\color{gray} \textbf{1º:} 750-Iniciativa-Convencional-Constituyente-del-cc-Raul-Celis-sobre-Estados-de-Excepcion-01-02.pdf}
\newline {\color{gray} (Emb: 0.743, TF-IDF: 0.405)}
\newline {\color{gray} \textbf{2º:} 325-6-Iniciativa-Convencional-del-cc-Tomas-Laibe-sobre-Jurisdiccion-Constitucional.pdf}
\newline {\color{gray} (Emb: 0.661, TF-IDF: 0.401)}

Las requisiciones que se practiquen darán lugar a indemnizaciones en conformidad a la ley. 
\newline {\color{gray} \textbf{1º:} 239-1-Iniciativa-Convencional-de-la-cc-Tania-Madriaga-sobre-Poder-Ejecutivo-1146-hrs.pdf}
\newline {\color{gray} (Emb: 0.909, TF-IDF: 0.917)}
\newline {\color{gray} \textbf{2º:} 90-6-Iniciativa-Convencional-Constituyente-del-cc-Tomas-Laibe-y-otros.pdf}
\newline {\color{gray} (Emb: 0.709, TF-IDF: 0.310)}


\item \textbf{Artículo} \newline
Chile es un Estado social y democrático de derecho. 
\newline {\color{gray} \textbf{1º:} 7-2-Iniciativa-Convencional-Constituyente-del-cc-Luis-Barceló-y-otros.pdf}
\newline {\color{gray} (Emb: 0.939, TF-IDF: 0.932)}
\newline {\color{gray} \textbf{2º:} 59-2-Iniciativa-Convencional-Constituyente-de-la-cc-Ericka-Portilla-y-otros.pdf}
\newline {\color{gray} (Emb: 0.895, TF-IDF: 0.870)}

Es plurinacional, intercultural y ecológico. 
\newline {\color{gray} \textbf{1º:} 94-1-Iniciativa-de-la-cc-Rosa-Catrileo-Establece-el-reconocimiento-de-los-Pueblos-Indigenas-2.pdf}
\newline {\color{gray} (Emb: 0.554, TF-IDF: 0.626)}
\newline {\color{gray} \textbf{2º:} 363-4-Iniciativa-Convencional-Constituyente-de-la-cc-Janis-Meneses-sobre-Derecho-a-la-Educacion-Publica-1844-hrs-21-01.pdf}
\newline {\color{gray} (Emb: 0.539, TF-IDF: 0.574)}

Se constituye como una República solidaria, su democracia es paritaria y reconoce como valores intrínsecos e irrenunciables la dignidad, la libertad, la igualdad sustantiva de los seres humanos y su relación indisoluble con la naturaleza. 
\newline {\color{gray} \textbf{1º:} 58-2-Iniciativa-Convencional-Constituyente-de-la-cc-Loreto-Vallejos-y-otros.pdf}
\newline {\color{gray} (Emb: 0.823, TF-IDF: 0.484)}
\newline {\color{gray} \textbf{2º:} 59-2-Iniciativa-Convencional-Constituyente-de-la-cc-Ericka-Portilla-y-otros.pdf}
\newline {\color{gray} (Emb: 0.637, TF-IDF: 0.414)}

La protección y garantía de los derechos humanos individuales y colectivos son el fundamento del Estado y orientan toda su actividad. 
\newline {\color{gray} \textbf{1º:} 237-1-Iniciativa-Convencional-de-la-cc-Tania-Madriaga-sobre-Estado-Plurinacional-y-Libre-Determinacion-1146-hrs.pdf}
\newline {\color{gray} (Emb: 0.759, TF-IDF: 0.340)}
\newline {\color{gray} \textbf{2º:} 394-3-Iniciativa-Convencional-Constituyente-de-la-cc-Ramona-Reyes-sobre-Comuna-Autonoma-1525-24-01.pdf}
\newline {\color{gray} (Emb: 0.745, TF-IDF: 0.322)}

Es deber del Estado generar las condiciones necesarias y proveer los bienes y servicios para asegurar el igual goce de los derechos y la integración de las personas en la vida política, económica, social y cultural para su pleno desarrollo. 
\newline {\color{gray} \textbf{1º:} 60-2-Iniciativa-Convencional-Constituyente-del-cc-Jorge-Baradit-y-otros.pdf}
\newline {\color{gray} (Emb: 0.694, TF-IDF: 0.458)}
\newline {\color{gray} \textbf{2º:} 210-1-c-Iniciativa-Convencional-del-cc-Cristián-Monckeberg-sobre-Estado-Intercultural-1953-hrs.pdf}
\newline {\color{gray} (Emb: 0.680, TF-IDF: 0.346)}


\item \textbf{Artículo} \newline
Para su protección, las personas gozarán de todas las garantías eficaces, oportunas, pertinentes y universales, nacionales e internacionales. 
\newline {\color{gray} \textbf{1º:} 378-2-Iniciativa-Convencional-Constituyente-del-cc-Alvin-Saldana-sobre-Integracion-de-DDHH-1046-hrs-24-01.pdf}
\newline {\color{gray} (Emb: 0.804, TF-IDF: 0.700)}
\newline {\color{gray} \textbf{2º:} 60-2-Iniciativa-Convencional-Constituyente-del-cc-Jorge-Baradit-y-otros.pdf}
\newline {\color{gray} (Emb: 0.802, TF-IDF: 0.700)}

En Chile, las personas nacen y permanecen libres, interdependientes e iguales en dignidad y derechos. 
\newline {\color{gray} \textbf{1º:} 60-2-Iniciativa-Convencional-Constituyente-del-cc-Jorge-Baradit-y-otros.pdf}
\newline {\color{gray} (Emb: 0.981, TF-IDF: 0.905)}
\newline {\color{gray} \textbf{2º:} 38-2-Iniciativa-Convencional-Constituyente-del-cc-Martín-Arrau-y-otros.pdf}
\newline {\color{gray} (Emb: 0.873, TF-IDF: 0.749)}

El Estado debe respetar, promover, proteger y garantizar los derechos fundamentales reconocidos en esta Constitución y en los tratados internacionales ratificados por Chile y que se encuentren vigentes. 
\newline {\color{gray} \textbf{1º:} 84-2-Iniciativa-Convencional-Constituyente-del-cc-Martin-Arrau-y-otros.pdf}
\newline {\color{gray} (Emb: 0.869, TF-IDF: 0.778)}
\newline {\color{gray} \textbf{2º:} 30-2-Iniciativa-Convencional-Constituyente-del-cc-Martín-Arrau-y-otros.pdf}
\newline {\color{gray} (Emb: 0.869, TF-IDF: 0.769)}


\item \textbf{Artículo} \newline
Se ejerce democráticamente, de manera directa y mediante representantes, de conformidad a lo dispuesto en esta Constitución y las leyes. 
\newline {\color{gray} \textbf{1º:} 7-2-Iniciativa-Convencional-Constituyente-del-cc-Luis-Barceló-y-otros.pdf}
\newline {\color{gray} (Emb: 0.899, TF-IDF: 0.416)}
\newline {\color{gray} \textbf{2º:} 7-2-Iniciativa-Convencional-Constituyente-del-cc-Luis-Barceló-y-otros.pdf}
\newline {\color{gray} (Emb: 0.761, TF-IDF: 0.344)}

Ningún sector del pueblo ni individuo alguno puede atribuirse su ejercicio. 
\newline {\color{gray} \textbf{1º:} 84-2-Iniciativa-Convencional-Constituyente-del-cc-Martin-Arrau-y-otros.pdf}
\newline {\color{gray} (Emb: 1.000, TF-IDF: 1.000)}
\newline {\color{gray} \textbf{2º:} 30-2-Iniciativa-Convencional-Constituyente-del-cc-Martín-Arrau-y-otros.pdf}
\newline {\color{gray} (Emb: 1.000, TF-IDF: 1.000)}

La soberanía reside en el Pueblo de Chile, conformado por diversas naciones. 
\newline {\color{gray} \textbf{1º:} 60-2-Iniciativa-Convencional-Constituyente-del-cc-Jorge-Baradit-y-otros.pdf}
\newline {\color{gray} (Emb: 0.925, TF-IDF: 0.569)}
\newline {\color{gray} \textbf{2º:} 59-2-Iniciativa-Convencional-Constituyente-de-la-cc-Ericka-Portilla-y-otros.pdf}
\newline {\color{gray} (Emb: 0.725, TF-IDF: 0.530)}

El ejercicio de la soberanía reconoce como limitación los derechos humanos en cuanto atributo que deriva de la dignidad humana. 
\newline {\color{gray} \textbf{1º:} 526-2-Iniciativa-Convencional-Constituyente-de-cc-Marcos-Barraza-sobre-recepcion-de-TTII-en-DDHH-1140-hrs.-01-02.pdf}
\newline {\color{gray} (Emb: 0.994, TF-IDF: 0.798)}
\newline {\color{gray} \textbf{2º:} 772-Iniciativa-Convencional-Constituyente-de-la-cc-Angelica-Tepper-sobre-Fuentes-del-Derecho-Internacional.pdf}
\newline {\color{gray} (Emb: 0.933, TF-IDF: 0.691)}


\item \textbf{Artículo} \newline
La actividad política organizada contribuye a la expresión de la voluntad popular, y su funcionamiento respetará los principios de independencia, probidad, transparencia financiera y democracia interna. 
\newline {\color{gray} \textbf{1º:} 71-2-Iniciativa-Convencional-Constitutente-de-Maria-Jose-Oyarzun-y-otros.pdf}
\newline {\color{gray} (Emb: 0.825, TF-IDF: 0.902)}
\newline {\color{gray} \textbf{2º:} 447-Iniciativa-Convencional-Constituyente-del-cc-Wilfredo-Bacian-sobre-Estatuto-de-las-Ues.-Estatales-1501-28-01.pdf}
\newline {\color{gray} (Emb: 0.638, TF-IDF: 0.293)}

El Estado deberá asegurar la prevalencia del interés general, y el carácter electivo de los cargos de representación política con responsabilidad de quienes ejercen el poder. 
\newline {\color{gray} \textbf{1º:} 60-2-Iniciativa-Convencional-Constituyente-del-cc-Jorge-Baradit-y-otros.pdf}
\newline {\color{gray} (Emb: 0.876, TF-IDF: 0.779)}
\newline {\color{gray} \textbf{2º:} 94-1-Iniciativa-de-la-cc-Rosa-Catrileo-Establece-el-reconocimiento-de-los-Pueblos-Indigenas-2.pdf}
\newline {\color{gray} (Emb: 0.619, TF-IDF: 0.246)}

Se ejerce en forma directa, participativa, comunitaria y representativa. 
\newline {\color{gray} \textbf{1º:} 377-2-Iniciativa-Convencional-Constituyente-del-cc-Alvin-Saldana-sobre-Mecanismos-de-Democracia-1045-hrs-24-01.pdf}
\newline {\color{gray} (Emb: 0.566, TF-IDF: 0.725)}
\newline {\color{gray} \textbf{2º:} 71-2-Iniciativa-Convencional-Constitutente-de-Maria-Jose-Oyarzun-y-otros.pdf}
\newline {\color{gray} (Emb: 0.533, TF-IDF: 0.449)}

Es deber del Estado promover y garantizar la adopción de medidas para la participación efectiva de toda la sociedad en el proceso político y el pleno ejercicio de la democracia. 
\newline {\color{gray} \textbf{1º:} 927-Iniciativa-Convencional-Constituyente-de-la-cc-Alondra-Carrillo-sobre-Economia-y-Desarrollo-Plurinacional.pdf}
\newline {\color{gray} (Emb: 0.629, TF-IDF: 0.369)}
\newline {\color{gray} \textbf{2º:} 403-7-Iniciativa-Convencional-Constituyente-de-la-cc-Cristina-Dorador-sobre-Neurodivergencia-1931-24-01.pdf}
\newline {\color{gray} (Emb: 0.618, TF-IDF: 0.350)}

En Chile, la democracia es inclusiva y paritaria. 
\newline {\color{gray} \textbf{1º:} 59-2-Iniciativa-Convencional-Constituyente-de-la-cc-Ericka-Portilla-y-otros.pdf}
\newline {\color{gray} (Emb: 0.820, TF-IDF: 0.592)}
\newline {\color{gray} \textbf{2º:} 377-2-Iniciativa-Convencional-Constituyente-del-cc-Alvin-Saldana-sobre-Mecanismos-de-Democracia-1045-hrs-24-01.pdf}
\newline {\color{gray} (Emb: 0.790, TF-IDF: 0.379)}


\item \textbf{Artículo} \newline
La Constitución asegura a todas las personas la igualdad sustantiva, en tanto garantía de igualdad de trato y oportunidades para el reconocimiento, goce y ejercicio de los derechos humanos y las libertades fundamentales, con pleno respeto a la diversidad, la inclusión social y la integración de los grupos oprimidos e históricamente excluidos. 
\newline {\color{gray} \textbf{1º:} 27-4-Iniciativa-Convencional-Constituyente-de-la-cc-Adriana-Cancino-y-otros.pdf}
\newline {\color{gray} (Emb: 0.673, TF-IDF: 0.393)}
\newline {\color{gray} \textbf{2º:} 674-Iniciativa-Convencional-Constituyente-del-cc-Cesar-Valenzuela-sobre-Seguridad-Social-121101-02.pdf}
\newline {\color{gray} (Emb: 0.644, TF-IDF: 0.362)}

La Constitución asegura la igualdad sustantiva de género, obligándose a garantizar el mismo trato y condiciones para las mujeres, niñas y diversidades y disidencias sexogenéricas ante todos los órganos estatales y espacios de organización de la sociedad civil. 
\newline {\color{gray} \textbf{1º:} 116-1-c-Iniciativa-de-la-cc-Alondra-Carrillo-Democracia-Paritaria.pdf}
\newline {\color{gray} (Emb: 0.775, TF-IDF: 0.658)}
\newline {\color{gray} \textbf{2º:} 237-1-Iniciativa-Convencional-de-la-cc-Tania-Madriaga-sobre-Estado-Plurinacional-y-Libre-Determinacion-1146-hrs.pdf}
\newline {\color{gray} (Emb: 0.668, TF-IDF: 0.446)}


\item \textbf{Artículo} \newline
El Estado reconoce y protege a las familias en sus diversas formas, expresiones y modos de vida, no restringiéndose a vínculos exclusivamente filiativos y consanguíneos. 
\newline {\color{gray} \textbf{1º:} 7-2-Iniciativa-Convencional-Constituyente-del-cc-Luis-Barceló-y-otros.pdf}
\newline {\color{gray} (Emb: 0.779, TF-IDF: 0.373)}
\newline {\color{gray} \textbf{2º:} 59-2-Iniciativa-Convencional-Constituyente-de-la-cc-Ericka-Portilla-y-otros.pdf}
\newline {\color{gray} (Emb: 0.708, TF-IDF: 0.315)}

El Estado debe garantizar a las familias una vida digna, procurando que los trabajos de cuidados no representen una desventaja para quienes los ejercen. 
\newline {\color{gray} \textbf{1º:} 774-Iniciativa-Convencional-Constituyente-de-la-cc-Barbara-Rebolledo-sobre-Derechos-de-las-Mujeres.pdf}
\newline {\color{gray} (Emb: 0.619, TF-IDF: 0.443)}
\newline {\color{gray} \textbf{2º:} 947-Iniciativa-Convencional-constituyente-de-la-cc-Barbara-Rebolledo-reconocimiento-y-valoracion-del-trabajo-domestico.pdf}
\newline {\color{gray} (Emb: 0.616, TF-IDF: 0.441)}


\item \textbf{Artículo} \newline
Las personas y los pueblos son interdependientes con la naturaleza y forman, con ella, un conjunto inseparable. 
\newline {\color{gray} \textbf{1º:} 771-Iniciativa-Convencional-Constituyente-del-cc-Fernando-Salinas-sobre-Principio-de-Interdependencia.pdf}
\newline {\color{gray} (Emb: 0.633, TF-IDF: 0.267)}
\newline {\color{gray} \textbf{2º:} 1001-Iniciativa-Convencional-Constituyente-del-cc-Juan-Jose-Martin-sobre-Principio-Ecocentrico.pdf}
\newline {\color{gray} (Emb: 0.624, TF-IDF: 0.226)}

La naturaleza tiene derechos. 
\newline {\color{gray} \textbf{1º:} 71-2-Iniciativa-Convencional-Constitutente-de-Maria-Jose-Oyarzun-y-otros.pdf}
\newline {\color{gray} (Emb: 1.000, TF-IDF: 1.000)}
\newline {\color{gray} \textbf{2º:} 792-Iniciativa-Convencional-Constituyente-de-la-cc-Elsa-Labrana-sobre-Derechos-de-la-Naturaleza.pdf}
\newline {\color{gray} (Emb: 0.759, TF-IDF: 0.485)}

El Estado y la sociedad tienen el deber de protegerlos y respetarlos. 
\newline {\color{gray} \textbf{1º:} 858-Iniciativa-Convencional-Constituyente-del-cc-Jorge-Abarca-sobre-Estado-Ecologico-de-Derecho.pdf}
\newline {\color{gray} (Emb: 0.745, TF-IDF: 0.335)}
\newline {\color{gray} \textbf{2º:} 587-Iniciativa-Convencional-Constituyente-de-cc-Marcos-Barraza-sobre-Derecho-al-Trabajo-2351-hrs.-01-02.pdf}
\newline {\color{gray} (Emb: 0.721, TF-IDF: 0.327)}

El Estado debe adoptar una administración ecológicamente responsable y promover la educación ambiental y científica mediante procesos de formación y aprendizaje permanentes. 
\newline {\color{gray} \textbf{1º:} 858-Iniciativa-Convencional-Constituyente-del-cc-Jorge-Abarca-sobre-Estado-Ecologico-de-Derecho.pdf}
\newline {\color{gray} (Emb: 0.859, TF-IDF: 0.614)}
\newline {\color{gray} \textbf{2º:} 857-Iniciativa-Convencional-Constituyente-del-cc-Jorge-Abarca-sobre-Derecho-a-un-Ambiente-Sano.pdf}
\newline {\color{gray} (Emb: 0.835, TF-IDF: 0.533)}


\item \textbf{Artículo} \newline
El Estado reconoce y promueve una relación de equilibrio armónico entre las personas, la naturaleza y la organización de la sociedad. 
\newline {\color{gray} \textbf{1º:} 796-Iniciativa-Convencional-Constituyente-del-cc-Juan-Jose-Martin-sobre-Principio-Pro-Natura.pdf}
\newline {\color{gray} (Emb: 0.781, TF-IDF: 0.416)}
\newline {\color{gray} \textbf{2º:} 72-2-Iniciativa-Convencional-Constituyente-de-la-cc-Elisa-Loncon-y-otros.pdf}
\newline {\color{gray} (Emb: 0.677, TF-IDF: 0.335)}


\item \textbf{Artículo} \newline
Quien dañe el medio ambiente tendrá el deber de repararlo, sin perjuicio de las sanciones administrativas, penales y civiles que correspondan en conformidad a la constitución y las leyes. 
\newline {\color{gray} \textbf{1º:} 672-Iniciativa-Convencional-Constituyente-del-cc-Carolina-Sepulveda-sobre-Principios-Ambientales-121101-02.pdf}
\newline {\color{gray} (Emb: 0.998, TF-IDF: 0.971)}
\newline {\color{gray} \textbf{2º:} 971-Iniciativa-Convencional-Constituyente-de-la-cc-Ivanna-Olivares-sobre-Relaves.pdf}
\newline {\color{gray} (Emb: 0.691, TF-IDF: 0.411)}


\item \textbf{Artículo} \newline
Es deber integral del Estado la conservación, preservación y cuidado de los ecosistemas marinos y costeros continentales, insulares y antárticos. 
\newline {\color{gray} \textbf{1º:} 761-Iniciativa-Convencinal-Constituyente-del-cc-Alvin-Saldana-sobre-Principio-de-Estado-Oceanico.pdf}
\newline {\color{gray} (Emb: 0.951, TF-IDF: 0.907)}
\newline {\color{gray} \textbf{2º:} 647-Iniciativa-Convencional-Constituyente-de-la-cc-Camila-Zarate-sobre-Derecho-a-la-Salud-181101-02.pdf}
\newline {\color{gray} (Emb: 0.805, TF-IDF: 0.398)}


\item \textbf{Artículo} \newline
Los derechos y obligaciones establecidos en los tratados internacionales de derechos humanos ratificados por Chile y que se encuentren vigentes, los principios generales del derecho internacional de los derechos humanos y el derecho internacional consuetudinario de la misma materia forman parte integral de esta constitución y gozan de rango constitucional. 
\newline {\color{gray} \textbf{1º:} 378-2-Iniciativa-Convencional-Constituyente-del-cc-Alvin-Saldana-sobre-Integracion-de-DDHH-1046-hrs-24-01.pdf}
\newline {\color{gray} (Emb: 0.937, TF-IDF: 0.818)}
\newline {\color{gray} \textbf{2º:} 836-Iniciativa-Convencional-Constituyente-de-la-cc-Elisa-Loncon-sobre-Jerarquia-e-interpretacion-de-DDHH.pdf}
\newline {\color{gray} (Emb: 0.874, TF-IDF: 0.681)}

El Estado tiene la obligación de promover, respetar, proteger y garantizar los derechos humanos conforme a las disposiciones y principios del derecho internacional de los derechos humanos. 
\newline {\color{gray} \textbf{1º:} 378-2-Iniciativa-Convencional-Constituyente-del-cc-Alvin-Saldana-sobre-Integracion-de-DDHH-1046-hrs-24-01.pdf}
\newline {\color{gray} (Emb: 0.757, TF-IDF: 0.548)}
\newline {\color{gray} \textbf{2º:} 813-Iniciativa-Convencional-Constituyente-del-cc-Manuel-Woldarsky-crea-la-Agencia-de-DDHH.pdf}
\newline {\color{gray} (Emb: 0.754, TF-IDF: 0.480)}

Asimismo, debe prevenir, investigar, sancionar y reparar integralmente las violaciones a los derechos humanos. 
\newline {\color{gray} \textbf{1º:} 256-4-Iniciativa-Convencional-de-la-cc-Elsa-Labrana-sobre-Parte-General-de-los-DDFF.pdf}
\newline {\color{gray} (Emb: 0.837, TF-IDF: 0.774)}
\newline {\color{gray} \textbf{2º:} 451-4-Iniciativa-Convencional-Constituyente-de-la-cc-Carolina-Videla-sobre-Tortura-y-desaparicion-1409-31-01.pdf}
\newline {\color{gray} (Emb: 0.695, TF-IDF: 0.640)}


\item \textbf{Artículo} \newline
El Estado es intercultural. 
\newline {\color{gray} \textbf{1º:} 9-2-Iniciativa-Convencional-Constituyente-del-cc-Ignacio-Achurra-y-otros-cta.pdf}
\newline {\color{gray} (Emb: 0.690, TF-IDF: 0.842)}
\newline {\color{gray} \textbf{2º:} 59-2-Iniciativa-Convencional-Constituyente-de-la-cc-Ericka-Portilla-y-otros.pdf}
\newline {\color{gray} (Emb: 0.643, TF-IDF: 0.753)}

Reconocerá, valorará y promoverá el diálogo horizontal y transversal entre las diversas cosmovisiones de los pueblos y naciones que conviven en el país con dignidad y respeto recíproco. 
\newline {\color{gray} \textbf{1º:} 72-2-Iniciativa-Convencional-Constituyente-de-la-cc-Elisa-Loncon-y-otros.pdf}
\newline {\color{gray} (Emb: 0.663, TF-IDF: 0.366)}
\newline {\color{gray} \textbf{2º:} 188-7-c-Iniciativa-Convenciona-del-cc-Ignacio-Achurra-sobre-Rol-del-Estado-en-las-Culturas-1126-hrs.pdf}
\newline {\color{gray} (Emb: 0.576, TF-IDF: 0.361)}

El Estado deberá garantizar los mecanismos institucionales que permitan ese diálogo superando las asimetrías existentes en el acceso, distribución y ejercicio del poder y en todos los ámbitos de la vida en sociedad. 
\newline {\color{gray} \textbf{1º:} 72-2-Iniciativa-Convencional-Constituyente-de-la-cc-Elisa-Loncon-y-otros.pdf}
\newline {\color{gray} (Emb: 0.725, TF-IDF: 0.390)}
\newline {\color{gray} \textbf{2º:} 226-6-Iniciativa-Convencional-de-la-cc-Manuela-Royo-sobre-Justicia-Local-1140-hrs.pdf}
\newline {\color{gray} (Emb: 0.549, TF-IDF: 0.276)}


\item \textbf{Artículo} \newline
Chile es un Estado plurilingüe, su idioma oficial es el castellano y los idiomas de los pueblos indígenas serán oficiales en sus territorios y en zonas de alta densidad poblacional de cada pueblo indígena. 
\newline {\color{gray} \textbf{1º:} 404-4-Iniciativa-Convencional-Constituyente-de-la-cc-Lidia-Gonzalez-sobre-Derechos-Linguisticos-1940-24-01.pdf}
\newline {\color{gray} (Emb: 0.915, TF-IDF: 0.720)}
\newline {\color{gray} \textbf{2º:} 72-2-Iniciativa-Convencional-Constituyente-de-la-cc-Elisa-Loncon-y-otros.pdf}
\newline {\color{gray} (Emb: 0.799, TF-IDF: 0.510)}

El Estado promueve el conocimiento, revitalización, valoración y respeto de las lenguas indígenas de todos los pueblos del Estado Plurinacional. 
\newline {\color{gray} \textbf{1º:} 404-4-Iniciativa-Convencional-Constituyente-de-la-cc-Lidia-Gonzalez-sobre-Derechos-Linguisticos-1940-24-01.pdf}
\newline {\color{gray} (Emb: 0.897, TF-IDF: 0.780)}
\newline {\color{gray} \textbf{2º:} 718-Iniciativa-Convencional-Constituyente-de-la-cc-Maria-Elisa-Quinteros-Personas-en-Situacion-de-Discapacidad-01-02.pdf}
\newline {\color{gray} (Emb: 0.658, TF-IDF: 0.304)}

El Estado reconoce la lengua de señas chilena como lengua natural y oficial de las personas sordas, así como sus derechos lingüísticos en todos los ámbitos de la vida social. 
\newline {\color{gray} \textbf{1º:} 366-4-Iniciativa-Convencional-Constituyente-del-cc-Marco-Arellano-sobre-Derechos-Linguisticos-2232-hrs-21-01.pdf}
\newline {\color{gray} (Emb: 0.757, TF-IDF: 0.798)}
\newline {\color{gray} \textbf{2º:} 1026-Iniciatva-Convencional-Constituyente-del-cc-Mariela-Serey-sobre-Personas-Discapacitadas.pdf}
\newline {\color{gray} (Emb: 0.710, TF-IDF: 0.762)}


\item \textbf{Artículo} \newline
Chile es un Estado Laico, donde se respeta y garantiza la libertad de religión y de creencias espirituales. 
\newline {\color{gray} \textbf{1º:} 549-Iniciativa-Convencional-Constituyente-de-la-cc-Barbara-Sepulveda-sobre-libertad-de-conciencia-1701-01-02.pdf}
\newline {\color{gray} (Emb: 0.826, TF-IDF: 0.502)}
\newline {\color{gray} \textbf{2º:} 258-4-Iniciativa-Convencional-de-la-cc-Janis-Meneses-sobre-Libertad-de-Conciencia-y-Religion-1152-hrs.pdf}
\newline {\color{gray} (Emb: 0.758, TF-IDF: 0.502)}

Ninguna religión, ni creencia en particular es la oficial del Estado, sin perjuicio de su reconocimiento y libre ejercicio, el cual no tiene más limitación que lo dispuesto por esta Constitución. 
\newline {\color{gray} \textbf{1º:} 7-2-Iniciativa-Convencional-Constituyente-del-cc-Luis-Barceló-y-otros.pdf}
\newline {\color{gray} (Emb: 0.739, TF-IDF: 0.301)}
\newline {\color{gray} \textbf{2º:} 252-4-Iniciativa-Convencional-de-la-cc-Tatiana-Urrutia-sobre-Libertad-de-Conciencia-y-Culto-1149-hrs.pdf}
\newline {\color{gray} (Emb: 0.686, TF-IDF: 0.301)}


\item \textbf{Artículo} \newline
Asimismo, podrá considerar otras medidas apropiadas para resolverlos. 
\newline {\color{gray} \textbf{1º:} 1015-Iniciativa-Convencional-Constituyente-de-la-cc-Paulina-Valenzuela-sobre-Probidad.pdf}
\newline {\color{gray} (Emb: 1.000, TF-IDF: 1.000)}
\newline {\color{gray} \textbf{2º:} 323-1-Iniciativa-Convencional-Constituyente-de-la-cc-Barbara-Sepulveda-sobre-Buen-Gobierno.pdf}
\newline {\color{gray} (Emb: 0.709, TF-IDF: 0.642)}

Una ley regulará los casos y las condiciones en las que los funcionarios, funcionarias y autoridades deleguen a terceros la administración de aquellos bienes y obligaciones que supongan un conflicto de interés en el ejercicio de la función pública. 
\newline {\color{gray} \textbf{1º:} 1015-Iniciativa-Convencional-Constituyente-de-la-cc-Paulina-Valenzuela-sobre-Probidad.pdf}
\newline {\color{gray} (Emb: 0.965, TF-IDF: 0.928)}
\newline {\color{gray} \textbf{2º:} 323-1-Iniciativa-Convencional-Constituyente-de-la-cc-Barbara-Sepulveda-sobre-Buen-Gobierno.pdf}
\newline {\color{gray} (Emb: 0.931, TF-IDF: 0.749)}

Esta obligación abarca el deber de perseguir administrativa y judicialmente la aplicación de las sanciones administrativas, civiles y penales que correspondan, en la forma que determine la ley. 
\newline {\color{gray} \textbf{1º:} 40-2-Iniciativa-Convencional-Constituyente-de-la-cc-Bernardo-de-la-Maza-y-otros.pdf}
\newline {\color{gray} (Emb: 0.958, TF-IDF: 0.966)}
\newline {\color{gray} \textbf{2º:} 108-4-c-Iniciativa-de-la-cc-Giovanna-Grandon-Derecho-a-la-Sindicalizacion.pdf}
\newline {\color{gray} (Emb: 0.726, TF-IDF: 0.408)}

Es deber del Estado promover la integridad de la función pública y erradicar la corrupción en todas sus formas, tanto en el sector público como privado. 
\newline {\color{gray} \textbf{1º:} 40-2-Iniciativa-Convencional-Constituyente-de-la-cc-Bernardo-de-la-Maza-y-otros.pdf}
\newline {\color{gray} (Emb: 1.000, TF-IDF: 1.000)}
\newline {\color{gray} \textbf{2º:} 194-1-c-Iniciativa-Convenciona-de-la-cc-Francisca-Arauna-sobre-Pp.-Buen-Gobierno-y-Probidad-1402-hrs.pdf}
\newline {\color{gray} (Emb: 0.717, TF-IDF: 0.485)}

El ejercicio de las funciones públicas obliga a sus titulares a dar estricto cumplimiento a los principios de probidad, transparencia y rendición de cuentas en todas sus actuaciones, con primacía del interés general por sobre el particular. 
\newline {\color{gray} \textbf{1º:} 40-2-Iniciativa-Convencional-Constituyente-de-la-cc-Bernardo-de-la-Maza-y-otros.pdf}
\newline {\color{gray} (Emb: 0.918, TF-IDF: 0.854)}
\newline {\color{gray} \textbf{2º:} 70-2-Iniciativa-Convencional-Constituyente-de-la-cc-Paulina-Veloso-y-otros-2.pdf}
\newline {\color{gray} (Emb: 0.878, TF-IDF: 0.778)}

En cumplimiento de lo anterior, deberá adoptar medidas eficaces para prevenir, detectar y sancionar los actos de corrupción. 
\newline {\color{gray} \textbf{1º:} 40-2-Iniciativa-Convencional-Constituyente-de-la-cc-Bernardo-de-la-Maza-y-otros.pdf}
\newline {\color{gray} (Emb: 0.931, TF-IDF: 0.919)}
\newline {\color{gray} \textbf{2º:} 672-Iniciativa-Convencional-Constituyente-del-cc-Carolina-Sepulveda-sobre-Principios-Ambientales-121101-02.pdf}
\newline {\color{gray} (Emb: 0.605, TF-IDF: 0.491)}


\item \textbf{Artículo} \newline
Chile es un Estado fundado en el principio de la supremacía constitucional y el respeto irrestricto a los derechos humanos. 
\newline {\color{gray} \textbf{1º:} 84-2-Iniciativa-Convencional-Constituyente-del-cc-Martin-Arrau-y-otros.pdf}
\newline {\color{gray} (Emb: 0.854, TF-IDF: 0.812)}
\newline {\color{gray} \textbf{2º:} 31-2-Iniciativa-Convencional-Constituyente-del-cc-Martín-Arrau-y-otros.pdf}
\newline {\color{gray} (Emb: 0.854, TF-IDF: 0.812)}

Los preceptos de esta Constitución obligan igualmente a toda persona, institución, autoridad o grupo. 
\newline {\color{gray} \textbf{1º:} 70-2-Iniciativa-Convencional-Constituyente-de-la-cc-Paulina-Veloso-y-otros-2.pdf}
\newline {\color{gray} (Emb: 0.860, TF-IDF: 0.714)}
\newline {\color{gray} \textbf{2º:} 7-2-Iniciativa-Convencional-Constituyente-del-cc-Luis-Barceló-y-otros.pdf}
\newline {\color{gray} (Emb: 0.774, TF-IDF: 0.640)}

Los órganos del Estado y sus titulares e integrantes, actúan previa investidura regular y someten su actuar a la Constitución y a las normas dictadas conforme a esta, dentro de los límites y competencias por ellas establecidos. 
\newline {\color{gray} \textbf{1º:} 70-2-Iniciativa-Convencional-Constituyente-de-la-cc-Paulina-Veloso-y-otros-2.pdf}
\newline {\color{gray} (Emb: 0.791, TF-IDF: 0.564)}
\newline {\color{gray} \textbf{2º:} 31-2-Iniciativa-Convencional-Constituyente-del-cc-Martín-Arrau-y-otros.pdf}
\newline {\color{gray} (Emb: 0.739, TF-IDF: 0.564)}

Ninguna magistratura, ninguna persona ni grupo de personas pueden atribuirse, ni aun a pretexto de circunstancias extraordinarias, autoridad, derechos o facultades distintas a las expresamente conferidas en virtud de la Constitución o las leyes. 
\newline {\color{gray} \textbf{1º:} 32-2-Iniciativa-Convencional-Constituyente-del-cc-Martín-Arrau-y-otros.pdf}
\newline {\color{gray} (Emb: 1.000, TF-IDF: 1.000)}
\newline {\color{gray} \textbf{2º:} 84-2-Iniciativa-Convencional-Constituyente-del-cc-Martin-Arrau-y-otros.pdf}
\newline {\color{gray} (Emb: 1.000, TF-IDF: 1.000)}

Todo acto en contravención a este artículo es nulo y originará las responsabilidades y sanciones que la ley señale. 
\newline {\color{gray} \textbf{1º:} 70-2-Iniciativa-Convencional-Constituyente-de-la-cc-Paulina-Veloso-y-otros-2.pdf}
\newline {\color{gray} (Emb: 0.794, TF-IDF: 0.411)}
\newline {\color{gray} \textbf{2º:} 84-2-Iniciativa-Convencional-Constituyente-del-cc-Martin-Arrau-y-otros.pdf}
\newline {\color{gray} (Emb: 0.729, TF-IDF: 0.338)}

La acción de nulidad se ejercerá en los plazos y condiciones establecidos por esta Constitución y la ley. 
\newline {\color{gray} \textbf{1º:} 408-4-Iniciativa-Convencional-Constituyente-de-la-cc-Tatiana-Urrutia-sobre-Sufragio-0041-17-01.pdf}
\newline {\color{gray} (Emb: 0.767, TF-IDF: 0.489)}
\newline {\color{gray} \textbf{2º:} 1014-Iniciativa-Convencional-Constituyente-cc-Adriana-Ampuero-Haciendas-territoriales-y-autonomia-financiera.pdf}
\newline {\color{gray} (Emb: 0.738, TF-IDF: 0.295)}

Ninguna magistratura, persona ni grupo de personas, civiles o militares, pueden atribuirse otra autoridad, competencia o derechos que los que expresamente se les haya conferido en virtud de la Constitución y las leyes, ni aun a pretexto de circunstancias extraordinarias. 
\newline {\color{gray} \textbf{1º:} 7-2-Iniciativa-Convencional-Constituyente-del-cc-Luis-Barceló-y-otros.pdf}
\newline {\color{gray} (Emb: 0.910, TF-IDF: 0.952)}
\newline {\color{gray} \textbf{2º:} 70-2-Iniciativa-Convencional-Constituyente-de-la-cc-Paulina-Veloso-y-otros-2.pdf}
\newline {\color{gray} (Emb: 0.905, TF-IDF: 0.891)}


\item \textbf{Artículo} \newline
Son emblemas nacionales de Chile la bandera, el escudo y el himno nacional. 
\newline {\color{gray} \textbf{1º:} 84-2-Iniciativa-Convencional-Constituyente-del-cc-Martin-Arrau-y-otros.pdf}
\newline {\color{gray} (Emb: 0.953, TF-IDF: 0.912)}
\newline {\color{gray} \textbf{2º:} 63-2-Iniciativa-Convencional-Constituyente-del-cc-Eduardo-Cretton-y-otros.pdf}
\newline {\color{gray} (Emb: 0.923, TF-IDF: 0.890)}

El Estado reconoce los símbolos y emblemas de los distintos pueblos indígenas. 
\newline {\color{gray} \textbf{1º:} 60-2-Iniciativa-Convencional-Constituyente-del-cc-Jorge-Baradit-y-otros.pdf}
\newline {\color{gray} (Emb: 0.628, TF-IDF: 0.472)}
\newline {\color{gray} \textbf{2º:} 70-2-Iniciativa-Convencional-Constituyente-de-la-cc-Paulina-Veloso-y-otros-2.pdf}
\newline {\color{gray} (Emb: 0.625, TF-IDF: 0.472)}


\item \textbf{Artículo} \newline
Las finanzas públicas se conducirán de conformidad a los principios de sostenibilidad y responsabilidad fiscal, los que guiarán el actuar del Estado en todas sus instituciones y en todos sus niveles. 
\newline {\color{gray} \textbf{1º:} 867-Iniciativa-Convencional-Constituyente-del-cc-Rodrigo-Alvarez-Principio-de-Responsabilidad-Fiscal.pdf}
\newline {\color{gray} (Emb: 0.830, TF-IDF: 0.460)}
\newline {\color{gray} \textbf{2º:} 957-5-Iniciativa-Convencional-Constituyente-de-la-cc-Ivanna-Olivares-sobre-Nuevo-Modelo-Economico.pdf}
\newline {\color{gray} (Emb: 0.726, TF-IDF: 0.451)}


\item \textbf{Artículo} \newline
Será deber de cada órgano del Estado disponer de los mecanismos para promover y asegurar la participación y deliberación ciudadana incidente en la gestión de asuntos públicos, incluyendo medios digitales. 
\newline {\color{gray} \textbf{1º:} 602-Iniciativa-Convencional-Constituyente-de-cc-Jorge-Baradit-sobre-Mecanismos-de-Demoracia-Directa-y-Semidirecta.pdf}
\newline {\color{gray} (Emb: 1.000, TF-IDF: 1.000)}
\newline {\color{gray} \textbf{2º:} 582-Iniciativa-Convencional-Constituyente-de-cc-Francisco-Caamano-sobre-Derecho-a-la-diversidad-de-opiniones.pdf}
\newline {\color{gray} (Emb: 0.562, TF-IDF: 0.319)}

Los poderes públicos deberán facilitar la participación del pueblo en la vida política, económica, cultural y social del país. 
\newline {\color{gray} \textbf{1º:} 602-Iniciativa-Convencional-Constituyente-de-cc-Jorge-Baradit-sobre-Mecanismos-de-Demoracia-Directa-y-Semidirecta.pdf}
\newline {\color{gray} (Emb: 1.000, TF-IDF: 1.000)}
\newline {\color{gray} \textbf{2º:} 753-Iniciativa-Convencional-Constituyente-del-cc-Raul-Celis-sobre-Gobiernos-Locales.pdf}
\newline {\color{gray} (Emb: 0.652, TF-IDF: 0.438)}

La ciudadanía tiene el derecho a participar de manera incidente o vinculante en los asuntos de interés público. 
\newline {\color{gray} \textbf{1º:} 998-Iniciativa-Convencional-Constituyente-del-cc-Francisco-Caamano-sobre-Derechos-de-Autor.pdf}
\newline {\color{gray} (Emb: 0.652, TF-IDF: 0.556)}
\newline {\color{gray} \textbf{2º:} 420-7-Iniciativa-Convencional-del-cc-Francisco-Caamano-sobre-Derechos-de-Autor-1200-26-01.pdf}
\newline {\color{gray} (Emb: 0.652, TF-IDF: 0.556)}

Es deber del Estado dar adecuada publicidad a los mecanismos de democracia, tendiendo a favorecer una amplia deliberación de las personas, en conformidad a esta Constitución y las leyes. 
\newline {\color{gray} \textbf{1º:} 320-2-Iniciativa-Convencional-de-la-cc-Valentina-Miranda-sobre-Democracia-Participativa-10-07-hrs.pdf}
\newline {\color{gray} (Emb: 0.775, TF-IDF: 0.511)}
\newline {\color{gray} \textbf{2º:} 243-4-Iniciativa-Convencional-de-la-cc-Elsa-Labrana-sobre-Derecho-a-la-Rebelion-1147-hrs.pdf}
\newline {\color{gray} (Emb: 0.731, TF-IDF: 0.349)}


\item \textbf{Artículo} \newline
El estado deberá garantizar a toda la ciudadanía, sin discriminación de ningún tipo, el ejercicio pleno de una democracia participativa, a través de mecanismos de democracia directa. 
\newline {\color{gray} \textbf{1º:} 399-2-Iniciativa-Convencional-Constituyente-de-la-cc-Constanza-San-Juan-sobre-Democracia-Directa-1850-24-01.pdf}
\newline {\color{gray} (Emb: 0.736, TF-IDF: 0.441)}
\newline {\color{gray} \textbf{2º:} 344-3-Iniciativa-Convencional-Constituyente-del-cc-Hernan-Larrain-sobre-Reforma-Administrativa-y-Modernizacion-del-Estado.pdf}
\newline {\color{gray} (Emb: 0.714, TF-IDF: 0.426)}


\item \textbf{Artículo} \newline
La ley regulará la utilización de herramientas digitales en la implementación de los mecanismos de participación establecidos en esta Constitución y que sean distintos al sufragio, buscando que su uso promueva la más alta participación posible en dichos procesos, al igual que la más amplia información, transparencia, seguridad y accesibilidad del proceso para todas las personas sin distinción. 
\newline {\color{gray} \textbf{1º:} 541-Iniciativa-Convencional-Constituyente-de-la-cc-Barbara-Rebolledo-sobre-servicio-nacional-fe-publica-1547-01-02-002.pdf}
\newline {\color{gray} (Emb: 0.562, TF-IDF: 0.500)}
\newline {\color{gray} \textbf{2º:} 416-7-Iniciativa-Convencional-del-cc-Francisco-Caamano-sobre-Proteccion-de-datos-de-caracter-personal.pdf}
\newline {\color{gray} (Emb: 0.545, TF-IDF: 0.286)}


\item \textbf{Artículo} \newline
El órgano legislativo deberá informar cada seis meses sobre el avance de la tramitación de estas iniciativas. 
\newline {\color{gray} \textbf{1º:} 915-Iniciativa-Convencional-Constituyente-del-cc-Luis-Jimenez-sobre-Plurinacionalidad.pdf}
\newline {\color{gray} (Emb: 0.658, TF-IDF: 0.321)}
\newline {\color{gray} \textbf{2º:} 409-6-Iniciativa-Convencional-Constituyente-de-la-cc-Ingrid-Villena-sobre-Defensoria-del-Pueblo-2239-24-01.pdf}
\newline {\color{gray} (Emb: 0.657, TF-IDF: 0.314)}

Un grupo ciudadanos habilitados para sufragar, equivalente al tres por ciento del último padrón electoral, podrá presentar una iniciativa popular de ley para su tramitación legislativa. 
\newline {\color{gray} \textbf{1º:} 602-Iniciativa-Convencional-Constituyente-de-cc-Jorge-Baradit-sobre-Mecanismos-de-Demoracia-Directa-y-Semidirecta.pdf}
\newline {\color{gray} (Emb: 0.743, TF-IDF: 0.398)}
\newline {\color{gray} \textbf{2º:} 602-Iniciativa-Convencional-Constituyente-de-cc-Jorge-Baradit-sobre-Mecanismos-de-Demoracia-Directa-y-Semidirecta.pdf}
\newline {\color{gray} (Emb: 0.723, TF-IDF: 0.392)}

Se contará con un plazo de ciento ochenta días desde su registro ante el Servicio Electoral para que la propuesta sea conocida por la ciudadanía y pueda reunir los patrocinios exigidos. 
\newline {\color{gray} \textbf{1º:} 602-Iniciativa-Convencional-Constituyente-de-cc-Jorge-Baradit-sobre-Mecanismos-de-Demoracia-Directa-y-Semidirecta.pdf}
\newline {\color{gray} (Emb: 0.996, TF-IDF: 0.981)}
\newline {\color{gray} \textbf{2º:} 602-Iniciativa-Convencional-Constituyente-de-cc-Jorge-Baradit-sobre-Mecanismos-de-Demoracia-Directa-y-Semidirecta.pdf}
\newline {\color{gray} (Emb: 0.953, TF-IDF: 0.862)}

En caso de reunir el apoyo requerido, el Servicio Electoral remitirá la propuesta al Congreso, para que ésta dé inicio al proceso de formación de ley. 
\newline {\color{gray} \textbf{1º:} 602-Iniciativa-Convencional-Constituyente-de-cc-Jorge-Baradit-sobre-Mecanismos-de-Demoracia-Directa-y-Semidirecta.pdf}
\newline {\color{gray} (Emb: 0.790, TF-IDF: 0.522)}
\newline {\color{gray} \textbf{2º:} 374-2-Iniciativa-Convencional-Constituyente-de-la-cc-Loreto-Vallejos-sobre-Democracia-Directa-0900-hrs-24-01.pdf}
\newline {\color{gray} (Emb: 0.684, TF-IDF: 0.378)}

Las iniciativas populares de ley ingresarán a la agenda legislativa con la urgencia determinada por la ley. 
\newline {\color{gray} \textbf{1º:} 399-2-Iniciativa-Convencional-Constituyente-de-la-cc-Constanza-San-Juan-sobre-Democracia-Directa-1850-24-01.pdf}
\newline {\color{gray} (Emb: 0.830, TF-IDF: 0.464)}
\newline {\color{gray} \textbf{2º:} 399-2-Iniciativa-Convencional-Constituyente-de-la-cc-Constanza-San-Juan-sobre-Democracia-Directa-1850-24-01.pdf}
\newline {\color{gray} (Emb: 0.781, TF-IDF: 0.387)}

La iniciativa popular de ley no podrá referirse a tributos, alterar la administración presupuestaria del Estado ni limitar derechos fundamentales de personas o pueblos reconocidos en esta Constitución y las leyes. 
\newline {\color{gray} \textbf{1º:} 240-1-Iniciativa-Convencional-de-la-cc-Tania-Madriaga-sobre-Poder-Legislativo-1146-hrs.pdf}
\newline {\color{gray} (Emb: 0.647, TF-IDF: 0.368)}
\newline {\color{gray} \textbf{2º:} 320-2-Iniciativa-Convencional-de-la-cc-Valentina-Miranda-sobre-Democracia-Participativa-10-07-hrs.pdf}
\newline {\color{gray} (Emb: 0.618, TF-IDF: 0.349)}


\item \textbf{Artículo} \newline
No serán admisibles las propuestas sobre materias que digan relación con tributos y la administración presupuestaria del Estado. 
\newline {\color{gray} \textbf{1º:} 602-Iniciativa-Convencional-Constituyente-de-cc-Jorge-Baradit-sobre-Mecanismos-de-Demoracia-Directa-y-Semidirecta.pdf}
\newline {\color{gray} (Emb: 0.921, TF-IDF: 0.901)}
\newline {\color{gray} \textbf{2º:} 1014-Iniciativa-Convencional-Constituyente-cc-Adriana-Ampuero-Haciendas-territoriales-y-autonomia-financiera.pdf}
\newline {\color{gray} (Emb: 0.599, TF-IDF: 0.398)}

Un grupo de ciudadanos habilitados para sufragar, equivalente al cinco por ciento del último padrón electoral, podrá presentar una iniciativa de derogación total o parcial de una o más leyes promulgadas bajo la vigencia de esta Constitución para que sea votada mediante referéndum nacional. 
\newline {\color{gray} \textbf{1º:} 602-Iniciativa-Convencional-Constituyente-de-cc-Jorge-Baradit-sobre-Mecanismos-de-Demoracia-Directa-y-Semidirecta.pdf}
\newline {\color{gray} (Emb: 0.822, TF-IDF: 0.566)}
\newline {\color{gray} \textbf{2º:} 602-Iniciativa-Convencional-Constituyente-de-cc-Jorge-Baradit-sobre-Mecanismos-de-Demoracia-Directa-y-Semidirecta.pdf}
\newline {\color{gray} (Emb: 0.798, TF-IDF: 0.499)}


\item \textbf{Artículo} \newline
La planificación presupuestaria de las distintas entidades territoriales deberá siempre considerar elementos de participación incidente de la población. 
\newline {\color{gray} \textbf{1º:} 712-Iniciativa-Convencional-Constituyente-de-la-cc-Lisette-Vergara-sobre-Administracion-Comunal.pdf}
\newline {\color{gray} (Emb: 0.546, TF-IDF: 0.284)}
\newline {\color{gray} \textbf{2º:} 1014-Iniciativa-Convencional-Constituyente-cc-Adriana-Ampuero-Haciendas-territoriales-y-autonomia-financiera.pdf}
\newline {\color{gray} (Emb: 0.529, TF-IDF: 0.284)}

Deberán considerar, al menos, la implementación de iniciativas populares de normas locales a nivel regional y municipal, de carácter vinculante, así como consultas ciudadanas incidentes. 
\newline {\color{gray} \textbf{1º:} 320-2-Iniciativa-Convencional-de-la-cc-Valentina-Miranda-sobre-Democracia-Participativa-10-07-hrs.pdf}
\newline {\color{gray} (Emb: 0.580, TF-IDF: 0.411)}
\newline {\color{gray} \textbf{2º:} 686-Iniciativa-Convencional-Constituyente-del-cc-Mario-Vrgas-sobre-Democracia-Digital-181101-02.pdf}
\newline {\color{gray} (Emb: 0.578, TF-IDF: 0.379)}

El Estatuto Regional deberá considerar mecanismos de democracia directa o semidirecta, que aseguren la participación incidente o vinculante de la población, según corresponda. 
\newline {\color{gray} \textbf{1º:} 399-2-Iniciativa-Convencional-Constituyente-de-la-cc-Constanza-San-Juan-sobre-Democracia-Directa-1850-24-01.pdf}
\newline {\color{gray} (Emb: 0.620, TF-IDF: 0.306)}
\newline {\color{gray} \textbf{2º:} 344-3-Iniciativa-Convencional-Constituyente-del-cc-Hernan-Larrain-sobre-Reforma-Administrativa-y-Modernizacion-del-Estado.pdf}
\newline {\color{gray} (Emb: 0.596, TF-IDF: 0.303)}


\item \textbf{Artículo} \newline
Se podrán someter a referéndum las materias de competencia de los gobiernos regionales y locales en conformidad a lo dispuesto en la ley y Estatuto Regional respectivo. 
\newline {\color{gray} \textbf{1º:} 120-3-c-Iniciativa-de-la-cc-Tammy-Pustilnick-atribuciones-exclusivas-de-la-Asamblea-Regional.pdf}
\newline {\color{gray} (Emb: 0.780, TF-IDF: 0.399)}
\newline {\color{gray} \textbf{2º:} 120-3-c-Iniciativa-de-la-cc-Tammy-Pustilnick-atribuciones-exclusivas-de-la-Asamblea-Regional.pdf}
\newline {\color{gray} (Emb: 0.767, TF-IDF: 0.371)}

Una ley deberá señalar los requisitos mínimos para solicitarlos o convocarlos, la época en que se podrán llevar a cabo, los mecanismos de votación, escrutinio y los casos y condiciones en que sus resultados serán vinculantes. 
\newline {\color{gray} \textbf{1º:} 322-1-Iniciativa-Convencional-Constituyente-de-la-cc-Barbara-Sepulveda-sobre-Formacion-de-la-Ley.pdf}
\newline {\color{gray} (Emb: 0.545, TF-IDF: 0.242)}
\newline {\color{gray} \textbf{2º:} 916-Iniciativa-Convencional-Constituyente-del-cc-Luis-Jimenez-sobre-Pueblo-Tribal-Afrodescendiente.pdf}
\newline {\color{gray} (Emb: 0.543, TF-IDF: 0.232)}


\item \textbf{Artículo} \newline
En el Congreso y en los órganos representativos a nivel regional y local se deberán realizar audiencias públicas en las oportunidades y formas que la ley disponga, en el que las personas y la sociedad civil puedan dar a conocer argumentos y propuestas. 
\newline {\color{gray} \textbf{1º:} 602-Iniciativa-Convencional-Constituyente-de-cc-Jorge-Baradit-sobre-Mecanismos-de-Demoracia-Directa-y-Semidirecta.pdf}
\newline {\color{gray} (Emb: 0.699, TF-IDF: 0.571)}
\newline {\color{gray} \textbf{2º:} 899-Iniciativa-Convencional-Constituyente-del-cc-Cesar-Uribe-Sobre-Participacion-Ciudadana.pdf}
\newline {\color{gray} (Emb: 0.583, TF-IDF: 0.236)}


\item \textbf{Artículo} \newline
La ley podrá crear procedimientos más favorables para la nacionalización de personas apátridas. 
\newline {\color{gray} \textbf{1º:} 603-Iniciativa-Convencional-Constituyente-de-cc-Jorge-Baradit-sobre-Nacionalidad-y-Ciudadania-2105-hrs.-01-02.pdf}
\newline {\color{gray} (Emb: 0.799, TF-IDF: 0.488)}
\newline {\color{gray} \textbf{2º:} 485-2-Iniciativa-Convencional-Constituyente-del-cc-Jorge-Baradit-sobre-Nacionalidad-2113-31-01.pdf}
\newline {\color{gray} (Emb: 0.799, TF-IDF: 0.488)}

Toda persona tiene derecho a la nacionalidad en la forma y condiciones que señala este artículo. 
\newline {\color{gray} \textbf{1º:} 367-2-Iniciativa-Convencional-Constituyente-del-cc-Eduardo-Castillo-sobre-Ciudadania-y-Migracion-0900-hrs-24-01.pdf}
\newline {\color{gray} (Emb: 0.776, TF-IDF: 0.592)}
\newline {\color{gray} \textbf{2º:} 507-2-Iniciativa-Convencional-Constituyente-de-la-cc-Giovanna-Grandon-sobre-Nacionalidad-1213-01-02.pdf}
\newline {\color{gray} (Emb: 0.775, TF-IDF: 0.440)}

No se exigirá renuncia a la nacionalidad anterior para obtener la carta de nacionalización chilena. 
\newline {\color{gray} \textbf{1º:} 485-2-Iniciativa-Convencional-Constituyente-del-cc-Jorge-Baradit-sobre-Nacionalidad-2113-31-01.pdf}
\newline {\color{gray} (Emb: 1.000, TF-IDF: 1.000)}
\newline {\color{gray} \textbf{2º:} 603-Iniciativa-Convencional-Constituyente-de-cc-Jorge-Baradit-sobre-Nacionalidad-y-Ciudadania-2105-hrs.-01-02.pdf}
\newline {\color{gray} (Emb: 1.000, TF-IDF: 1.000)}

Toda persona podrá exigir que en cualquier documento oficial de identificación sea consignada, además de la nacionalidad chilena, su pertenencia a alguno de los pueblos originarios del país. 
\newline {\color{gray} \textbf{1º:} 834-Iniciativa-Convencional-Constituyente-de-la-cc-Cristina-Dorador-sobre-Nacionalidad-y-Ciudadania.pdf}
\newline {\color{gray} (Emb: 0.763, TF-IDF: 0.275)}
\newline {\color{gray} \textbf{2º:} 485-2-Iniciativa-Convencional-Constituyente-del-cc-Jorge-Baradit-sobre-Nacionalidad-2113-31-01.pdf}
\newline {\color{gray} (Emb: 0.755, TF-IDF: 0.246)}

Sean hijas o hijos de padre o madre chilenos, nacidos en territorio extranjero. 
\newline {\color{gray} \textbf{1º:} 485-2-Iniciativa-Convencional-Constituyente-del-cc-Jorge-Baradit-sobre-Nacionalidad-2113-31-01.pdf}
\newline {\color{gray} (Emb: 1.000, TF-IDF: 1.000)}
\newline {\color{gray} \textbf{2º:} 603-Iniciativa-Convencional-Constituyente-de-cc-Jorge-Baradit-sobre-Nacionalidad-y-Ciudadania-2105-hrs.-01-02.pdf}
\newline {\color{gray} (Emb: 1.000, TF-IDF: 1.000)}

Hayan nacido en el territorio de Chile, con excepción de las hijas e hijos de personas extranjeras que se encuentren en Chile en servicio de su Gobierno, quienes podrán optar por la nacionalidad chilena. 
\newline {\color{gray} \textbf{1º:} 603-Iniciativa-Convencional-Constituyente-de-cc-Jorge-Baradit-sobre-Nacionalidad-y-Ciudadania-2105-hrs.-01-02.pdf}
\newline {\color{gray} (Emb: 1.000, TF-IDF: 1.000)}
\newline {\color{gray} \textbf{2º:} 485-2-Iniciativa-Convencional-Constituyente-del-cc-Jorge-Baradit-sobre-Nacionalidad-2113-31-01.pdf}
\newline {\color{gray} (Emb: 1.000, TF-IDF: 1.000)}

Son chilenas y chilenos, aquellas personas que: 1. 
\newline {\color{gray} \textbf{1º:} 485-2-Iniciativa-Convencional-Constituyente-del-cc-Jorge-Baradit-sobre-Nacionalidad-2113-31-01.pdf}
\newline {\color{gray} (Emb: 1.000, TF-IDF: 1.000)}
\newline {\color{gray} \textbf{2º:} 603-Iniciativa-Convencional-Constituyente-de-cc-Jorge-Baradit-sobre-Nacionalidad-y-Ciudadania-2105-hrs.-01-02.pdf}
\newline {\color{gray} (Emb: 1.000, TF-IDF: 1.000)}

Obtuvieren especial gracia de nacionalización por ley. 
\newline {\color{gray} \textbf{1º:} 603-Iniciativa-Convencional-Constituyente-de-cc-Jorge-Baradit-sobre-Nacionalidad-y-Ciudadania-2105-hrs.-01-02.pdf}
\newline {\color{gray} (Emb: 1.000, TF-IDF: 1.000)}
\newline {\color{gray} \textbf{2º:} 485-2-Iniciativa-Convencional-Constituyente-del-cc-Jorge-Baradit-sobre-Nacionalidad-2113-31-01.pdf}
\newline {\color{gray} (Emb: 1.000, TF-IDF: 1.000)}


\item \textbf{Artículo} \newline
La nacionalidad chilena confiere el derecho incondicional a residir en el territorio chileno y a retornar a él. 
\newline {\color{gray} \textbf{1º:} 485-2-Iniciativa-Convencional-Constituyente-del-cc-Jorge-Baradit-sobre-Nacionalidad-2113-31-01.pdf}
\newline {\color{gray} (Emb: 0.974, TF-IDF: 0.938)}
\newline {\color{gray} \textbf{2º:} 603-Iniciativa-Convencional-Constituyente-de-cc-Jorge-Baradit-sobre-Nacionalidad-y-Ciudadania-2105-hrs.-01-02.pdf}
\newline {\color{gray} (Emb: 0.974, TF-IDF: 0.938)}

Concede, además, el derecho a la protección diplomática por parte del Estado de Chile y todos los demás derechos que la Constitución y las leyes vinculen al estatuto de nacionalidad. 
\newline {\color{gray} \textbf{1º:} 485-2-Iniciativa-Convencional-Constituyente-del-cc-Jorge-Baradit-sobre-Nacionalidad-2113-31-01.pdf}
\newline {\color{gray} (Emb: 0.982, TF-IDF: 0.941)}
\newline {\color{gray} \textbf{2º:} 603-Iniciativa-Convencional-Constituyente-de-cc-Jorge-Baradit-sobre-Nacionalidad-y-Ciudadania-2105-hrs.-01-02.pdf}
\newline {\color{gray} (Emb: 0.982, TF-IDF: 0.941)}


\item \textbf{Artículo} \newline
Todas las personas que tengan la nacionalidad chilena serán ciudadanas y ciudadanos de Chile. 
\newline {\color{gray} \textbf{1º:} 304-4-Iniciativa-Convencional-de-la-cc-Valentina-Miranda-sobre-Derechos-Fundamentales-1301-hrs.pdf}
\newline {\color{gray} (Emb: 0.940, TF-IDF: 0.669)}
\newline {\color{gray} \textbf{2º:} 485-2-Iniciativa-Convencional-Constituyente-del-cc-Jorge-Baradit-sobre-Nacionalidad-2113-31-01.pdf}
\newline {\color{gray} (Emb: 0.884, TF-IDF: 0.595)}

Asimismo, serán ciudadanas y ciudadanos las personas extranjeras avecindadas en Chile por al menos cinco años. 
\newline {\color{gray} \textbf{1º:} 830-Iniciativa-Convencional-Constituyente-de-la-cc-Barbara-Rebolledo-sobre-Sufragio.pdf}
\newline {\color{gray} (Emb: 0.715, TF-IDF: 0.414)}
\newline {\color{gray} \textbf{2º:} 485-2-Iniciativa-Convencional-Constituyente-del-cc-Jorge-Baradit-sobre-Nacionalidad-2113-31-01.pdf}
\newline {\color{gray} (Emb: 0.707, TF-IDF: 0.345)}

El sufragio será personal, igualitario, secreto y obligatorio. 
\newline {\color{gray} \textbf{1º:} 485-2-Iniciativa-Convencional-Constituyente-del-cc-Jorge-Baradit-sobre-Nacionalidad-2113-31-01.pdf}
\newline {\color{gray} (Emb: 1.000, TF-IDF: 1.000)}
\newline {\color{gray} \textbf{2º:} 367-2-Iniciativa-Convencional-Constituyente-del-cc-Eduardo-Castillo-sobre-Ciudadania-y-Migracion-0900-hrs-24-01.pdf}
\newline {\color{gray} (Emb: 1.000, TF-IDF: 1.000)}

No será obligatorio para las y los chilenos que vivan en el extranjero y para las y los mayores de dieciséis y menores de dieciocho años. 
\newline {\color{gray} \textbf{1º:} 485-2-Iniciativa-Convencional-Constituyente-del-cc-Jorge-Baradit-sobre-Nacionalidad-2113-31-01.pdf}
\newline {\color{gray} (Emb: 1.000, TF-IDF: 1.000)}
\newline {\color{gray} \textbf{2º:} 603-Iniciativa-Convencional-Constituyente-de-cc-Jorge-Baradit-sobre-Nacionalidad-y-Ciudadania-2105-hrs.-01-02.pdf}
\newline {\color{gray} (Emb: 1.000, TF-IDF: 1.000)}

Ninguna autoridad u órgano podrá impedir el efectivo ejercicio de este derecho, debiendo a su vez proporcionar todos los medios necesarios para que las personas habilitadas para sufragar puedan ejercerlo. 
\newline {\color{gray} \textbf{1º:} 603-Iniciativa-Convencional-Constituyente-de-cc-Jorge-Baradit-sobre-Nacionalidad-y-Ciudadania-2105-hrs.-01-02.pdf}
\newline {\color{gray} (Emb: 1.000, TF-IDF: 1.000)}
\newline {\color{gray} \textbf{2º:} 485-2-Iniciativa-Convencional-Constituyente-del-cc-Jorge-Baradit-sobre-Nacionalidad-2113-31-01.pdf}
\newline {\color{gray} (Emb: 1.000, TF-IDF: 1.000)}

El Estado promoverá el ejercicio activo y progresivo, a través de los distintos mecanismos de participación, de los derechos derivados de la ciudadanía, en especial en favor de niños, niñas, adolescentes, personas privadas de libertad, personas con discapacidad, personas mayores y personas cuyas circunstancias o capacidades personales disminuyan sus posibilidades de ejercicio. 
\newline {\color{gray} \textbf{1º:} 603-Iniciativa-Convencional-Constituyente-de-cc-Jorge-Baradit-sobre-Nacionalidad-y-Ciudadania-2105-hrs.-01-02.pdf}
\newline {\color{gray} (Emb: 1.000, TF-IDF: 1.000)}
\newline {\color{gray} \textbf{2º:} 485-2-Iniciativa-Convencional-Constituyente-del-cc-Jorge-Baradit-sobre-Nacionalidad-2113-31-01.pdf}
\newline {\color{gray} (Emb: 1.000, TF-IDF: 1.000)}


\item \textbf{Artículo} \newline
Las y los extranjeros avecindados en Chile por más de cinco años, y que cumplan con los requisitos señalados en el artículo 20, podrán ejercer el derecho de sufragio activo en los casos y formas que determine la ley. 
\newline {\color{gray} \textbf{1º:} 830-Iniciativa-Convencional-Constituyente-de-la-cc-Barbara-Rebolledo-sobre-Sufragio.pdf}
\newline {\color{gray} (Emb: 0.795, TF-IDF: 0.805)}
\newline {\color{gray} \textbf{2º:} 367-2-Iniciativa-Convencional-Constituyente-del-cc-Eduardo-Castillo-sobre-Ciudadania-y-Migracion-0900-hrs-24-01.pdf}
\newline {\color{gray} (Emb: 0.734, TF-IDF: 0.436)}


\item \textbf{Artículo} \newline
Por ley que revoque la nacionalización concedida por gracia. 
\newline {\color{gray} \textbf{1º:} 603-Iniciativa-Convencional-Constituyente-de-cc-Jorge-Baradit-sobre-Nacionalidad-y-Ciudadania-2105-hrs.-01-02.pdf}
\newline {\color{gray} (Emb: 1.000, TF-IDF: 1.000)}
\newline {\color{gray} \textbf{2º:} 485-2-Iniciativa-Convencional-Constituyente-del-cc-Jorge-Baradit-sobre-Nacionalidad-2113-31-01.pdf}
\newline {\color{gray} (Emb: 1.000, TF-IDF: 1.000)}

En los restantes casos, sólo podrán ser rehabilitados por ley. 
\newline {\color{gray} \textbf{1º:} 603-Iniciativa-Convencional-Constituyente-de-cc-Jorge-Baradit-sobre-Nacionalidad-y-Ciudadania-2105-hrs.-01-02.pdf}
\newline {\color{gray} (Emb: 1.000, TF-IDF: 1.000)}
\newline {\color{gray} \textbf{2º:} 485-2-Iniciativa-Convencional-Constituyente-del-cc-Jorge-Baradit-sobre-Nacionalidad-2113-31-01.pdf}
\newline {\color{gray} (Emb: 1.000, TF-IDF: 1.000)}

Esto último no será aplicable a niños, niñas y adolescentes; 3. 
\newline {\color{gray} \textbf{1º:} 603-Iniciativa-Convencional-Constituyente-de-cc-Jorge-Baradit-sobre-Nacionalidad-y-Ciudadania-2105-hrs.-01-02.pdf}
\newline {\color{gray} (Emb: 1.000, TF-IDF: 1.000)}
\newline {\color{gray} \textbf{2º:} 485-2-Iniciativa-Convencional-Constituyente-del-cc-Jorge-Baradit-sobre-Nacionalidad-2113-31-01.pdf}
\newline {\color{gray} (Emb: 1.000, TF-IDF: 1.000)}

En el caso del número 1, la nacionalidad podrá recuperarse en conformidad al número 3 del artículo 1. 
\newline {\color{gray} \textbf{1º:} 758-Iniciativa-Convencional-Constituyente-Irrenunciabilidad-de-la-Nacionalidad-Saldana.pdf}
\newline {\color{gray} (Emb: 0.495, TF-IDF: 0.444)}
\newline {\color{gray} \textbf{2º:} 319-6-Iniciativa-Convencional-del-cc-Mauricio-Daza-sobre-el-Sistema-Nacional-de-Justicia17-09-hrs.pdf}
\newline {\color{gray} (Emb: 0.455, TF-IDF: 0.435)}

Esta renuncia sólo producirá efectos si la persona, previamente, se ha nacionalizado en país extranjero; 2. 
\newline {\color{gray} \textbf{1º:} 603-Iniciativa-Convencional-Constituyente-de-cc-Jorge-Baradit-sobre-Nacionalidad-y-Ciudadania-2105-hrs.-01-02.pdf}
\newline {\color{gray} (Emb: 1.000, TF-IDF: 1.000)}
\newline {\color{gray} \textbf{2º:} 485-2-Iniciativa-Convencional-Constituyente-del-cc-Jorge-Baradit-sobre-Nacionalidad-2113-31-01.pdf}
\newline {\color{gray} (Emb: 1.000, TF-IDF: 1.000)}

Por renuncia voluntaria manifestada ante autoridad chilena competente. 
\newline {\color{gray} \textbf{1º:} 603-Iniciativa-Convencional-Constituyente-de-cc-Jorge-Baradit-sobre-Nacionalidad-y-Ciudadania-2105-hrs.-01-02.pdf}
\newline {\color{gray} (Emb: 1.000, TF-IDF: 1.000)}
\newline {\color{gray} \textbf{2º:} 485-2-Iniciativa-Convencional-Constituyente-del-cc-Jorge-Baradit-sobre-Nacionalidad-2113-31-01.pdf}
\newline {\color{gray} (Emb: 1.000, TF-IDF: 1.000)}

La nacionalidad chilena se pierde, exclusivamente: 1. 
\newline {\color{gray} \textbf{1º:} 485-2-Iniciativa-Convencional-Constituyente-del-cc-Jorge-Baradit-sobre-Nacionalidad-2113-31-01.pdf}
\newline {\color{gray} (Emb: 1.000, TF-IDF: 1.000)}
\newline {\color{gray} \textbf{2º:} 603-Iniciativa-Convencional-Constituyente-de-cc-Jorge-Baradit-sobre-Nacionalidad-y-Ciudadania-2105-hrs.-01-02.pdf}
\newline {\color{gray} (Emb: 1.000, TF-IDF: 1.000)}

Por cancelación de la carta de nacionalización, siempre que la persona no se convirtiera en apátrida, salvo que se hubiera obtenido por declaración falsa o por fraude. 
\newline {\color{gray} \textbf{1º:} 603-Iniciativa-Convencional-Constituyente-de-cc-Jorge-Baradit-sobre-Nacionalidad-y-Ciudadania-2105-hrs.-01-02.pdf}
\newline {\color{gray} (Emb: 1.000, TF-IDF: 1.000)}
\newline {\color{gray} \textbf{2º:} 485-2-Iniciativa-Convencional-Constituyente-del-cc-Jorge-Baradit-sobre-Nacionalidad-2113-31-01.pdf}
\newline {\color{gray} (Emb: 1.000, TF-IDF: 1.000)}


\item \textbf{Artículo} \newline
La pérdida de la nacionalidad sólo puede producirse por causales establecidas en esta Constitución, y siempre que la persona afectada no quede en condición de apátrida. 
\newline {\color{gray} \textbf{1º:} 834-Iniciativa-Convencional-Constituyente-de-la-cc-Cristina-Dorador-sobre-Nacionalidad-y-Ciudadania.pdf}
\newline {\color{gray} (Emb: 0.879, TF-IDF: 0.769)}
\newline {\color{gray} \textbf{2º:} 507-2-Iniciativa-Convencional-Constituyente-de-la-cc-Giovanna-Grandon-sobre-Nacionalidad-1213-01-02.pdf}
\newline {\color{gray} (Emb: 0.683, TF-IDF: 0.424)}


\item \textbf{Artículo} \newline
La calidad de ciudadano se pierde: 1º. 
\newline {\color{gray} \textbf{1º:} 485-2-Iniciativa-Convencional-Constituyente-del-cc-Jorge-Baradit-sobre-Nacionalidad-2113-31-01.pdf}
\newline {\color{gray} (Emb: 0.649, TF-IDF: 0.632)}
\newline {\color{gray} \textbf{2º:} 603-Iniciativa-Convencional-Constituyente-de-cc-Jorge-Baradit-sobre-Nacionalidad-y-Ciudadania-2105-hrs.-01-02.pdf}
\newline {\color{gray} (Emb: 0.649, TF-IDF: 0.632)}

- Por pérdida de la nacionalidad chilena;. 
\newline {\color{gray} \textbf{1º:} 246-1-Iniciativa-Convencional-de-la-cc-Barbara-Sepulveda-sobre-Sufragio-Obligatorio-1148-hrs.pdf}
\newline {\color{gray} (Emb: 0.981, TF-IDF: 0.599)}
\newline {\color{gray} \textbf{2º:} 319-6-Iniciativa-Convencional-del-cc-Mauricio-Daza-sobre-el-Sistema-Nacional-de-Justicia17-09-hrs.pdf}
\newline {\color{gray} (Emb: 0.899, TF-IDF: 0.593)}


\item \textbf{Artículo} \newline
La persona afectada por acto o resolución de autoridad administrativa que la prive o desconozca de su nacionalidad chilena, podrá recurrir, por sí o por cualquiera a su nombre, ante cualquier Corte de Apelaciones, conforme al procedimiento establecido en la ley. 
\newline {\color{gray} \textbf{1º:} 485-2-Iniciativa-Convencional-Constituyente-del-cc-Jorge-Baradit-sobre-Nacionalidad-2113-31-01.pdf}
\newline {\color{gray} (Emb: 1.000, TF-IDF: 1.000)}
\newline {\color{gray} \textbf{2º:} 603-Iniciativa-Convencional-Constituyente-de-cc-Jorge-Baradit-sobre-Nacionalidad-y-Ciudadania-2105-hrs.-01-02.pdf}
\newline {\color{gray} (Emb: 1.000, TF-IDF: 1.000)}

La interposición de la acción constitucional suspenderá los efectos del acto o resolución recurridos. 
\newline {\color{gray} \textbf{1º:} 603-Iniciativa-Convencional-Constituyente-de-cc-Jorge-Baradit-sobre-Nacionalidad-y-Ciudadania-2105-hrs.-01-02.pdf}
\newline {\color{gray} (Emb: 1.000, TF-IDF: 1.000)}
\newline {\color{gray} \textbf{2º:} 485-2-Iniciativa-Convencional-Constituyente-del-cc-Jorge-Baradit-sobre-Nacionalidad-2113-31-01.pdf}
\newline {\color{gray} (Emb: 1.000, TF-IDF: 1.000)}


\item \textbf{Artículo} \newline
Ninguna persona que resida en Chile cumpliendo los requisitos que establece esta Constitución y las leyes puede ser desterrado, exiliado o relegado. 
\newline {\color{gray} \textbf{1º:} 485-2-Iniciativa-Convencional-Constituyente-del-cc-Jorge-Baradit-sobre-Nacionalidad-2113-31-01.pdf}
\newline {\color{gray} (Emb: 1.000, TF-IDF: 1.000)}
\newline {\color{gray} \textbf{2º:} 603-Iniciativa-Convencional-Constituyente-de-cc-Jorge-Baradit-sobre-Nacionalidad-y-Ciudadania-2105-hrs.-01-02.pdf}
\newline {\color{gray} (Emb: 1.000, TF-IDF: 1.000)}

La ley establecerá medidas para la recuperación de la nacionalidad chilena en favor de quienes perdieron o tuvieron que renunciar a ella como consecuencia del exilio. 
\newline {\color{gray} \textbf{1º:} 834-Iniciativa-Convencional-Constituyente-de-la-cc-Cristina-Dorador-sobre-Nacionalidad-y-Ciudadania.pdf}
\newline {\color{gray} (Emb: 0.797, TF-IDF: 0.413)}
\newline {\color{gray} \textbf{2º:} 758-Iniciativa-Convencional-Constituyente-Irrenunciabilidad-de-la-Nacionalidad-Saldana.pdf}
\newline {\color{gray} (Emb: 0.736, TF-IDF: 0.341)}

Este derecho también se reconocerá a hijas e hijos de dichas personas. 
\newline {\color{gray} \textbf{1º:} 935-Iniciativa-de-la-Convencional-Constituyente-de-la-cc-Barbara-Rebolledo-sobre-Derechos-de-los-NNA.pdf}
\newline {\color{gray} (Emb: 0.562, TF-IDF: 0.473)}
\newline {\color{gray} \textbf{2º:} 840-Iniciativa-Convencional-Constituyente-de-la-cc-Elsa-Labrana-sobre-Derechos-de-los-NNA.pdf}
\newline {\color{gray} (Emb: 0.545, TF-IDF: 0.359)}


\item \textbf{Artículo} \newline
Las personas mayores son titulares y plenos sujetos de derecho. 
\newline {\color{gray} \textbf{1º:} 456-4-Iniciativa-Convencional-Constituyente-del-cc-Benito-Baranda-sobre-Derechos-de-las-personas-mayores-1729-31-01.pdf}
\newline {\color{gray} (Emb: 0.779, TF-IDF: 0.895)}
\newline {\color{gray} \textbf{2º:} 342-4-Iniciativa-Convencional-Constituyente-del-cc-Jorge-Baradit-sobre-Derechos-de-las-Personas-Mayores.pdf}
\newline {\color{gray} (Emb: 0.779, TF-IDF: 0.895)}

Tienen derecho a envejecer con dignidad y a ejercer todos los derechos consagrados en esta Constitución y en los tratados internacionales de derechos humanos ratificados por Chile y que se encuentren vigentes, en igualdad de condiciones que el resto de la población. 
\newline {\color{gray} \textbf{1º:} 342-4-Iniciativa-Convencional-Constituyente-del-cc-Jorge-Baradit-sobre-Derechos-de-las-Personas-Mayores.pdf}
\newline {\color{gray} (Emb: 0.930, TF-IDF: 0.742)}
\newline {\color{gray} \textbf{2º:} 456-4-Iniciativa-Convencional-Constituyente-del-cc-Benito-Baranda-sobre-Derechos-de-las-personas-mayores-1729-31-01.pdf}
\newline {\color{gray} (Emb: 0.930, TF-IDF: 0.742)}

Especialmente, las personas mayores tienen derecho a obtener prestaciones de seguridad social suficientes para una vida digna; a la accesibilidad al entorno físico, social, económico, cultural y digital; a la participación política y social; a una vida libre de maltrato por motivos de edad; a la autonomía e independencia y al pleno ejercicio de su capacidad jurídica con los apoyos y salvaguardias que correspondan. 
\newline {\color{gray} \textbf{1º:} 475-4-Iniciativa-Convencional-Constituyente-de-la-cc-Janis-Meneses-sobre-Derechos-de-las-Personas-Mayores-2011-31-01.pdf}
\newline {\color{gray} (Emb: 0.708, TF-IDF: 0.413)}
\newline {\color{gray} \textbf{2º:} 8-4-Iniciativa-Convencional-Constituyente-de-la-cc-Cristina-Dorador-y-otros.pdf}
\newline {\color{gray} (Emb: 0.629, TF-IDF: 0.413)}


\item \textbf{Artículo} \newline
El Estado deberá adoptar las medidas necesarias para erradicar todo tipo de violencia de género y los patrones socioculturales que la posibilitan, actuando con la debida diligencia para prevenir, investigar y sancionar dicha violencia, así como brindar atención, protección y reparación integral a las víctimas, considerando especialmente las situaciones de vulnerabilidad en que puedan hallarse. 
\newline {\color{gray} \textbf{1º:} 2-4-Iniciativa-Convencional-Constituyente-de-la-cc-Bárbara-Sepúlveda-y-otras-1.pdf}
\newline {\color{gray} (Emb: 0.716, TF-IDF: 0.455)}
\newline {\color{gray} \textbf{2º:} 844-Iniciativa-Convencional-Constituyente-de-la-cc-Francisca-Linconao-sobre-Derecho-de-las-Mujeres-Indigenas.pdf}
\newline {\color{gray} (Emb: 0.658, TF-IDF: 0.441)}

El Estado garantiza y promueve el derecho de las mujeres, niñas, diversidades y disidencias sexogenéricas a una vida libre de violencia de género en todas sus manifestaciones, tanto en el ámbito público como privado, sea que provenga de particulares, instituciones o agentes del Estado. 
\newline {\color{gray} \textbf{1º:} 2-4-Iniciativa-Convencional-Constituyente-de-la-cc-Bárbara-Sepúlveda-y-otras-1.pdf}
\newline {\color{gray} (Emb: 0.830, TF-IDF: 0.756)}
\newline {\color{gray} \textbf{2º:} 844-Iniciativa-Convencional-Constituyente-de-la-cc-Francisca-Linconao-sobre-Derecho-de-las-Mujeres-Indigenas.pdf}
\newline {\color{gray} (Emb: 0.663, TF-IDF: 0.461)}


\item \textbf{Artículo} \newline
La Constitución reconoce a las personas con discapacidad como sujetos de derechos que esta Constitución y los tratados internacionales ratificados y vigentes les reconocen, en igualdad de condiciones con los demás y garantiza el goce y ejercicio de su capacidad jurídica, con apoyos y salvaguardias, según corresponda. 
\newline {\color{gray} \textbf{1º:} 1026-Iniciatva-Convencional-Constituyente-del-cc-Mariela-Serey-sobre-Personas-Discapacitadas.pdf}
\newline {\color{gray} (Emb: 0.730, TF-IDF: 0.560)}
\newline {\color{gray} \textbf{2º:} 210-1-c-Iniciativa-Convencional-del-cc-Cristián-Monckeberg-sobre-Estado-Intercultural-1953-hrs.pdf}
\newline {\color{gray} (Emb: 0.724, TF-IDF: 0.545)}

Las personas con discapacidad tienen derecho a la accesibilidad universal, así como también la inclusión social, inserción laboral, su participación política, económica, social y cultural. 
\newline {\color{gray} \textbf{1º:} 718-Iniciativa-Convencional-Constituyente-de-la-cc-Maria-Elisa-Quinteros-Personas-en-Situacion-de-Discapacidad-01-02.pdf}
\newline {\color{gray} (Emb: 0.643, TF-IDF: 0.402)}
\newline {\color{gray} \textbf{2º:} 27-4-Iniciativa-Convencional-Constituyente-de-la-cc-Adriana-Cancino-y-otros.pdf}
\newline {\color{gray} (Emb: 0.629, TF-IDF: 0.352)}

Existirá, de conformidad a la ley, un sistema nacional a través del cual se elaborarán, coordinarán y ejecutarán las políticas y programas destinados a atender las necesidades de trabajo, educación, vivienda, salud y cuidado de las personas con discapacidad. 
\newline {\color{gray} \textbf{1º:} 27-4-Iniciativa-Convencional-Constituyente-de-la-cc-Adriana-Cancino-y-otros.pdf}
\newline {\color{gray} (Emb: 1.000, TF-IDF: 0.927)}
\newline {\color{gray} \textbf{2º:} 773-Iniciativa-Convencional-Constituyente-de-la-cc-Adriana-Cancino-sobre-Derecho-a-la-Alimentacion-Adecuada.pdf}
\newline {\color{gray} (Emb: 0.696, TF-IDF: 0.283)}

La ley garantizará que la elaboración, ejecución y supervisión de dichos planes y programas cuente con la participación activa y vinculante de las personas con discapacidad y de las organizaciones que las representan. 
\newline {\color{gray} \textbf{1º:} 27-4-Iniciativa-Convencional-Constituyente-de-la-cc-Adriana-Cancino-y-otros.pdf}
\newline {\color{gray} (Emb: 1.000, TF-IDF: 1.000)}
\newline {\color{gray} \textbf{2º:} 107-4-c-Iniciativa-de-la-cc-Giovanna-Grandon-Derecho-al-Trabajo.pdf}
\newline {\color{gray} (Emb: 0.522, TF-IDF: 0.291)}

La ley arbitrará los medios necesarios para identificar y remover las barreras físicas, sociales, culturales, actitudinales, de comunicación y de otra índole para facilitar a las personas con discapacidad el ejercicio de sus derechos. 
\newline {\color{gray} \textbf{1º:} 27-4-Iniciativa-Convencional-Constituyente-de-la-cc-Adriana-Cancino-y-otros.pdf}
\newline {\color{gray} (Emb: 0.965, TF-IDF: 0.896)}
\newline {\color{gray} \textbf{2º:} 1026-Iniciatva-Convencional-Constituyente-del-cc-Mariela-Serey-sobre-Personas-Discapacitadas.pdf}
\newline {\color{gray} (Emb: 0.757, TF-IDF: 0.436)}

El Estado garantizará los derechos lingüísticos e identidades culturales de las personas con discapacidad, los que incluyen el derecho a expresarse y comunicarse a través de sus lenguas y el acceso a mecanismos, medios y formas alternativas de comunicación. 
\newline {\color{gray} \textbf{1º:} 1026-Iniciatva-Convencional-Constituyente-del-cc-Mariela-Serey-sobre-Personas-Discapacitadas.pdf}
\newline {\color{gray} (Emb: 0.926, TF-IDF: 0.829)}
\newline {\color{gray} \textbf{2º:} 718-Iniciativa-Convencional-Constituyente-de-la-cc-Maria-Elisa-Quinteros-Personas-en-Situacion-de-Discapacidad-01-02.pdf}
\newline {\color{gray} (Emb: 0.656, TF-IDF: 0.282)}

Asimismo, garantizará la autonomía lingüística de las personas sordas en todos los ámbitos de la vida. 
\newline {\color{gray} \textbf{1º:} 366-4-Iniciativa-Convencional-Constituyente-del-cc-Marco-Arellano-sobre-Derechos-Linguisticos-2232-hrs-21-01.pdf}
\newline {\color{gray} (Emb: 0.555, TF-IDF: 0.483)}
\newline {\color{gray} \textbf{2º:} 404-4-Iniciativa-Convencional-Constituyente-de-la-cc-Lidia-Gonzalez-sobre-Derechos-Linguisticos-1940-24-01.pdf}
\newline {\color{gray} (Emb: 0.539, TF-IDF: 0.381)}


\item \textbf{Artículo} \newline
Toda persona tiene derecho a buscar y recibir asilo, de acuerdo con la legislación nacional y los tratados internacionales suscritos y ratificados por Chile, y que se encuentren vigentes. 
\newline {\color{gray} \textbf{1º:} 725-Iniciativa-Convencional-Constituyente-de-la-cc-Tatiana-Urrutia-sobre-Derecho-al-Asilo-01-02.pdf}
\newline {\color{gray} (Emb: 1.000, TF-IDF: 1.000)}
\newline {\color{gray} \textbf{2º:} 515-4-Iniciativa-Convencional-Constituyente-de-la-cc-Giovanna-Grandon-sobre-Derecho-a-Migrar-1245-01-02.pdf}
\newline {\color{gray} (Emb: 0.843, TF-IDF: 0.753)}

Una ley regulará el procedimiento de solicitud y reconocimiento de la condición de refugiado, así como las garantías y protecciones específicas que se establezcan en favor de las personas solicitantes de asilo o refugiadas. 
\newline {\color{gray} \textbf{1º:} 725-Iniciativa-Convencional-Constituyente-de-la-cc-Tatiana-Urrutia-sobre-Derecho-al-Asilo-01-02.pdf}
\newline {\color{gray} (Emb: 1.000, TF-IDF: 1.000)}
\newline {\color{gray} \textbf{2º:} 246-1-Iniciativa-Convencional-de-la-cc-Barbara-Sepulveda-sobre-Sufragio-Obligatorio-1148-hrs.pdf}
\newline {\color{gray} (Emb: 0.597, TF-IDF: 0.347)}


\item \textbf{Artículo} \newline
Ninguna persona solicitante de asilo o refugiada será regresada por la fuerza a las fronteras del Estado donde su vida o libertad pueden verse amenazadas, corra riesgo de persecución o graves violaciones de derechos humanos. 
\newline {\color{gray} \textbf{1º:} 725-Iniciativa-Convencional-Constituyente-de-la-cc-Tatiana-Urrutia-sobre-Derecho-al-Asilo-01-02.pdf}
\newline {\color{gray} (Emb: 0.768, TF-IDF: 0.375)}
\newline {\color{gray} \textbf{2º:} 171-4-c-sobre-Refugio-y-asilo-político-2111.pdf}
\newline {\color{gray} (Emb: 0.737, TF-IDF: 0.355)}


\item \textbf{Artículo} \newline
El Estado asegurará por medio de este sistema, que ante amenaza o vulneración de derechos, existan mecanismos para su restitución, sanción y reparación. 
\newline {\color{gray} \textbf{1º:} 433-2-Iniciativa-Convencional-del-cc-Jorge-Abarca-sobre-Preambulo-de-la-Constitucion-1155-27-01.pdf}
\newline {\color{gray} (Emb: 0.587, TF-IDF: 0.276)}
\newline {\color{gray} \textbf{2º:} 655-Iniciativa-Convencional-Constituyente-de-la-cc-Jeniffer-Mella-sobre-Trabajo-y-Seguridad-Social-121101-02.pdf}
\newline {\color{gray} (Emb: 0.571, TF-IDF: 0.248)}

La ley establecerá un sistema de protección integral de garantías de los derechos de niños, niñas y adolescentes, a través del cual establecerá responsabilidades específicas de los poderes y órganos del Estado y su deber de trabajo intersectorial y coordinado para asegurar la prevención de la violencia contra niños, niñas, y adolescentes y la promoción y protección efectiva de los derechos de estos. 
\newline {\color{gray} \textbf{1º:} 391-4-Iniciativa-Convencional-Constituyente-de-la-cc-Valentina-Miranda-sobre-Ninos-Ninas-y-Adolescentes-1440-24-01.pdf}
\newline {\color{gray} (Emb: 0.715, TF-IDF: 0.710)}
\newline {\color{gray} \textbf{2º:} 622-Iniciativa-Convencional-Constituyente-de-cc-Felipe-Harboe-Reconocimiento-y-proteccion-integral-de-derechos-de-NNA.pdf}
\newline {\color{gray} (Emb: 0.696, TF-IDF: 0.549)}

La erradicación de la violencia contra la niñez será declarada un asunto de la más alta prioridad del Estado, y para ello diseñará estrategias y acciones para abordar situaciones que impliquen un menoscabo de la integridad personal de niñas, niños y adolescentes, sea que la violencia provenga de las familias, del propio Estado, o de terceros. 
\newline {\color{gray} \textbf{1º:} 689-Iniciativa-Convencional-Constituyente-del-cc-Matias-Orellana-sobre-Derechos-de-NNA-181101-02.pdf}
\newline {\color{gray} (Emb: 0.644, TF-IDF: 0.345)}
\newline {\color{gray} \textbf{2º:} 689-Iniciativa-Convencional-Constituyente-del-cc-Matias-Orellana-sobre-Derechos-de-NNA-181101-02.pdf}
\newline {\color{gray} (Emb: 0.526, TF-IDF: 0.255)}

Los niños, niñas y adolescentes tienen derecho a ser protegidos contra toda forma de violencia, maltrato, abuso, explotación, acoso y negligencia. 
\newline {\color{gray} \textbf{1º:} 413-4-Iniciativa-Convencional-Constituyente-del-cc-Pollyana-Rivera-sobre-Derechos-de-NNA-1255-25-01.pdf}
\newline {\color{gray} (Emb: 0.774, TF-IDF: 0.478)}
\newline {\color{gray} \textbf{2º:} 622-Iniciativa-Convencional-Constituyente-de-cc-Felipe-Harboe-Reconocimiento-y-proteccion-integral-de-derechos-de-NNA.pdf}
\newline {\color{gray} (Emb: 0.726, TF-IDF: 0.434)}

Las niñas, niños y adolescentes son titulares de todos los derechos y garantías establecidas en esta Constitución, en las leyes y tratados internacionales ratificados y vigentes en Chile. 
\newline {\color{gray} \textbf{1º:} 419-4-Iniciativa-Convencional-de-la-cc-Tammy-Pustilnick-sobre-Derecho-de-NNA.pdf}
\newline {\color{gray} (Emb: 1.000, TF-IDF: 1.000)}
\newline {\color{gray} \textbf{2º:} 391-4-Iniciativa-Convencional-Constituyente-de-la-cc-Valentina-Miranda-sobre-Ninos-Ninas-y-Adolescentes-1440-24-01.pdf}
\newline {\color{gray} (Emb: 0.906, TF-IDF: 0.601)}

Niñas, niños y adolescentes tienen derecho a vivir en condiciones familiares y ambientales que les permitan el pleno y armonioso desarrollo de su personalidad. 
\newline {\color{gray} \textbf{1º:} 935-Iniciativa-de-la-Convencional-Constituyente-de-la-cc-Barbara-Rebolledo-sobre-Derechos-de-los-NNA.pdf}
\newline {\color{gray} (Emb: 1.000, TF-IDF: 1.000)}
\newline {\color{gray} \textbf{2º:} 391-4-Iniciativa-Convencional-Constituyente-de-la-cc-Valentina-Miranda-sobre-Ninos-Ninas-y-Adolescentes-1440-24-01.pdf}
\newline {\color{gray} (Emb: 0.674, TF-IDF: 0.387)}

El Estado tiene el deber prioritario de promover, respetar y garantizar, sin discriminación y en todo su actuar, los derechos de niñas, niños y adolescentes, resguardando su interés superior, su autonomía progresiva, su desarrollo integral y a ser escuchados y a participar e influir en todos los asuntos que les afecten en el grado que corresponda a su nivel de desarrollo en la vida familiar, comunitaria y social. 
\newline {\color{gray} \textbf{1º:} 419-4-Iniciativa-Convencional-de-la-cc-Tammy-Pustilnick-sobre-Derecho-de-NNA.pdf}
\newline {\color{gray} (Emb: 0.825, TF-IDF: 0.684)}
\newline {\color{gray} \textbf{2º:} 391-4-Iniciativa-Convencional-Constituyente-de-la-cc-Valentina-Miranda-sobre-Ninos-Ninas-y-Adolescentes-1440-24-01.pdf}
\newline {\color{gray} (Emb: 0.697, TF-IDF: 0.445)}

El Estado deberá velar porque no sean separados de sus familias salvo como medida temporal y último recurso en resguardo de su interés superior, en cuyo caso se priorizará un acogimiento familiar por sobre el residencial, debiendo adoptar todas las medidas que sean necesarias para asegurar su bienestar y resguardar el ejercicio de sus derechos y libertades. 
\newline {\color{gray} \textbf{1º:} 935-Iniciativa-de-la-Convencional-Constituyente-de-la-cc-Barbara-Rebolledo-sobre-Derechos-de-los-NNA.pdf}
\newline {\color{gray} (Emb: 1.000, TF-IDF: 1.000)}
\newline {\color{gray} \textbf{2º:} 1031-Iniciativa-Convencional-Constituyente-del-cc-Tomas-Laibe-sobre-Personas-Privadas-de-Libertad.pdf}
\newline {\color{gray} (Emb: 0.642, TF-IDF: 0.244)}


\item \textbf{Artículo} \newline
Chile es un Estado Regional, plurinacional e intercultural conformado por entidades territoriales autónomas, en un marco de equidad y solidaridad entre todas ellas, preservando la unidad e integridad del Estado. 
\newline {\color{gray} \textbf{1º:} 43-3-Iniciativa-Convencional-Constituyente-de-la-cc-Tammy-Pustilnick-y-otros.pdf}
\newline {\color{gray} (Emb: 0.922, TF-IDF: 0.944)}
\newline {\color{gray} \textbf{2º:} 91-3-Iniciativa-del-cc-Wilfredo-Bacian-que-establece-la-Forma-de-Estado-Regional-2.pdf}
\newline {\color{gray} (Emb: 0.828, TF-IDF: 0.483)}

El Estado promoverá la cooperación, la integración armónica y el desarrollo adecuado y justo entre las diversas entidades territoriales. 
\newline {\color{gray} \textbf{1º:} 91-3-Iniciativa-del-cc-Wilfredo-Bacian-que-establece-la-Forma-de-Estado-Regional-2.pdf}
\newline {\color{gray} (Emb: 0.841, TF-IDF: 0.539)}
\newline {\color{gray} \textbf{2º:} 43-3-Iniciativa-Convencional-Constituyente-de-la-cc-Tammy-Pustilnick-y-otros.pdf}
\newline {\color{gray} (Emb: 0.812, TF-IDF: 0.486)}


\item \textbf{Artículo} \newline
El Estado se organiza territorialmente en regiones autónomas, comunas autónomas, autonomías territoriales indígenas y territorios especiales. 
\newline {\color{gray} \textbf{1º:} 43-3-Iniciativa-Convencional-Constituyente-de-la-cc-Tammy-Pustilnick-y-otros.pdf}
\newline {\color{gray} (Emb: 0.973, TF-IDF: 0.875)}
\newline {\color{gray} \textbf{2º:} 91-3-Iniciativa-del-cc-Wilfredo-Bacian-que-establece-la-Forma-de-Estado-Regional-2.pdf}
\newline {\color{gray} (Emb: 0.818, TF-IDF: 0.707)}

Las entidades territoriales autónomas tienen personalidad jurídica y patrimonio propio y las potestades y competencias necesarias para gobernarse en atención al interés general de la República, de acuerdo a la Constitución y la ley, teniendo como límites los derechos humanos y de la Naturaleza. 
\newline {\color{gray} \textbf{1º:} 43-3-Iniciativa-Convencional-Constituyente-de-la-cc-Tammy-Pustilnick-y-otros.pdf}
\newline {\color{gray} (Emb: 0.838, TF-IDF: 0.565)}
\newline {\color{gray} \textbf{2º:} 929-Iniciativa-Convencional-Constituyente-de-la-cc-Maria-Magdalena-Rivera-sobre-Region-Exterior.pdf}
\newline {\color{gray} (Emb: 0.740, TF-IDF: 0.410)}

La creación, modificación, delimitación y supresión de las entidades territoriales deberá considerar criterios objetivos en función de antecedentes históricos, geográficos, sociales, culturales, ecosistémicos y económicos, garantizando la participación popular, democrática y vinculante de sus habitantes, de acuerdo con la Constitución y la ley. 
\newline {\color{gray} \textbf{1º:} 43-3-Iniciativa-Convencional-Constituyente-de-la-cc-Tammy-Pustilnick-y-otros.pdf}
\newline {\color{gray} (Emb: 0.916, TF-IDF: 0.931)}
\newline {\color{gray} \textbf{2º:} 656-Iniciativa-Convencional-Constituyente-de-la-cc-Lisette-Vergara-sobre-Unidades-Vecinales-121101-02.pdf}
\newline {\color{gray} (Emb: 0.579, TF-IDF: 0.282)}


\item \textbf{Artículo} \newline
Chile, en su diversidad geográfica, natural, histórica y cultural, forma un territorio único e indivisible. 
\newline {\color{gray} \textbf{1º:} 213-1-c-Iniciativa-Convencional-del-cc-Jaime-Bassa-sobre-Congreso-plurinacional-2058-hrs.pdf}
\newline {\color{gray} (Emb: 0.962, TF-IDF: 0.950)}
\newline {\color{gray} \textbf{2º:} 211-1-c-Iniciativa-convencional-del-cc-Cristián-Monckeberg-sobre-Régimen-de-Gobierno-y-Formación-de-la-Ley-1953-hrs.pdf}
\newline {\color{gray} (Emb: 0.940, TF-IDF: 0.833)}

La soberanía y jurisdicción sobre el territorio se ejerce de acuerdo a la Constitución, la ley y el derecho internacional. 
\newline {\color{gray} \textbf{1º:} 983-Iniciativa-Convencional-Constituyente-de-la-cc-Carolina-Vilches-sobre-Territorio-suelos-y-Agua.pdf}
\newline {\color{gray} (Emb: 0.811, TF-IDF: 0.681)}
\newline {\color{gray} \textbf{2º:} 99-3-c-Iniciativa-de-la-cc-Tammy-Pustilnick-Disposiciones-del-Estado-Regional.pdf}
\newline {\color{gray} (Emb: 0.811, TF-IDF: 0.681)}


\item \textbf{Artículo} \newline
La ley establecerá su ordenación espacial y gestión integrada, mediante un trato diferenciado, autónomo y descentralizado, según corresponda, en base a la equidad y justicia territorial. 
\newline {\color{gray} \textbf{1º:} 99-3-c-Iniciativa-de-la-cc-Tammy-Pustilnick-Disposiciones-del-Estado-Regional.pdf}
\newline {\color{gray} (Emb: 0.971, TF-IDF: 0.861)}
\newline {\color{gray} \textbf{2º:} 898-Iniciativa-Convencional-Constituyente-del-cc-Cesar-Uribe-sobre-Ordenamiento-y-Planificacion.pdf}
\newline {\color{gray} (Emb: 0.687, TF-IDF: 0.655)}

Es deber del Estado proteger los espacios y ecosistemas marinos y marino-costeros, propiciando las diversas vocaciones y usos asociados a ellos, y asegurando, en todo caso, su preservación, conservación y restauración ecológica. 
\newline {\color{gray} \textbf{1º:} 710-Iniciativa-Convencional-Constituyente-de-la-cc-Gloria-Alvarado-sobre-Maritorio-01-02.pdf}
\newline {\color{gray} (Emb: 0.793, TF-IDF: 0.749)}
\newline {\color{gray} \textbf{2º:} 99-3-c-Iniciativa-de-la-cc-Tammy-Pustilnick-Disposiciones-del-Estado-Regional.pdf}
\newline {\color{gray} (Emb: 0.762, TF-IDF: 0.340)}


\item \textbf{Artículo} \newline
Las regiones autónomas, comunas autónomas y autonomías territoriales indígenas están dotadas de autonomía política, administrativa y financiera para la realización de sus fines e intereses en los términos establecidos por la presente Constitución y la ley. 
\newline {\color{gray} \textbf{1º:} 99-3-c-Iniciativa-de-la-cc-Tammy-Pustilnick-Disposiciones-del-Estado-Regional.pdf}
\newline {\color{gray} (Emb: 0.927, TF-IDF: 0.798)}
\newline {\color{gray} \textbf{2º:} 99-3-c-Iniciativa-de-la-cc-Tammy-Pustilnick-Disposiciones-del-Estado-Regional.pdf}
\newline {\color{gray} (Emb: 0.779, TF-IDF: 0.488)}

En ningún caso el ejercicio de la autonomía podrá atentar en contra del carácter único e indivisible del Estado de Chile, ni permitirá la secesión territorial. 
\newline {\color{gray} \textbf{1º:} 99-3-c-Iniciativa-de-la-cc-Tammy-Pustilnick-Disposiciones-del-Estado-Regional.pdf}
\newline {\color{gray} (Emb: 0.772, TF-IDF: 0.665)}
\newline {\color{gray} \textbf{2º:} 91-3-Iniciativa-del-cc-Wilfredo-Bacian-que-establece-la-Forma-de-Estado-Regional-2.pdf}
\newline {\color{gray} (Emb: 0.682, TF-IDF: 0.367)}


\item \textbf{Artículo} \newline
Las entidades territoriales se coordinan y asocian en relaciones de solidaridad, cooperación, reciprocidad y apoyo mutuo, evitando la duplicidad de funciones, en conformidad a los mecanismos que establezca la ley. 
\newline {\color{gray} \textbf{1º:} 99-3-c-Iniciativa-de-la-cc-Tammy-Pustilnick-Disposiciones-del-Estado-Regional.pdf}
\newline {\color{gray} (Emb: 0.784, TF-IDF: 0.813)}
\newline {\color{gray} \textbf{2º:} 210-1-c-Iniciativa-Convencional-del-cc-Cristián-Monckeberg-sobre-Estado-Intercultural-1953-hrs.pdf}
\newline {\color{gray} (Emb: 0.598, TF-IDF: 0.282)}

Dos o más entidades territoriales, con o sin continuidad territorial, podrán pactar convenios y constituir asociaciones territoriales con la finalidad de lograr objetivos comunes, promover la cohesión social, mejorar la prestación de los servicios públicos, incrementar la eficiencia y eficacia en el ejercicio de sus competencias y potenciar el desarrollo social, cultural, económico sostenible y equilibrado. 
\newline {\color{gray} \textbf{1º:} 99-3-c-Iniciativa-de-la-cc-Tammy-Pustilnick-Disposiciones-del-Estado-Regional.pdf}
\newline {\color{gray} (Emb: 1.000, TF-IDF: 1.000)}
\newline {\color{gray} \textbf{2º:} 907-Iniciativa-Convencional-Constituyente-del-cc-Hugo-Gutierrez-Sobre-Forma-del-Estado.pdf}
\newline {\color{gray} (Emb: 0.577, TF-IDF: 0.270)}

El Estado promoverá y apoyará la cooperación y asociatividad con las entidades territoriales y entre ellas. 
\newline {\color{gray} \textbf{1º:} 99-3-c-Iniciativa-de-la-cc-Tammy-Pustilnick-Disposiciones-del-Estado-Regional.pdf}
\newline {\color{gray} (Emb: 0.876, TF-IDF: 0.869)}
\newline {\color{gray} \textbf{2º:} 99-3-c-Iniciativa-de-la-cc-Tammy-Pustilnick-Disposiciones-del-Estado-Regional.pdf}
\newline {\color{gray} (Emb: 0.825, TF-IDF: 0.287)}

La ley establecerá las bases generales para la creación y funcionamiento de estas asociaciones, en concordancia con la normativa regional respectiva. 
\newline {\color{gray} \textbf{1º:} 99-3-c-Iniciativa-de-la-cc-Tammy-Pustilnick-Disposiciones-del-Estado-Regional.pdf}
\newline {\color{gray} (Emb: 0.900, TF-IDF: 0.609)}
\newline {\color{gray} \textbf{2º:} 394-3-Iniciativa-Convencional-Constituyente-de-la-cc-Ramona-Reyes-sobre-Comuna-Autonoma-1525-24-01.pdf}
\newline {\color{gray} (Emb: 0.680, TF-IDF: 0.314)}

Las asociaciones de entidades territoriales, en ningún caso, alterarán la organización territorial del Estado. 
\newline {\color{gray} \textbf{1º:} 99-3-c-Iniciativa-de-la-cc-Tammy-Pustilnick-Disposiciones-del-Estado-Regional.pdf}
\newline {\color{gray} (Emb: 1.000, TF-IDF: 1.000)}
\newline {\color{gray} \textbf{2º:} 400-1-Iniciativa-Convencional-Constituyente-de-la-cc-Constanza-Hube-sobre-Servicio-y-Registro-Electoral-1905-24-01.pdf}
\newline {\color{gray} (Emb: 0.617, TF-IDF: 0.371)}


\item \textbf{Artículo} \newline
Las entidades territoriales garantizan el derecho de sus habitantes a participar, individual o colectivamente en las decisiones públicas, comprendiendo en ella la formulación, ejecución, evaluación, fiscalización y control democrático de la función pública, con arreglo a la Constitución y las leyes. 
\newline {\color{gray} \textbf{1º:} 99-3-c-Iniciativa-de-la-cc-Tammy-Pustilnick-Disposiciones-del-Estado-Regional.pdf}
\newline {\color{gray} (Emb: 0.994, TF-IDF: 0.969)}
\newline {\color{gray} \textbf{2º:} 641-Iniciativa-Convencional-Constituyente-del-cc-Mauricio-Daza-sobre-Contraloria-General-1730-01-02.pdf}
\newline {\color{gray} (Emb: 0.654, TF-IDF: 0.303)}

Los pueblos y naciones preexistentes al Estado deberán ser consultados y otorgar el consentimiento libre, previo e informado en aquellas materias o asuntos que les afecten en sus derechos reconocidos en esta Constitución. 
\newline {\color{gray} \textbf{1º:} 99-3-c-Iniciativa-de-la-cc-Tammy-Pustilnick-Disposiciones-del-Estado-Regional.pdf}
\newline {\color{gray} (Emb: 0.685, TF-IDF: 0.466)}
\newline {\color{gray} \textbf{2º:} 788-Iniciativa-Convencional-Constituyente-de-la-cc-Camila-Zarate-sobre-Democracia-Ecologica.pdf}
\newline {\color{gray} (Emb: 0.650, TF-IDF: 0.442)}


\item \textbf{Artículo} \newline
Las entidades territoriales considerarán para su planificación social, política, administrativa, cultural, territorial y económica los criterios de suficiencia presupuestaria, inclusión e interculturalidad, integración socioespacial, perspectiva de género, enfoque socio ecosistémico, enfoque en derechos humanos y los demás que establezca esta Constitución. 
\newline {\color{gray} \textbf{1º:} 99-3-c-Iniciativa-de-la-cc-Tammy-Pustilnick-Disposiciones-del-Estado-Regional.pdf}
\newline {\color{gray} (Emb: 0.874, TF-IDF: 0.849)}
\newline {\color{gray} \textbf{2º:} 898-Iniciativa-Convencional-Constituyente-del-cc-Cesar-Uribe-sobre-Ordenamiento-y-Planificacion.pdf}
\newline {\color{gray} (Emb: 0.590, TF-IDF: 0.293)}

Es deber de las entidades territoriales, en el ámbito de sus competencias, establecer una política permanente de equidad territorial de desarrollo sostenible y armónico con la naturaleza. 
\newline {\color{gray} \textbf{1º:} 99-3-c-Iniciativa-de-la-cc-Tammy-Pustilnick-Disposiciones-del-Estado-Regional.pdf}
\newline {\color{gray} (Emb: 0.961, TF-IDF: 0.870)}
\newline {\color{gray} \textbf{2º:} 1014-Iniciativa-Convencional-Constituyente-cc-Adriana-Ampuero-Haciendas-territoriales-y-autonomia-financiera.pdf}
\newline {\color{gray} (Emb: 0.904, TF-IDF: 0.854)}


\item \textbf{Artículo} \newline
El Estado de Chile promoverá un desarrollo territorial equitativo, armónico y solidario que permita una integración efectiva de las distintas localidades, tanto urbanas como rurales, promoviendo la equidad horizontal en la provisión de bienes y servicios. 
\newline {\color{gray} \textbf{1º:} 269-3-Iniciativa-Convencional-del-cc-Felipe-Mena-sobre-Organizacion-Territorial-del-Estado-17-01-1154-hrs.pdf}
\newline {\color{gray} (Emb: 0.960, TF-IDF: 0.930)}
\newline {\color{gray} \textbf{2º:} 154-3-c-Iniciativa-del-cc-Felipe-Mena-sobre-Organizacion-Territorial-del-Estado.pdf}
\newline {\color{gray} (Emb: 0.960, TF-IDF: 0.930)}

El Estado garantiza un tratamiento equitativo y un desarrollo armónico y solidario entre las diversas entidades territoriales, propendiendo al interés general, no pudiendo establecer diferencias arbitrarias entre ellas, asegurando a su vez, las mismas condiciones de acceso a los servicios públicos, al empleo y a todas las prestaciones estatales, sin perjuicio del lugar que habiten en el territorio, estableciendo de ser necesario, acciones afirmativas en favor de los grupos empobrecidos e históricamente vulnerados. 
\newline {\color{gray} \textbf{1º:} 99-3-c-Iniciativa-de-la-cc-Tammy-Pustilnick-Disposiciones-del-Estado-Regional.pdf}
\newline {\color{gray} (Emb: 0.892, TF-IDF: 0.801)}
\newline {\color{gray} \textbf{2º:} 72-2-Iniciativa-Convencional-Constituyente-de-la-cc-Elisa-Loncon-y-otros.pdf}
\newline {\color{gray} (Emb: 0.580, TF-IDF: 0.330)}


\item \textbf{Artículo} \newline
Las entidades territoriales y sus órganos reconocen, garantizan y promueven en todo su actuar el reconocimiento político y jurídico de los pueblos y naciones preexistentes al Estado que habitan sus territorios; su supervivencia, existencia y desarrollo armónico e integral; la distribución equitativa del poder y de los espacios de participación política; el uso, reconocimiento y promoción de las lenguas indígenas que se hablan en ellas, propiciando el entendimiento intercultural, el respeto de formas diversas de ver, organizar y concebir el mundo y de relacionarse con la naturaleza; la protección y el respeto de los derechos de autodeterminación y de autonomía de los territorios indígenas, en coordinación con el resto de las entidades territoriales. 
\newline {\color{gray} \textbf{1º:} 99-3-c-Iniciativa-de-la-cc-Tammy-Pustilnick-Disposiciones-del-Estado-Regional.pdf}
\newline {\color{gray} (Emb: 0.919, TF-IDF: 0.912)}
\newline {\color{gray} \textbf{2º:} 670-Iniciativa-Convencional-Constituyente-del-cc-Alvin-Saldana-sobre-Pueblos-Originarios-121101-02.pdf}
\newline {\color{gray} (Emb: 0.688, TF-IDF: 0.393)}


\item \textbf{Artículo} \newline
La elección de las y los representantes por votación popular de las entidades territoriales se efectuará asegurando la paridad de género, la probidad, la representatividad territorial, la pertenencia territorial, avecindamiento y la representación efectiva de los pueblos y naciones preexistentes al Estado. 
\newline {\color{gray} \textbf{1º:} 99-3-c-Iniciativa-de-la-cc-Tammy-Pustilnick-Disposiciones-del-Estado-Regional.pdf}
\newline {\color{gray} (Emb: 0.883, TF-IDF: 0.727)}
\newline {\color{gray} \textbf{2º:} 94-1-Iniciativa-de-la-cc-Rosa-Catrileo-Establece-el-reconocimiento-de-los-Pueblos-Indigenas-2.pdf}
\newline {\color{gray} (Emb: 0.655, TF-IDF: 0.359)}

La Constitución y la ley establecerán los requisitos para la postulación y las causales de cesación de dichos cargos. 
\newline {\color{gray} \textbf{1º:} 1014-Iniciativa-Convencional-Constituyente-cc-Adriana-Ampuero-Haciendas-territoriales-y-autonomia-financiera.pdf}
\newline {\color{gray} (Emb: 0.834, TF-IDF: 0.613)}
\newline {\color{gray} \textbf{2º:} 377-2-Iniciativa-Convencional-Constituyente-del-cc-Alvin-Saldana-sobre-Mecanismos-de-Democracia-1045-hrs-24-01.pdf}
\newline {\color{gray} (Emb: 0.776, TF-IDF: 0.404)}

La calificación y procedencia de estas causales de cesación se realizará a través de un procedimiento expedito ante la justicia electoral, en conformidad a la ley. 
\newline {\color{gray} \textbf{1º:} 99-3-c-Iniciativa-de-la-cc-Tammy-Pustilnick-Disposiciones-del-Estado-Regional.pdf}
\newline {\color{gray} (Emb: 1.000, TF-IDF: 1.000)}
\newline {\color{gray} \textbf{2º:} 711-Iniciativa-Convencional-Constituyente-de-la-cc-Ingrid-Villena-sobre-Tricel.pdf}
\newline {\color{gray} (Emb: 0.750, TF-IDF: 0.356)}


\item \textbf{Artículo} \newline
Ninguna entidad territorial podrá ejercer tutela sobre otra entidad territorial, sin perjuicio de la aplicación de los principios de coordinación, de asociatividad, de solidaridad, y los conflictos de competencias que puedan ocasionarse. 
\newline {\color{gray} \textbf{1º:} 157-3-c-Iniciativa-de-la-cc-Tammy-Pustilnick-Disposiciones-Generales-del-Estado.pdf}
\newline {\color{gray} (Emb: 0.907, TF-IDF: 0.807)}
\newline {\color{gray} \textbf{2º:} 751-Iniciativa-Convencional-Constituyente-del-cc-Raul-Celis-sobre-Fuerzas-Armadas-01-02.pdf}
\newline {\color{gray} (Emb: 0.610, TF-IDF: 0.343)}


\item \textbf{Artículo} \newline
Sin perjuicio de las competencias que establece esta Constitución y la ley, el Estado podrá transferir a las entidades territoriales aquellas competencias de titularidad estatal que por su propia naturaleza son susceptibles de transferencia. 
\newline {\color{gray} \textbf{1º:} 157-3-c-Iniciativa-de-la-cc-Tammy-Pustilnick-Disposiciones-Generales-del-Estado.pdf}
\newline {\color{gray} (Emb: 1.000, TF-IDF: 1.000)}
\newline {\color{gray} \textbf{2º:} 157-3-c-Iniciativa-de-la-cc-Tammy-Pustilnick-Disposiciones-Generales-del-Estado.pdf}
\newline {\color{gray} (Emb: 0.775, TF-IDF: 0.750)}

Estas transferencias deberán ir acompañadas siempre por el personal y los recursos financieros suficientes y oportunos para su adecuada ejecución. 
\newline {\color{gray} \textbf{1º:} 157-3-c-Iniciativa-de-la-cc-Tammy-Pustilnick-Disposiciones-Generales-del-Estado.pdf}
\newline {\color{gray} (Emb: 0.685, TF-IDF: 0.686)}
\newline {\color{gray} \textbf{2º:} 1014-Iniciativa-Convencional-Constituyente-cc-Adriana-Ampuero-Haciendas-territoriales-y-autonomia-financiera.pdf}
\newline {\color{gray} (Emb: 0.617, TF-IDF: 0.656)}

Una ley regulará el régimen jurídico del procedimiento de transferencia de competencias y sus sistemas de evaluación y control. 
\newline {\color{gray} \textbf{1º:} 120-3-c-Iniciativa-de-la-cc-Tammy-Pustilnick-atribuciones-exclusivas-de-la-Asamblea-Regional.pdf}
\newline {\color{gray} (Emb: 1.000, TF-IDF: 1.000)}
\newline {\color{gray} \textbf{2º:} 641-Iniciativa-Convencional-Constituyente-del-cc-Mauricio-Daza-sobre-Contraloria-General-1730-01-02.pdf}
\newline {\color{gray} (Emb: 0.632, TF-IDF: 0.471)}


\item \textbf{Artículo} \newline
La ley establecerá el procedimiento para resolución de las distintas contiendas de competencia que se susciten entre el Estado y las entidades territoriales, o entre ellas, las que serán conocidas por el órgano encargado de la justicia constitucional. 
\newline {\color{gray} \textbf{1º:} 151-3-c-Iniciativa-de-la-cc-Angelica-Tepper-Competencias-de-los-Gobiernos-Regionales.pdf}
\newline {\color{gray} (Emb: 0.716, TF-IDF: 0.561)}
\newline {\color{gray} \textbf{2º:} 472-6-Iniciativa-Convencional-Constituyente-del-cc-Daniel-Bravo-sobre-Corte-Constitucional-2003-31-01.pdf}
\newline {\color{gray} (Emb: 0.676, TF-IDF: 0.485)}


\item \textbf{Artículo} \newline
Las funciones públicas deberán radicarse priorizando la entidad local sobre la regional y ésta última sobre el Estado, sin perjuicio de aquellas competencias que la propia Constitución o las leyes reserven a cada una de estas entidades territoriales. 
\newline {\color{gray} \textbf{1º:} 641-Iniciativa-Convencional-Constituyente-del-cc-Mauricio-Daza-sobre-Contraloria-General-1730-01-02.pdf}
\newline {\color{gray} (Emb: 0.721, TF-IDF: 0.361)}
\newline {\color{gray} \textbf{2º:} 384-3-Iniciativa-Convencional-Constituyente-del-cc-Felipe-Mena-sobre-Gobiernos-Regionales-1159-24-01.pdf}
\newline {\color{gray} (Emb: 0.671, TF-IDF: 0.316)}

La Región Autónoma o el Estado, cuando así lo exija el interés general, podrán subrogar de manera transitoria y supletoria las competencias que no puedan ser asumidas por la entidad local. 
\newline {\color{gray} \textbf{1º:} 394-3-Iniciativa-Convencional-Constituyente-de-la-cc-Ramona-Reyes-sobre-Comuna-Autonoma-1525-24-01.pdf}
\newline {\color{gray} (Emb: 0.969, TF-IDF: 0.905)}
\newline {\color{gray} \textbf{2º:} 394-3-Iniciativa-Convencional-Constituyente-de-la-cc-Ramona-Reyes-sobre-Comuna-Autonoma-1525-24-01.pdf}
\newline {\color{gray} (Emb: 0.698, TF-IDF: 0.339)}


\item \textbf{Artículo} \newline
La ley establecerá los criterios y requisitos para la aplicación de diferencias territoriales, así como los mecanismos de solidaridad y equidad que compensen las desigualdades entre los distintos niveles territoriales. 
\newline {\color{gray} \textbf{1º:} 633-3-Iniciativa-Convencional-Constituyente-de-la-Yarela-Gomez-sobre-Regimen-Tributario-1739-01-02.pdf}
\newline {\color{gray} (Emb: 0.587, TF-IDF: 0.301)}
\newline {\color{gray} \textbf{2º:} 394-3-Iniciativa-Convencional-Constituyente-de-la-cc-Ramona-Reyes-sobre-Comuna-Autonoma-1525-24-01.pdf}
\newline {\color{gray} (Emb: 0.583, TF-IDF: 0.273)}

El Estado deberá generar políticas públicas diferenciadas y transferir las competencias que mejor se ajusten a las necesidades y particularidades de los entes territoriales, con los respectivos recursos. 
\newline {\color{gray} \textbf{1º:} 269-3-Iniciativa-Convencional-del-cc-Felipe-Mena-sobre-Organizacion-Territorial-del-Estado-17-01-1154-hrs.pdf}
\newline {\color{gray} (Emb: 0.708, TF-IDF: 0.247)}
\newline {\color{gray} \textbf{2º:} 154-3-c-Iniciativa-del-cc-Felipe-Mena-sobre-Organizacion-Territorial-del-Estado.pdf}
\newline {\color{gray} (Emb: 0.708, TF-IDF: 0.239)}


\item \textbf{Artículo} \newline
Las Regiones autónomas son entidades políticas y territoriales dotadas de personalidad jurídica de derecho público y patrimonio propio que gozan de autonomía para el desarrollo de los intereses regionales, la gestión de sus recursos económicos y el ejercicio de las atribuciones legislativa, reglamentaria, ejecutiva y fiscalizadora a través de sus órganos en el ámbito de sus competencias, con arreglo a lo dispuesto en la Constitución y la ley. 
\newline {\color{gray} \textbf{1º:} 159-3-c-Iniciativa-de-la-cc-Jennifer-Mella-.pdf}
\newline {\color{gray} (Emb: 0.712, TF-IDF: 0.554)}
\newline {\color{gray} \textbf{2º:} 122-3-c-Iniciativa-de-la-cc-Jennifer-Mella-Forma-del-Estado.pdf}
\newline {\color{gray} (Emb: 0.712, TF-IDF: 0.352)}


\item \textbf{Artículo} \newline
Las competencias no expresamente conferidas a la Región autónoma corresponden al Estado, sin perjuicio de las transferencias de competencia que regula la Constitución y la ley. 
\newline {\color{gray} \textbf{1º:} 157-3-c-Iniciativa-de-la-cc-Tammy-Pustilnick-Disposiciones-Generales-del-Estado.pdf}
\newline {\color{gray} (Emb: 1.000, TF-IDF: 1.000)}
\newline {\color{gray} \textbf{2º:} 157-3-c-Iniciativa-de-la-cc-Tammy-Pustilnick-Disposiciones-Generales-del-Estado.pdf}
\newline {\color{gray} (Emb: 0.775, TF-IDF: 0.438)}


\item \textbf{Artículo} \newline
Cada Región Autónoma establecerá su organización administrativa y funcionamiento interno, en el marco de las competencias fiscalizadoras, normativas, resolutivas, administrativas y las demás establecidas en la Constitución y las leyes. 
\newline {\color{gray} \textbf{1º:} 384-3-Iniciativa-Convencional-Constituyente-del-cc-Felipe-Mena-sobre-Gobiernos-Regionales-1159-24-01.pdf}
\newline {\color{gray} (Emb: 0.745, TF-IDF: 0.386)}
\newline {\color{gray} \textbf{2º:} 384-3-Iniciativa-Convencional-Constituyente-del-cc-Felipe-Mena-sobre-Gobiernos-Regionales-1159-24-01.pdf}
\newline {\color{gray} (Emb: 0.736, TF-IDF: 0.386)}

El Estatuto Regional debe respetar los derechos fundamentales y los principios del Estado social y democrático de derecho reconocidos en los términos establecidos en la Constitución. 
\newline {\color{gray} \textbf{1º:} 122-3-c-Iniciativa-de-la-cc-Jennifer-Mella-Forma-del-Estado.pdf}
\newline {\color{gray} (Emb: 0.824, TF-IDF: 0.495)}
\newline {\color{gray} \textbf{2º:} 159-3-c-Iniciativa-de-la-cc-Jennifer-Mella-.pdf}
\newline {\color{gray} (Emb: 0.824, TF-IDF: 0.495)}


\item \textbf{Artículo} \newline
El proyecto de Estatuto Regional será propuesto por la Gobernadora o Gobernador Regional a la Asamblea Regional respectiva, para su deliberación y acuerdo, el cual será aprobado por la mayoría absoluta de sus miembros en ejercicio. 
\newline {\color{gray} \textbf{1º:} 120-3-c-Iniciativa-de-la-cc-Tammy-Pustilnick-atribuciones-exclusivas-de-la-Asamblea-Regional.pdf}
\newline {\color{gray} (Emb: 0.709, TF-IDF: 0.596)}
\newline {\color{gray} \textbf{2º:} 159-3-c-Iniciativa-de-la-cc-Jennifer-Mella-.pdf}
\newline {\color{gray} (Emb: 0.674, TF-IDF: 0.442)}

El proceso de elaboración y reforma del Estatuto Regional deberá garantizar la participación popular, democrática y vinculante de sus habitantes. 
\newline {\color{gray} \textbf{1º:} 394-3-Iniciativa-Convencional-Constituyente-de-la-cc-Ramona-Reyes-sobre-Comuna-Autonoma-1525-24-01.pdf}
\newline {\color{gray} (Emb: 0.911, TF-IDF: 0.624)}
\newline {\color{gray} \textbf{2º:} 374-2-Iniciativa-Convencional-Constituyente-de-la-cc-Loreto-Vallejos-sobre-Democracia-Directa-0900-hrs-24-01.pdf}
\newline {\color{gray} (Emb: 0.603, TF-IDF: 0.451)}


\item \textbf{Artículo} \newline
La organización institucional de las Regiones Autónomas se compone del Gobernador o Gobernadora Regional y de la Asamblea Regional. 
\newline {\color{gray} \textbf{1º:} 119-3-c-Iniciativa-de-la-cc-Tammy-Pustilnick-competencias-de-las-Regiones-Autonomas.pdf}
\newline {\color{gray} (Emb: 0.786, TF-IDF: 0.755)}
\newline {\color{gray} \textbf{2º:} 159-3-c-Iniciativa-de-la-cc-Jennifer-Mella-.pdf}
\newline {\color{gray} (Emb: 0.730, TF-IDF: 0.528)}


\item \textbf{Artículo} \newline
En este caso, se considerará que se ha ejercido el cargo durante un período cuando el Gobernador o Gobernadora Regional haya cumplido más de la mitad del mandato. 
\newline {\color{gray} \textbf{1º:} 400-1-Iniciativa-Convencional-Constituyente-de-la-cc-Constanza-Hube-sobre-Servicio-y-Registro-Electoral-1905-24-01.pdf}
\newline {\color{gray} (Emb: 0.586, TF-IDF: 0.487)}
\newline {\color{gray} \textbf{2º:} 120-3-c-Iniciativa-de-la-cc-Tammy-Pustilnick-atribuciones-exclusivas-de-la-Asamblea-Regional.pdf}
\newline {\color{gray} (Emb: 0.578, TF-IDF: 0.375)}

La Gobernadora o Gobernador Regional ejercerá sus funciones por el término de cuatro años, pudiendo ser reelegido o reelegida consecutivamente sólo una vez para el período siguiente. 
\newline {\color{gray} \textbf{1º:} 384-3-Iniciativa-Convencional-Constituyente-del-cc-Felipe-Mena-sobre-Gobiernos-Regionales-1159-24-01.pdf}
\newline {\color{gray} (Emb: 0.903, TF-IDF: 0.806)}
\newline {\color{gray} \textbf{2º:} 384-3-Iniciativa-Convencional-Constituyente-del-cc-Felipe-Mena-sobre-Gobiernos-Regionales-1159-24-01.pdf}
\newline {\color{gray} (Emb: 0.816, TF-IDF: 0.527)}

En la elección de Gobernadora o Gobernador Regional, resultará electo quien obtenga la mayoría de los votos válidamente emitidos, pero si ningún candidato logra al menos el cuarenta por ciento de los votos se producirá una segunda votación entre los candidatos o candidatas que hayan obtenido las dos más altas mayorías, resultando elegido el que obtuviere la mayoría de los votos válidamente emitidos. 
\newline {\color{gray} \textbf{1º:} 286-1-Iniciativa-Convencional-de-la-cc-Barbara-Sepulveda-sobre-Poder-Ejecutivo-1210-hrs.pdf}
\newline {\color{gray} (Emb: 0.747, TF-IDF: 0.422)}
\newline {\color{gray} \textbf{2º:} 236-1-Iniciativa-Convencional-de-la-cc-Barbara-Sepulveda-sobre-Poder-Ejecutivo-1146-hrs.pdf}
\newline {\color{gray} (Emb: 0.747, TF-IDF: 0.421)}

La Gobernadora o Gobernador regional, será elegido en votación directa, en conformidad con lo dispuesto en la Constitución y la ley. 
\newline {\color{gray} \textbf{1º:} 384-3-Iniciativa-Convencional-Constituyente-del-cc-Felipe-Mena-sobre-Gobiernos-Regionales-1159-24-01.pdf}
\newline {\color{gray} (Emb: 0.761, TF-IDF: 0.515)}
\newline {\color{gray} \textbf{2º:} 120-3-c-Iniciativa-de-la-cc-Tammy-Pustilnick-atribuciones-exclusivas-de-la-Asamblea-Regional.pdf}
\newline {\color{gray} (Emb: 0.716, TF-IDF: 0.433)}

Una Gobernadora o Gobernador Regional dirigirá el Gobierno Regional, ejerciendo la función administrativa y reglamentaria y representará a la Región autónoma ante las demás autoridades nacionales e internacionales, en el marco de la política nacional de relaciones internacionales con funciones de coordinación e intermediación entre el gobierno central y la región. 
\newline {\color{gray} \textbf{1º:} 384-3-Iniciativa-Convencional-Constituyente-del-cc-Felipe-Mena-sobre-Gobiernos-Regionales-1159-24-01.pdf}
\newline {\color{gray} (Emb: 0.651, TF-IDF: 0.488)}
\newline {\color{gray} \textbf{2º:} 159-3-c-Iniciativa-de-la-cc-Jennifer-Mella-.pdf}
\newline {\color{gray} (Emb: 0.646, TF-IDF: 0.488)}

El Gobierno Regional es el órgano ejecutivo de la Región Autónoma. 
\newline {\color{gray} \textbf{1º:} 907-Iniciativa-Convencional-Constituyente-del-cc-Hugo-Gutierrez-Sobre-Forma-del-Estado.pdf}
\newline {\color{gray} (Emb: 0.789, TF-IDF: 0.536)}
\newline {\color{gray} \textbf{2º:} 907-Iniciativa-Convencional-Constituyente-del-cc-Hugo-Gutierrez-Sobre-Forma-del-Estado.pdf}
\newline {\color{gray} (Emb: 0.748, TF-IDF: 0.536)}

La Gobernadora o Gobernador regional tendrá la representación judicial y extrajudicial de la región. 
\newline {\color{gray} \textbf{1º:} 120-3-c-Iniciativa-de-la-cc-Tammy-Pustilnick-atribuciones-exclusivas-de-la-Asamblea-Regional.pdf}
\newline {\color{gray} (Emb: 0.730, TF-IDF: 0.538)}
\newline {\color{gray} \textbf{2º:} 929-Iniciativa-Convencional-Constituyente-de-la-cc-Maria-Magdalena-Rivera-sobre-Region-Exterior.pdf}
\newline {\color{gray} (Emb: 0.684, TF-IDF: 0.431)}


\item \textbf{Artículo} \newline
El Consejo de Alcaldes y Alcaldesas es un órgano de carácter consultivo que estará integrado por los alcaldes y alcaldesas de todas las comunas de la región autónoma y de las ciudades respectivas, el cual será coordinado por quien determinen sus integrantes por mayoría absoluta. 
\newline {\color{gray} \textbf{1º:} 384-3-Iniciativa-Convencional-Constituyente-del-cc-Felipe-Mena-sobre-Gobiernos-Regionales-1159-24-01.pdf}
\newline {\color{gray} (Emb: 0.881, TF-IDF: 0.679)}
\newline {\color{gray} \textbf{2º:} 385-3-Iniciativa-Convencional-Constituyente-del-cc-Felipe-Mena-sobre-Municipalidades-1200-24-01.pdf}
\newline {\color{gray} (Emb: 0.737, TF-IDF: 0.284)}

El Consejo deberá sesionar y abordar las problemáticas de la región autónoma, promover una coordinación efectiva entre los distintos órganos con presencia regional y fomentar una cooperación eficaz entre los gobiernos locales en la forma que determine la ley. 
\newline {\color{gray} \textbf{1º:} 384-3-Iniciativa-Convencional-Constituyente-del-cc-Felipe-Mena-sobre-Gobiernos-Regionales-1159-24-01.pdf}
\newline {\color{gray} (Emb: 0.804, TF-IDF: 0.844)}
\newline {\color{gray} \textbf{2º:} 84-2-Iniciativa-Convencional-Constituyente-del-cc-Martin-Arrau-y-otros.pdf}
\newline {\color{gray} (Emb: 0.635, TF-IDF: 0.279)}


\item \textbf{Artículo} \newline
La Asamblea Regional es el órgano colegiado de representación regional que, en conformidad a la Constitución, está dotado de potestades normativas, resolutivas y fiscalizadoras. 
\newline {\color{gray} \textbf{1º:} 120-3-c-Iniciativa-de-la-cc-Tammy-Pustilnick-atribuciones-exclusivas-de-la-Asamblea-Regional.pdf}
\newline {\color{gray} (Emb: 0.767, TF-IDF: 0.697)}
\newline {\color{gray} \textbf{2º:} 120-3-c-Iniciativa-de-la-cc-Tammy-Pustilnick-atribuciones-exclusivas-de-la-Asamblea-Regional.pdf}
\newline {\color{gray} (Emb: 0.761, TF-IDF: 0.357)}

Una ley determinará los requisitos generales para acceder al cargo de Asambleísta Regional y su número en proporción a la población regional. 
\newline {\color{gray} \textbf{1º:} 400-1-Iniciativa-Convencional-Constituyente-de-la-cc-Constanza-Hube-sobre-Servicio-y-Registro-Electoral-1905-24-01.pdf}
\newline {\color{gray} (Emb: 0.689, TF-IDF: 0.341)}
\newline {\color{gray} \textbf{2º:} 219-1-c-Iniciativa-Convencional-de-la-cc-Rosa-Catrileo-sobre-Parlamento-Bicameral-2215-hrs.pdf}
\newline {\color{gray} (Emb: 0.685, TF-IDF: 0.341)}

La elección de Asambleístas Regionales será por sufragio universal, directo y secreto. 
\newline {\color{gray} \textbf{1º:} 907-Iniciativa-Convencional-Constituyente-del-cc-Hugo-Gutierrez-Sobre-Forma-del-Estado.pdf}
\newline {\color{gray} (Emb: 0.801, TF-IDF: 0.720)}
\newline {\color{gray} \textbf{2º:} 394-3-Iniciativa-Convencional-Constituyente-de-la-cc-Ramona-Reyes-sobre-Comuna-Autonoma-1525-24-01.pdf}
\newline {\color{gray} (Emb: 0.766, TF-IDF: 0.559)}

Los y las Asambleístas Regionales ejercerán sus funciones por el término de cuatro años, pudiendo ser reelegidos consecutivamente sólo una vez para el período inmediatamente siguiente. 
\newline {\color{gray} \textbf{1º:} 384-3-Iniciativa-Convencional-Constituyente-del-cc-Felipe-Mena-sobre-Gobiernos-Regionales-1159-24-01.pdf}
\newline {\color{gray} (Emb: 0.871, TF-IDF: 0.839)}
\newline {\color{gray} \textbf{2º:} 384-3-Iniciativa-Convencional-Constituyente-del-cc-Felipe-Mena-sobre-Gobiernos-Regionales-1159-24-01.pdf}
\newline {\color{gray} (Emb: 0.831, TF-IDF: 0.596)}

En este caso, se considerará que se ha ejercido el cargo durante un período cuando hayan cumplido más de la mitad de su mandato. 
\newline {\color{gray} \textbf{1º:} 400-1-Iniciativa-Convencional-Constituyente-de-la-cc-Constanza-Hube-sobre-Servicio-y-Registro-Electoral-1905-24-01.pdf}
\newline {\color{gray} (Emb: 0.675, TF-IDF: 0.562)}
\newline {\color{gray} \textbf{2º:} 196-2-c-Iniciativa-Convenciona-del-cc-Marco-Arellano-sobre-Plebiscito-e-iniciativa-de-ley-1552.pdf}
\newline {\color{gray} (Emb: 0.575, TF-IDF: 0.243)}


\item \textbf{Artículo} \newline
El Gobernador o Gobernadora Regional y las jefaturas de los servicios públicos regionales deberán rendir cuenta ante el Consejo Social Regional, a lo menos, una vez al año de la ejecución presupuestaria y el desarrollo de proyectos en los términos prescritos por el Estatuto Regional. 
\newline {\color{gray} \textbf{1º:} 384-3-Iniciativa-Convencional-Constituyente-del-cc-Felipe-Mena-sobre-Gobiernos-Regionales-1159-24-01.pdf}
\newline {\color{gray} (Emb: 0.784, TF-IDF: 0.656)}
\newline {\color{gray} \textbf{2º:} 384-3-Iniciativa-Convencional-Constituyente-del-cc-Felipe-Mena-sobre-Gobiernos-Regionales-1159-24-01.pdf}
\newline {\color{gray} (Emb: 0.784, TF-IDF: 0.656)}

La Constitución y la ley establecerán las bases de los mecanismos y procedimientos de participación popular, velando por un involucramiento efectivo de las personas y sus organizaciones dentro de la Región Autónoma. 
\newline {\color{gray} \textbf{1º:} 384-3-Iniciativa-Convencional-Constituyente-del-cc-Felipe-Mena-sobre-Gobiernos-Regionales-1159-24-01.pdf}
\newline {\color{gray} (Emb: 0.805, TF-IDF: 0.733)}
\newline {\color{gray} \textbf{2º:} 385-3-Iniciativa-Convencional-Constituyente-del-cc-Felipe-Mena-sobre-Municipalidades-1200-24-01.pdf}
\newline {\color{gray} (Emb: 0.739, TF-IDF: 0.676)}

El Consejo Social Regional es el encargado de promover la participación popular en los asuntos públicos regionales de carácter participativo y consultivo. 
\newline {\color{gray} \textbf{1º:} 394-3-Iniciativa-Convencional-Constituyente-de-la-cc-Ramona-Reyes-sobre-Comuna-Autonoma-1525-24-01.pdf}
\newline {\color{gray} (Emb: 0.597, TF-IDF: 0.410)}
\newline {\color{gray} \textbf{2º:} 118-3-c-Iniciativa-del-cc-Cristobal-Andrade-Regiones-Autonomas.pdf}
\newline {\color{gray} (Emb: 0.593, TF-IDF: 0.401)}

Su integración y competencias serán determinadas por ley. 
\newline {\color{gray} \textbf{1º:} 806-Iniciativa-Convencional-Constituyente-del-cc-Hugo-Gutierrez-sobre-Gobiernos-Regionales.pdf}
\newline {\color{gray} (Emb: 0.795, TF-IDF: 0.522)}
\newline {\color{gray} \textbf{2º:} 957-5-Iniciativa-Convencional-Constituyente-de-la-cc-Ivanna-Olivares-sobre-Nuevo-Modelo-Economico.pdf}
\newline {\color{gray} (Emb: 0.711, TF-IDF: 0.508)}


\item \textbf{Artículo} \newline
El ejercicio de estas competencias por la Región Autónoma no excluye la concurrencia y desarrollo coordinado con otros órganos del Estado, conforme a la Constitución y la ley. 
\newline {\color{gray} \textbf{1º:} 157-3-c-Iniciativa-de-la-cc-Tammy-Pustilnick-Disposiciones-Generales-del-Estado.pdf}
\newline {\color{gray} (Emb: 0.725, TF-IDF: 0.534)}
\newline {\color{gray} \textbf{2º:} 119-3-c-Iniciativa-de-la-cc-Tammy-Pustilnick-competencias-de-las-Regiones-Autonomas.pdf}
\newline {\color{gray} (Emb: 0.660, TF-IDF: 0.534)}

- Las demás competencias que determine la Constitución y ley nacional. 
\newline {\color{gray} \textbf{1º:} 119-3-c-Iniciativa-de-la-cc-Tammy-Pustilnick-competencias-de-las-Regiones-Autonomas.pdf}
\newline {\color{gray} (Emb: 1.000, TF-IDF: 1.000)}
\newline {\color{gray} \textbf{2º:} 118-3-c-Iniciativa-del-cc-Cristobal-Andrade-Regiones-Autonomas.pdf}
\newline {\color{gray} (Emb: 1.000, TF-IDF: 1.000)}

- Promover la participación popular en asuntos de interés regional. 
\newline {\color{gray} \textbf{1º:} 118-3-c-Iniciativa-del-cc-Cristobal-Andrade-Regiones-Autonomas.pdf}
\newline {\color{gray} (Emb: 0.662, TF-IDF: 0.419)}
\newline {\color{gray} \textbf{2º:} 119-3-c-Iniciativa-de-la-cc-Tammy-Pustilnick-competencias-de-las-Regiones-Autonomas.pdf}
\newline {\color{gray} (Emb: 0.662, TF-IDF: 0.419)}

- Ejercer autónomamente la administración y coordinación de todos los servicios públicos de su dependencia. 
\newline {\color{gray} \textbf{1º:} 151-3-c-Iniciativa-de-la-cc-Angelica-Tepper-Competencias-de-los-Gobiernos-Regionales.pdf}
\newline {\color{gray} (Emb: 0.736, TF-IDF: 0.385)}
\newline {\color{gray} \textbf{2º:} 11-4-Iniciativa-Convencional-Constituyente-de-la-cc-María-Elisa-Quinteros-y-otras.pdf}
\newline {\color{gray} (Emb: 0.606, TF-IDF: 0.274)}

- Establecer una política permanente de desarrollo sostenible y armónico con la naturaleza. 
\newline {\color{gray} \textbf{1º:} 1014-Iniciativa-Convencional-Constituyente-cc-Adriana-Ampuero-Haciendas-territoriales-y-autonomia-financiera.pdf}
\newline {\color{gray} (Emb: 0.738, TF-IDF: 0.723)}
\newline {\color{gray} \textbf{2º:} 99-3-c-Iniciativa-de-la-cc-Tammy-Pustilnick-Disposiciones-del-Estado-Regional.pdf}
\newline {\color{gray} (Emb: 0.701, TF-IDF: 0.696)}

La creación de empresas públicas regionales por parte de los órganos de la Región Autónoma competentes, en conformidad a los procedimientos regulados en la Constitución y la ley. 
\newline {\color{gray} \textbf{1º:} 119-3-c-Iniciativa-de-la-cc-Tammy-Pustilnick-competencias-de-las-Regiones-Autonomas.pdf}
\newline {\color{gray} (Emb: 0.829, TF-IDF: 0.537)}
\newline {\color{gray} \textbf{2º:} 118-3-c-Iniciativa-del-cc-Cristobal-Andrade-Regiones-Autonomas.pdf}
\newline {\color{gray} (Emb: 0.829, TF-IDF: 0.537)}

Establecer contribuciones y tasas dentro de su territorio previa autorización por ley. 
\newline {\color{gray} \textbf{1º:} 1014-Iniciativa-Convencional-Constituyente-cc-Adriana-Ampuero-Haciendas-territoriales-y-autonomia-financiera.pdf}
\newline {\color{gray} (Emb: 0.669, TF-IDF: 0.515)}
\newline {\color{gray} \textbf{2º:} 120-3-c-Iniciativa-de-la-cc-Tammy-Pustilnick-atribuciones-exclusivas-de-la-Asamblea-Regional.pdf}
\newline {\color{gray} (Emb: 0.644, TF-IDF: 0.435)}

Coordinar y delegar las competencias constitucionales compartidas con las demás entidades territoriales. 
\newline {\color{gray} \textbf{1º:} 119-3-c-Iniciativa-de-la-cc-Tammy-Pustilnick-competencias-de-las-Regiones-Autonomas.pdf}
\newline {\color{gray} (Emb: 1.000, TF-IDF: 1.000)}
\newline {\color{gray} \textbf{2º:} 118-3-c-Iniciativa-del-cc-Cristobal-Andrade-Regiones-Autonomas.pdf}
\newline {\color{gray} (Emb: 1.000, TF-IDF: 1.000)}

La promoción y ordenación del turismo en el ámbito territorial de la región autónoma, en coordinación con la Comuna Autónoma. 
\newline {\color{gray} \textbf{1º:} 118-3-c-Iniciativa-del-cc-Cristobal-Andrade-Regiones-Autonomas.pdf}
\newline {\color{gray} (Emb: 0.921, TF-IDF: 0.880)}
\newline {\color{gray} \textbf{2º:} 119-3-c-Iniciativa-de-la-cc-Tammy-Pustilnick-competencias-de-las-Regiones-Autonomas.pdf}
\newline {\color{gray} (Emb: 0.921, TF-IDF: 0.880)}

La regulación y administración de los bosques, las reservas y los parques de las áreas silvestres protegidas y cualquier otro predio fiscal que se considere necesario para el cuidado de los servicios ecosistémicos que se otorgan a las comunidades, en el ámbito de sus competencias. 
\newline {\color{gray} \textbf{1º:} 119-3-c-Iniciativa-de-la-cc-Tammy-Pustilnick-competencias-de-las-Regiones-Autonomas.pdf}
\newline {\color{gray} (Emb: 0.618, TF-IDF: 0.521)}
\newline {\color{gray} \textbf{2º:} 118-3-c-Iniciativa-del-cc-Cristobal-Andrade-Regiones-Autonomas.pdf}
\newline {\color{gray} (Emb: 0.618, TF-IDF: 0.521)}

La planificación e implementación de la conectividad física y digital. 
\newline {\color{gray} \textbf{1º:} 119-3-c-Iniciativa-de-la-cc-Tammy-Pustilnick-competencias-de-las-Regiones-Autonomas.pdf}
\newline {\color{gray} (Emb: 1.000, TF-IDF: 1.000)}
\newline {\color{gray} \textbf{2º:} 118-3-c-Iniciativa-del-cc-Cristobal-Andrade-Regiones-Autonomas.pdf}
\newline {\color{gray} (Emb: 1.000, TF-IDF: 1.000)}

La promoción y fomento del deporte, el ocio y la recreación. 
\newline {\color{gray} \textbf{1º:} 118-3-c-Iniciativa-del-cc-Cristobal-Andrade-Regiones-Autonomas.pdf}
\newline {\color{gray} (Emb: 1.000, TF-IDF: 1.000)}
\newline {\color{gray} \textbf{2º:} 119-3-c-Iniciativa-de-la-cc-Tammy-Pustilnick-competencias-de-las-Regiones-Autonomas.pdf}
\newline {\color{gray} (Emb: 1.000, TF-IDF: 1.000)}

La política regional de vivienda, urbanismo, salud, transporte y educación en coordinación con las políticas, planes y programas nacionales, respetando la universalidad de los derechos garantizados por esta Constitución. 
\newline {\color{gray} \textbf{1º:} 119-3-c-Iniciativa-de-la-cc-Tammy-Pustilnick-competencias-de-las-Regiones-Autonomas.pdf}
\newline {\color{gray} (Emb: 0.879, TF-IDF: 0.837)}
\newline {\color{gray} \textbf{2º:} 118-3-c-Iniciativa-del-cc-Cristobal-Andrade-Regiones-Autonomas.pdf}
\newline {\color{gray} (Emb: 0.879, TF-IDF: 0.837)}

La planificación, ordenamiento territorial y manejo integrado de cuencas. 
\newline {\color{gray} \textbf{1º:} 119-3-c-Iniciativa-de-la-cc-Tammy-Pustilnick-competencias-de-las-Regiones-Autonomas.pdf}
\newline {\color{gray} (Emb: 1.000, TF-IDF: 1.000)}
\newline {\color{gray} \textbf{2º:} 118-3-c-Iniciativa-del-cc-Cristobal-Andrade-Regiones-Autonomas.pdf}
\newline {\color{gray} (Emb: 1.000, TF-IDF: 1.000)}

El fomento y la protección de las culturas, las artes, el patrimonio histórico, inmaterial arqueológico, lingüístico y arquitectónico; y la formación artística en su territorio. 
\newline {\color{gray} \textbf{1º:} 118-3-c-Iniciativa-del-cc-Cristobal-Andrade-Regiones-Autonomas.pdf}
\newline {\color{gray} (Emb: 0.998, TF-IDF: 0.917)}
\newline {\color{gray} \textbf{2º:} 119-3-c-Iniciativa-de-la-cc-Tammy-Pustilnick-competencias-de-las-Regiones-Autonomas.pdf}
\newline {\color{gray} (Emb: 0.998, TF-IDF: 0.917)}

Aprobar, mediando procesos de participación ciudadana, los planes de descontaminación ambientales de la región autónoma. 
\newline {\color{gray} \textbf{1º:} 118-3-c-Iniciativa-del-cc-Cristobal-Andrade-Regiones-Autonomas.pdf}
\newline {\color{gray} (Emb: 0.804, TF-IDF: 0.830)}
\newline {\color{gray} \textbf{2º:} 119-3-c-Iniciativa-de-la-cc-Tammy-Pustilnick-competencias-de-las-Regiones-Autonomas.pdf}
\newline {\color{gray} (Emb: 0.767, TF-IDF: 0.801)}

La conservación, preservación, protección y restauración de la naturaleza, del equilibrio ecológico y el uso racional del agua y los demás elementos naturales de su territorio. 
\newline {\color{gray} \textbf{1º:} 119-3-c-Iniciativa-de-la-cc-Tammy-Pustilnick-competencias-de-las-Regiones-Autonomas.pdf}
\newline {\color{gray} (Emb: 1.000, TF-IDF: 1.000)}
\newline {\color{gray} \textbf{2º:} 118-3-c-Iniciativa-del-cc-Cristobal-Andrade-Regiones-Autonomas.pdf}
\newline {\color{gray} (Emb: 1.000, TF-IDF: 1.000)}

El desarrollo de la investigación, tecnología y las ciencias en materias correspondientes a la competencia regional. 
\newline {\color{gray} \textbf{1º:} 119-3-c-Iniciativa-de-la-cc-Tammy-Pustilnick-competencias-de-las-Regiones-Autonomas.pdf}
\newline {\color{gray} (Emb: 0.931, TF-IDF: 0.887)}
\newline {\color{gray} \textbf{2º:} 118-3-c-Iniciativa-del-cc-Cristobal-Andrade-Regiones-Autonomas.pdf}
\newline {\color{gray} (Emb: 0.931, TF-IDF: 0.887)}

- Participar en acciones de cooperación internacional, dentro de los marcos establecidos por los tratados y los convenios vigentes, en conformidad a los procedimientos establecidos en la Constitución y las leyes. 
\newline {\color{gray} \textbf{1º:} 118-3-c-Iniciativa-del-cc-Cristobal-Andrade-Regiones-Autonomas.pdf}
\newline {\color{gray} (Emb: 0.929, TF-IDF: 0.811)}
\newline {\color{gray} \textbf{2º:} 119-3-c-Iniciativa-de-la-cc-Tammy-Pustilnick-competencias-de-las-Regiones-Autonomas.pdf}
\newline {\color{gray} (Emb: 0.929, TF-IDF: 0.811)}

Fomentar el desarrollo social, productivo y económico de la Región autónoma en el ámbito de sus competencias, en coordinación con las políticas, planes y programas nacionales. 
\newline {\color{gray} \textbf{1º:} 119-3-c-Iniciativa-de-la-cc-Tammy-Pustilnick-competencias-de-las-Regiones-Autonomas.pdf}
\newline {\color{gray} (Emb: 1.000, TF-IDF: 1.000)}
\newline {\color{gray} \textbf{2º:} 118-3-c-Iniciativa-del-cc-Cristobal-Andrade-Regiones-Autonomas.pdf}
\newline {\color{gray} (Emb: 1.000, TF-IDF: 1.000)}

La organización político-administrativa y financiera de la Región autónoma, en función de la responsabilidad y eficiencia económica, con arreglo a la Constitución y las leyes. 
\newline {\color{gray} \textbf{1º:} 119-3-c-Iniciativa-de-la-cc-Tammy-Pustilnick-competencias-de-las-Regiones-Autonomas.pdf}
\newline {\color{gray} (Emb: 1.000, TF-IDF: 1.000)}
\newline {\color{gray} \textbf{2º:} 118-3-c-Iniciativa-del-cc-Cristobal-Andrade-Regiones-Autonomas.pdf}
\newline {\color{gray} (Emb: 1.000, TF-IDF: 1.000)}

La organización del Gobierno Regional, en conformidad con la Constitución y su Estatuto. 
\newline {\color{gray} \textbf{1º:} 118-3-c-Iniciativa-del-cc-Cristobal-Andrade-Regiones-Autonomas.pdf}
\newline {\color{gray} (Emb: 1.000, TF-IDF: 1.000)}
\newline {\color{gray} \textbf{2º:} 119-3-c-Iniciativa-de-la-cc-Tammy-Pustilnick-competencias-de-las-Regiones-Autonomas.pdf}
\newline {\color{gray} (Emb: 1.000, TF-IDF: 1.000)}

Son competencias de la Región autónoma: 1. 
\newline {\color{gray} \textbf{1º:} 119-3-c-Iniciativa-de-la-cc-Tammy-Pustilnick-competencias-de-las-Regiones-Autonomas.pdf}
\newline {\color{gray} (Emb: 1.000, TF-IDF: 1.000)}
\newline {\color{gray} \textbf{2º:} 118-3-c-Iniciativa-del-cc-Cristobal-Andrade-Regiones-Autonomas.pdf}
\newline {\color{gray} (Emb: 1.000, TF-IDF: 1.000)}

Las obras públicas de interés ejecutadas en el territorio de la región autónoma. 
\newline {\color{gray} \textbf{1º:} 118-3-c-Iniciativa-del-cc-Cristobal-Andrade-Regiones-Autonomas.pdf}
\newline {\color{gray} (Emb: 1.000, TF-IDF: 1.000)}
\newline {\color{gray} \textbf{2º:} 119-3-c-Iniciativa-de-la-cc-Tammy-Pustilnick-competencias-de-las-Regiones-Autonomas.pdf}
\newline {\color{gray} (Emb: 1.000, TF-IDF: 1.000)}


\item \textbf{Artículo} \newline
La ley determinará cuáles Servicios Públicos, Instituciones Autónomas o Empresas del Estado, en virtud de sus fines fiscalizadores, o por razones de eficiencia y de interés general, mantendrán una organización centralizada y desconcentrada en todo el territorio de la República. 
\newline {\color{gray} \textbf{1º:} 122-3-c-Iniciativa-de-la-cc-Jennifer-Mella-Forma-del-Estado.pdf}
\newline {\color{gray} (Emb: 1.000, TF-IDF: 1.000)}
\newline {\color{gray} \textbf{2º:} 159-3-c-Iniciativa-de-la-cc-Jennifer-Mella-.pdf}
\newline {\color{gray} (Emb: 1.000, TF-IDF: 1.000)}


\item \textbf{Artículo} \newline
g) Las demás que establezcan la Constitución y la ley. 
\newline {\color{gray} \textbf{1º:} 151-3-c-Iniciativa-de-la-cc-Angelica-Tepper-Competencias-de-los-Gobiernos-Regionales.pdf}
\newline {\color{gray} (Emb: 0.982, TF-IDF: 0.662)}
\newline {\color{gray} \textbf{2º:} 151-3-c-Iniciativa-de-la-cc-Angelica-Tepper-Competencias-de-los-Gobiernos-Regionales.pdf}
\newline {\color{gray} (Emb: 0.982, TF-IDF: 0.662)}

f) Acordar la creación de comisiones o grupos de trabajo para el estudio de asuntos de interés común. 
\newline {\color{gray} \textbf{1º:} 257-4-Iniciativa-Convencional-de-la-cc-Maria-Magdalena-Rivera-sobre-Derecho-a-Huelga-Solidaria-17-01-1151-hrs.pdf}
\newline {\color{gray} (Emb: 0.432, TF-IDF: 0.246)}
\newline {\color{gray} \textbf{2º:} 240-1-Iniciativa-Convencional-de-la-cc-Tania-Madriaga-sobre-Poder-Legislativo-1146-hrs.pdf}
\newline {\color{gray} (Emb: 0.414, TF-IDF: 0.235)}

Son facultades del Consejo de Gobernaciones: a) La coordinación, la complementación y la colaboración en la ejecución de políticas públicas en las Regiones; b) La coordinación económica y presupuestaria entre el Estado y las Regiones Autónomas; c) Debatir sobre las actuaciones conjuntas de carácter estratégico, que afecten a los ámbitos competenciales estatal y regional, así como velar por el respeto de las autonomías de las entidades territoriales; d) Velar por la correcta aplicación de los principios de equidad, solidaridad y justicia territorial, y de los mecanismos de compensación económica interterritorial, en conformidad con la Constitución y la ley. 
\newline {\color{gray} \textbf{1º:} 122-3-c-Iniciativa-de-la-cc-Jennifer-Mella-Forma-del-Estado.pdf}
\newline {\color{gray} (Emb: 0.759, TF-IDF: 0.532)}
\newline {\color{gray} \textbf{2º:} 159-3-c-Iniciativa-de-la-cc-Jennifer-Mella-.pdf}
\newline {\color{gray} (Emb: 0.759, TF-IDF: 0.532)}

El Consejo de Gobernaciones, presidido por el Presidente de la República y conformado por las y los Gobernadores de cada Región, coordinará las relaciones entre el Estado Central y las entidades territoriales, velando por el bienestar social y económico equilibrado de la República en su conjunto. 
\newline {\color{gray} \textbf{1º:} 122-3-c-Iniciativa-de-la-cc-Jennifer-Mella-Forma-del-Estado.pdf}
\newline {\color{gray} (Emb: 0.976, TF-IDF: 0.894)}
\newline {\color{gray} \textbf{2º:} 159-3-c-Iniciativa-de-la-cc-Jennifer-Mella-.pdf}
\newline {\color{gray} (Emb: 0.976, TF-IDF: 0.894)}

e) Convocar encuentros sectoriales entre entidades territoriales. 
\newline {\color{gray} \textbf{1º:} 196-2-c-Iniciativa-Convenciona-del-cc-Marco-Arellano-sobre-Plebiscito-e-iniciativa-de-ley-1552.pdf}
\newline {\color{gray} (Emb: 0.691, TF-IDF: 0.236)}
\newline {\color{gray} \textbf{2º:} 122-3-c-Iniciativa-de-la-cc-Jennifer-Mella-Forma-del-Estado.pdf}
\newline {\color{gray} (Emb: 0.619, TF-IDF: 0.234)}


\item \textbf{Artículo} \newline
Las Regiones Autónomas contarán con las competencias para coordinarse con las y los representantes de Ministerios y Servicios Públicos con presencia en la Región Autónoma. 
\newline {\color{gray} \textbf{1º:} 122-3-c-Iniciativa-de-la-cc-Jennifer-Mella-Forma-del-Estado.pdf}
\newline {\color{gray} (Emb: 0.763, TF-IDF: 0.553)}
\newline {\color{gray} \textbf{2º:} 159-3-c-Iniciativa-de-la-cc-Jennifer-Mella-.pdf}
\newline {\color{gray} (Emb: 0.763, TF-IDF: 0.553)}

El Gobierno Regional podrá solicitar al Estado la transferencia de competencias de Ministerios y Servicios Públicos. 
\newline {\color{gray} \textbf{1º:} 120-3-c-Iniciativa-de-la-cc-Tammy-Pustilnick-atribuciones-exclusivas-de-la-Asamblea-Regional.pdf}
\newline {\color{gray} (Emb: 0.725, TF-IDF: 0.563)}
\newline {\color{gray} \textbf{2º:} 159-3-c-Iniciativa-de-la-cc-Jennifer-Mella-.pdf}
\newline {\color{gray} (Emb: 0.699, TF-IDF: 0.440)}

A su vez, las Municipalidades podrán solicitar al Gobierno Regional la transferencia de competencias. 
\newline {\color{gray} \textbf{1º:} 120-3-c-Iniciativa-de-la-cc-Tammy-Pustilnick-atribuciones-exclusivas-de-la-Asamblea-Regional.pdf}
\newline {\color{gray} (Emb: 0.733, TF-IDF: 0.413)}
\newline {\color{gray} \textbf{2º:} 801-Iniciativa-Convencional-Constituyente-del-cc-Christian-Viera-sobre-Consejo-de-Contiendas-de-Trabajo.pdf}
\newline {\color{gray} (Emb: 0.705, TF-IDF: 0.385)}

La ley regulará este procedimiento. 
\newline {\color{gray} \textbf{1º:} 515-4-Iniciativa-Convencional-Constituyente-de-la-cc-Giovanna-Grandon-sobre-Derecho-a-Migrar-1245-01-02.pdf}
\newline {\color{gray} (Emb: 0.795, TF-IDF: 0.575)}
\newline {\color{gray} \textbf{2º:} 976-Iniciativa-Convencional-Constituye-de-la-cc-Tania-Madriaga-sobre-Puertos.pdf}
\newline {\color{gray} (Emb: 0.787, TF-IDF: 0.508)}


\item \textbf{Artículo} \newline
Concurrir, en conjunto con el Gobernador Regional, en el ejercicio de la potestad reglamentaria, en la forma prescrita por la Constitución y las leyes. 
\newline {\color{gray} \textbf{1º:} 120-3-c-Iniciativa-de-la-cc-Tammy-Pustilnick-atribuciones-exclusivas-de-la-Asamblea-Regional.pdf}
\newline {\color{gray} (Emb: 0.896, TF-IDF: 0.384)}
\newline {\color{gray} \textbf{2º:} 120-3-c-Iniciativa-de-la-cc-Tammy-Pustilnick-atribuciones-exclusivas-de-la-Asamblea-Regional.pdf}
\newline {\color{gray} (Emb: 0.829, TF-IDF: 0.355)}

Las demás atribuciones que determine la Constitución y la ley. 
\newline {\color{gray} \textbf{1º:} 466-6-Iniciativa-Convencional-Constituyente-de-la-cc-Adriana-Cancino-sobre-Defensoria-de-los-DDHH-1933-31-01.pdf}
\newline {\color{gray} (Emb: 0.964, TF-IDF: 0.855)}
\newline {\color{gray} \textbf{2º:} 394-3-Iniciativa-Convencional-Constituyente-de-la-cc-Ramona-Reyes-sobre-Comuna-Autonoma-1525-24-01.pdf}
\newline {\color{gray} (Emb: 0.947, TF-IDF: 0.783)}

Solicitar al Congreso la transferencia de la potestad legislativa en materias de interés de la Región Autónoma respectiva, en conformidad a la ley. 
\newline {\color{gray} \textbf{1º:} 120-3-c-Iniciativa-de-la-cc-Tammy-Pustilnick-atribuciones-exclusivas-de-la-Asamblea-Regional.pdf}
\newline {\color{gray} (Emb: 0.687, TF-IDF: 0.433)}
\newline {\color{gray} \textbf{2º:} 157-3-c-Iniciativa-de-la-cc-Tammy-Pustilnick-Disposiciones-Generales-del-Estado.pdf}
\newline {\color{gray} (Emb: 0.645, TF-IDF: 0.397)}

Iniciar el trámite legislativo ante el Consejo Territorial en materias de interés regional. 
\newline {\color{gray} \textbf{1º:} 325-6-Iniciativa-Convencional-del-cc-Tomas-Laibe-sobre-Jurisdiccion-Constitucional.pdf}
\newline {\color{gray} (Emb: 0.724, TF-IDF: 0.422)}
\newline {\color{gray} \textbf{2º:} 120-3-c-Iniciativa-de-la-cc-Tammy-Pustilnick-atribuciones-exclusivas-de-la-Asamblea-Regional.pdf}
\newline {\color{gray} (Emb: 0.703, TF-IDF: 0.325)}

Dictar las normas regionales que hagan aplicables las leyes de acuerdo regional. 
\newline {\color{gray} \textbf{1º:} 120-3-c-Iniciativa-de-la-cc-Tammy-Pustilnick-atribuciones-exclusivas-de-la-Asamblea-Regional.pdf}
\newline {\color{gray} (Emb: 0.808, TF-IDF: 0.297)}
\newline {\color{gray} \textbf{2º:} 325-6-Iniciativa-Convencional-del-cc-Tomas-Laibe-sobre-Jurisdiccion-Constitucional.pdf}
\newline {\color{gray} (Emb: 0.793, TF-IDF: 0.278)}

Ejercer la potestad reglamentaria de ejecución de ley cuando esta lo encomiende y dictar los demás reglamentos en materias de competencia de la región autónoma. 
\newline {\color{gray} \textbf{1º:} 631-Iniciativa-Convencional-Constituyente-de-cc-Ingrid-Villena-sobre-Contraloria-General-de-la-Republica.pdf}
\newline {\color{gray} (Emb: 0.699, TF-IDF: 0.482)}
\newline {\color{gray} \textbf{2º:} 120-3-c-Iniciativa-de-la-cc-Tammy-Pustilnick-atribuciones-exclusivas-de-la-Asamblea-Regional.pdf}
\newline {\color{gray} (Emb: 0.685, TF-IDF: 0.421)}

Aprobar, a propuesta del Gobernador o Gobernadora Regional y previa ratificación del Consejo Territorial, la creación de empresas públicas regionales o la participación en empresas regionales. 
\newline {\color{gray} \textbf{1º:} 120-3-c-Iniciativa-de-la-cc-Tammy-Pustilnick-atribuciones-exclusivas-de-la-Asamblea-Regional.pdf}
\newline {\color{gray} (Emb: 0.669, TF-IDF: 0.522)}
\newline {\color{gray} \textbf{2º:} 120-3-c-Iniciativa-de-la-cc-Tammy-Pustilnick-atribuciones-exclusivas-de-la-Asamblea-Regional.pdf}
\newline {\color{gray} (Emb: 0.655, TF-IDF: 0.479)}

Dictar su reglamento interno de funcionamiento. 
\newline {\color{gray} \textbf{1º:} 120-3-c-Iniciativa-de-la-cc-Tammy-Pustilnick-atribuciones-exclusivas-de-la-Asamblea-Regional.pdf}
\newline {\color{gray} (Emb: 1.000, TF-IDF: 1.000)}
\newline {\color{gray} \textbf{2º:} 959-1-Iniciativa-Convencional-Constituyente-de-la-cc-Rosa-Catrileo-sobre-Defensa-Plurinacional-1.pdf}
\newline {\color{gray} (Emb: 0.621, TF-IDF: 0.415)}

Aprobar, modificar o rechazar el Plan Regional de manejo integrado de cuencas. 
\newline {\color{gray} \textbf{1º:} 120-3-c-Iniciativa-de-la-cc-Tammy-Pustilnick-atribuciones-exclusivas-de-la-Asamblea-Regional.pdf}
\newline {\color{gray} (Emb: 0.778, TF-IDF: 0.759)}
\newline {\color{gray} \textbf{2º:} 119-3-c-Iniciativa-de-la-cc-Tammy-Pustilnick-competencias-de-las-Regiones-Autonomas.pdf}
\newline {\color{gray} (Emb: 0.712, TF-IDF: 0.507)}

Aprobar, modificar o rechazar el Presupuesto Regional, el Plan de Desarrollo Regional y los Planes de Ordenamiento Territorial. 
\newline {\color{gray} \textbf{1º:} 120-3-c-Iniciativa-de-la-cc-Tammy-Pustilnick-atribuciones-exclusivas-de-la-Asamblea-Regional.pdf}
\newline {\color{gray} (Emb: 0.825, TF-IDF: 0.640)}
\newline {\color{gray} \textbf{2º:} 120-3-c-Iniciativa-de-la-cc-Tammy-Pustilnick-atribuciones-exclusivas-de-la-Asamblea-Regional.pdf}
\newline {\color{gray} (Emb: 0.752, TF-IDF: 0.607)}

Solicitar al Gobernador o Gobernadora Regional rendir cuenta sobre su participación en el Consejo de Gobernaciones. 
\newline {\color{gray} \textbf{1º:} 120-3-c-Iniciativa-de-la-cc-Tammy-Pustilnick-atribuciones-exclusivas-de-la-Asamblea-Regional.pdf}
\newline {\color{gray} (Emb: 0.804, TF-IDF: 0.519)}
\newline {\color{gray} \textbf{2º:} 907-Iniciativa-Convencional-Constituyente-del-cc-Hugo-Gutierrez-Sobre-Forma-del-Estado.pdf}
\newline {\color{gray} (Emb: 0.735, TF-IDF: 0.356)}

Fiscalizar los actos de la administración regional, para lo cual podrá requerir información de autoridades o jefaturas que desempeñen sus funciones en la Región Autónoma, citar a funcionarios públicos o autoridades regionales y crear comisiones especiales. 
\newline {\color{gray} \textbf{1º:} 120-3-c-Iniciativa-de-la-cc-Tammy-Pustilnick-atribuciones-exclusivas-de-la-Asamblea-Regional.pdf}
\newline {\color{gray} (Emb: 0.924, TF-IDF: 0.954)}
\newline {\color{gray} \textbf{2º:} 120-3-c-Iniciativa-de-la-cc-Tammy-Pustilnick-atribuciones-exclusivas-de-la-Asamblea-Regional.pdf}
\newline {\color{gray} (Emb: 0.597, TF-IDF: 0.262)}

Fiscalizar los actos del Gobierno Regional de acuerdo con el procedimiento establecido en el Estatuto Regional. 
\newline {\color{gray} \textbf{1º:} 120-3-c-Iniciativa-de-la-cc-Tammy-Pustilnick-atribuciones-exclusivas-de-la-Asamblea-Regional.pdf}
\newline {\color{gray} (Emb: 1.000, TF-IDF: 1.000)}
\newline {\color{gray} \textbf{2º:} 120-3-c-Iniciativa-de-la-cc-Tammy-Pustilnick-atribuciones-exclusivas-de-la-Asamblea-Regional.pdf}
\newline {\color{gray} (Emb: 0.790, TF-IDF: 0.587)}

Son atribuciones de la Asamblea Regional, en conformidad a la Constitución, la ley y el Estatuto Regional: 1. 
\newline {\color{gray} \textbf{1º:} 871-Iniciativa-Convencional-Constituyente-de-la-cc-Amaya-Alvez-sobre-Asambleas-Regionales.pdf}
\newline {\color{gray} (Emb: 0.884, TF-IDF: 0.616)}
\newline {\color{gray} \textbf{2º:} 871-Iniciativa-Convencional-Constituyente-de-la-cc-Amaya-Alvez-sobre-Asambleas-Regionales.pdf}
\newline {\color{gray} (Emb: 0.810, TF-IDF: 0.598)}


\item \textbf{Artículo} \newline
Las demás atribuciones que señalen la Constitución, el Estatuto Regional y las leyes. 
\newline {\color{gray} \textbf{1º:} 120-3-c-Iniciativa-de-la-cc-Tammy-Pustilnick-atribuciones-exclusivas-de-la-Asamblea-Regional.pdf}
\newline {\color{gray} (Emb: 1.000, TF-IDF: 1.000)}
\newline {\color{gray} \textbf{2º:} 120-3-c-Iniciativa-de-la-cc-Tammy-Pustilnick-atribuciones-exclusivas-de-la-Asamblea-Regional.pdf}
\newline {\color{gray} (Emb: 0.933, TF-IDF: 0.390)}

Establecer sistemas de gestión de crisis entre los órganos que tienen asiento en la Región Autónoma, que incluyan, a lo menos, su preparación, prevención, administración y manejo. 
\newline {\color{gray} \textbf{1º:} 118-3-c-Iniciativa-del-cc-Cristobal-Andrade-Regiones-Autonomas.pdf}
\newline {\color{gray} (Emb: 0.507, TF-IDF: 0.265)}
\newline {\color{gray} \textbf{2º:} 119-3-c-Iniciativa-de-la-cc-Tammy-Pustilnick-competencias-de-las-Regiones-Autonomas.pdf}
\newline {\color{gray} (Emb: 0.507, TF-IDF: 0.265)}

Convocar a referéndum y plebiscitos regionales en virtud de lo previsto en la Constitución, el Estatuto Regional y la ley. 
\newline {\color{gray} \textbf{1º:} 120-3-c-Iniciativa-de-la-cc-Tammy-Pustilnick-atribuciones-exclusivas-de-la-Asamblea-Regional.pdf}
\newline {\color{gray} (Emb: 1.000, TF-IDF: 1.000)}
\newline {\color{gray} \textbf{2º:} 120-3-c-Iniciativa-de-la-cc-Tammy-Pustilnick-atribuciones-exclusivas-de-la-Asamblea-Regional.pdf}
\newline {\color{gray} (Emb: 0.862, TF-IDF: 0.334)}

Promover la innovación, la competitividad y la inversión en la respectiva región autónoma. 
\newline {\color{gray} \textbf{1º:} 151-3-c-Iniciativa-de-la-cc-Angelica-Tepper-Competencias-de-los-Gobiernos-Regionales.pdf}
\newline {\color{gray} (Emb: 0.944, TF-IDF: 0.945)}
\newline {\color{gray} \textbf{2º:} 151-3-c-Iniciativa-de-la-cc-Angelica-Tepper-Competencias-de-los-Gobiernos-Regionales.pdf}
\newline {\color{gray} (Emb: 0.840, TF-IDF: 0.831)}

Celebrar y ejecutar acciones de cooperación internacional, dentro de los marcos establecidos por los tratados y convenios que el país celebre al efecto y en conformidad a los procedimientos regulados en la ley. 
\newline {\color{gray} \textbf{1º:} 120-3-c-Iniciativa-de-la-cc-Tammy-Pustilnick-atribuciones-exclusivas-de-la-Asamblea-Regional.pdf}
\newline {\color{gray} (Emb: 1.000, TF-IDF: 1.000)}
\newline {\color{gray} \textbf{2º:} 118-3-c-Iniciativa-del-cc-Cristobal-Andrade-Regiones-Autonomas.pdf}
\newline {\color{gray} (Emb: 0.878, TF-IDF: 0.662)}

Celebrar y ejecutar convenios con los Gobiernos de otras regiones autónomas para efectos de implementar programas y políticas públicas interregionales, así como toda otra forma de asociatividad territorial. 
\newline {\color{gray} \textbf{1º:} 120-3-c-Iniciativa-de-la-cc-Tammy-Pustilnick-atribuciones-exclusivas-de-la-Asamblea-Regional.pdf}
\newline {\color{gray} (Emb: 1.000, TF-IDF: 1.000)}
\newline {\color{gray} \textbf{2º:} 117-3-c-Iniciativa-del-cc-Bastian-Labbe-sobre-Asamblea-Social-Regional.pdf}
\newline {\color{gray} (Emb: 0.622, TF-IDF: 0.307)}

Proponer a la Asamblea Regional la creación de empresas públicas regionales o la participación en empresas regionales para la gestión de servicios de su competencia, según lo dispuesto en la Constitución, la ley y el Estatuto Regional. 
\newline {\color{gray} \textbf{1º:} 120-3-c-Iniciativa-de-la-cc-Tammy-Pustilnick-atribuciones-exclusivas-de-la-Asamblea-Regional.pdf}
\newline {\color{gray} (Emb: 0.976, TF-IDF: 0.946)}
\newline {\color{gray} \textbf{2º:} 120-3-c-Iniciativa-de-la-cc-Tammy-Pustilnick-atribuciones-exclusivas-de-la-Asamblea-Regional.pdf}
\newline {\color{gray} (Emb: 0.942, TF-IDF: 0.857)}

Organizar, administrar, supervigilar y fiscalizar los servicios públicos de la Región Autónoma y coordinarse con el Gobierno respecto de aquellos que detenten un carácter nacional y que funcionen en la Región. 
\newline {\color{gray} \textbf{1º:} 120-3-c-Iniciativa-de-la-cc-Tammy-Pustilnick-atribuciones-exclusivas-de-la-Asamblea-Regional.pdf}
\newline {\color{gray} (Emb: 1.000, TF-IDF: 1.000)}
\newline {\color{gray} \textbf{2º:} 633-3-Iniciativa-Convencional-Constituyente-de-la-Yarela-Gomez-sobre-Regimen-Tributario-1739-01-02.pdf}
\newline {\color{gray} (Emb: 0.652, TF-IDF: 0.334)}

Ejercer la potestad reglamentaria en todas aquellas materias que se encuentren dentro del ámbito de sus competencias, en conformidad a la Constitución, la ley y el Estatuto Regional. 
\newline {\color{gray} \textbf{1º:} 120-3-c-Iniciativa-de-la-cc-Tammy-Pustilnick-atribuciones-exclusivas-de-la-Asamblea-Regional.pdf}
\newline {\color{gray} (Emb: 0.830, TF-IDF: 0.885)}
\newline {\color{gray} \textbf{2º:} 120-3-c-Iniciativa-de-la-cc-Tammy-Pustilnick-atribuciones-exclusivas-de-la-Asamblea-Regional.pdf}
\newline {\color{gray} (Emb: 0.784, TF-IDF: 0.885)}

Preparar y presentar ante la Asamblea Regional el plan regional de ordenamiento territorial, los planes de desarrollo urbano de las áreas metropolitanas y los planes de manejo integrado de cuencas, en conformidad al Estatuto Regional y la ley. 
\newline {\color{gray} \textbf{1º:} 120-3-c-Iniciativa-de-la-cc-Tammy-Pustilnick-atribuciones-exclusivas-de-la-Asamblea-Regional.pdf}
\newline {\color{gray} (Emb: 0.877, TF-IDF: 0.817)}
\newline {\color{gray} \textbf{2º:} 120-3-c-Iniciativa-de-la-cc-Tammy-Pustilnick-atribuciones-exclusivas-de-la-Asamblea-Regional.pdf}
\newline {\color{gray} (Emb: 0.828, TF-IDF: 0.680)}

Administrar y ejecutar el Presupuesto Regional, realizar actos y contratos en los que tenga interés, ejercer competencias fiscales propias conforme a la ley, y elaborar la planificación presupuestaria sobre la destinación y uso del presupuesto regional. 
\newline {\color{gray} \textbf{1º:} 120-3-c-Iniciativa-de-la-cc-Tammy-Pustilnick-atribuciones-exclusivas-de-la-Asamblea-Regional.pdf}
\newline {\color{gray} (Emb: 1.000, TF-IDF: 1.000)}
\newline {\color{gray} \textbf{2º:} 120-3-c-Iniciativa-de-la-cc-Tammy-Pustilnick-atribuciones-exclusivas-de-la-Asamblea-Regional.pdf}
\newline {\color{gray} (Emb: 0.611, TF-IDF: 0.397)}

Preparar y presentar ante la Asamblea Regional el proyecto de Presupuesto Regional, en conformidad a esta Constitución y el Estatuto Regional. 
\newline {\color{gray} \textbf{1º:} 120-3-c-Iniciativa-de-la-cc-Tammy-Pustilnick-atribuciones-exclusivas-de-la-Asamblea-Regional.pdf}
\newline {\color{gray} (Emb: 0.962, TF-IDF: 0.960)}
\newline {\color{gray} \textbf{2º:} 120-3-c-Iniciativa-de-la-cc-Tammy-Pustilnick-atribuciones-exclusivas-de-la-Asamblea-Regional.pdf}
\newline {\color{gray} (Emb: 0.722, TF-IDF: 0.799)}

Preparar y presentar ante la Asamblea Regional el Plan de Desarrollo Regional, en conformidad al Estatuto Regional. 
\newline {\color{gray} \textbf{1º:} 120-3-c-Iniciativa-de-la-cc-Tammy-Pustilnick-atribuciones-exclusivas-de-la-Asamblea-Regional.pdf}
\newline {\color{gray} (Emb: 0.943, TF-IDF: 0.958)}
\newline {\color{gray} \textbf{2º:} 120-3-c-Iniciativa-de-la-cc-Tammy-Pustilnick-atribuciones-exclusivas-de-la-Asamblea-Regional.pdf}
\newline {\color{gray} (Emb: 0.763, TF-IDF: 0.800)}

Son atribuciones exclusivas de los Gobiernos Regionales las siguientes: 1. 
\newline {\color{gray} \textbf{1º:} 120-3-c-Iniciativa-de-la-cc-Tammy-Pustilnick-atribuciones-exclusivas-de-la-Asamblea-Regional.pdf}
\newline {\color{gray} (Emb: 1.000, TF-IDF: 1.000)}
\newline {\color{gray} \textbf{2º:} 151-3-c-Iniciativa-de-la-cc-Angelica-Tepper-Competencias-de-los-Gobiernos-Regionales.pdf}
\newline {\color{gray} (Emb: 0.789, TF-IDF: 0.655)}

Adoptar e implementar políticas públicas que fomenten y promocionen el desarrollo social, productivo, económico y cultural de la región autónoma, especialmente en ámbitos de competencia de la región autónoma. 
\newline {\color{gray} \textbf{1º:} 120-3-c-Iniciativa-de-la-cc-Tammy-Pustilnick-atribuciones-exclusivas-de-la-Asamblea-Regional.pdf}
\newline {\color{gray} (Emb: 1.000, TF-IDF: 1.000)}
\newline {\color{gray} \textbf{2º:} 119-3-c-Iniciativa-de-la-cc-Tammy-Pustilnick-competencias-de-las-Regiones-Autonomas.pdf}
\newline {\color{gray} (Emb: 0.676, TF-IDF: 0.507)}


\item \textbf{Artículo} \newline
(Inciso tercero) La ley clasificará las comunas en distintos tipos, las que deberán ser consideradas por los órganos del Estado para el establecimiento de regímenes administrativos y económico-fiscales diferenciados, la implementación de políticas, planes y programas atendiendo a las diversas realidades locales, y en especial, para el traspaso de competencias y recursos. 
\newline {\color{gray} \textbf{1º:} 385-3-Iniciativa-Convencional-Constituyente-del-cc-Felipe-Mena-sobre-Municipalidades-1200-24-01.pdf}
\newline {\color{gray} (Emb: 0.724, TF-IDF: 0.437)}
\newline {\color{gray} \textbf{2º:} 154-3-c-Iniciativa-del-cc-Felipe-Mena-sobre-Organizacion-Territorial-del-Estado.pdf}
\newline {\color{gray} (Emb: 0.658, TF-IDF: 0.437)}

La comuna autónoma es la entidad territorial base del Estado regional, dotada de personalidad jurídica de derecho público y patrimonio propio, que goza de autonomía para el cumplimiento de sus fines y el ejercicio de sus competencias, con arreglo a lo dispuesto en la Constitución y la ley. 
\newline {\color{gray} \textbf{1º:} 394-3-Iniciativa-Convencional-Constituyente-de-la-cc-Ramona-Reyes-sobre-Comuna-Autonoma-1525-24-01.pdf}
\newline {\color{gray} (Emb: 0.831, TF-IDF: 0.508)}
\newline {\color{gray} \textbf{2º:} 43-3-Iniciativa-Convencional-Constituyente-de-la-cc-Tammy-Pustilnick-y-otros.pdf}
\newline {\color{gray} (Emb: 0.813, TF-IDF: 0.383)}

El establecimiento de los tipos comunales deberá considerar, a lo menos, criterios demográficos, económicos, culturales, geográficos, socioambientales, urbanos y rurales. 
\newline {\color{gray} \textbf{1º:} 385-3-Iniciativa-Convencional-Constituyente-del-cc-Felipe-Mena-sobre-Municipalidades-1200-24-01.pdf}
\newline {\color{gray} (Emb: 0.843, TF-IDF: 0.690)}
\newline {\color{gray} \textbf{2º:} 394-3-Iniciativa-Convencional-Constituyente-de-la-cc-Ramona-Reyes-sobre-Comuna-Autonoma-1525-24-01.pdf}
\newline {\color{gray} (Emb: 0.657, TF-IDF: 0.445)}


\item \textbf{Artículo} \newline
Para el gobierno comunal se observará como principio básico la búsqueda de un desarrollo territorial armónico y equitativo, propendiendo a que todas las personas tengan acceso a igual nivel y calidad de servicios públicos municipales, sin distingo del lugar que habiten. 
\newline {\color{gray} \textbf{1º:} 931-Iniciativa-Convencional-Constituyente-del-cc-Felipe-Mena-sobre-Descentralizacion-Fiscal.pdf}
\newline {\color{gray} (Emb: 0.670, TF-IDF: 0.280)}
\newline {\color{gray} \textbf{2º:} 344-3-Iniciativa-Convencional-Constituyente-del-cc-Hernan-Larrain-sobre-Reforma-Administrativa-y-Modernizacion-del-Estado.pdf}
\newline {\color{gray} (Emb: 0.617, TF-IDF: 0.242)}

El Estado garantizará a la municipalidad el financiamiento y recursos suficientes, para el justo y equitativo desarrollo de cada comuna, conforme a los mecanismos que señale la Constitución y la ley. 
\newline {\color{gray} \textbf{1º:} 753-Iniciativa-Convencional-Constituyente-del-cc-Raul-Celis-sobre-Gobiernos-Locales.pdf}
\newline {\color{gray} (Emb: 0.623, TF-IDF: 0.349)}
\newline {\color{gray} \textbf{2º:} 753-Iniciativa-Convencional-Constituyente-del-cc-Raul-Celis-sobre-Gobiernos-Locales.pdf}
\newline {\color{gray} (Emb: 0.600, TF-IDF: 0.325)}


\item \textbf{Artículo} \newline
La creación, división o fusión de comunas autónomas, o la modificación de sus límites o denominación, se determinará por ley, respetando en todo caso criterios objetivos, según lo dispuesto en el artículo 2 de esta Constitución. 
\newline {\color{gray} \textbf{1º:} 397-3-Iniciativa-Convencional-Constituyente-del-cc-Cesar-Uribe-sobre-Gobiernos-Locales-1722-24-01.pdf}
\newline {\color{gray} (Emb: 0.703, TF-IDF: 0.720)}
\newline {\color{gray} \textbf{2º:} 656-Iniciativa-Convencional-Constituyente-de-la-cc-Lisette-Vergara-sobre-Unidades-Vecinales-121101-02.pdf}
\newline {\color{gray} (Emb: 0.608, TF-IDF: 0.278)}

Una ley regulará la administración transitoria de las comunas que se creen, el procedimiento de instalación de las nuevas municipalidades, de traspaso del personal municipal y de los servicios y los resguardos necesarios para cautelar el uso y disposición de los bienes que se encuentren situados en los territorios de las nuevas comunas. 
\newline {\color{gray} \textbf{1º:} 385-3-Iniciativa-Convencional-Constituyente-del-cc-Felipe-Mena-sobre-Municipalidades-1200-24-01.pdf}
\newline {\color{gray} (Emb: 0.596, TF-IDF: 0.269)}
\newline {\color{gray} \textbf{2º:} 931-Iniciativa-Convencional-Constituyente-del-cc-Felipe-Mena-sobre-Descentralizacion-Fiscal.pdf}
\newline {\color{gray} (Emb: 0.545, TF-IDF: 0.255)}


\item \textbf{Artículo} \newline
En los términos que establezca la ley, las regiones y comunas autónomas ubicadas en zonas fronterizas podrán vincularse con las entidades territoriales limítrofes del país vecino, de igual nivel, a través de sus respectivas autoridades, para establecer programas de cooperación e integración, dirigidos a fomentar el desarrollo comunitario, la prestación de servicios públicos y la conservación del medio ambiente. 
\newline {\color{gray} \textbf{1º:} 397-3-Iniciativa-Convencional-Constituyente-del-cc-Cesar-Uribe-sobre-Gobiernos-Locales-1722-24-01.pdf}
\newline {\color{gray} (Emb: 1.000, TF-IDF: 1.000)}
\newline {\color{gray} \textbf{2º:} 898-Iniciativa-Convencional-Constituyente-del-cc-Cesar-Uribe-sobre-Ordenamiento-y-Planificacion.pdf}
\newline {\color{gray} (Emb: 0.641, TF-IDF: 0.261)}


\item \textbf{Artículo} \newline
Las municipalidades, para el cumplimiento de sus funciones, podrán establecer sus plantas de personal y los órganos o unidades de su estructura interna, en conformidad a la Constitución y la ley. 
\newline {\color{gray} \textbf{1º:} 712-Iniciativa-Convencional-Constituyente-de-la-cc-Lisette-Vergara-sobre-Administracion-Comunal.pdf}
\newline {\color{gray} (Emb: 0.664, TF-IDF: 0.353)}
\newline {\color{gray} \textbf{2º:} 753-Iniciativa-Convencional-Constituyente-del-cc-Raul-Celis-sobre-Gobiernos-Locales.pdf}
\newline {\color{gray} (Emb: 0.659, TF-IDF: 0.340)}

Estas facultades se ejercerán cautelando su debido financiamiento y el carácter técnico y profesional de dichos empleos. 
\newline {\color{gray} \textbf{1º:} 394-3-Iniciativa-Convencional-Constituyente-de-la-cc-Ramona-Reyes-sobre-Comuna-Autonoma-1525-24-01.pdf}
\newline {\color{gray} (Emb: 0.807, TF-IDF: 0.893)}
\newline {\color{gray} \textbf{2º:} 127-4-c-Iniciativa-de-la-cc-Rocio-Cantuarias-Derecho-a-la-eduacion-y-libertad-de-ensenanza.pdf}
\newline {\color{gray} (Emb: 0.580, TF-IDF: 0.259)}


\item \textbf{Artículo} \newline
Las municipalidades tienen el deber de promover y garantizar la participación ciudadana de la comunidad local en la gestión, en la construcción de políticas de desarrollo local y en la planificación del territorio, así como en los casos que esta Constitución, la ley y los estatutos regionales o comunales señalen. 
\newline {\color{gray} \textbf{1º:} 899-Iniciativa-Convencional-Constituyente-del-cc-Cesar-Uribe-Sobre-Participacion-Ciudadana.pdf}
\newline {\color{gray} (Emb: 0.808, TF-IDF: 0.522)}
\newline {\color{gray} \textbf{2º:} 924-Iniciativa-Convencional-Constituyente-de-la-cc-Constanza-Schonhaut-sobre-Administracion-del-Estado.pdf}
\newline {\color{gray} (Emb: 0.670, TF-IDF: 0.384)}

Las municipalidades proveerán los mecanismos, espacios, recursos, alfabetización digital, formación y educación cívica y todo aquello que sea necesario para concretar dicha participación que será consultiva, incidente y/o vinculante de acuerdo a la legislación respectiva. 
\newline {\color{gray} \textbf{1º:} 899-Iniciativa-Convencional-Constituyente-del-cc-Cesar-Uribe-Sobre-Participacion-Ciudadana.pdf}
\newline {\color{gray} (Emb: 0.942, TF-IDF: 0.981)}
\newline {\color{gray} \textbf{2º:} 899-Iniciativa-Convencional-Constituyente-del-cc-Cesar-Uribe-Sobre-Participacion-Ciudadana.pdf}
\newline {\color{gray} (Emb: 0.485, TF-IDF: 0.254)}


\item \textbf{Artículo} \newline
El gobierno de la comuna autónoma reside en la municipalidad, la que estará constituida por el alcalde o alcaldesa y el concejo municipal, con la participación de la comunidad que habita en su territorio. 
\newline {\color{gray} \textbf{1º:} 394-3-Iniciativa-Convencional-Constituyente-de-la-cc-Ramona-Reyes-sobre-Comuna-Autonoma-1525-24-01.pdf}
\newline {\color{gray} (Emb: 0.820, TF-IDF: 0.621)}
\newline {\color{gray} \textbf{2º:} 656-Iniciativa-Convencional-Constituyente-de-la-cc-Lisette-Vergara-sobre-Unidades-Vecinales-121101-02.pdf}
\newline {\color{gray} (Emb: 0.768, TF-IDF: 0.592)}


\item \textbf{Artículo} \newline
Los concejales o concejalas ejercerán sus funciones por el término de cuatro años, y podrán ser reelegidos o reelegidas consecutivamente sólo una vez para el período siguiente. 
\newline {\color{gray} \textbf{1º:} 513-3-Iniciativa-Convencional-Constituyente-del-cc-Adolfo-Millabur-sobre-gobiernos-locales-1115-01-02.pdf}
\newline {\color{gray} (Emb: 0.871, TF-IDF: 0.705)}
\newline {\color{gray} \textbf{2º:} 730-Iniciativa-Convencional-Constituyente-del-cc-Adolfo-Millabur-sobre-Gobiernos-Locales.pdf}
\newline {\color{gray} (Emb: 0.871, TF-IDF: 0.512)}

Será igualmente necesario el acuerdo del Concejo para la aprobación del plan regulador comunal. 
\newline {\color{gray} \textbf{1º:} 374-2-Iniciativa-Convencional-Constituyente-de-la-cc-Loreto-Vallejos-sobre-Democracia-Directa-0900-hrs-24-01.pdf}
\newline {\color{gray} (Emb: 0.602, TF-IDF: 0.474)}
\newline {\color{gray} \textbf{2º:} 652-Iniciativa-Convencional-Constituyente-de-la-cc-Ericka-Portilla-sobre-Trabajo-Decente-151101-02.pdf}
\newline {\color{gray} (Emb: 0.531, TF-IDF: 0.422)}

La ley establecerá un régimen de inhabilidades e incompatibilidades. 
\newline {\color{gray} \textbf{1º:} 915-Iniciativa-Convencional-Constituyente-del-cc-Luis-Jimenez-sobre-Plurinacionalidad.pdf}
\newline {\color{gray} (Emb: 0.837, TF-IDF: 0.718)}
\newline {\color{gray} \textbf{2º:} 472-6-Iniciativa-Convencional-Constituyente-del-cc-Daniel-Bravo-sobre-Corte-Constitucional-2003-31-01.pdf}
\newline {\color{gray} (Emb: 0.768, TF-IDF: 0.694)}

Los Concejales y Concejalas dispondrán de las condiciones y recursos necesarios para el desempeño eficiente y probo del cargo. 
\newline {\color{gray} \textbf{1º:} 179-6-c-Iniciativa-Convencional-DEL-CC-Rodrigo-Álvarez-Ministerio-Público-1044-hrs.pdf}
\newline {\color{gray} (Emb: 0.612, TF-IDF: 0.347)}
\newline {\color{gray} \textbf{2º:} 892-Iniciativa-Convencional-Constituyente-del-cc-Benito-Baranda-Sobre-Deberes-y-Garantias-del-Contribuyente.pdf}
\newline {\color{gray} (Emb: 0.598, TF-IDF: 0.297)}

Será necesario el acuerdo del concejo para la aprobación del plan comunal de desarrollo, del presupuesto municipal y de los proyectos de inversión respectivos, y otros que determine la ley. 
\newline {\color{gray} \textbf{1º:} 385-3-Iniciativa-Convencional-Constituyente-del-cc-Felipe-Mena-sobre-Municipalidades-1200-24-01.pdf}
\newline {\color{gray} (Emb: 0.849, TF-IDF: 0.834)}
\newline {\color{gray} \textbf{2º:} 385-3-Iniciativa-Convencional-Constituyente-del-cc-Felipe-Mena-sobre-Municipalidades-1200-24-01.pdf}
\newline {\color{gray} (Emb: 0.635, TF-IDF: 0.381)}

La elección de concejales y concejalas será por sufragio universal, directo y secreto, en conformidad a la ley. 
\newline {\color{gray} \textbf{1º:} 394-3-Iniciativa-Convencional-Constituyente-de-la-cc-Ramona-Reyes-sobre-Comuna-Autonoma-1525-24-01.pdf}
\newline {\color{gray} (Emb: 1.000, TF-IDF: 1.000)}
\newline {\color{gray} \textbf{2º:} 730-Iniciativa-Convencional-Constituyente-del-cc-Adolfo-Millabur-sobre-Gobiernos-Locales.pdf}
\newline {\color{gray} (Emb: 0.792, TF-IDF: 0.689)}

Para estos efectos se entenderá que los concejales y concejalas han ejercido su cargo durante un período cuando hayan cumplido más de la mitad de su mandato. 
\newline {\color{gray} \textbf{1º:} 400-1-Iniciativa-Convencional-Constituyente-de-la-cc-Constanza-Hube-sobre-Servicio-y-Registro-Electoral-1905-24-01.pdf}
\newline {\color{gray} (Emb: 0.899, TF-IDF: 0.777)}
\newline {\color{gray} \textbf{2º:} 98-6-Iniciativa-del-cc-Ruggero-Cozzi-Funcion-y-Principios-de-la-Jurisdiccion.pdf}
\newline {\color{gray} (Emb: 0.496, TF-IDF: 0.296)}

El concejo municipal estará integrado por el número de personas que determine la ley, en proporción a la población de la comuna, según los criterios de inclusión, paridad de género y escaños reservados para pueblos y naciones indígenas considerando su población dentro de la jurisdicción electoral respectiva. 
\newline {\color{gray} \textbf{1º:} 394-3-Iniciativa-Convencional-Constituyente-de-la-cc-Ramona-Reyes-sobre-Comuna-Autonoma-1525-24-01.pdf}
\newline {\color{gray} (Emb: 0.895, TF-IDF: 0.656)}
\newline {\color{gray} \textbf{2º:} 899-Iniciativa-Convencional-Constituyente-del-cc-Cesar-Uribe-Sobre-Participacion-Ciudadana.pdf}
\newline {\color{gray} (Emb: 0.841, TF-IDF: 0.538)}

El concejo municipal es el órgano colegiado de representación popular y vecinal, dotado de funciones normativas, resolutivas y fiscalizadoras, en conformidad a la Constitución y la ley. 
\newline {\color{gray} \textbf{1º:} 394-3-Iniciativa-Convencional-Constituyente-de-la-cc-Ramona-Reyes-sobre-Comuna-Autonoma-1525-24-01.pdf}
\newline {\color{gray} (Emb: 0.996, TF-IDF: 1.000)}
\newline {\color{gray} \textbf{2º:} 753-Iniciativa-Convencional-Constituyente-del-cc-Raul-Celis-sobre-Gobiernos-Locales.pdf}
\newline {\color{gray} (Emb: 0.683, TF-IDF: 0.319)}

La ley y el estatuto comunal determinarán las normas sobre organización y funcionamiento del concejo. 
\newline {\color{gray} \textbf{1º:} 325-6-Iniciativa-Convencional-del-cc-Tomas-Laibe-sobre-Jurisdiccion-Constitucional.pdf}
\newline {\color{gray} (Emb: 0.738, TF-IDF: 0.458)}
\newline {\color{gray} \textbf{2º:} 88-6-Iniciativa-Convencional-Constituyente-del-cc-Christian-Viera-y-otros.pdf}
\newline {\color{gray} (Emb: 0.726, TF-IDF: 0.428)}


\item \textbf{Artículo} \newline
El alcalde o alcaldesa será elegido en votación directa, en conformidad con lo dispuesto en la Constitución y la ley. 
\newline {\color{gray} \textbf{1º:} 513-3-Iniciativa-Convencional-Constituyente-del-cc-Adolfo-Millabur-sobre-gobiernos-locales-1115-01-02.pdf}
\newline {\color{gray} (Emb: 0.775, TF-IDF: 0.438)}
\newline {\color{gray} \textbf{2º:} 730-Iniciativa-Convencional-Constituyente-del-cc-Adolfo-Millabur-sobre-Gobiernos-Locales.pdf}
\newline {\color{gray} (Emb: 0.775, TF-IDF: 0.431)}

El alcalde o alcaldesa es la máxima autoridad ejecutiva del gobierno comunal, integra el concejo municipal y representa judicial y extrajudicialmente a la comuna. 
\newline {\color{gray} \textbf{1º:} 730-Iniciativa-Convencional-Constituyente-del-cc-Adolfo-Millabur-sobre-Gobiernos-Locales.pdf}
\newline {\color{gray} (Emb: 0.805, TF-IDF: 0.514)}
\newline {\color{gray} \textbf{2º:} 513-3-Iniciativa-Convencional-Constituyente-del-cc-Adolfo-Millabur-sobre-gobiernos-locales-1115-01-02.pdf}
\newline {\color{gray} (Emb: 0.805, TF-IDF: 0.514)}

El alcalde o alcaldesa ejercerá sus funciones por el término de cuatros años, pudiendo ser reelegido o reelegida consecutivamente sólo una vez para el período siguiente. 
\newline {\color{gray} \textbf{1º:} 384-3-Iniciativa-Convencional-Constituyente-del-cc-Felipe-Mena-sobre-Gobiernos-Regionales-1159-24-01.pdf}
\newline {\color{gray} (Emb: 0.811, TF-IDF: 0.668)}
\newline {\color{gray} \textbf{2º:} 216-1-c-Iniciativa-Convencional-de-la-cc-Rosa-Catrileo-sobre-Sistema-de-Gobierno-2126-hrs.pdf}
\newline {\color{gray} (Emb: 0.806, TF-IDF: 0.482)}

Para estos efectos se entenderá que el alcalde o alcaldesa ha ejercido su cargo durante un período cuando haya cumplido más de la mitad de su mandato. 
\newline {\color{gray} \textbf{1º:} 400-1-Iniciativa-Convencional-Constituyente-de-la-cc-Constanza-Hube-sobre-Servicio-y-Registro-Electoral-1905-24-01.pdf}
\newline {\color{gray} (Emb: 0.750, TF-IDF: 0.646)}
\newline {\color{gray} \textbf{2º:} 234-1-Iniciativa-Convencional-del-cc-Jaime-Bassa-sobre-Justicia-Complementaria-1144-hrs.pdf}
\newline {\color{gray} (Emb: 0.523, TF-IDF: 0.284)}

El alcalde o alcaldesa ejercerá la presidencia del concejo municipal. 
\newline {\color{gray} \textbf{1º:} 377-2-Iniciativa-Convencional-Constituyente-del-cc-Alvin-Saldana-sobre-Mecanismos-de-Democracia-1045-hrs-24-01.pdf}
\newline {\color{gray} (Emb: 0.748, TF-IDF: 0.572)}
\newline {\color{gray} \textbf{2º:} 801-Iniciativa-Convencional-Constituyente-del-cc-Christian-Viera-sobre-Consejo-de-Contiendas-de-Trabajo.pdf}
\newline {\color{gray} (Emb: 0.729, TF-IDF: 0.387)}


\item \textbf{Artículo} \newline
El alcalde o alcaldesa, con aprobación del Concejo Municipal, podrá establecer delegaciones para el ejercicio de las facultades de la comuna autónoma en los casos y formas que determine la ley. 
\newline {\color{gray} \textbf{1º:} 385-3-Iniciativa-Convencional-Constituyente-del-cc-Felipe-Mena-sobre-Municipalidades-1200-24-01.pdf}
\newline {\color{gray} (Emb: 0.827, TF-IDF: 0.499)}
\newline {\color{gray} \textbf{2º:} 753-Iniciativa-Convencional-Constituyente-del-cc-Raul-Celis-sobre-Gobiernos-Locales.pdf}
\newline {\color{gray} (Emb: 0.684, TF-IDF: 0.478)}


\item \textbf{Artículo} \newline
La ley dispondrá la forma de determinar el territorio de las unidades vecinales, el procedimiento de constitución de las juntas vecinales y uniones comunales y sus atribuciones. 
\newline {\color{gray} \textbf{1º:} 938-Iniciativa-Convencional-Constituyente-de-la-cc-Elisa-Giustinianovich-sobre-Juntas-de-Vecinos.pdf}
\newline {\color{gray} (Emb: 0.686, TF-IDF: 0.737)}
\newline {\color{gray} \textbf{2º:} 753-Iniciativa-Convencional-Constituyente-del-cc-Raul-Celis-sobre-Gobiernos-Locales.pdf}
\newline {\color{gray} (Emb: 0.660, TF-IDF: 0.319)}

En Comunas Autónomas con población rural, podrá constituirse además una Unión Comunal de Juntas Vecinales de carácter rural. 
\newline {\color{gray} \textbf{1º:} 938-Iniciativa-Convencional-Constituyente-de-la-cc-Elisa-Giustinianovich-sobre-Juntas-de-Vecinos.pdf}
\newline {\color{gray} (Emb: 0.706, TF-IDF: 0.488)}
\newline {\color{gray} \textbf{2º:} 656-Iniciativa-Convencional-Constituyente-de-la-cc-Lisette-Vergara-sobre-Unidades-Vecinales-121101-02.pdf}
\newline {\color{gray} (Emb: 0.641, TF-IDF: 0.473)}

Las comunas autónomas establecerán en el ámbito de sus competencias, territorios denominados unidades vecinales. 
\newline {\color{gray} \textbf{1º:} 656-Iniciativa-Convencional-Constituyente-de-la-cc-Lisette-Vergara-sobre-Unidades-Vecinales-121101-02.pdf}
\newline {\color{gray} (Emb: 0.738, TF-IDF: 0.582)}
\newline {\color{gray} \textbf{2º:} 394-3-Iniciativa-Convencional-Constituyente-de-la-cc-Ramona-Reyes-sobre-Comuna-Autonoma-1525-24-01.pdf}
\newline {\color{gray} (Emb: 0.717, TF-IDF: 0.518)}

Dentro de ellas, se constituirá una junta vecinal, representativa de las personas que residen en una misma unidad vecinal, que gozará de personalidad jurídica y será sin fines de lucro, cuyo objeto será hacer efectiva la participación popular en la gestión comunal y en el desarrollo de la comunidad, y las demás atribuciones que determine la ley. 
\newline {\color{gray} \textbf{1º:} 938-Iniciativa-Convencional-Constituyente-de-la-cc-Elisa-Giustinianovich-sobre-Juntas-de-Vecinos.pdf}
\newline {\color{gray} (Emb: 0.749, TF-IDF: 0.413)}
\newline {\color{gray} \textbf{2º:} 730-Iniciativa-Convencional-Constituyente-del-cc-Adolfo-Millabur-sobre-Gobiernos-Locales.pdf}
\newline {\color{gray} (Emb: 0.582, TF-IDF: 0.348)}


\item \textbf{Artículo} \newline
La asamblea social comunal tiene la finalidad de promover la participación popular y ciudadana en los asuntos públicos de la comuna autónoma, de carácter consultivo, incidente y representativo de las organizaciones de la comuna. 
\newline {\color{gray} \textbf{1º:} 394-3-Iniciativa-Convencional-Constituyente-de-la-cc-Ramona-Reyes-sobre-Comuna-Autonoma-1525-24-01.pdf}
\newline {\color{gray} (Emb: 0.905, TF-IDF: 0.745)}
\newline {\color{gray} \textbf{2º:} 117-3-c-Iniciativa-del-cc-Bastian-Labbe-sobre-Asamblea-Social-Regional.pdf}
\newline {\color{gray} (Emb: 0.635, TF-IDF: 0.488)}

Su integración, organización, funcionamiento y atribuciones serán establecidas por ley y complementada por el Estatuto Regional. 
\newline {\color{gray} \textbf{1º:} 394-3-Iniciativa-Convencional-Constituyente-de-la-cc-Ramona-Reyes-sobre-Comuna-Autonoma-1525-24-01.pdf}
\newline {\color{gray} (Emb: 0.804, TF-IDF: 0.518)}
\newline {\color{gray} \textbf{2º:} 631-Iniciativa-Convencional-Constituyente-de-cc-Ingrid-Villena-sobre-Contraloria-General-de-la-Republica.pdf}
\newline {\color{gray} (Emb: 0.722, TF-IDF: 0.511)}


\item \textbf{Artículo} \newline
Cada Comuna Autónoma tendrá un Estatuto Comunal elaborado y discutido por el Concejo Municipal correspondiente, que establecerá la organización administrativa y funcionamiento de los órganos comunales, los mecanismos de democracia vecinal y las normas de elaboración de ordenanzas comunales. 
\newline {\color{gray} \textbf{1º:} 394-3-Iniciativa-Convencional-Constituyente-de-la-cc-Ramona-Reyes-sobre-Comuna-Autonoma-1525-24-01.pdf}
\newline {\color{gray} (Emb: 0.899, TF-IDF: 0.907)}
\newline {\color{gray} \textbf{2º:} 329-3-Iniciativa-Convencional-Constituyente-de-la-cc-Tania-Madriaga-sobre-Gobiernos-Locales.pdf}
\newline {\color{gray} (Emb: 0.664, TF-IDF: 0.432)}

Todo lo anterior se entenderá sin perjuicio de los mínimos generales que establezca la ley respectiva para todas las comunas autónomas. 
\newline {\color{gray} \textbf{1º:} 210-1-c-Iniciativa-Convencional-del-cc-Cristián-Monckeberg-sobre-Estado-Intercultural-1953-hrs.pdf}
\newline {\color{gray} (Emb: 0.566, TF-IDF: 0.375)}
\newline {\color{gray} \textbf{2º:} 220-6-c-Iniciativa-Convencional-del-cc-Daniel-Bravo-sobre-organizacion-de-tribunales-2315-hrs.pdf}
\newline {\color{gray} (Emb: 0.547, TF-IDF: 0.358)}


\item \textbf{Artículo} \newline
Las demás competencias que determinen la Constitución y la ley. 
\newline {\color{gray} \textbf{1º:} 394-3-Iniciativa-Convencional-Constituyente-de-la-cc-Ramona-Reyes-sobre-Comuna-Autonoma-1525-24-01.pdf}
\newline {\color{gray} (Emb: 0.998, TF-IDF: 0.651)}
\newline {\color{gray} \textbf{2º:} 466-6-Iniciativa-Convencional-Constituyente-de-la-cc-Adriana-Cancino-sobre-Defensoria-de-los-DDHH-1933-31-01.pdf}
\newline {\color{gray} (Emb: 0.934, TF-IDF: 0.597)}

La dictación de normas generales y obligatorias en materias de carácter comunal, con arreglo a la Constitución y las leyes. 
\newline {\color{gray} \textbf{1º:} 753-Iniciativa-Convencional-Constituyente-del-cc-Raul-Celis-sobre-Gobiernos-Locales.pdf}
\newline {\color{gray} (Emb: 0.823, TF-IDF: 0.777)}
\newline {\color{gray} \textbf{2º:} 151-3-c-Iniciativa-de-la-cc-Angelica-Tepper-Competencias-de-los-Gobiernos-Regionales.pdf}
\newline {\color{gray} (Emb: 0.799, TF-IDF: 0.403)}

El fomento de la reintegración y reinserción de las personas en situación de calle que así lo requieran, mediante la planificación, coordinación y ejecución de programas al efecto. 
\newline {\color{gray} \textbf{1º:} 340-4-Iniciativa-Convencional-Constituyente-de-la-cc-Tania-Madriaga-sobre-Derecho-a-la-Vivienda.pdf}
\newline {\color{gray} (Emb: 0.559, TF-IDF: 0.371)}
\newline {\color{gray} \textbf{2º:} 67-2-Iniciativa-Convencional-Constituyente-del-cc-Cristián-Monckeberg-y-otros.pdf}
\newline {\color{gray} (Emb: 0.521, TF-IDF: 0.311)}

Ejercer las acciones pertinentes en resguardo de la Naturaleza y sus derechos reconocidos por esta Constitución y la ley. 
\newline {\color{gray} \textbf{1º:} 601-Iniciativa-Convencional-Constituyente-de-cc-Jorge-Abarca-Deber-de-Proteccion-Ambiental-de-los-Gobiernos-Locales.pdf}
\newline {\color{gray} (Emb: 0.956, TF-IDF: 0.604)}
\newline {\color{gray} \textbf{2º:} 14-4-Iniciativa-Convencional-Constituyente-de-la-cc-Aurora-Delgado-y-otros.pdf}
\newline {\color{gray} (Emb: 0.807, TF-IDF: 0.347)}

Las leyes deberán reconocer las diferencias existentes entre los distintos tipos de comunas y municipalidades, velando por la equidad, inclusión y cohesión territorial. 
\newline {\color{gray} \textbf{1º:} 394-3-Iniciativa-Convencional-Constituyente-de-la-cc-Ramona-Reyes-sobre-Comuna-Autonoma-1525-24-01.pdf}
\newline {\color{gray} (Emb: 1.000, TF-IDF: 1.000)}
\newline {\color{gray} \textbf{2º:} 385-3-Iniciativa-Convencional-Constituyente-del-cc-Felipe-Mena-sobre-Municipalidades-1200-24-01.pdf}
\newline {\color{gray} (Emb: 0.652, TF-IDF: 0.272)}

La promoción de la Seguridad ciudadana. 
\newline {\color{gray} \textbf{1º:} 196-2-c-Iniciativa-Convenciona-del-cc-Marco-Arellano-sobre-Plebiscito-e-iniciativa-de-ley-1552.pdf}
\newline {\color{gray} (Emb: 0.432, TF-IDF: 0.380)}
\newline {\color{gray} \textbf{2º:} 160-4-c-Iniciativa-de-la-cc-Valentina-Miranda-sobre-contenido-de-los-Derechos-Fundamentales.pdf}
\newline {\color{gray} (Emb: 0.387, TF-IDF: 0.352)}

Desarrollar el aseo y ornato de la comuna. 
\newline {\color{gray} \textbf{1º:} 514-4-Iniciativa-Convencional-Constituyente-del-cc-Felipe-Harboe-sobre-Derecho-a-la-Privacidad-1245-01-02.pdf}
\newline {\color{gray} (Emb: 0.422, TF-IDF: 0.310)}
\newline {\color{gray} \textbf{2º:} 440-Iniciativa-Convencional-Constituyente-del-cc-Felipe-Harboe-sobre-Principios-del-Debido-Proceso-1401-28-01.pdf}
\newline {\color{gray} (Emb: 0.422, TF-IDF: 0.310)}

La creación, organización y administración de los servicios públicos municipales en el ámbito de sus funciones, conforme a la Constitución y la ley. 
\newline {\color{gray} \textbf{1º:} 159-3-c-Iniciativa-de-la-cc-Jennifer-Mella-.pdf}
\newline {\color{gray} (Emb: 0.714, TF-IDF: 0.368)}
\newline {\color{gray} \textbf{2º:} 122-3-c-Iniciativa-de-la-cc-Jennifer-Mella-Forma-del-Estado.pdf}
\newline {\color{gray} (Emb: 0.714, TF-IDF: 0.366)}

Las Comunas Autónomas, a través de sus órganos de gobierno y administración, tendrán competencias preeminentes sobre las Regiones Autónomas y el Estado, en relación a las funciones de gobierno local que puedan ser cumplidas de modo adecuado y eficaz, sin perjuicio de una necesaria coordinación para su ejercicio y la distribución de competencias establecida en esta Constitución y las leyes. 
\newline {\color{gray} \textbf{1º:} 151-3-c-Iniciativa-de-la-cc-Angelica-Tepper-Competencias-de-los-Gobiernos-Regionales.pdf}
\newline {\color{gray} (Emb: 0.722, TF-IDF: 0.745)}
\newline {\color{gray} \textbf{2º:} 697-Iniciativa-Convencional-Constituyente-de-la-cc-Maria-Trinidad-Castillo-sobre-Educacion-Plurinacional-01-02.pdf}
\newline {\color{gray} (Emb: 0.568, TF-IDF: 0.383)}

A fin de garantizar el respeto, protección y realización progresiva de los derechos económicos y sociales en igualdad de condiciones, las comunas autónomas podrán encomendar temporalmente una o más competencias a las región autónoma respectiva o al Estado central, conforme lo establecido en la ley. 
\newline {\color{gray} \textbf{1º:} 196-2-c-Iniciativa-Convenciona-del-cc-Marco-Arellano-sobre-Plebiscito-e-iniciativa-de-ley-1552.pdf}
\newline {\color{gray} (Emb: 0.432, TF-IDF: 0.380)}
\newline {\color{gray} \textbf{2º:} 160-4-c-Iniciativa-de-la-cc-Valentina-Miranda-sobre-contenido-de-los-Derechos-Fundamentales.pdf}
\newline {\color{gray} (Emb: 0.387, TF-IDF: 0.352)}

A petición del alcalde o alcaldesa con acuerdo del concejo municipal, la región autónoma o el Estado, cuando así lo exija el interés general, podrán subrogar de manera transitoria y supletoria las competencias que no puedan ser asumidas por la comuna autónoma. 
\newline {\color{gray} \textbf{1º:} 208-4-c-Iniciativa-Convencional-del-cc-Martín-Arrau-sobre-Derecho-a-la-seguridad-1944-hrs.pdf}
\newline {\color{gray} (Emb: 0.702, TF-IDF: 0.758)}
\newline {\color{gray} \textbf{2º:} 159-3-c-Iniciativa-de-la-cc-Jennifer-Mella-.pdf}
\newline {\color{gray} (Emb: 0.529, TF-IDF: 0.417)}

Gestionar la reducción de riesgos frente a desastres. 
\newline {\color{gray} \textbf{1º:} 196-2-c-Iniciativa-Convenciona-del-cc-Marco-Arellano-sobre-Plebiscito-e-iniciativa-de-ley-1552.pdf}
\newline {\color{gray} (Emb: 0.432, TF-IDF: 0.380)}
\newline {\color{gray} \textbf{2º:} 160-4-c-Iniciativa-de-la-cc-Valentina-Miranda-sobre-contenido-de-los-Derechos-Fundamentales.pdf}
\newline {\color{gray} (Emb: 0.387, TF-IDF: 0.352)}

Fomentar las actividades productivas. 
\newline {\color{gray} \textbf{1º:} 151-3-c-Iniciativa-de-la-cc-Angelica-Tepper-Competencias-de-los-Gobiernos-Regionales.pdf}
\newline {\color{gray} (Emb: 0.825, TF-IDF: 0.808)}
\newline {\color{gray} \textbf{2º:} 151-3-c-Iniciativa-de-la-cc-Angelica-Tepper-Competencias-de-los-Gobiernos-Regionales.pdf}
\newline {\color{gray} (Emb: 0.771, TF-IDF: 0.780)}

La ejecución de los mecanismos y acciones de protección ambiental en la forma que determine la Constitución, la ley, los instrumentos de gestión ambiental y normas afines. 
\newline {\color{gray} \textbf{1º:} 601-Iniciativa-Convencional-Constituyente-de-cc-Jorge-Abarca-Deber-de-Proteccion-Ambiental-de-los-Gobiernos-Locales.pdf}
\newline {\color{gray} (Emb: 0.996, TF-IDF: 0.912)}
\newline {\color{gray} \textbf{2º:} 601-Iniciativa-Convencional-Constituyente-de-cc-Jorge-Abarca-Deber-de-Proteccion-Ambiental-de-los-Gobiernos-Locales.pdf}
\newline {\color{gray} (Emb: 0.761, TF-IDF: 0.350)}

Desarrollar, con el nivel regional y central, actividades y servicios en materias de educación, salud, vivienda, turismo, recreación, deporte y las demás que establezca la ley. 
\newline {\color{gray} \textbf{1º:} 151-3-c-Iniciativa-de-la-cc-Angelica-Tepper-Competencias-de-los-Gobiernos-Regionales.pdf}
\newline {\color{gray} (Emb: 0.928, TF-IDF: 0.822)}
\newline {\color{gray} \textbf{2º:} 151-3-c-Iniciativa-de-la-cc-Angelica-Tepper-Competencias-de-los-Gobiernos-Regionales.pdf}
\newline {\color{gray} (Emb: 0.827, TF-IDF: 0.676)}

Proteger los ecosistemas comunales y los derechos de la naturaleza. 
\newline {\color{gray} \textbf{1º:} 394-3-Iniciativa-Convencional-Constituyente-de-la-cc-Ramona-Reyes-sobre-Comuna-Autonoma-1525-24-01.pdf}
\newline {\color{gray} (Emb: 0.885, TF-IDF: 1.000)}
\newline {\color{gray} \textbf{2º:} 792-Iniciativa-Convencional-Constituyente-de-la-cc-Elsa-Labrana-sobre-Derechos-de-la-Naturaleza.pdf}
\newline {\color{gray} (Emb: 0.752, TF-IDF: 0.420)}

Fomento y protección a las culturas, las artes y los patrimonios culturales y naturales, así como la investigación y la formación artística en sus territorios. 
\newline {\color{gray} \textbf{1º:} 394-3-Iniciativa-Convencional-Constituyente-de-la-cc-Ramona-Reyes-sobre-Comuna-Autonoma-1525-24-01.pdf}
\newline {\color{gray} (Emb: 1.000, TF-IDF: 1.000)}
\newline {\color{gray} \textbf{2º:} 188-7-c-Iniciativa-Convenciona-del-cc-Ignacio-Achurra-sobre-Rol-del-Estado-en-las-Culturas-1126-hrs.pdf}
\newline {\color{gray} (Emb: 0.759, TF-IDF: 0.672)}

La conservación, custodia y resguardo de los patrimonios culturales y naturales. 
\newline {\color{gray} \textbf{1º:} 394-3-Iniciativa-Convencional-Constituyente-de-la-cc-Ramona-Reyes-sobre-Comuna-Autonoma-1525-24-01.pdf}
\newline {\color{gray} (Emb: 0.870, TF-IDF: 1.000)}
\newline {\color{gray} \textbf{2º:} 71-2-Iniciativa-Convencional-Constitutente-de-Maria-Jose-Oyarzun-y-otros.pdf}
\newline {\color{gray} (Emb: 0.844, TF-IDF: 0.499)}

El desarrollo sostenible e integral de la comuna. 
\newline {\color{gray} \textbf{1º:} 394-3-Iniciativa-Convencional-Constituyente-de-la-cc-Ramona-Reyes-sobre-Comuna-Autonoma-1525-24-01.pdf}
\newline {\color{gray} (Emb: 0.774, TF-IDF: 0.701)}
\newline {\color{gray} \textbf{2º:} 734-Iniciativa-Convencional-Constituyente-del-cc-Eduardo-Castillo-sobre-Asamble-Ciudadana.pdf}
\newline {\color{gray} (Emb: 0.561, TF-IDF: 0.557)}

Ejercer funciones de gobierno y administración dentro de la comuna y en el ámbito de sus competencias. 
\newline {\color{gray} \textbf{1º:} 151-3-c-Iniciativa-de-la-cc-Angelica-Tepper-Competencias-de-los-Gobiernos-Regionales.pdf}
\newline {\color{gray} (Emb: 1.000, TF-IDF: 1.000)}
\newline {\color{gray} \textbf{2º:} 151-3-c-Iniciativa-de-la-cc-Angelica-Tepper-Competencias-de-los-Gobiernos-Regionales.pdf}
\newline {\color{gray} (Emb: 0.861, TF-IDF: 0.842)}

El fomento del comercio local. 
\newline {\color{gray} \textbf{1º:} 394-3-Iniciativa-Convencional-Constituyente-de-la-cc-Ramona-Reyes-sobre-Comuna-Autonoma-1525-24-01.pdf}
\newline {\color{gray} (Emb: 0.690, TF-IDF: 1.000)}
\newline {\color{gray} \textbf{2º:} 931-Iniciativa-Convencional-Constituyente-del-cc-Felipe-Mena-sobre-Descentralizacion-Fiscal.pdf}
\newline {\color{gray} (Emb: 0.655, TF-IDF: 0.367)}

Garantizar la participación popular y el fortalecimiento de la democracia. 
\newline {\color{gray} \textbf{1º:} 394-3-Iniciativa-Convencional-Constituyente-de-la-cc-Ramona-Reyes-sobre-Comuna-Autonoma-1525-24-01.pdf}
\newline {\color{gray} (Emb: 0.892, TF-IDF: 1.000)}
\newline {\color{gray} \textbf{2º:} 652-Iniciativa-Convencional-Constituyente-de-la-cc-Ericka-Portilla-sobre-Trabajo-Decente-151101-02.pdf}
\newline {\color{gray} (Emb: 0.678, TF-IDF: 0.422)}

La planificación del territorio mediante el plan regulador comunal acordado de forma participativa con la comunidad de su respectivo territorio. 
\newline {\color{gray} \textbf{1º:} 196-2-c-Iniciativa-Convenciona-del-cc-Marco-Arellano-sobre-Plebiscito-e-iniciativa-de-ley-1552.pdf}
\newline {\color{gray} (Emb: 0.667, TF-IDF: 0.447)}
\newline {\color{gray} \textbf{2º:} 151-3-c-Iniciativa-de-la-cc-Angelica-Tepper-Competencias-de-los-Gobiernos-Regionales.pdf}
\newline {\color{gray} (Emb: 0.610, TF-IDF: 0.290)}

Construir las obras que demande el progreso local en el marco de sus atribuciones. 
\newline {\color{gray} \textbf{1º:} 394-3-Iniciativa-Convencional-Constituyente-de-la-cc-Ramona-Reyes-sobre-Comuna-Autonoma-1525-24-01.pdf}
\newline {\color{gray} (Emb: 0.828, TF-IDF: 1.000)}
\newline {\color{gray} \textbf{2º:} 120-3-c-Iniciativa-de-la-cc-Tammy-Pustilnick-atribuciones-exclusivas-de-la-Asamblea-Regional.pdf}
\newline {\color{gray} (Emb: 0.514, TF-IDF: 0.415)}

La prestación de los servicios públicos que determine la ley. 
\newline {\color{gray} \textbf{1º:} 394-3-Iniciativa-Convencional-Constituyente-de-la-cc-Ramona-Reyes-sobre-Comuna-Autonoma-1525-24-01.pdf}
\newline {\color{gray} (Emb: 0.825, TF-IDF: 1.000)}
\newline {\color{gray} \textbf{2º:} 921-Iniciativa-Convencional-Constituyente-de-la-cc-Yarela-Gomez-sobre-Modernizacion-del-Estado.pdf}
\newline {\color{gray} (Emb: 0.691, TF-IDF: 0.433)}

El desarrollo estratégico de la comuna mediante el plan de desarrollo comunal. 
\newline {\color{gray} \textbf{1º:} 394-3-Iniciativa-Convencional-Constituyente-de-la-cc-Ramona-Reyes-sobre-Comuna-Autonoma-1525-24-01.pdf}
\newline {\color{gray} (Emb: 0.692, TF-IDF: 0.465)}
\newline {\color{gray} \textbf{2º:} 117-3-c-Iniciativa-del-cc-Bastian-Labbe-sobre-Asamblea-Social-Regional.pdf}
\newline {\color{gray} (Emb: 0.608, TF-IDF: 0.415)}

Son competencias esenciales de la comuna autónoma: 1. 
\newline {\color{gray} \textbf{1º:} 394-3-Iniciativa-Convencional-Constituyente-de-la-cc-Ramona-Reyes-sobre-Comuna-Autonoma-1525-24-01.pdf}
\newline {\color{gray} (Emb: 1.000, TF-IDF: 1.000)}
\newline {\color{gray} \textbf{2º:} 119-3-c-Iniciativa-de-la-cc-Tammy-Pustilnick-competencias-de-las-Regiones-Autonomas.pdf}
\newline {\color{gray} (Emb: 0.790, TF-IDF: 0.599)}

La comuna autónoma cuenta con todas las potestades y competencias de autogobierno para satisfacer las necesidades de la comunidad local. 
\newline {\color{gray} \textbf{1º:} 394-3-Iniciativa-Convencional-Constituyente-de-la-cc-Ramona-Reyes-sobre-Comuna-Autonoma-1525-24-01.pdf}
\newline {\color{gray} (Emb: 0.947, TF-IDF: 0.935)}
\newline {\color{gray} \textbf{2º:} 91-3-Iniciativa-del-cc-Wilfredo-Bacian-que-establece-la-Forma-de-Estado-Regional-2.pdf}
\newline {\color{gray} (Emb: 0.711, TF-IDF: 0.397)}


\item \textbf{Artículo} \newline
Sin perjuicio de lo dispuesto en el inciso precedente, las asociaciones quedarán sujetas a la fiscalización de la entidad contralora y deberán cumplir con la normativa de probidad administrativa y de transparencia en el ejercicio de la función que desarrollan. 
\newline {\color{gray} \textbf{1º:} 394-3-Iniciativa-Convencional-Constituyente-de-la-cc-Ramona-Reyes-sobre-Comuna-Autonoma-1525-24-01.pdf}
\newline {\color{gray} (Emb: 1.000, TF-IDF: 1.000)}
\newline {\color{gray} \textbf{2º:} 159-3-c-Iniciativa-de-la-cc-Jennifer-Mella-.pdf}
\newline {\color{gray} (Emb: 0.748, TF-IDF: 0.491)}

Las comunas autónomas podrán asociarse entre sí, de manera permanente o temporal, pudiendo dichas organizaciones contar con personalidad jurídica de derecho privado, rigiéndose por la normativa propia de dicho sector. 
\newline {\color{gray} \textbf{1º:} 696-Iniciativa-Convencional-Consituyente-del-cc-Alvaro-Jofre-sobre-Descentralizacion-Fiscal-01-02.pdf}
\newline {\color{gray} (Emb: 0.683, TF-IDF: 0.719)}
\newline {\color{gray} \textbf{2º:} 36-3-Iniciativa-Convencional-Constituyente-del-cc-Martin-Arrau-y-otros.pdf}
\newline {\color{gray} (Emb: 0.669, TF-IDF: 0.356)}


\item \textbf{Artículo} \newline
Las comunas autónomas, previa autorización por ley general o especial, podrán establecer empresas, o participar en ellas, ya sea individualmente o asociadas con otras entidades públicas o privadas, a fin de cumplir con las funciones y ejercer las atribuciones que les asignan la Constitución y las leyes. 
\newline {\color{gray} \textbf{1º:} 394-3-Iniciativa-Convencional-Constituyente-de-la-cc-Ramona-Reyes-sobre-Comuna-Autonoma-1525-24-01.pdf}
\newline {\color{gray} (Emb: 0.831, TF-IDF: 0.856)}
\newline {\color{gray} \textbf{2º:} 899-Iniciativa-Convencional-Constituyente-del-cc-Cesar-Uribe-Sobre-Participacion-Ciudadana.pdf}
\newline {\color{gray} (Emb: 0.672, TF-IDF: 0.380)}

Las empresas públicas municipales tendrán personalidad jurídica y patrimonio propio y se regirán por las normas del derecho común. 
\newline {\color{gray} \textbf{1º:} 394-3-Iniciativa-Convencional-Constituyente-de-la-cc-Ramona-Reyes-sobre-Comuna-Autonoma-1525-24-01.pdf}
\newline {\color{gray} (Emb: 0.806, TF-IDF: 0.700)}
\newline {\color{gray} \textbf{2º:} 385-3-Iniciativa-Convencional-Constituyente-del-cc-Felipe-Mena-sobre-Municipalidades-1200-24-01.pdf}
\newline {\color{gray} (Emb: 0.640, TF-IDF: 0.445)}


\item \textbf{Artículo} \newline
La provincia es una división territorial establecida con fines administrativos y está compuesta por una agrupación de comunas autónomas, conforme a lo establecido en la Constitución y la ley. 
\newline {\color{gray} \textbf{1º:} 863-Iniciativa-Convencional-Constituyente-del-cc-Marcos-Barraza-sobre-Fuerzas-Armadas.pdf}
\newline {\color{gray} (Emb: 0.662, TF-IDF: 0.296)}
\newline {\color{gray} \textbf{2º:} 560-Iniciativa-Convencional-Constituyente-de-cc-Andres-Cruz-sobre-Ministerio-Publico-2016-hrs.-01-02.pdf}
\newline {\color{gray} (Emb: 0.654, TF-IDF: 0.273)}


\item \textbf{Artículo} \newline
Las Autonomías Territoriales Indígenas son entidades territoriales dotadas de personalidad jurídica de derecho público y patrimonio propio, donde los pueblos y naciones indígenas ejercen derechos de autonomía, en coordinación con las demás entidades territoriales que integran el Estado Regional de conformidad a la Constitución y la ley. 
\newline {\color{gray} \textbf{1º:} 394-3-Iniciativa-Convencional-Constituyente-de-la-cc-Ramona-Reyes-sobre-Comuna-Autonoma-1525-24-01.pdf}
\newline {\color{gray} (Emb: 0.670, TF-IDF: 0.257)}
\newline {\color{gray} \textbf{2º:} 555-Iniciativa-Convencional-Constituyente-del-convencional-Bernardo-Fontaine-sobre-libertad-de-trabajo-1737-01-02.pdf}
\newline {\color{gray} (Emb: 0.661, TF-IDF: 0.249)}

Es deber del Estado reconocer, promover y garantizar las Autonomías Territoriales Indígenas, para el cumplimiento de sus propios fines. 
\newline {\color{gray} \textbf{1º:} 683-Iniciativa-Convencional-Constituyente-del-cc-Jorge-Abarca-sobre-Pueblo-Tribal-150001-02.pdf}
\newline {\color{gray} (Emb: 0.841, TF-IDF: 0.597)}
\newline {\color{gray} \textbf{2º:} 489-3-Iniciativa-Convencional-Constituyente-del-cc-Adolfo-Millabur-sobre-Autonomia-Territorial-Indigena-0925-01-02.pdf}
\newline {\color{gray} (Emb: 0.841, TF-IDF: 0.597)}


\item \textbf{Artículo} \newline
La ley, mediante un proceso de participación y consulta previa, creará un procedimiento oportuno, eficiente y transparente para la constitución de las Autonomías Territoriales Indígenas. 
\newline {\color{gray} \textbf{1º:} 683-Iniciativa-Convencional-Constituyente-del-cc-Jorge-Abarca-sobre-Pueblo-Tribal-150001-02.pdf}
\newline {\color{gray} (Emb: 0.718, TF-IDF: 0.362)}
\newline {\color{gray} \textbf{2º:} 159-3-c-Iniciativa-de-la-cc-Jennifer-Mella-.pdf}
\newline {\color{gray} (Emb: 0.712, TF-IDF: 0.302)}

Dicho procedimiento deberá iniciarse a requerimiento de los pueblos y naciones indígenas interesados, a través de sus autoridades representativas. 
\newline {\color{gray} \textbf{1º:} 489-3-Iniciativa-Convencional-Constituyente-del-cc-Adolfo-Millabur-sobre-Autonomia-Territorial-Indigena-0925-01-02.pdf}
\newline {\color{gray} (Emb: 0.800, TF-IDF: 0.590)}
\newline {\color{gray} \textbf{2º:} 683-Iniciativa-Convencional-Constituyente-del-cc-Jorge-Abarca-sobre-Pueblo-Tribal-150001-02.pdf}
\newline {\color{gray} (Emb: 0.770, TF-IDF: 0.371)}


\item \textbf{Artículo} \newline
La ley deberá establecer las competencias exclusivas de las Autonomías Territoriales Indígenas y las compartidas con las demás entidades territoriales, de conformidad con lo que establece esta Constitución. 
\newline {\color{gray} \textbf{1º:} 768-Iniciativa-Convencional-Constituyente-del-cc-Eric-Chinga-sobre-autonomia-territorial-indigena.pdf}
\newline {\color{gray} (Emb: 0.731, TF-IDF: 0.432)}
\newline {\color{gray} \textbf{2º:} 633-3-Iniciativa-Convencional-Constituyente-de-la-Yarela-Gomez-sobre-Regimen-Tributario-1739-01-02.pdf}
\newline {\color{gray} (Emb: 0.711, TF-IDF: 0.431)}

Las Autonomías Territoriales Indígenas deberán tener las competencias y el financiamiento necesario para el adecuado ejercicio del derecho de libre determinación de los pueblos y naciones indígenas. 
\newline {\color{gray} \textbf{1º:} 913-Iniciativa-Convencional-Constituyente-del-cc-Luis-Jimenez-que-crea-la-Defensoría-de-los-Pueblos-Indígenas.pdf}
\newline {\color{gray} (Emb: 0.607, TF-IDF: 0.379)}
\newline {\color{gray} \textbf{2º:} 768-Iniciativa-Convencional-Constituyente-del-cc-Eric-Chinga-sobre-autonomia-territorial-indigena.pdf}
\newline {\color{gray} (Emb: 0.599, TF-IDF: 0.365)}


\item \textbf{Artículo} \newline
El Estado de Chile reconoce la existencia del maritorio como una categoría jurídica que, al igual que el territorio, debe contar con regulación normativa específica, que reconozca sus características propias en los ámbitos social, cultural, medioambiental y económico. 
\newline {\color{gray} \textbf{1º:} 1017-Iniciativa-Convencional-Constituyente-de-la-cc-Lisette-Vergara-sobre-Defensoria-Regional-de-DDHH-y-PPOO-RESPALDO.pdf}
\newline {\color{gray} (Emb: 0.610, TF-IDF: 0.382)}
\newline {\color{gray} \textbf{2º:} 644-Iniciativiva-Convencional-Constituyene-de-la-cc-Isabel-Godoy-sobre-Derechos-Linguisticos-1738-01-02.pdf}
\newline {\color{gray} (Emb: 0.592, TF-IDF: 0.329)}

Una ley establecerá la división administrativa del maritorio y los principios básicos que deberán informar los cuerpos legales que materialicen su institucionalización. 
\newline {\color{gray} \textbf{1º:} 683-Iniciativa-Convencional-Constituyente-del-cc-Jorge-Abarca-sobre-Pueblo-Tribal-150001-02.pdf}
\newline {\color{gray} (Emb: 0.721, TF-IDF: 0.495)}
\newline {\color{gray} \textbf{2º:} 159-3-c-Iniciativa-de-la-cc-Jennifer-Mella-.pdf}
\newline {\color{gray} (Emb: 0.711, TF-IDF: 0.495)}


\item \textbf{Artículo} \newline
En los territorios especiales, la ley podrá establecer regímenes económicos y administrativos diferenciados, así como su duración, teniendo en consideración las características y particularidades propias de estas entidades. 
\newline {\color{gray} \textbf{1º:} 75-3-Iniciativa-Convencional-Constituyente-del-cc-Harry-Jurgensen-y-otros.pdf}
\newline {\color{gray} (Emb: 0.960, TF-IDF: 0.957)}
\newline {\color{gray} \textbf{2º:} 714-Iniciativa-Convencional-Constituyente-de-la-cc-Lisette-Vergara-sobre-Empleo-Fiscal.pdf}
\newline {\color{gray} (Emb: 0.604, TF-IDF: 0.245)}

Son territorios especiales Rapa Nui y el Archipiélago Juan Fernández, los cuales estarán regidos por sus respectivos estatutos. 
\newline {\color{gray} \textbf{1º:} 190-6-c-Iniciativa-Convencional-Constituyente-de-la-cc-Natividad-LLanquileo-sobre-Pluralismo-Jurídico-1240-hrs.pdf}
\newline {\color{gray} (Emb: 0.738, TF-IDF: 0.443)}
\newline {\color{gray} \textbf{2º:} 237-1-Iniciativa-Convencional-de-la-cc-Tania-Madriaga-sobre-Estado-Plurinacional-y-Libre-Determinacion-1146-hrs.pdf}
\newline {\color{gray} (Emb: 0.649, TF-IDF: 0.387)}

Sin perjuicio de lo establecido en esta Constitución, la ley podrá crear territorios especiales en virtud de las particularidades geográficas, climáticas, ambientales, económicas, sociales y culturales de una determinada entidad territorial o parte de esta. 
\newline {\color{gray} \textbf{1º:} 75-3-Iniciativa-Convencional-Constituyente-del-cc-Harry-Jurgensen-y-otros.pdf}
\newline {\color{gray} (Emb: 0.993, TF-IDF: 0.965)}
\newline {\color{gray} \textbf{2º:} 269-3-Iniciativa-Convencional-del-cc-Felipe-Mena-sobre-Organizacion-Territorial-del-Estado-17-01-1154-hrs.pdf}
\newline {\color{gray} (Emb: 0.882, TF-IDF: 0.466)}


\item \textbf{Artículo} \newline
Para el cumplimiento de los fines establecidos en la creación de territorios especiales, el Estado y las entidades territoriales autónomas deberán destinar recursos de sus presupuestos respectivos, en conformidad a la Constitución y la ley. 
\newline {\color{gray} \textbf{1º:} 154-3-c-Iniciativa-del-cc-Felipe-Mena-sobre-Organizacion-Territorial-del-Estado.pdf}
\newline {\color{gray} (Emb: 0.712, TF-IDF: 0.457)}
\newline {\color{gray} \textbf{2º:} 269-3-Iniciativa-Convencional-del-cc-Felipe-Mena-sobre-Organizacion-Territorial-del-Estado-17-01-1154-hrs.pdf}
\newline {\color{gray} (Emb: 0.712, TF-IDF: 0.287)}


\item \textbf{Artículo} \newline
Se reconoce la titularidad colectiva de los derechos sobre el territorio al pueblo Rapa Nui con excepción de los derechos sobre tierras individuales de sus miembros. 
\newline {\color{gray} \textbf{1º:} 210-1-c-Iniciativa-Convencional-del-cc-Cristián-Monckeberg-sobre-Estado-Intercultural-1953-hrs.pdf}
\newline {\color{gray} (Emb: 0.686, TF-IDF: 0.290)}
\newline {\color{gray} \textbf{2º:} 384-3-Iniciativa-Convencional-Constituyente-del-cc-Felipe-Mena-sobre-Gobiernos-Regionales-1159-24-01.pdf}
\newline {\color{gray} (Emb: 0.615, TF-IDF: 0.268)}

En el territorio especial de Rapa Nui, el Estado garantiza el derecho a la libre determinación y autonomía del pueblo nación polinésico Rapa Nui, asegurando los medios para financiar y promover su desarrollo, protección y bienestar en virtud del Acuerdo de Voluntades firmado en 1888, por el cual se incorpora a Chile. 
\newline {\color{gray} \textbf{1º:} 402-3-Iniciativa-Convencional-Constituyente-de-la-cc-Elisa-Giustinianovich-sobre-Territorios-Especiales-1906-24-01.pdf}
\newline {\color{gray} (Emb: 0.723, TF-IDF: 0.263)}
\newline {\color{gray} \textbf{2º:} 304-4-Iniciativa-Convencional-de-la-cc-Valentina-Miranda-sobre-Derechos-Fundamentales-1301-hrs.pdf}
\newline {\color{gray} (Emb: 0.670, TF-IDF: 0.254)}

El territorio Rapa Nui se regulará por un estatuto de autonomía. 
\newline {\color{gray} \textbf{1º:} 402-3-Iniciativa-Convencional-Constituyente-de-la-cc-Elisa-Giustinianovich-sobre-Territorios-Especiales-1906-24-01.pdf}
\newline {\color{gray} (Emb: 0.983, TF-IDF: 0.926)}
\newline {\color{gray} \textbf{2º:} 633-3-Iniciativa-Convencional-Constituyente-de-la-Yarela-Gomez-sobre-Regimen-Tributario-1739-01-02.pdf}
\newline {\color{gray} (Emb: 0.713, TF-IDF: 0.359)}


\item \textbf{Artículo} \newline
El gobierno y administración de este territorio se regirá por los estatutos especiales que establezcan las leyes respectivas. 
\newline {\color{gray} \textbf{1º:} 370-4-Iniciativa-Convencional-Constituyente-de-la-cc-Constanza-San-Juan-sobre-Justicia-Transicional-0900-hrs-24-01.pdf}
\newline {\color{gray} (Emb: 0.678, TF-IDF: 0.395)}
\newline {\color{gray} \textbf{2º:} 94-1-Iniciativa-de-la-cc-Rosa-Catrileo-Establece-el-reconocimiento-de-los-Pueblos-Indigenas-2.pdf}
\newline {\color{gray} (Emb: 0.648, TF-IDF: 0.325)}

El Archipiélago Juan Fernández es un territorio especial, conformado por las islas Robinson Crusoe, Alejandro Selkirk, Santa Clara, San Félix y San Ambrosio, así como también el territorio marítimo adyacente a ellas. 
\newline {\color{gray} \textbf{1º:} 296-4-Iniciativa-Convencional-del-cc-Manuel-Woldarsky-sobre-Derecho-a-Defender-los-Derechos-Humanos-1931-hrs.pdf}
\newline {\color{gray} (Emb: 0.562, TF-IDF: 0.336)}
\newline {\color{gray} \textbf{2º:} 141-4-c-Iniciativa-de-la-cc-Rocio-Cantuarias-Incorpora-la-Libertad-de-asociacion.pdf}
\newline {\color{gray} (Emb: 0.535, TF-IDF: 0.317)}


\item \textbf{Artículo} \newline
Todas las personas y entidades deberán contribuir al sostenimiento de los gastos públicos mediante el pago de los impuestos, las tasas y las contribuciones que autorice la ley. 
\newline {\color{gray} \textbf{1º:} 99-3-c-Iniciativa-de-la-cc-Tammy-Pustilnick-Disposiciones-del-Estado-Regional.pdf}
\newline {\color{gray} (Emb: 0.711, TF-IDF: 0.342)}
\newline {\color{gray} \textbf{2º:} 43-3-Iniciativa-Convencional-Constituyente-de-la-cc-Tammy-Pustilnick-y-otros.pdf}
\newline {\color{gray} (Emb: 0.650, TF-IDF: 0.272)}

El Sistema tributario se funda en los principios de igualdad, progresividad, solidaridad y justicia material el cual, en ningún caso, tendrá alcance confiscatorio. 
\newline {\color{gray} \textbf{1º:} 470-3-Iniciativa-Convencional-Constituyente-del-cc-Jaime-Bassa-sobre-Reconocimiento-del-P.-Fernandeciano-1956-31-01.pdf}
\newline {\color{gray} (Emb: 0.935, TF-IDF: 0.896)}
\newline {\color{gray} \textbf{2º:} 470-3-Iniciativa-Convencional-Constituyente-del-cc-Jaime-Bassa-sobre-Reconocimiento-del-P.-Fernandeciano-1956-31-01.pdf}
\newline {\color{gray} (Emb: 0.483, TF-IDF: 0.357)}

El sistema tributario tendrá dentro de sus objetivos la reducción de las desigualdades y la pobreza. 
\newline {\color{gray} \textbf{1º:} 983-Iniciativa-Convencional-Constituyente-de-la-cc-Carolina-Vilches-sobre-Territorio-suelos-y-Agua.pdf}
\newline {\color{gray} (Emb: 0.860, TF-IDF: 0.525)}
\newline {\color{gray} \textbf{2º:} 924-Iniciativa-Convencional-Constituyente-de-la-cc-Constanza-Schonhaut-sobre-Administracion-del-Estado.pdf}
\newline {\color{gray} (Emb: 0.718, TF-IDF: 0.388)}

Los tributos y los beneficios tributarios se crean, modifican o suprimen por ley, salvo las excepciones que establezca esta Constitución. 
\newline {\color{gray} \textbf{1º:} 1014-Iniciativa-Convencional-Constituyente-cc-Adriana-Ampuero-Haciendas-territoriales-y-autonomia-financiera.pdf}
\newline {\color{gray} (Emb: 0.791, TF-IDF: 0.545)}
\newline {\color{gray} \textbf{2º:} 84-2-Iniciativa-Convencional-Constituyente-del-cc-Martin-Arrau-y-otros.pdf}
\newline {\color{gray} (Emb: 0.782, TF-IDF: 0.356)}

El ejercicio de la potestad tributaria admite la creación de tributos que respondan principalmente a fines distintos de la recaudación, debiendo tener en consideración límites tales como la necesidad, la razonabilidad y la transparencia. 
\newline {\color{gray} \textbf{1º:} 633-3-Iniciativa-Convencional-Constituyente-de-la-Yarela-Gomez-sobre-Regimen-Tributario-1739-01-02.pdf}
\newline {\color{gray} (Emb: 0.757, TF-IDF: 0.623)}
\newline {\color{gray} \textbf{2º:} 723-Iniciativa-Convencional-Constituyente-de-la-cc-Maria-Trinidad-Castillo-sobre-Principios-Tributarios-01-02.pdf}
\newline {\color{gray} (Emb: 0.757, TF-IDF: 0.623)}


\item \textbf{Artículo} \newline
Los Gobiernos Regionales y las Municipalidades gozan de autonomía financiera para el cumplimiento de sus funciones, dentro del marco establecido por esta Constitución y las leyes. 
\newline {\color{gray} \textbf{1º:} 84-2-Iniciativa-Convencional-Constituyente-del-cc-Martin-Arrau-y-otros.pdf}
\newline {\color{gray} (Emb: 0.531, TF-IDF: 0.253)}
\newline {\color{gray} \textbf{2º:} 85-3Iniciativa-del-cc-Martin-Arrau-Principios-y-Cargas-Tributarias.pdf}
\newline {\color{gray} (Emb: 0.531, TF-IDF: 0.241)}

La Ley de Presupuestos de la Nación deberá propender a que, progresivamente, una parte significativa del gasto público sea ejecutado a través de los gobiernos subnacionales, en función de las responsabilidades propias que debe asumir cada nivel de gobierno. 
\newline {\color{gray} \textbf{1º:} 1014-Iniciativa-Convencional-Constituyente-cc-Adriana-Ampuero-Haciendas-territoriales-y-autonomia-financiera.pdf}
\newline {\color{gray} (Emb: 0.738, TF-IDF: 0.584)}
\newline {\color{gray} \textbf{2º:} 119-3-c-Iniciativa-de-la-cc-Tammy-Pustilnick-competencias-de-las-Regiones-Autonomas.pdf}
\newline {\color{gray} (Emb: 0.699, TF-IDF: 0.437)}

El deber y la facultad de velar por la estabilidad macroeconómica y fiscal será centralizada, conforme a lo dispuesto en esta Constitución. 
\newline {\color{gray} \textbf{1º:} 40-2-Iniciativa-Convencional-Constituyente-de-la-cc-Bernardo-de-la-Maza-y-otros.pdf}
\newline {\color{gray} (Emb: 0.602, TF-IDF: 0.529)}
\newline {\color{gray} \textbf{2º:} 70-2-Iniciativa-Convencional-Constituyente-de-la-cc-Paulina-Veloso-y-otros-2.pdf}
\newline {\color{gray} (Emb: 0.587, TF-IDF: 0.252)}


\item \textbf{Artículo} \newline
Los tributos que se recauden, cualquiera sea su naturaleza, ingresarán al erario público del Estado o a las entidades territoriales según corresponda conforme a la Constitución. 
\newline {\color{gray} \textbf{1º:} 931-Iniciativa-Convencional-Constituyente-del-cc-Felipe-Mena-sobre-Descentralizacion-Fiscal.pdf}
\newline {\color{gray} (Emb: 1.000, TF-IDF: 1.000)}
\newline {\color{gray} \textbf{2º:} 1014-Iniciativa-Convencional-Constituyente-cc-Adriana-Ampuero-Haciendas-territoriales-y-autonomia-financiera.pdf}
\newline {\color{gray} (Emb: 0.809, TF-IDF: 0.365)}

Excepcionalmente, la ley podrá crear tributos en favor de las entidades territoriales que graven las actividades o bienes con una clara identificación con los territorios. 
\newline {\color{gray} \textbf{1º:} 931-Iniciativa-Convencional-Constituyente-del-cc-Felipe-Mena-sobre-Descentralizacion-Fiscal.pdf}
\newline {\color{gray} (Emb: 1.000, TF-IDF: 1.000)}
\newline {\color{gray} \textbf{2º:} 931-Iniciativa-Convencional-Constituyente-del-cc-Felipe-Mena-sobre-Descentralizacion-Fiscal.pdf}
\newline {\color{gray} (Emb: 0.687, TF-IDF: 0.480)}


\item \textbf{Artículo} \newline
No procederán iniciativas populares ni plebiscito y referéndum en materia tributaria. 
\newline {\color{gray} \textbf{1º:} 890-Iniciativa-Convencional-Constituyente-del-cc-Hugo-Gutierrez-que-Asegura-la-vigencia-de-la-Constitucion.pdf}
\newline {\color{gray} (Emb: 0.679, TF-IDF: 0.508)}
\newline {\color{gray} \textbf{2º:} 723-Iniciativa-Convencional-Constituyente-de-la-cc-Maria-Trinidad-Castillo-sobre-Principios-Tributarios-01-02.pdf}
\newline {\color{gray} (Emb: 0.652, TF-IDF: 0.504)}

La ley de Presupuestos no puede crear tributos ni beneficios tributarios. 
\newline {\color{gray} \textbf{1º:} 931-Iniciativa-Convencional-Constituyente-del-cc-Felipe-Mena-sobre-Descentralizacion-Fiscal.pdf}
\newline {\color{gray} (Emb: 1.000, TF-IDF: 1.000)}
\newline {\color{gray} \textbf{2º:} 957-5-Iniciativa-Convencional-Constituyente-de-la-cc-Ivanna-Olivares-sobre-Nuevo-Modelo-Economico.pdf}
\newline {\color{gray} (Emb: 0.636, TF-IDF: 0.348)}


\item \textbf{Artículo} \newline
La ley creará y regulará la administración de un Fondo para Territorios Especiales, cuyos recursos serán destinados exclusivamente a los fines para los cuales fueron creados. 
\newline {\color{gray} \textbf{1º:} 753-Iniciativa-Convencional-Constituyente-del-cc-Raul-Celis-sobre-Gobiernos-Locales.pdf}
\newline {\color{gray} (Emb: 0.772, TF-IDF: 0.594)}
\newline {\color{gray} \textbf{2º:} 200-3-c-Iniciativa-Convencional-del-cc-Felipe-Harboe-sobre-Igualdad-Cargas-Tributarias-1655-hrs.pdf}
\newline {\color{gray} (Emb: 0.718, TF-IDF: 0.547)}


\item \textbf{Artículo} \newline
Las entidades territoriales mencionadas en el artículo 5° de esta Constitución, gozarán de autonomía financiera en sus ingresos y gastos para el cumplimiento de sus competencias, la cual deberá ajustarse a los principios de suficiencia, coordinación, equilibrio presupuestario, solidaridad y compensación interterritorial, sostenibilidad, responsabilidad y eficiencia económica. 
\newline {\color{gray} \textbf{1º:} 1014-Iniciativa-Convencional-Constituyente-cc-Adriana-Ampuero-Haciendas-territoriales-y-autonomia-financiera.pdf}
\newline {\color{gray} (Emb: 1.000, TF-IDF: 1.000)}
\newline {\color{gray} \textbf{2º:} 892-Iniciativa-Convencional-Constituyente-del-cc-Benito-Baranda-Sobre-Deberes-y-Garantias-del-Contribuyente.pdf}
\newline {\color{gray} (Emb: 0.688, TF-IDF: 0.413)}


\item \textbf{Artículo} \newline
Las donaciones, herencias y legados que reciban conforme a la ley. 
\newline {\color{gray} \textbf{1º:} 957-5-Iniciativa-Convencional-Constituyente-de-la-cc-Ivanna-Olivares-sobre-Nuevo-Modelo-Economico.pdf}
\newline {\color{gray} (Emb: 0.613, TF-IDF: 0.386)}
\newline {\color{gray} \textbf{2º:} 698-Iniciativa-Convencional-Constituyente-de-la-cc-Alejandra-Flores-sobre-Educacion-01-02.pdf}
\newline {\color{gray} (Emb: 0.571, TF-IDF: 0.351)}

Las entidades territoriales, de conformidad a la Constitución y las leyes, tendrán las siguientes fuentes de ingresos: 1. 
\newline {\color{gray} \textbf{1º:} 1014-Iniciativa-Convencional-Constituyente-cc-Adriana-Ampuero-Haciendas-territoriales-y-autonomia-financiera.pdf}
\newline {\color{gray} (Emb: 0.858, TF-IDF: 0.625)}
\newline {\color{gray} \textbf{2º:} 892-Iniciativa-Convencional-Constituyente-del-cc-Benito-Baranda-Sobre-Deberes-y-Garantias-del-Contribuyente.pdf}
\newline {\color{gray} (Emb: 0.614, TF-IDF: 0.452)}

Los recursos asignados por la Ley de Presupuestos del Estado. 
\newline {\color{gray} \textbf{1º:} 633-3-Iniciativa-Convencional-Constituyente-de-la-Yarela-Gomez-sobre-Regimen-Tributario-1739-01-02.pdf}
\newline {\color{gray} (Emb: 0.921, TF-IDF: 0.885)}
\newline {\color{gray} \textbf{2º:} 633-3-Iniciativa-Convencional-Constituyente-de-la-Yarela-Gomez-sobre-Regimen-Tributario-1739-01-02.pdf}
\newline {\color{gray} (Emb: 0.696, TF-IDF: 0.411)}

Los impuestos en favor de la entidad territorial 3. 
\newline {\color{gray} \textbf{1º:} 1014-Iniciativa-Convencional-Constituyente-cc-Adriana-Ampuero-Haciendas-territoriales-y-autonomia-financiera.pdf}
\newline {\color{gray} (Emb: 0.476, TF-IDF: 0.308)}
\newline {\color{gray} \textbf{2º:} 1012-Iniciativa-Convencional-Constituyente-del-cc-Nicolas-Nunez-sobre-Responsabilidad-Fiscal.pdf}
\newline {\color{gray} (Emb: 0.474, TF-IDF: 0.307)}

La distribución de los impuestos establecida en la Ley de Presupuestos. 
\newline {\color{gray} \textbf{1º:} 1014-Iniciativa-Convencional-Constituyente-cc-Adriana-Ampuero-Haciendas-territoriales-y-autonomia-financiera.pdf}
\newline {\color{gray} (Emb: 0.796, TF-IDF: 0.902)}
\newline {\color{gray} \textbf{2º:} 120-3-c-Iniciativa-de-la-cc-Tammy-Pustilnick-atribuciones-exclusivas-de-la-Asamblea-Regional.pdf}
\newline {\color{gray} (Emb: 0.720, TF-IDF: 0.337)}

Las tasas y contribuciones. 
\newline {\color{gray} \textbf{1º:} 671-Iniciativa-Convencional-Constituyente-del-cc-Bernardo-Fontaine-Formacion-de-la-Ley-121101-02.pdf}
\newline {\color{gray} (Emb: 0.798, TF-IDF: 0.734)}
\newline {\color{gray} \textbf{2º:} 1014-Iniciativa-Convencional-Constituyente-cc-Adriana-Ampuero-Haciendas-territoriales-y-autonomia-financiera.pdf}
\newline {\color{gray} (Emb: 0.775, TF-IDF: 0.375)}

La distribución de los fondos solidarios. 
\newline {\color{gray} \textbf{1º:} 159-3-c-Iniciativa-de-la-cc-Jennifer-Mella-.pdf}
\newline {\color{gray} (Emb: 0.617, TF-IDF: 0.393)}
\newline {\color{gray} \textbf{2º:} 122-3-c-Iniciativa-de-la-cc-Jennifer-Mella-Forma-del-Estado.pdf}
\newline {\color{gray} (Emb: 0.617, TF-IDF: 0.338)}

La transferencia fiscal interterritorial. 
\newline {\color{gray} \textbf{1º:} 671-Iniciativa-Convencional-Constituyente-del-cc-Bernardo-Fontaine-Formacion-de-la-Ley-121101-02.pdf}
\newline {\color{gray} (Emb: 0.669, TF-IDF: 0.302)}
\newline {\color{gray} \textbf{2º:} 1014-Iniciativa-Convencional-Constituyente-cc-Adriana-Ampuero-Haciendas-territoriales-y-autonomia-financiera.pdf}
\newline {\color{gray} (Emb: 0.661, TF-IDF: 0.294)}

La administración y aprovechamiento de su patrimonio. 
\newline {\color{gray} \textbf{1º:} 1014-Iniciativa-Convencional-Constituyente-cc-Adriana-Ampuero-Haciendas-territoriales-y-autonomia-financiera.pdf}
\newline {\color{gray} (Emb: 0.531, TF-IDF: 0.656)}
\newline {\color{gray} \textbf{2º:} 713-Iniciativa-Convencional-Constituyente-de-la-cc-Lisette-Vergara-sobre-Economia.pdf}
\newline {\color{gray} (Emb: 0.479, TF-IDF: 0.579)}

Otras que determine la Constitución y la ley. 
\newline {\color{gray} \textbf{1º:} 636-6-Iniciativa-Convencional-Constituyente-de-la-cc-Manuela-Royo-sobre-funcion-notarial-y-registral.pdf}
\newline {\color{gray} (Emb: 0.537, TF-IDF: 0.330)}
\newline {\color{gray} \textbf{2º:} 713-Iniciativa-Convencional-Constituyente-de-la-cc-Lisette-Vergara-sobre-Economia.pdf}
\newline {\color{gray} (Emb: 0.496, TF-IDF: 0.272)}


\item \textbf{Artículo} \newline
Las entidades territoriales solo podrán establecer tasas y contribuciones dentro de su territorio conforme a una ley marco que establecerá el hecho gravado. 
\newline {\color{gray} \textbf{1º:} 1014-Iniciativa-Convencional-Constituyente-cc-Adriana-Ampuero-Haciendas-territoriales-y-autonomia-financiera.pdf}
\newline {\color{gray} (Emb: 0.861, TF-IDF: 0.965)}
\newline {\color{gray} \textbf{2º:} 366-4-Iniciativa-Convencional-Constituyente-del-cc-Marco-Arellano-sobre-Derechos-Linguisticos-2232-hrs-21-01.pdf}
\newline {\color{gray} (Emb: 0.621, TF-IDF: 0.181)}

Sólo la ley podrá crear, modificar y suprimir impuestos y beneficios tributarios aplicables a estos. 
\newline {\color{gray} \textbf{1º:} 1014-Iniciativa-Convencional-Constituyente-cc-Adriana-Ampuero-Haciendas-territoriales-y-autonomia-financiera.pdf}
\newline {\color{gray} (Emb: 0.797, TF-IDF: 0.456)}
\newline {\color{gray} \textbf{2º:} 122-3-c-Iniciativa-de-la-cc-Jennifer-Mella-Forma-del-Estado.pdf}
\newline {\color{gray} (Emb: 0.710, TF-IDF: 0.353)}


\item \textbf{Artículo} \newline
Los ingresos fiscales generados por impuestos serán distribuidos entre el Estado y las entidades territoriales en la forma establecida en la Ley de Presupuestos. 
\newline {\color{gray} \textbf{1º:} 466-6-Iniciativa-Convencional-Constituyente-de-la-cc-Adriana-Cancino-sobre-Defensoria-de-los-DDHH-1933-31-01.pdf}
\newline {\color{gray} (Emb: 0.986, TF-IDF: 0.668)}
\newline {\color{gray} \textbf{2º:} 151-3-c-Iniciativa-de-la-cc-Angelica-Tepper-Competencias-de-los-Gobiernos-Regionales.pdf}
\newline {\color{gray} (Emb: 0.953, TF-IDF: 0.567)}

Durante el trámite legislativo presupuestario, el organismo competente sugerirá una fórmula de distribución de ingresos fiscales, la cual considerará los criterios de distribución establecidos por la ley. 
\newline {\color{gray} \textbf{1º:} 1014-Iniciativa-Convencional-Constituyente-cc-Adriana-Ampuero-Haciendas-territoriales-y-autonomia-financiera.pdf}
\newline {\color{gray} (Emb: 1.000, TF-IDF: 0.926)}
\newline {\color{gray} \textbf{2º:} 892-Iniciativa-Convencional-Constituyente-del-cc-Benito-Baranda-Sobre-Deberes-y-Garantias-del-Contribuyente.pdf}
\newline {\color{gray} (Emb: 0.789, TF-IDF: 0.449)}


\item \textbf{Artículo} \newline
La autonomía financiera de las entidades territoriales implica la facultad de ordenar y gestionar sus finanzas públicas en el marco de la Constitución y las leyes, en beneficio de sus habitantes, bajo los criterios de responsabilidad y sostenibilidad financiera. 
\newline {\color{gray} \textbf{1º:} 120-3-c-Iniciativa-de-la-cc-Tammy-Pustilnick-atribuciones-exclusivas-de-la-Asamblea-Regional.pdf}
\newline {\color{gray} (Emb: 0.680, TF-IDF: 0.496)}
\newline {\color{gray} \textbf{2º:} 1014-Iniciativa-Convencional-Constituyente-cc-Adriana-Ampuero-Haciendas-territoriales-y-autonomia-financiera.pdf}
\newline {\color{gray} (Emb: 0.669, TF-IDF: 0.496)}

(Inciso tercero) La suficiencia financiera se determinará bajo criterios objetivos tales como correspondencia entre competencias y recursos necesarios para su cumplimiento, equilibrio presupuestario, coordinación, no discriminación arbitraria entre entidades territoriales, igualdad en las prestaciones sociales, desarrollo armónico de los territorios, unidad, objetividad, razonabilidad, oportunidad y transparencia. 
\newline {\color{gray} \textbf{1º:} 120-3-c-Iniciativa-de-la-cc-Tammy-Pustilnick-atribuciones-exclusivas-de-la-Asamblea-Regional.pdf}
\newline {\color{gray} (Emb: 0.655, TF-IDF: 0.294)}
\newline {\color{gray} \textbf{2º:} 447-Iniciativa-Convencional-Constituyente-del-cc-Wilfredo-Bacian-sobre-Estatuto-de-las-Ues.-Estatales-1501-28-01.pdf}
\newline {\color{gray} (Emb: 0.639, TF-IDF: 0.270)}


\item \textbf{Artículo} \newline
La actividad financiera de las entidades territoriales se realizará coordinadamente entre ellas, el Estado y las autoridades competentes, las cuales deberán cooperar y colaborar entre sí y evitar la duplicidad e interferencia de funciones, velando en todo momento por la satisfacción del interés general. 
\newline {\color{gray} \textbf{1º:} 1014-Iniciativa-Convencional-Constituyente-cc-Adriana-Ampuero-Haciendas-territoriales-y-autonomia-financiera.pdf}
\newline {\color{gray} (Emb: 0.553, TF-IDF: 0.244)}
\newline {\color{gray} \textbf{2º:} 871-Iniciativa-Convencional-Constituyente-de-la-cc-Amaya-Alvez-sobre-Asambleas-Regionales.pdf}
\newline {\color{gray} (Emb: 0.535, TF-IDF: 0.231)}

Este principio se aplicará también respecto de todas las competencias o potestades que se atribuyan a las entidades territoriales. 
\newline {\color{gray} \textbf{1º:} 1014-Iniciativa-Convencional-Constituyente-cc-Adriana-Ampuero-Haciendas-territoriales-y-autonomia-financiera.pdf}
\newline {\color{gray} (Emb: 0.913, TF-IDF: 0.929)}
\newline {\color{gray} \textbf{2º:} 931-Iniciativa-Convencional-Constituyente-del-cc-Felipe-Mena-sobre-Descentralizacion-Fiscal.pdf}
\newline {\color{gray} (Emb: 0.737, TF-IDF: 0.340)}


\item \textbf{Artículo} \newline
f) Estos recursos no podrán ser destinados a remuneraciones ni a gasto corriente. 
\newline {\color{gray} \textbf{1º:} 239-1-Iniciativa-Convencional-de-la-cc-Tania-Madriaga-sobre-Poder-Ejecutivo-1146-hrs.pdf}
\newline {\color{gray} (Emb: 0.562, TF-IDF: 0.276)}
\newline {\color{gray} \textbf{2º:} 569-Iniciativa-Convencional-Constituyente-de-cc-Roberto-Celedon-sobre-Derecho-al-trabajo-2024-hrs.-01-02.pdf}
\newline {\color{gray} (Emb: 0.523, TF-IDF: 0.271)}

Los gobiernos regionales y locales podrán emitir deuda en conformidad a lo que disponga la ley, general o especial, la que establecerá al menos las siguientes regulaciones: a) La prohibición de destinar los fondos recaudados mediante emisión de deuda o empréstitos al financiamiento de gasto corriente. 
\newline {\color{gray} \textbf{1º:} 1014-Iniciativa-Convencional-Constituyente-cc-Adriana-Ampuero-Haciendas-territoriales-y-autonomia-financiera.pdf}
\newline {\color{gray} (Emb: 0.950, TF-IDF: 0.958)}
\newline {\color{gray} \textbf{2º:} 970-Iniciativa-Convencional-Consttuyente-del-cc-Nicolas-Nunez-sobre-Presupuestos.pdf}
\newline {\color{gray} (Emb: 0.612, TF-IDF: 0.224)}

b) Los mecanismos que garanticen que la deuda sea íntegra y debidamente servida por el deudor. 
\newline {\color{gray} \textbf{1º:} 1014-Iniciativa-Convencional-Constituyente-cc-Adriana-Ampuero-Haciendas-territoriales-y-autonomia-financiera.pdf}
\newline {\color{gray} (Emb: 0.931, TF-IDF: 0.898)}
\newline {\color{gray} \textbf{2º:} 154-3-c-Iniciativa-del-cc-Felipe-Mena-sobre-Organizacion-Territorial-del-Estado.pdf}
\newline {\color{gray} (Emb: 0.603, TF-IDF: 0.266)}

c) La prohibición del establecimiento de garantías o cauciones del fisco. 
\newline {\color{gray} \textbf{1º:} 1014-Iniciativa-Convencional-Constituyente-cc-Adriana-Ampuero-Haciendas-territoriales-y-autonomia-financiera.pdf}
\newline {\color{gray} (Emb: 1.000, TF-IDF: 1.000)}
\newline {\color{gray} \textbf{2º:} 344-3-Iniciativa-Convencional-Constituyente-del-cc-Hernan-Larrain-sobre-Reforma-Administrativa-y-Modernizacion-del-Estado.pdf}
\newline {\color{gray} (Emb: 0.628, TF-IDF: 0.338)}

d) El establecimiento de límites máximos de endeudamiento como porcentaje del presupuesto anual del gobierno regional y municipal respectivo y la obligación de mantener una clasificación de riesgo actualizada. 
\newline {\color{gray} \textbf{1º:} 176-2-c-Iniciativa-Convencional-del-cc-Rodrigo-Álvarez-sobre-Responsabilidad-Fiscal-1044-hrs.pdf}
\newline {\color{gray} (Emb: 0.660, TF-IDF: 0.284)}
\newline {\color{gray} \textbf{2º:} 84-2-Iniciativa-Convencional-Constituyente-del-cc-Martin-Arrau-y-otros.pdf}
\newline {\color{gray} (Emb: 0.660, TF-IDF: 0.284)}

e) Restricciones en períodos electorales. 
\newline {\color{gray} \textbf{1º:} 344-3-Iniciativa-Convencional-Constituyente-del-cc-Hernan-Larrain-sobre-Reforma-Administrativa-y-Modernizacion-del-Estado.pdf}
\newline {\color{gray} (Emb: 0.453, TF-IDF: 0.242)}
\newline {\color{gray} \textbf{2º:} 302-4-Iniciativa-Convencional-del-cc-Fuad-Chahin-sobre-Derecho-al-trabajo-18-01.-1141-hrs.pdf}
\newline {\color{gray} (Emb: 0.433, TF-IDF: 0.242)}


\item \textbf{Artículo} \newline
Para estos efectos se deberá considerar la participación y representación de las entidades territoriales. 
\newline {\color{gray} \textbf{1º:} 159-3-c-Iniciativa-de-la-cc-Jennifer-Mella-.pdf}
\newline {\color{gray} (Emb: 0.575, TF-IDF: 0.277)}
\newline {\color{gray} \textbf{2º:} 122-3-c-Iniciativa-de-la-cc-Jennifer-Mella-Forma-del-Estado.pdf}
\newline {\color{gray} (Emb: 0.575, TF-IDF: 0.273)}

La ley definirá el organismo encargado de recopilar y sistematizar la información necesaria para proponer al Poder Legislativo las fórmulas de distribución de los ingresos fiscales, de compensación fiscal entre entidades territoriales y de los recursos a integrar en los diversos fondos. 
\newline {\color{gray} \textbf{1º:} 176-2-c-Iniciativa-Convencional-del-cc-Rodrigo-Álvarez-sobre-Responsabilidad-Fiscal-1044-hrs.pdf}
\newline {\color{gray} (Emb: 0.582, TF-IDF: 0.331)}
\newline {\color{gray} \textbf{2º:} 84-2-Iniciativa-Convencional-Constituyente-del-cc-Martin-Arrau-y-otros.pdf}
\newline {\color{gray} (Emb: 0.582, TF-IDF: 0.256)}


\item \textbf{Artículo} \newline
El Estado y las entidades territoriales deben Contribuir a la corrección de las desigualdades que existan entre ellas. 
\newline {\color{gray} \textbf{1º:} 240-1-Iniciativa-Convencional-de-la-cc-Tania-Madriaga-sobre-Poder-Legislativo-1146-hrs.pdf}
\newline {\color{gray} (Emb: 0.583, TF-IDF: 0.269)}
\newline {\color{gray} \textbf{2º:} 176-2-c-Iniciativa-Convencional-del-cc-Rodrigo-Álvarez-sobre-Responsabilidad-Fiscal-1044-hrs.pdf}
\newline {\color{gray} (Emb: 0.580, TF-IDF: 0.227)}

La ley establecerá fondos de compensación para las entidades territoriales con una menor capacidad fiscal. 
\newline {\color{gray} \textbf{1º:} 633-3-Iniciativa-Convencional-Constituyente-de-la-Yarela-Gomez-sobre-Regimen-Tributario-1739-01-02.pdf}
\newline {\color{gray} (Emb: 0.664, TF-IDF: 0.373)}
\newline {\color{gray} \textbf{2º:} 1014-Iniciativa-Convencional-Constituyente-cc-Adriana-Ampuero-Haciendas-territoriales-y-autonomia-financiera.pdf}
\newline {\color{gray} (Emb: 0.606, TF-IDF: 0.312)}

El organismo competente, sobre la base de criterios objetivos, sugerirá al legislador los recursos que deberán ser integrados a estos fondos. 
\newline {\color{gray} \textbf{1º:} 1008-Iniciativa-Convencional-Constituyente-de-la-cc-Claudia-Castro-sobre-Propiedad.pdf}
\newline {\color{gray} (Emb: 0.562, TF-IDF: 0.359)}
\newline {\color{gray} \textbf{2º:} 414-5-Iniciativa-Convencional-del-cc-Bernardo-Fontaine-sobre-Estatuto-de-los-Minerales-1500-25-01.pdf}
\newline {\color{gray} (Emb: 0.562, TF-IDF: 0.323)}

El Estado deberá realizar transferencias directas incondicionales a las entidades territoriales que cuenten con ingresos fiscales inferiores a la mitad del promedio ponderado de estas. 
\newline {\color{gray} \textbf{1º:} 931-Iniciativa-Convencional-Constituyente-del-cc-Felipe-Mena-sobre-Descentralizacion-Fiscal.pdf}
\newline {\color{gray} (Emb: 0.632, TF-IDF: 0.388)}
\newline {\color{gray} \textbf{2º:} 91-3-Iniciativa-del-cc-Wilfredo-Bacian-que-establece-la-Forma-de-Estado-Regional-2.pdf}
\newline {\color{gray} (Emb: 0.626, TF-IDF: 0.255)}

La ley establecerá un fondo de contingencia y estabilización macroeconómica para garantizar los recursos de las entidades territoriales ante fluctuaciones de ingresos ordinarios. 
\newline {\color{gray} \textbf{1º:} 1014-Iniciativa-Convencional-Constituyente-cc-Adriana-Ampuero-Haciendas-territoriales-y-autonomia-financiera.pdf}
\newline {\color{gray} (Emb: 0.671, TF-IDF: 0.626)}
\newline {\color{gray} \textbf{2º:} 633-3-Iniciativa-Convencional-Constituyente-de-la-Yarela-Gomez-sobre-Regimen-Tributario-1739-01-02.pdf}
\newline {\color{gray} (Emb: 0.648, TF-IDF: 0.457)}


\item \textbf{Artículo} \newline
Las regiones y comunas que cuenten con ingresos por sobre el promedio ponderado de ingresos fiscales, transferirán recursos a aquellas equivalentes con ingresos bajo el promedio. 
\newline {\color{gray} \textbf{1º:} 753-Iniciativa-Convencional-Constituyente-del-cc-Raul-Celis-sobre-Gobiernos-Locales.pdf}
\newline {\color{gray} (Emb: 0.601, TF-IDF: 0.304)}
\newline {\color{gray} \textbf{2º:} 713-Iniciativa-Convencional-Constituyente-de-la-cc-Lisette-Vergara-sobre-Economia.pdf}
\newline {\color{gray} (Emb: 0.550, TF-IDF: 0.237)}

El organismo competente sugerirá una fórmula al legislador para realizar tales transferencias. 
\newline {\color{gray} \textbf{1º:} 633-3-Iniciativa-Convencional-Constituyente-de-la-Yarela-Gomez-sobre-Regimen-Tributario-1739-01-02.pdf}
\newline {\color{gray} (Emb: 0.623, TF-IDF: 0.361)}
\newline {\color{gray} \textbf{2º:} 633-3-Iniciativa-Convencional-Constituyente-de-la-Yarela-Gomez-sobre-Regimen-Tributario-1739-01-02.pdf}
\newline {\color{gray} (Emb: 0.620, TF-IDF: 0.349)}


\item \textbf{Artículo} \newline
Es deber del Estado y de las entidades territoriales, en el ámbito de sus competencias financieras, establecer una política permanente de desarrollo sostenible y armónico con la naturaleza. 
\newline {\color{gray} \textbf{1º:} 633-3-Iniciativa-Convencional-Constituyente-de-la-Yarela-Gomez-sobre-Regimen-Tributario-1739-01-02.pdf}
\newline {\color{gray} (Emb: 0.568, TF-IDF: 0.333)}
\newline {\color{gray} \textbf{2º:} 931-Iniciativa-Convencional-Constituyente-del-cc-Felipe-Mena-sobre-Descentralizacion-Fiscal.pdf}
\newline {\color{gray} (Emb: 0.567, TF-IDF: 0.322)}

Con el objeto de contar con recursos para el cuidado y la reparación de los ecosistemas, la ley podrá establecer tributos sobre actividades que afecten al medio ambiente. 
\newline {\color{gray} \textbf{1º:} 633-3-Iniciativa-Convencional-Constituyente-de-la-Yarela-Gomez-sobre-Regimen-Tributario-1739-01-02.pdf}
\newline {\color{gray} (Emb: 0.825, TF-IDF: 0.546)}
\newline {\color{gray} \textbf{2º:} 633-3-Iniciativa-Convencional-Constituyente-de-la-Yarela-Gomez-sobre-Regimen-Tributario-1739-01-02.pdf}
\newline {\color{gray} (Emb: 0.802, TF-IDF: 0.544)}

Asimismo, la ley podrá establecer tributos sobre el uso de bienes comunes naturales, bienes nacionales de uso público o bienes fiscales. 
\newline {\color{gray} \textbf{1º:} 544-Iniciativa-Convencional-Constituyente-del-cc-Andres-Cruz-sobre-reforma-constitucional-1623-01-02.pdf}
\newline {\color{gray} (Emb: 0.562, TF-IDF: 0.351)}
\newline {\color{gray} \textbf{2º:} 1029-Iniciativa-Convencional-Consituyente-de-la-cc-Carolina-Vilches-sobre-Regeneracion-de-la-Vida.pdf}
\newline {\color{gray} (Emb: 0.559, TF-IDF: 0.320)}

Cuando dichas actividades estén territorialmente circunscritas, la ley debe distribuir recursos a la entidad territorial que corresponda. 
\newline {\color{gray} \textbf{1º:} 1014-Iniciativa-Convencional-Constituyente-cc-Adriana-Ampuero-Haciendas-territoriales-y-autonomia-financiera.pdf}
\newline {\color{gray} (Emb: 0.965, TF-IDF: 0.990)}
\newline {\color{gray} \textbf{2º:} 99-3-c-Iniciativa-de-la-cc-Tammy-Pustilnick-Disposiciones-del-Estado-Regional.pdf}
\newline {\color{gray} (Emb: 0.865, TF-IDF: 0.841)}


\item \textbf{Artículo} \newline
Las entidades territoriales, sus representantes y sus autoridades que incumplan con sus obligaciones en materia financiera, deberán asumir, en la parte que les sea imputable, las responsabilidades que de tal incumplimiento se deriven con arreglo a la Constitución y las leyes. 
\newline {\color{gray} \textbf{1º:} 1014-Iniciativa-Convencional-Constituyente-cc-Adriana-Ampuero-Haciendas-territoriales-y-autonomia-financiera.pdf}
\newline {\color{gray} (Emb: 0.674, TF-IDF: 0.357)}
\newline {\color{gray} \textbf{2º:} 799-Iniciativa-Convencional-Constituyente-del-cc-Juan-Jose-Martin-sobre-Estatuto-Oceanico.pdf}
\newline {\color{gray} (Emb: 0.664, TF-IDF: 0.357)}

Sin perjuicio de los distintos tipos de responsabilidad a que pueda dar lugar el incumplimiento de las obligaciones en materia financiera, la ley deberá establecer mecanismos para un resarcimiento efectivo del patrimonio fiscal o de la entidad territorial respectiva. 
\newline {\color{gray} \textbf{1º:} 159-3-c-Iniciativa-de-la-cc-Jennifer-Mella-.pdf}
\newline {\color{gray} (Emb: 0.702, TF-IDF: 0.695)}
\newline {\color{gray} \textbf{2º:} 122-3-c-Iniciativa-de-la-cc-Jennifer-Mella-Forma-del-Estado.pdf}
\newline {\color{gray} (Emb: 0.702, TF-IDF: 0.695)}


\item \textbf{Artículo} \newline
El principio de eficiencia económica implica que las entidades territoriales deberán usar sus recursos de forma económicamente razonable, óptima y eficaz, en beneficio de sus habitantes y en función de los objetivos que la Constitución y las leyes les impongan. 
\newline {\color{gray} \textbf{1º:} 633-3-Iniciativa-Convencional-Constituyente-de-la-Yarela-Gomez-sobre-Regimen-Tributario-1739-01-02.pdf}
\newline {\color{gray} (Emb: 0.621, TF-IDF: 0.275)}
\newline {\color{gray} \textbf{2º:} 633-3-Iniciativa-Convencional-Constituyente-de-la-Yarela-Gomez-sobre-Regimen-Tributario-1739-01-02.pdf}
\newline {\color{gray} (Emb: 0.600, TF-IDF: 0.234)}


\item \textbf{Artículo} \newline
Anualmente la autoridad competente publicará los ingresos afectos a impuestos y las cargas tributarias estatales, regionales y comunales, así como los beneficios tributarios, subsidios, subvenciones o bonificaciones de fomento a la actividad empresarial, incluyendo personas naturales y jurídicas. 
\newline {\color{gray} \textbf{1º:} 1014-Iniciativa-Convencional-Constituyente-cc-Adriana-Ampuero-Haciendas-territoriales-y-autonomia-financiera.pdf}
\newline {\color{gray} (Emb: 0.997, TF-IDF: 0.962)}
\newline {\color{gray} \textbf{2º:} 1014-Iniciativa-Convencional-Constituyente-cc-Adriana-Ampuero-Haciendas-territoriales-y-autonomia-financiera.pdf}
\newline {\color{gray} (Emb: 0.695, TF-IDF: 0.327)}

También deberá estimarse anualmente en la Ley de Presupuestos y publicarse el costo de estos beneficios fiscales. 
\newline {\color{gray} \textbf{1º:} 1014-Iniciativa-Convencional-Constituyente-cc-Adriana-Ampuero-Haciendas-territoriales-y-autonomia-financiera.pdf}
\newline {\color{gray} (Emb: 1.000, TF-IDF: 1.000)}
\newline {\color{gray} \textbf{2º:} 931-Iniciativa-Convencional-Constituyente-del-cc-Felipe-Mena-sobre-Descentralizacion-Fiscal.pdf}
\newline {\color{gray} (Emb: 0.598, TF-IDF: 0.268)}

La ley determinará la información que deberá ser publicada y la forma de llevarla a cabo. 
\newline {\color{gray} \textbf{1º:} 1014-Iniciativa-Convencional-Constituyente-cc-Adriana-Ampuero-Haciendas-territoriales-y-autonomia-financiera.pdf}
\newline {\color{gray} (Emb: 1.000, TF-IDF: 1.000)}
\newline {\color{gray} \textbf{2º:} 256-4-Iniciativa-Convencional-de-la-cc-Elsa-Labrana-sobre-Parte-General-de-los-DDFF.pdf}
\newline {\color{gray} (Emb: 0.613, TF-IDF: 0.347)}


\item \textbf{Artículo} \newline
Las entidades territoriales deberán promover, fomentar y garantizar los mecanismos de participación en las políticas públicas, planes y programas que se implementen en cada nivel territorial, en los casos que esta Constitución, la ley y los estatutos regionales señalen. 
\newline {\color{gray} \textbf{1º:} 715-Iniciativa-Convencional-Constituyente-de-la-cc-Lisette-Vergara-sobre-Empresas-Publicas.pdf}
\newline {\color{gray} (Emb: 0.667, TF-IDF: 0.278)}
\newline {\color{gray} \textbf{2º:} 633-3-Iniciativa-Convencional-Constituyente-de-la-Yarela-Gomez-sobre-Regimen-Tributario-1739-01-02.pdf}
\newline {\color{gray} (Emb: 0.651, TF-IDF: 0.246)}


\item \textbf{Artículo} \newline
En el ejercicio de la función pública se deberá observar una conducta funcionaria intachable y responsable, desempeñando la función o el cargo correspondiente en forma leal, honesta, objetiva e imparcial, sin incurrir en discriminaciones de ningún tipo, con preeminencia del interés general por sobre el particular. 
\newline {\color{gray} \textbf{1º:} 163-3-c-Iniciativa-de-la-cc-Rocio-Cantuarias-sobre-Tributos.pdf}
\newline {\color{gray} (Emb: 0.775, TF-IDF: 0.525)}
\newline {\color{gray} \textbf{2º:} 1014-Iniciativa-Convencional-Constituyente-cc-Adriana-Ampuero-Haciendas-territoriales-y-autonomia-financiera.pdf}
\newline {\color{gray} (Emb: 0.755, TF-IDF: 0.501)}

La función pública se deberá brindar con pertinencia territorial, cultural y lingüística. 
\newline {\color{gray} \textbf{1º:} 633-3-Iniciativa-Convencional-Constituyente-de-la-Yarela-Gomez-sobre-Regimen-Tributario-1739-01-02.pdf}
\newline {\color{gray} (Emb: 0.964, TF-IDF: 0.973)}
\newline {\color{gray} \textbf{2º:} 238-1-Iniciativa-Convencional-de-la-cc-Tania-Madriaga-sobre-Buen-Gobierno-1146-hrs.pdf}
\newline {\color{gray} (Emb: 0.680, TF-IDF: 0.488)}


\item \textbf{Artículo} \newline
Es deber del Estado proveer de servicios públicos universales y de calidad a todas las personas que habiten en su territorio, los cuales contarán con un financiamiento suficiente. 
\newline {\color{gray} \textbf{1º:} 899-Iniciativa-Convencional-Constituyente-del-cc-Cesar-Uribe-Sobre-Participacion-Ciudadana.pdf}
\newline {\color{gray} (Emb: 0.826, TF-IDF: 0.676)}
\newline {\color{gray} \textbf{2º:} 924-Iniciativa-Convencional-Constituyente-de-la-cc-Constanza-Schonhaut-sobre-Administracion-del-Estado.pdf}
\newline {\color{gray} (Emb: 0.679, TF-IDF: 0.279)}

El Estado planificará y coordinará de manera intersectorial la provisión, prestación y cobertura de estos servicios, bajo los principios de generalidad, uniformidad, regularidad y pertinencia territorial. 
\newline {\color{gray} \textbf{1º:} 469-3-Iniciativa-Convencional-Constituyente-del-cc-Felipe-Mena-sobre-Modernizacion-del-Estado-1952-31-01.pdf}
\newline {\color{gray} (Emb: 0.774, TF-IDF: 0.766)}
\newline {\color{gray} \textbf{2º:} 60-2-Iniciativa-Convencional-Constituyente-del-cc-Jorge-Baradit-y-otros.pdf}
\newline {\color{gray} (Emb: 0.712, TF-IDF: 0.530)}


\item \textbf{Artículo} \newline
Los órganos de la Administración tienen por objeto satisfacer las necesidades de las personas y las comunidades. 
\newline {\color{gray} \textbf{1º:} 646-Iniciativa-Convencional-Constituyente-de-la-cc-Maria-Trinidad-Castillo-sobre-Derecho-a-la-Educacion121101-02.pdf}
\newline {\color{gray} (Emb: 0.771, TF-IDF: 0.666)}
\newline {\color{gray} \textbf{2º:} 921-Iniciativa-Convencional-Constituyente-de-la-cc-Yarela-Gomez-sobre-Modernizacion-del-Estado.pdf}
\newline {\color{gray} (Emb: 0.670, TF-IDF: 0.392)}

La Administración Pública ejecutará políticas públicas, planes y programas, y proveerá o garantizará, en su caso, la prestación de servicios públicos en forma continua y permanente. 
\newline {\color{gray} \textbf{1º:} 945-Iniciativa-Convencional-Constituyente-del-cc-Francisco-Caamano-sobre-Derecho-de-Atencion-en-era-digital.pdf}
\newline {\color{gray} (Emb: 0.651, TF-IDF: 0.372)}
\newline {\color{gray} \textbf{2º:} 358-4-Iniciativa-Convencional-Constituyente-del-cc-Eduardo-Castillo-sobre-derecho-a-la-Seguridad-Social-1425-hrs-21-01.pdf}
\newline {\color{gray} (Emb: 0.630, TF-IDF: 0.372)}

La Administración Pública se somete en su organización y funcionamiento a los principios de juridicidad, publicidad, celeridad, objetividad, participación, control, jerarquía, eficiencia, eficacia, rendición de cuentas, buen trato, primacía del interés general y los demás principios que señale la Constitución y la ley. 
\newline {\color{gray} \textbf{1º:} 957-5-Iniciativa-Convencional-Constituyente-de-la-cc-Ivanna-Olivares-sobre-Nuevo-Modelo-Economico.pdf}
\newline {\color{gray} (Emb: 0.599, TF-IDF: 0.433)}
\newline {\color{gray} \textbf{2º:} 732-Iniciativa-Convencional-Constituyente-del-cc-Bastian-Labbe-crea-el-Servicio-de-proteccion-de-bienes-comunes.pdf}
\newline {\color{gray} (Emb: 0.592, TF-IDF: 0.326)}

Cualquier persona que hubiere sido vulnerada en sus derechos por la Administración Pública podrá reclamar ante las instancias administrativas y jurisdiccionales que establezcan esta Constitución y la ley. 
\newline {\color{gray} \textbf{1º:} 117-3-c-Iniciativa-del-cc-Bastian-Labbe-sobre-Asamblea-Social-Regional.pdf}
\newline {\color{gray} (Emb: 0.692, TF-IDF: 0.445)}
\newline {\color{gray} \textbf{2º:} 394-3-Iniciativa-Convencional-Constituyente-de-la-cc-Ramona-Reyes-sobre-Comuna-Autonoma-1525-24-01.pdf}
\newline {\color{gray} (Emb: 0.653, TF-IDF: 0.336)}


\item \textbf{Artículo} \newline
En ningún caso estas potestades implicarán ejercicio de jurisdicción. 
\newline {\color{gray} \textbf{1º:} 60-2-Iniciativa-Convencional-Constituyente-del-cc-Jorge-Baradit-y-otros.pdf}
\newline {\color{gray} (Emb: 0.746, TF-IDF: 0.411)}
\newline {\color{gray} \textbf{2º:} 119-3-c-Iniciativa-de-la-cc-Tammy-Pustilnick-competencias-de-las-Regiones-Autonomas.pdf}
\newline {\color{gray} (Emb: 0.744, TF-IDF: 0.407)}

La ley podrá conferir, a lo menos, potestades fiscalizadoras, instructoras, normativas, interpretativas y sancionatorias a los órganos de la Administración Pública. 
\newline {\color{gray} \textbf{1º:} 921-Iniciativa-Convencional-Constituyente-de-la-cc-Yarela-Gomez-sobre-Modernizacion-del-Estado.pdf}
\newline {\color{gray} (Emb: 0.699, TF-IDF: 0.448)}
\newline {\color{gray} \textbf{2º:} 304-4-Iniciativa-Convencional-de-la-cc-Valentina-Miranda-sobre-Derechos-Fundamentales-1301-hrs.pdf}
\newline {\color{gray} (Emb: 0.692, TF-IDF: 0.347)}

La ley establecerá la organización básica de la Administración Pública en el Estado y en las entidades territoriales. 
\newline {\color{gray} \textbf{1º:} 241-1-Iniciativa-Convencional-de-la-cc-Alejandra-Flores-sobre-Buen-Gobierno-1146-hrs.pdf}
\newline {\color{gray} (Emb: 0.868, TF-IDF: 0.787)}
\newline {\color{gray} \textbf{2º:} 921-Iniciativa-Convencional-Constituyente-de-la-cc-Yarela-Gomez-sobre-Modernizacion-del-Estado.pdf}
\newline {\color{gray} (Emb: 0.666, TF-IDF: 0.588)}

Cada autoridad y jefatura, dentro del ámbito de su competencia, podrá dictar normas, resoluciones e instrucciones para el mejor y más eficaz desarrollo de sus funciones. 
\newline {\color{gray} \textbf{1º:} 921-Iniciativa-Convencional-Constituyente-de-la-cc-Yarela-Gomez-sobre-Modernizacion-del-Estado.pdf}
\newline {\color{gray} (Emb: 0.724, TF-IDF: 0.666)}
\newline {\color{gray} \textbf{2º:} 344-3-Iniciativa-Convencional-Constituyente-del-cc-Hernan-Larrain-sobre-Reforma-Administrativa-y-Modernizacion-del-Estado.pdf}
\newline {\color{gray} (Emb: 0.670, TF-IDF: 0.432)}


\item \textbf{Artículo} \newline
Se exceptúan los nombramientos que se hagan en aplicación de las normas vigentes sobre ingreso o ascenso por méritos en cargos de carrera. 
\newline {\color{gray} \textbf{1º:} 921-Iniciativa-Convencional-Constituyente-de-la-cc-Yarela-Gomez-sobre-Modernizacion-del-Estado.pdf}
\newline {\color{gray} (Emb: 0.984, TF-IDF: 0.883)}
\newline {\color{gray} \textbf{2º:} 232-6-Iniciativa-Convencional-del-cc-Marco-Arellano-que-Crea-el-Consejo-Nacional-de-Justicia-1144-hrs.pdf}
\newline {\color{gray} (Emb: 0.668, TF-IDF: 0.533)}

La Administración Pública desarrolla sus funciones propias y habituales a través de funcionarias y funcionarios públicos. 
\newline {\color{gray} \textbf{1º:} 122-3-c-Iniciativa-de-la-cc-Jennifer-Mella-Forma-del-Estado.pdf}
\newline {\color{gray} (Emb: 0.575, TF-IDF: 0.330)}
\newline {\color{gray} \textbf{2º:} 159-3-c-Iniciativa-de-la-cc-Jennifer-Mella-.pdf}
\newline {\color{gray} (Emb: 0.575, TF-IDF: 0.330)}

El ingreso a estas funciones se realizará mediante un sistema abierto, transparente, imparcial, ágil y que privilegie el mérito, la especialidad e idoneidad para el cargo, observando en todo momento criterios objetivos y predeterminados. 
\newline {\color{gray} \textbf{1º:} 631-Iniciativa-Convencional-Constituyente-de-cc-Ingrid-Villena-sobre-Contraloria-General-de-la-Republica.pdf}
\newline {\color{gray} (Emb: 0.650, TF-IDF: 0.369)}
\newline {\color{gray} \textbf{2º:} 201-6-c-Iniciativa-Convencional-del-cc-Felipe-Harboe-sobre-la-Contraloria-1658-hrs.pdf}
\newline {\color{gray} (Emb: 0.567, TF-IDF: 0.348)}

El desarrollo, evaluación de desempeño y cese en estas funciones deberá respetar su carácter técnico y profesional. 
\newline {\color{gray} \textbf{1º:} 558-Iniciativa-Convencional-Constituyente-de-cc-Andres-Cruz-sobre-Contraloria-General-de-la-Republica-1755-hrs.-01-02.pdf}
\newline {\color{gray} (Emb: 0.738, TF-IDF: 0.461)}
\newline {\color{gray} \textbf{2º:} 695-Iniciativa-Convencional-del-cc-Felipe-Harboe-sobre-Ministerio-Publico-120001-02.pdf}
\newline {\color{gray} (Emb: 0.652, TF-IDF: 0.399)}

La ley regulará las bases de la carrera funcionaria, permitiendo la movilidad de los funcionarios dentro de toda la Administración Pública y la capacitación funcionaria, teniendo en cuenta la pertinencia territorial y cultural del lugar en el que se presta el servicio. 
\newline {\color{gray} \textbf{1º:} 151-3-c-Iniciativa-de-la-cc-Angelica-Tepper-Competencias-de-los-Gobiernos-Regionales.pdf}
\newline {\color{gray} (Emb: 0.728, TF-IDF: 0.290)}
\newline {\color{gray} \textbf{2º:} 122-3-c-Iniciativa-de-la-cc-Jennifer-Mella-Forma-del-Estado.pdf}
\newline {\color{gray} (Emb: 0.645, TF-IDF: 0.258)}

La ley establecerá un sistema de formación, capacitación y perfeccionamiento de las funcionarias y funcionarios públicos. 
\newline {\color{gray} \textbf{1º:} 921-Iniciativa-Convencional-Constituyente-de-la-cc-Yarela-Gomez-sobre-Modernizacion-del-Estado.pdf}
\newline {\color{gray} (Emb: 0.703, TF-IDF: 0.708)}
\newline {\color{gray} \textbf{2º:} 924-Iniciativa-Convencional-Constituyente-de-la-cc-Constanza-Schonhaut-sobre-Administracion-del-Estado.pdf}
\newline {\color{gray} (Emb: 0.673, TF-IDF: 0.583)}

Los cargos que esta Constitución o la ley califiquen como de exclusiva confianza, atendiendo a la naturaleza de sus funciones, son parte del gobierno y tendrán el régimen de ingreso, desempeño y cesación que establezca la ley. 
\newline {\color{gray} \textbf{1º:} 220-6-c-Iniciativa-Convencional-del-cc-Daniel-Bravo-sobre-organizacion-de-tribunales-2315-hrs.pdf}
\newline {\color{gray} (Emb: 0.535, TF-IDF: 0.301)}
\newline {\color{gray} \textbf{2º:} 343-4-Iniciativa-Convencional-Constituyente-del-cc-Manuel-Jose-Ossandon-sobre-Derecho-a-la-Educacion.pdf}
\newline {\color{gray} (Emb: 0.532, TF-IDF: 0.297)}

Las personas que tengan la calidad de cónyuge, conviviente civil o parientes hasta el cuarto grado de consanguinidad y segundo de afinidad inclusive, no podrán ser nombrados en cargos de la administración pública respecto de las autoridades y de los funcionarios directivos del organismo del Estado al que postulan. 
\newline {\color{gray} \textbf{1º:} 120-3-c-Iniciativa-de-la-cc-Tammy-Pustilnick-atribuciones-exclusivas-de-la-Asamblea-Regional.pdf}
\newline {\color{gray} (Emb: 0.628, TF-IDF: 0.580)}
\newline {\color{gray} \textbf{2º:} 384-3-Iniciativa-Convencional-Constituyente-del-cc-Felipe-Mena-sobre-Gobiernos-Regionales-1159-24-01.pdf}
\newline {\color{gray} (Emb: 0.622, TF-IDF: 0.413)}


\item \textbf{Artículo} \newline
El Estado deberá fomentar los mecanismos de participación, la relación con las personas y promover la gestión eficiente y moderna, acorde a las necesidades de las personas y comunidades. 
\newline {\color{gray} \textbf{1º:} 362-2-Iniciativa-Convencional-Constituyente-de-la-cc-Janis-Meneses-sobre-Derechos-Laborales-1836-hrs-21-01.pdf}
\newline {\color{gray} (Emb: 0.541, TF-IDF: 0.255)}
\newline {\color{gray} \textbf{2º:} 107-4-c-Iniciativa-de-la-cc-Giovanna-Grandon-Derecho-al-Trabajo.pdf}
\newline {\color{gray} (Emb: 0.518, TF-IDF: 0.243)}

El Estado deberá destinar recursos para que sus órganos adopten las medidas que resulten necesarias para la incorporación de avances tecnológicos, innovación y el mejor uso de los recursos que permitan optimizar la provisión de bienes y servicios públicos. 
\newline {\color{gray} \textbf{1º:} 347-3-Iniciativa-Convencional-Constituyente-del-cc-Ruth-Hurtado-sobre-Inhabilidades-para-ejercer-cargos-publicos.pdf}
\newline {\color{gray} (Emb: 0.717, TF-IDF: 0.572)}
\newline {\color{gray} \textbf{2º:} 1005-Iniciativa-Convencional-Constituyente-del-cc-Roberto-Celedon-sobre-Derechos-de-las-Victimas-de-violaciones-DDHH.pdf}
\newline {\color{gray} (Emb: 0.530, TF-IDF: 0.547)}

Es deber del Estado definir mecanismos de modernización de sus procesos y organización, ajustando su funcionamiento a las condiciones sociales, ambientales y culturales de cada localidad. 
\newline {\color{gray} \textbf{1º:} 196-2-c-Iniciativa-Convenciona-del-cc-Marco-Arellano-sobre-Plebiscito-e-iniciativa-de-ley-1552.pdf}
\newline {\color{gray} (Emb: 0.651, TF-IDF: 0.330)}
\newline {\color{gray} \textbf{2º:} 583-Iniciativa-Convencional-Constituyente-de-cc-Loreto-Vidal-sobre-el-Conocimiento-como-Bien-Comun-2351-hrs.-01-02.pdf}
\newline {\color{gray} (Emb: 0.580, TF-IDF: 0.289)}


\item \textbf{Artículo} \newline
La ley encomendará a un organismo la elaboración de planes para promover la modernización de la Administración del Estado, monitorear su implementación, elaborar diagnósticos periódicos sobre el funcionamiento de los servicios públicos y las demás funciones que establezca la ley. 
\newline {\color{gray} \textbf{1º:} 921-Iniciativa-Convencional-Constituyente-de-la-cc-Yarela-Gomez-sobre-Modernizacion-del-Estado.pdf}
\newline {\color{gray} (Emb: 0.854, TF-IDF: 0.817)}
\newline {\color{gray} \textbf{2º:} 186-7-c-Iniciativa-Convencional-de-la-cc-Malucha-Pinto-sobre-derechos-culturales-1126-hrs.pdf}
\newline {\color{gray} (Emb: 0.548, TF-IDF: 0.267)}

Este organismo contará con un consejo consultivo cuya integración considerará, entre otros, a las y los usuarios y funcionarios de los servicios públicos y las entidades territoriales. 
\newline {\color{gray} \textbf{1º:} 469-3-Iniciativa-Convencional-Constituyente-del-cc-Felipe-Mena-sobre-Modernizacion-del-Estado-1952-31-01.pdf}
\newline {\color{gray} (Emb: 0.751, TF-IDF: 0.591)}
\newline {\color{gray} \textbf{2º:} 344-3-Iniciativa-Convencional-Constituyente-del-cc-Hernan-Larrain-sobre-Reforma-Administrativa-y-Modernizacion-del-Estado.pdf}
\newline {\color{gray} (Emb: 0.583, TF-IDF: 0.281)}


\item \textbf{Artículo} \newline
El Estado reconoce la ruralidad como una expresión territorial donde las formas de vida y producción se desarrollan en torno a la relación directa de las personas y comunidades con la tierra, el agua y el mar. 
\newline {\color{gray} \textbf{1º:} 469-3-Iniciativa-Convencional-Constituyente-del-cc-Felipe-Mena-sobre-Modernizacion-del-Estado-1952-31-01.pdf}
\newline {\color{gray} (Emb: 0.541, TF-IDF: 0.335)}
\newline {\color{gray} \textbf{2º:} 927-Iniciativa-Convencional-Constituyente-de-la-cc-Alondra-Carrillo-sobre-Economia-y-Desarrollo-Plurinacional.pdf}
\newline {\color{gray} (Emb: 0.514, TF-IDF: 0.267)}

El Estado promoverá el desarrollo integral de los territorios rurales. 
\newline {\color{gray} \textbf{1º:} 344-3-Iniciativa-Convencional-Constituyente-del-cc-Hernan-Larrain-sobre-Reforma-Administrativa-y-Modernizacion-del-Estado.pdf}
\newline {\color{gray} (Emb: 0.680, TF-IDF: 0.250)}
\newline {\color{gray} \textbf{2º:} 580-Iniciativa-Convencional-Constituyente-de-cc-Daniel-Stingo-sobre-Contraloria-General-de-la-Republica-2318-hrs.-01-02.pdf}
\newline {\color{gray} (Emb: 0.565, TF-IDF: 0.248)}


\item \textbf{Artículo} \newline
El Estado y las entidades territoriales facilitarán la participación de las comunidades rurales a nivel local y regional en el diseño e implementación de programas y políticas públicas que les afectan o conciernen. 
\newline {\color{gray} \textbf{1º:} 876-Iniciativa-Convencional-Constituyente-de-la-cc-Elisa-Giustinianovich-sobre-Transicion-Productiva.pdf}
\newline {\color{gray} (Emb: 0.634, TF-IDF: 0.220)}
\newline {\color{gray} \textbf{2º:} 929-Iniciativa-Convencional-Constituyente-de-la-cc-Maria-Magdalena-Rivera-sobre-Region-Exterior.pdf}
\newline {\color{gray} (Emb: 0.617, TF-IDF: 0.212)}


\item \textbf{Artículo} \newline
El Estado fomentará los mercados locales, ferias libres y circuitos cortos de comercialización e intercambio de bienes y productos relacionados a la ruralidad. 
\newline {\color{gray} \textbf{1º:} 961-3-Iniciativa-Convencional-Constituyente-de-la-cc-Tania-Madriaga-sobre-Soberania-Territorial.pdf}
\newline {\color{gray} (Emb: 0.623, TF-IDF: 0.332)}
\newline {\color{gray} \textbf{2º:} 595-Iniciativa-Convencional-Constituyente-de-cc-Cesar-Uribe-sobre-Presencia-del-mundo-rural-2351-hrs.-01-02.pdf}
\newline {\color{gray} (Emb: 0.615, TF-IDF: 0.287)}


\item \textbf{Artículo} \newline
El Estado protegerá la función ecológica y social de la tierra. 
\newline {\color{gray} \textbf{1º:} 188-7-c-Iniciativa-Convenciona-del-cc-Ignacio-Achurra-sobre-Rol-del-Estado-en-las-Culturas-1126-hrs.pdf}
\newline {\color{gray} (Emb: 0.685, TF-IDF: 0.448)}
\newline {\color{gray} \textbf{2º:} 60-2-Iniciativa-Convencional-Constituyente-del-cc-Jorge-Baradit-y-otros.pdf}
\newline {\color{gray} (Emb: 0.648, TF-IDF: 0.374)}


\item \textbf{Artículo} \newline
El Estado reconoce y apoya la agricultura campesina e indígena, la recolección y la pesca artesanal, entre otros, como actividades fundamentales de la producción de alimentos. 
\newline {\color{gray} \textbf{1º:} 899-Iniciativa-Convencional-Constituyente-del-cc-Cesar-Uribe-Sobre-Participacion-Ciudadana.pdf}
\newline {\color{gray} (Emb: 0.690, TF-IDF: 0.357)}
\newline {\color{gray} \textbf{2º:} 961-3-Iniciativa-Convencional-Constituyente-de-la-cc-Tania-Madriaga-sobre-Soberania-Territorial.pdf}
\newline {\color{gray} (Emb: 0.639, TF-IDF: 0.299)}


\item \textbf{Artículo} \newline
El Estado tomará las medidas necesarias para prevenir la violencia y superar las desigualdades que afrontan mujeres y niñas rurales, promoviendo la implementación de políticas públicas para garantizar su acceso igualitario a los derechos que esta Constitución consagra. 
\newline {\color{gray} \textbf{1º:} 113-5-c-Iniciativa-del-cc-Elsa-Labrana-sobre-Soberania-Alimentaria.pdf}
\newline {\color{gray} (Emb: 0.572, TF-IDF: 0.427)}
\newline {\color{gray} \textbf{2º:} 849-Iniciativa-Convencional-Constituyente-de-la-cc-Patricia-Labra-sobre-Agricultura-esencial-y-estratégica.pdf}
\newline {\color{gray} (Emb: 0.567, TF-IDF: 0.403)}


\item \textbf{Artículo} \newline
El Estado es garante de la conectividad del país en coordinación con los gobiernos regionales. 
\newline {\color{gray} \textbf{1º:} 281-4-Iniciativa-Convencional-de-la-cc-Ivanna-Olivares-sobre-Derecho-de-Propiedad-1200-hrs.pdf}
\newline {\color{gray} (Emb: 0.698, TF-IDF: 0.347)}
\newline {\color{gray} \textbf{2º:} 991-Iniciativa-Convencional-Constituyente-de-la-cc-Ivanna-Olivares-sobre-Funcion-Ecologica-de-la-Propiedad.pdf}
\newline {\color{gray} (Emb: 0.678, TF-IDF: 0.347)}

Se fomentará la conectividad regional con especial atención a territorios aislados, rurales y de difícil acceso. 
\newline {\color{gray} \textbf{1º:} 113-5-c-Iniciativa-del-cc-Elsa-Labrana-sobre-Soberania-Alimentaria.pdf}
\newline {\color{gray} (Emb: 0.693, TF-IDF: 0.418)}
\newline {\color{gray} \textbf{2º:} 985-Iniciativa-Convencional-Constituente-de-la-cc-Elsa-Labrana-sobre-Soberania-Alimentaria.pdf}
\newline {\color{gray} (Emb: 0.570, TF-IDF: 0.374)}


\item \textbf{Artículo} \newline
La designación de las y los representantes de los Ministerios y Servicios Públicos con presencia en la Región Autónoma será decisión de la Presidencia de la República. 
\newline {\color{gray} \textbf{1º:} 366-4-Iniciativa-Convencional-Constituyente-del-cc-Marco-Arellano-sobre-Derechos-Linguisticos-2232-hrs-21-01.pdf}
\newline {\color{gray} (Emb: 0.701, TF-IDF: 0.306)}
\newline {\color{gray} \textbf{2º:} 482-3-Iniciativa-Convencional-Constituyente-de-la-cc-Francisca-Arauna-sobre-Derechos-de-Mujeres-rurales-2101-31-01.pdf}
\newline {\color{gray} (Emb: 0.682, TF-IDF: 0.261)}


\item \textbf{Artículo} \newline
Las regiones autónomas, para el cumplimiento de sus funciones, podrán establecer sus plantas de personal y las unidades de su estructura interna, en conformidad a la Constitución y la ley. 
\newline {\color{gray} \textbf{1º:} 159-3-c-Iniciativa-de-la-cc-Jennifer-Mella-.pdf}
\newline {\color{gray} (Emb: 0.710, TF-IDF: 0.378)}
\newline {\color{gray} \textbf{2º:} 122-3-c-Iniciativa-de-la-cc-Jennifer-Mella-Forma-del-Estado.pdf}
\newline {\color{gray} (Emb: 0.710, TF-IDF: 0.339)}

Estas facultades serán ejercidas por el Gobernador o Gobernadora Regional, previo acuerdo de la Asamblea Regional, cautelando la carrera funcionaria, su debido financiamiento y el carácter técnico y profesional de dichos empleos. 
\newline {\color{gray} \textbf{1º:} 43-3-Iniciativa-Convencional-Constituyente-de-la-cc-Tammy-Pustilnick-y-otros.pdf}
\newline {\color{gray} (Emb: 0.543, TF-IDF: 0.511)}
\newline {\color{gray} \textbf{2º:} 1023-Iniciativa-Convencional-Constituyente-de-la-cc-Carolina-Vilches-sobre-DDFF-con-perspectiva-Rural.pdf}
\newline {\color{gray} (Emb: 0.527, TF-IDF: 0.493)}


\item \textbf{Artículo} \newline
Son atribuciones de la Asamblea Regional, en conformidad a la Constitución, la ley y el Estatuto Regional: 1. 
\newline {\color{gray} \textbf{1º:} 159-3-c-Iniciativa-de-la-cc-Jennifer-Mella-.pdf}
\newline {\color{gray} (Emb: 0.626, TF-IDF: 0.483)}
\newline {\color{gray} \textbf{2º:} 122-3-c-Iniciativa-de-la-cc-Jennifer-Mella-Forma-del-Estado.pdf}
\newline {\color{gray} (Emb: 0.626, TF-IDF: 0.366)}

Pronunciarse sobre la convocatoria a consultas o plebiscitos regionales. 
\newline {\color{gray} \textbf{1º:} 120-3-c-Iniciativa-de-la-cc-Tammy-Pustilnick-atribuciones-exclusivas-de-la-Asamblea-Regional.pdf}
\newline {\color{gray} (Emb: 0.720, TF-IDF: 0.345)}
\newline {\color{gray} \textbf{2º:} 120-3-c-Iniciativa-de-la-cc-Tammy-Pustilnick-atribuciones-exclusivas-de-la-Asamblea-Regional.pdf}
\newline {\color{gray} (Emb: 0.659, TF-IDF: 0.341)}

Administrar sus bienes y patrimonio propio. 
\newline {\color{gray} \textbf{1º:} 394-3-Iniciativa-Convencional-Constituyente-de-la-cc-Ramona-Reyes-sobre-Comuna-Autonoma-1525-24-01.pdf}
\newline {\color{gray} (Emb: 0.600, TF-IDF: 0.574)}
\newline {\color{gray} \textbf{2º:} 560-Iniciativa-Convencional-Constituyente-de-cc-Andres-Cruz-sobre-Ministerio-Publico-2016-hrs.-01-02.pdf}
\newline {\color{gray} (Emb: 0.541, TF-IDF: 0.370)}

Aprobar, rechazar o modificar la inversión de los recursos de los fondos solidarios que se creen y otros recursos públicos que disponga la ley. 
\newline {\color{gray} \textbf{1º:} 871-Iniciativa-Convencional-Constituyente-de-la-cc-Amaya-Alvez-sobre-Asambleas-Regionales.pdf}
\newline {\color{gray} (Emb: 0.884, TF-IDF: 0.616)}
\newline {\color{gray} \textbf{2º:} 871-Iniciativa-Convencional-Constituyente-de-la-cc-Amaya-Alvez-sobre-Asambleas-Regionales.pdf}
\newline {\color{gray} (Emb: 0.810, TF-IDF: 0.598)}

Pronunciarse en conjunto con los órganos competentes respecto de los procedimientos de evaluación ambiental. 
\newline {\color{gray} \textbf{1º:} 907-Iniciativa-Convencional-Constituyente-del-cc-Hugo-Gutierrez-Sobre-Forma-del-Estado.pdf}
\newline {\color{gray} (Emb: 0.693, TF-IDF: 0.346)}
\newline {\color{gray} \textbf{2º:} 377-2-Iniciativa-Convencional-Constituyente-del-cc-Alvin-Saldana-sobre-Mecanismos-de-Democracia-1045-hrs-24-01.pdf}
\newline {\color{gray} (Emb: 0.689, TF-IDF: 0.335)}


\item \textbf{Artículo} \newline
El Estado y las entidades territoriales tienen el deber de ordenar y planificar el territorio nacional. 
\newline {\color{gray} \textbf{1º:} 1014-Iniciativa-Convencional-Constituyente-cc-Adriana-Ampuero-Haciendas-territoriales-y-autonomia-financiera.pdf}
\newline {\color{gray} (Emb: 0.607, TF-IDF: 0.379)}
\newline {\color{gray} \textbf{2º:} 606-Iniciativa-Convencional-Constituyente-de-cc-Marcos-Barraza-Renacionalziacion-Cobre-y-otro-Bienes-Publicos-Estrategicos.pdf}
\newline {\color{gray} (Emb: 0.526, TF-IDF: 0.379)}

Para esto utilizarán unidades de ordenación que consideren las cuencas hidrográficas. 
\newline {\color{gray} \textbf{1º:} 353-1-Iniciativa-Convencional-Constituyente-del-cc-Jaime-Bassa-sobre-Sistema-de-Gobierno-1158-21-01.pdf}
\newline {\color{gray} (Emb: 0.569, TF-IDF: 0.280)}
\newline {\color{gray} \textbf{2º:} 239-1-Iniciativa-Convencional-de-la-cc-Tania-Madriaga-sobre-Poder-Ejecutivo-1146-hrs.pdf}
\newline {\color{gray} (Emb: 0.569, TF-IDF: 0.255)}

Este deber tendrá como fin asegurar una adecuada localización de los asentamientos y las actividades productivas, que permitan tanto un manejo responsable de los ecosistemas como de las actividades humanas, con criterios de equidad y justicia territorial para el bienestar intergeneracional. 
\newline {\color{gray} \textbf{1º:} 914-Iniciativa-Convencional-Constituyente-del-cc-Luis-Jimenez-crea-la-Defensoria-de-la-Naturaleza.pdf}
\newline {\color{gray} (Emb: 0.661, TF-IDF: 0.362)}
\newline {\color{gray} \textbf{2º:} 971-Iniciativa-Convencional-Constituyente-de-la-cc-Ivanna-Olivares-sobre-Relaves.pdf}
\newline {\color{gray} (Emb: 0.519, TF-IDF: 0.331)}

La ordenación y planificación de los territorios será vinculante en las materias que la ley determine y realizada de manera coordinada, integrada y enfocada en el interés público, considerando procesos participativos en sus diferentes etapas. 
\newline {\color{gray} \textbf{1º:} 284-5-Iniciativa-Convencional-de-la-cc-Ivanna-Olivares-sobre-Proteccion-a-las-Pequenas-y-Medianas-Empresas-1209-hrs.pdf}
\newline {\color{gray} (Emb: 0.679, TF-IDF: 0.766)}
\newline {\color{gray} \textbf{2º:} 732-Iniciativa-Convencional-Constituyente-del-cc-Bastian-Labbe-crea-el-Servicio-de-proteccion-de-bienes-comunes.pdf}
\newline {\color{gray} (Emb: 0.653, TF-IDF: 0.364)}

Los planes de ordenamiento y planificación contemplarán los impactos que los usos de suelos causen en la disponibilidad y calidad de las aguas. 
\newline {\color{gray} \textbf{1º:} 954-5-Iniciativa-Convencional-Constituyente-de-la-cc-Carolina-Vilches-sobre-Estatuto-del-Agua.pdf}
\newline {\color{gray} (Emb: 0.585, TF-IDF: 0.435)}
\newline {\color{gray} \textbf{2º:} 983-Iniciativa-Convencional-Constituyente-de-la-cc-Carolina-Vilches-sobre-Territorio-suelos-y-Agua.pdf}
\newline {\color{gray} (Emb: 0.577, TF-IDF: 0.402)}

Estos podrán definir áreas de protección ambiental o cultural. 
\newline {\color{gray} \textbf{1º:} 692-Iniciativa-Convencional-Constituyente-del-cc-Pedro-Munoz-sobre-Agua-y-Buen-Vivir-120001-02.pdf}
\newline {\color{gray} (Emb: 0.621, TF-IDF: 0.428)}
\newline {\color{gray} \textbf{2º:} 554-Iniciativa-Convencional-Constituyente-del-cc-Pedro-Munoz-sobre-gestion-integrada-de-cuencas-1752-01-02.pdf}
\newline {\color{gray} (Emb: 0.621, TF-IDF: 0.402)}


\item \textbf{Artículo} \newline
En cada región existirá, al menos, una universidad estatal y una institución de formación técnico profesional de nivel superior estatal. 
\newline {\color{gray} \textbf{1º:} 120-3-c-Iniciativa-de-la-cc-Tammy-Pustilnick-atribuciones-exclusivas-de-la-Asamblea-Regional.pdf}
\newline {\color{gray} (Emb: 0.600, TF-IDF: 0.231)}
\newline {\color{gray} \textbf{2º:} 898-Iniciativa-Convencional-Constituyente-del-cc-Cesar-Uribe-sobre-Ordenamiento-y-Planificacion.pdf}
\newline {\color{gray} (Emb: 0.596, TF-IDF: 0.227)}

Estas se relacionarán de manera coordinada y preferente con las entidades territoriales y servicios públicos con presencia regional, de acuerdo a las necesidades locales. 
\newline {\color{gray} \textbf{1º:} 954-5-Iniciativa-Convencional-Constituyente-de-la-cc-Carolina-Vilches-sobre-Estatuto-del-Agua.pdf}
\newline {\color{gray} (Emb: 0.594, TF-IDF: 0.310)}
\newline {\color{gray} \textbf{2º:} 954-5-Iniciativa-Convencional-Constituyente-de-la-cc-Carolina-Vilches-sobre-Estatuto-del-Agua.pdf}
\newline {\color{gray} (Emb: 0.561, TF-IDF: 0.293)}


\item \textbf{Artículo} \newline
Los Cuerpos de Bomberos de Chile se sujetarán en todas sus actuaciones a los principios de probidad, transparencia y rendición de cuentas. 
\newline {\color{gray} \textbf{1º:} 315-5-Iniciativa-Convencional-de-la-cc-Isabel-Godoy-sobre-Derechos-de-la-Naturaleza-10-35-hrs.pdf}
\newline {\color{gray} (Emb: 0.465, TF-IDF: 0.223)}
\newline {\color{gray} \textbf{2º:} 890-Iniciativa-Convencional-Constituyente-del-cc-Hugo-Gutierrez-que-Asegura-la-vigencia-de-la-Constitucion.pdf}
\newline {\color{gray} (Emb: 0.464, TF-IDF: 0.196)}

La ley regulará el régimen de financiamiento y prestaciones sociales en época de vejez e invalidez. 
\newline {\color{gray} \textbf{1º:} 876-Iniciativa-Convencional-Constituyente-de-la-cc-Elisa-Giustinianovich-sobre-Transicion-Productiva.pdf}
\newline {\color{gray} (Emb: 0.706, TF-IDF: 0.248)}
\newline {\color{gray} \textbf{2º:} 122-3-c-Iniciativa-de-la-cc-Jennifer-Mella-Forma-del-Estado.pdf}
\newline {\color{gray} (Emb: 0.669, TF-IDF: 0.244)}

Los Cuerpos de Bomberos de Chile son una institución perteneciente al sistema de protección civil, cuyo objeto es atender las emergencias causadas por la naturaleza o el ser humano, sin perjuicio de la competencia específica que tengan otros organismos públicos y/o privados. 
\newline {\color{gray} \textbf{1º:} 304-4-Iniciativa-Convencional-de-la-cc-Valentina-Miranda-sobre-Derechos-Fundamentales-1301-hrs.pdf}
\newline {\color{gray} (Emb: 0.647, TF-IDF: 0.303)}
\newline {\color{gray} \textbf{2º:} 614-Iniciativa-Convencional-Constituyente-de-cc-Cesar-Uribe-sobre-Dano-Ambiental-y-Zonas-de-Sacrificio-2100-hrs.-01-02.pdf}
\newline {\color{gray} (Emb: 0.550, TF-IDF: 0.275)}

Será deber del Estado dar cobertura financiera para cubrir la totalidad de sus gastos operacionales, capacitación y equipos, como también otorgar cobertura médica a su personal por accidentes o enfermedades contraídas por actos de servicio. 
\newline {\color{gray} \textbf{1º:} 665-Iniciativa-Convencional-Constituyente-de-la-cc-Ramona-Reyes-sobre-Educacion-161101-02.pdf}
\newline {\color{gray} (Emb: 0.619, TF-IDF: 0.348)}
\newline {\color{gray} \textbf{2º:} 88-6-Iniciativa-Convencional-Constituyente-del-cc-Christian-Viera-y-otros.pdf}
\newline {\color{gray} (Emb: 0.537, TF-IDF: 0.327)}


\item \textbf{Artículo} \newline
Los derechos fundamentales son inherentes a la persona humana, universales, inalienables, indivisibles e interdependientes. 
\newline {\color{gray} \textbf{1º:} 587-Iniciativa-Convencional-Constituyente-de-cc-Marcos-Barraza-sobre-Derecho-al-Trabajo-2351-hrs.-01-02.pdf}
\newline {\color{gray} (Emb: 0.542, TF-IDF: 0.261)}
\newline {\color{gray} \textbf{2º:} 652-Iniciativa-Convencional-Constituyente-de-la-cc-Ericka-Portilla-sobre-Trabajo-Decente-151101-02.pdf}
\newline {\color{gray} (Emb: 0.542, TF-IDF: 0.261)}

El pleno ejercicio de estos derechos es esencial para la vida digna de las personas y los pueblos, la democracia, la paz y el equilibrio de la Naturaleza. 
\newline {\color{gray} \textbf{1º:} 914-Iniciativa-Convencional-Constituyente-del-cc-Luis-Jimenez-crea-la-Defensoria-de-la-Naturaleza.pdf}
\newline {\color{gray} (Emb: 0.651, TF-IDF: 0.324)}
\newline {\color{gray} \textbf{2º:} 387-4-Iniciativa-Convencional-Constituyente-de-la-cc-Francisca-Linconao-sobre-Derecho-a-la-Saud-1228-24-01.pdf}
\newline {\color{gray} (Emb: 0.641, TF-IDF: 0.283)}


\item \textbf{Artículo} \newline
El Estado debe respetar, proteger, garantizar y promover la plena satisfacción y ejercicio de los derechos fundamentales, así como adoptar las medidas necesarias para eliminar todos los obstáculos que pudieran limitar o entorpecer su realización. 
\newline {\color{gray} \textbf{1º:} 40-2-Iniciativa-Convencional-Constituyente-de-la-cc-Bernardo-de-la-Maza-y-otros.pdf}
\newline {\color{gray} (Emb: 0.773, TF-IDF: 0.553)}
\newline {\color{gray} \textbf{2º:} 70-2-Iniciativa-Convencional-Constituyente-de-la-cc-Paulina-Veloso-y-otros-2.pdf}
\newline {\color{gray} (Emb: 0.750, TF-IDF: 0.431)}

Toda persona, institución, asociación o grupo deberá respetar los derechos fundamentales, conforme a la Constitución y las leyes. 
\newline {\color{gray} \textbf{1º:} 12-4-Iniciativa-Convencional-Constituyente-de-la-cc-Tammy-Pustilnick-y-otros.pdf}
\newline {\color{gray} (Emb: 0.883, TF-IDF: 0.865)}
\newline {\color{gray} \textbf{2º:} 15-4-Iniciativa-Convencional-Constituyente-del-cc-Roberto-Celedón-y-otros.pdf}
\newline {\color{gray} (Emb: 0.791, TF-IDF: 0.512)}


\item \textbf{Artículo} \newline
El Estado debe adoptar todas las medidas necesarias para lograr de manera progresiva la plena satisfacción de los derechos fundamentales. 
\newline {\color{gray} \textbf{1º:} 1015-Iniciativa-Convencional-Constituyente-de-la-cc-Paulina-Valenzuela-sobre-Probidad.pdf}
\newline {\color{gray} (Emb: 0.714, TF-IDF: 0.380)}
\newline {\color{gray} \textbf{2º:} 11-4-Iniciativa-Convencional-Constituyente-de-la-cc-María-Elisa-Quinteros-y-otras.pdf}
\newline {\color{gray} (Emb: 0.700, TF-IDF: 0.365)}

Ninguna medida podrá tener un carácter regresivo que disminuya, menoscabe o impida injustificadamente su ejercicio. 
\newline {\color{gray} \textbf{1º:} 456-4-Iniciativa-Convencional-Constituyente-del-cc-Benito-Baranda-sobre-Derechos-de-las-personas-mayores-1729-31-01.pdf}
\newline {\color{gray} (Emb: 0.805, TF-IDF: 0.406)}
\newline {\color{gray} \textbf{2º:} 342-4-Iniciativa-Convencional-Constituyente-del-cc-Jorge-Baradit-sobre-Derechos-de-las-Personas-Mayores.pdf}
\newline {\color{gray} (Emb: 0.805, TF-IDF: 0.358)}


\item \textbf{Artículo} \newline
El financiamiento de las prestaciones estatales vinculadas al ejercicio de los derechos fundamentales propenderá a la progresividad. 
\newline {\color{gray} \textbf{1º:} 12-4-Iniciativa-Convencional-Constituyente-de-la-cc-Tammy-Pustilnick-y-otros.pdf}
\newline {\color{gray} (Emb: 0.803, TF-IDF: 0.502)}
\newline {\color{gray} \textbf{2º:} 12-4-Iniciativa-Convencional-Constituyente-de-la-cc-Tammy-Pustilnick-y-otros.pdf}
\newline {\color{gray} (Emb: 0.785, TF-IDF: 0.437)}


\item \textbf{Artículo} \newline
Las personas naturales son titulares de derechos fundamentales. 
\newline {\color{gray} \textbf{1º:} 652-Iniciativa-Convencional-Constituyente-de-la-cc-Ericka-Portilla-sobre-Trabajo-Decente-151101-02.pdf}
\newline {\color{gray} (Emb: 0.728, TF-IDF: 0.382)}
\newline {\color{gray} \textbf{2º:} 587-Iniciativa-Convencional-Constituyente-de-cc-Marcos-Barraza-sobre-Derecho-al-Trabajo-2351-hrs.-01-02.pdf}
\newline {\color{gray} (Emb: 0.728, TF-IDF: 0.352)}

Los derechos podrán ser ejercidos y exigidos individual o colectivamente. 
\newline {\color{gray} \textbf{1º:} 12-4-Iniciativa-Convencional-Constituyente-de-la-cc-Tammy-Pustilnick-y-otros.pdf}
\newline {\color{gray} (Emb: 0.684, TF-IDF: 0.581)}
\newline {\color{gray} \textbf{2º:} 256-4-Iniciativa-Convencional-de-la-cc-Elsa-Labrana-sobre-Parte-General-de-los-DDFF.pdf}
\newline {\color{gray} (Emb: 0.640, TF-IDF: 0.380)}

Los Pueblos y Naciones Indígenas son titulares de derechos fundamentales colectivos. 
\newline {\color{gray} \textbf{1º:} 11-4-Iniciativa-Convencional-Constituyente-de-la-cc-María-Elisa-Quinteros-y-otras.pdf}
\newline {\color{gray} (Emb: 0.727, TF-IDF: 0.529)}
\newline {\color{gray} \textbf{2º:} 797-Iniciativa-Convencional-Constituyente-de-la-cc-Lorena-Cespedes-sobre-Regimen-Publico-Economico.pdf}
\newline {\color{gray} (Emb: 0.629, TF-IDF: 0.249)}

La Naturaleza será titular de los derechos reconocidos en esta Constitución que le sean aplicables. 
\newline {\color{gray} \textbf{1º:} 14-4-Iniciativa-Convencional-Constituyente-de-la-cc-Aurora-Delgado-y-otros.pdf}
\newline {\color{gray} (Emb: 1.000, TF-IDF: 1.000)}
\newline {\color{gray} \textbf{2º:} 12-4-Iniciativa-Convencional-Constituyente-de-la-cc-Tammy-Pustilnick-y-otros.pdf}
\newline {\color{gray} (Emb: 1.000, TF-IDF: 1.000)}


\item \textbf{Artículo} \newline
Éstas no podrán perseguir fines de lucro y sus bienes deberán gestionarse de forma transparente de acuerdo con lo que establezca la ley. 
\newline {\color{gray} \textbf{1º:} 357-4-Iniciativa-Convencional-Constituyente-de-la-cc-Amaya-Alvez-sobre-minimo-vital-1254-hrs-21-01.pdf}
\newline {\color{gray} (Emb: 0.597, TF-IDF: 0.421)}
\newline {\color{gray} \textbf{2º:} 333-3-Iniciativa-Convencional-Constituyente-del-cc-Jorge-Arancibia-sobre-Autonomia-Territorial-y-Pueblos-Originarios.pdf}
\newline {\color{gray} (Emb: 0.505, TF-IDF: 0.344)}

Respetando los derechos, deberes y principios que esta Constitución establece. 
\newline {\color{gray} \textbf{1º:} 258-4-Iniciativa-Convencional-de-la-cc-Janis-Meneses-sobre-Libertad-de-Conciencia-y-Religion-1152-hrs.pdf}
\newline {\color{gray} (Emb: 0.735, TF-IDF: 0.600)}
\newline {\color{gray} \textbf{2º:} 277-2-Iniciativa-Convencional-de-la-cc-Ivanna-Olivares-sobre-Derecho-a-la-Espititualidad-y-a-la-Felicidad-1159-hrs.pdf}
\newline {\color{gray} (Emb: 0.625, TF-IDF: 0.329)}

Las agrupaciones religiosas y espirituales podrán organizarse como personas jurídicas de conformidad a la ley. 
\newline {\color{gray} \textbf{1º:} 59-2-Iniciativa-Convencional-Constituyente-de-la-cc-Ericka-Portilla-y-otros.pdf}
\newline {\color{gray} (Emb: 0.736, TF-IDF: 0.458)}
\newline {\color{gray} \textbf{2º:} 549-Iniciativa-Convencional-Constituyente-de-la-cc-Barbara-Sepulveda-sobre-libertad-de-conciencia-1701-01-02.pdf}
\newline {\color{gray} (Emb: 0.684, TF-IDF: 0.409)}

Ninguna religión, ni creencia es la oficial del Estado, sin perjuicio de su reconocimiento y libre ejercicio en el espacio público o privado, mediante el culto, la celebración de los ritos, las prácticas espirituales y la enseñanza. 
\newline {\color{gray} \textbf{1º:} 237-1-Iniciativa-Convencional-de-la-cc-Tania-Madriaga-sobre-Estado-Plurinacional-y-Libre-Determinacion-1146-hrs.pdf}
\newline {\color{gray} (Emb: 0.750, TF-IDF: 0.590)}
\newline {\color{gray} \textbf{2º:} 22-6-Iniciativa-Convencional-Constituyente-de-la-cc-Isabella-Mamani-y-otros.pdf}
\newline {\color{gray} (Emb: 0.709, TF-IDF: 0.583)}

Podrán erigir templos, dependencias y lugares para el culto; mantener, proteger y acceder a los lugares sagrados y aquellos de relevancia espiritual, rescatar y preservar los objetos de culto o que tengan un significado sagrado. 
\newline {\color{gray} \textbf{1º:} 14-4-Iniciativa-Convencional-Constituyente-de-la-cc-Aurora-Delgado-y-otros.pdf}
\newline {\color{gray} (Emb: 0.894, TF-IDF: 0.759)}
\newline {\color{gray} \textbf{2º:} 60-2-Iniciativa-Convencional-Constituyente-del-cc-Jorge-Baradit-y-otros.pdf}
\newline {\color{gray} (Emb: 0.764, TF-IDF: 0.387)}

Toda persona tiene derecho a la libertad de pensamiento, de conciencia, de religión y cosmovisión; este derecho incluye la libertad de profesar y cambiar de religión o creencias. 
\newline {\color{gray} \textbf{1º:} 14-4-Iniciativa-Convencional-Constituyente-de-la-cc-Aurora-Delgado-y-otros.pdf}
\newline {\color{gray} (Emb: 0.905, TF-IDF: 0.847)}
\newline {\color{gray} \textbf{2º:} 12-4-Iniciativa-Convencional-Constituyente-de-la-cc-Tammy-Pustilnick-y-otros.pdf}
\newline {\color{gray} (Emb: 0.905, TF-IDF: 0.847)}

El Estado reconoce la espiritualidad como elemento esencial del ser humano. 
\newline {\color{gray} \textbf{1º:} 635-4-Iniciativa-Convencional-Constituyente-del-cc-Cristobal-Andrade-sobre-Libertad-de-Conciencia-1730-01-02.pdf}
\newline {\color{gray} (Emb: 0.886, TF-IDF: 0.670)}
\newline {\color{gray} \textbf{2º:} 251-4-Iniciativa-Convencional-del-cc-Patricio-Fernandez-sobre-Derechos-Civiles-y-Politicos1150-hrs.pdf}
\newline {\color{gray} (Emb: 0.856, TF-IDF: 0.666)}


\item \textbf{Artículo} \newline
Toda persona, natural o jurídica, tiene derecho a la libertad de expresión y opinión, en cualquier forma y por cualquier medio. 
\newline {\color{gray} \textbf{1º:} 635-4-Iniciativa-Convencional-Constituyente-del-cc-Cristobal-Andrade-sobre-Libertad-de-Conciencia-1730-01-02.pdf}
\newline {\color{gray} (Emb: 0.852, TF-IDF: 0.455)}
\newline {\color{gray} \textbf{2º:} 252-4-Iniciativa-Convencional-de-la-cc-Tatiana-Urrutia-sobre-Libertad-de-Conciencia-y-Culto-1149-hrs.pdf}
\newline {\color{gray} (Emb: 0.602, TF-IDF: 0.262)}

Este derecho comprende la libertad de buscar, recibir y difundir informaciones e ideas de toda índole. 
\newline {\color{gray} \textbf{1º:} 807-Iniciativa-Convencional-Constituyente-del-cc-Jaime-Bassa-sobre-Formacion-de-la-Ley.pdf}
\newline {\color{gray} (Emb: 0.811, TF-IDF: 0.457)}
\newline {\color{gray} \textbf{2º:} 256-4-Iniciativa-Convencional-de-la-cc-Elsa-Labrana-sobre-Parte-General-de-los-DDFF.pdf}
\newline {\color{gray} (Emb: 0.770, TF-IDF: 0.335)}

No existirá censura previa sino únicamente las responsabilidades ulteriores que determine la ley. 
\newline {\color{gray} \textbf{1º:} 635-4-Iniciativa-Convencional-Constituyente-del-cc-Cristobal-Andrade-sobre-Libertad-de-Conciencia-1730-01-02.pdf}
\newline {\color{gray} (Emb: 0.704, TF-IDF: 0.566)}
\newline {\color{gray} \textbf{2º:} 865-Iniciativa-Convencional-Constituyente-del-cc-Marcos-Barraza-sobre-Seguridad-Publica.pdf}
\newline {\color{gray} (Emb: 0.584, TF-IDF: 0.356)}


\item \textbf{Artículo} \newline
Es deber del Estado proteger en forma equitativa el ejercicio de este derecho a todas las personas, a través de una política de prevención de la violencia y el delito que considerará especialmente las condiciones materiales, ambientales, sociales y el fortalecimiento comunitario de los territorios. 
\newline {\color{gray} \textbf{1º:} 251-4-Iniciativa-Convencional-del-cc-Patricio-Fernandez-sobre-Derechos-Civiles-y-Politicos1150-hrs.pdf}
\newline {\color{gray} (Emb: 0.835, TF-IDF: 0.570)}
\newline {\color{gray} \textbf{2º:} 261-4-Iniciativa-Convencional-de-la-cc-Janis-Meneses-sobre-Derecho-a-la-Libertad-de-Expresion-1152-hrs.pdf}
\newline {\color{gray} (Emb: 0.808, TF-IDF: 0.570)}

Las acciones de prevención, persecución y sanción de los delitos, así como la reinserción social de las personas condenadas, serán desarrolladas por los organismos públicos que señale esta Constitución y la ley, en forma coordinada y con irrestricto respeto a los derechos humanos. 
\newline {\color{gray} \textbf{1º:} 523-4-Iniciativa-Convencional-Constituyente-de-cc-Felipe-Harboe-sobre-Libertad-de-Expresion-e-informacion-1130-hrs.-01-02.pdf}
\newline {\color{gray} (Emb: 0.800, TF-IDF: 0.640)}
\newline {\color{gray} \textbf{2º:} 262-4-Iniciativa-Convencional-de-la-cc-Janis-Meneses-sobre-Derecho-a-la-Libertad-de-Opinion-1153-hrs.pdf}
\newline {\color{gray} (Emb: 0.782, TF-IDF: 0.615)}


\item \textbf{Artículo} \newline
(Inciso cuarto) Ninguna persona puede ser arrestada o detenida, sujeta a prisión preventiva o presa, sino en su domicilio o en los lugares públicos destinados a este objeto. 
\newline {\color{gray} \textbf{1º:} 131-4-c-Iniciativa-de-la-cc-Rocio-Cantuarias-Establece-la-Libertad-Personal-y-la-Seguridad-Individual.pdf}
\newline {\color{gray} (Emb: 0.673, TF-IDF: 0.319)}
\newline {\color{gray} \textbf{2º:} 440-Iniciativa-Convencional-Constituyente-del-cc-Felipe-Harboe-sobre-Principios-del-Debido-Proceso-1401-28-01.pdf}
\newline {\color{gray} (Emb: 0.608, TF-IDF: 0.319)}

(Inciso sexto) No existirá la detención por deudas, salvo en caso de incumplimiento de deberes alimentarios. 
\newline {\color{gray} \textbf{1º:} 271-4-Iniciativa-Convencional-de-la-cc-Natalia-Henriquez-sobre-Libertad-Personal-17-01-1154-hrs.pdf}
\newline {\color{gray} (Emb: 0.667, TF-IDF: 0.314)}
\newline {\color{gray} \textbf{2º:} 850-Iniciativa-Convencional-Constituyente-de-la-cc-Patricia-Labra-sobre-Ministerio-Publico.pdf}
\newline {\color{gray} (Emb: 0.642, TF-IDF: 0.271)}

Su ingreso deberá constar en un registro público. 
\newline {\color{gray} \textbf{1º:} 261-4-Iniciativa-Convencional-de-la-cc-Janis-Meneses-sobre-Derecho-a-la-Libertad-de-Expresion-1152-hrs.pdf}
\newline {\color{gray} (Emb: 0.647, TF-IDF: 0.277)}
\newline {\color{gray} \textbf{2º:} 440-Iniciativa-Convencional-Constituyente-del-cc-Felipe-Harboe-sobre-Principios-del-Debido-Proceso-1401-28-01.pdf}
\newline {\color{gray} (Emb: 0.637, TF-IDF: 0.277)}

Además, tendrá derecho a comunicarse con su abogado o quien estime pertinente. 
\newline {\color{gray} \textbf{1º:} 514-4-Iniciativa-Convencional-Constituyente-del-cc-Felipe-Harboe-sobre-Derecho-a-la-Privacidad-1245-01-02.pdf}
\newline {\color{gray} (Emb: 0.848, TF-IDF: 0.451)}
\newline {\color{gray} \textbf{2º:} 440-Iniciativa-Convencional-Constituyente-del-cc-Felipe-Harboe-sobre-Principios-del-Debido-Proceso-1401-28-01.pdf}
\newline {\color{gray} (Emb: 0.848, TF-IDF: 0.304)}

(Inciso tercero) La persona arrestada o detenida deberá ser puesta a disposición del tribunal competente en un plazo máximo de veinticuatro horas. 
\newline {\color{gray} \textbf{1º:} 303-4-Iniciativa-Convencional-del-cc-Cesar-Valenzuela-sobre-Derecho-a-Vivir-en-un-entorno-seguro-1301-hrs.pdf}
\newline {\color{gray} (Emb: 0.992, TF-IDF: 0.904)}
\newline {\color{gray} \textbf{2º:} 1003-Iniciativa-Convencional-Constituyente-del-cc-Luis-Mayol-sobre-Derecho-de-las-Personas-Privadas-de-Libertad.pdf}
\newline {\color{gray} (Emb: 0.704, TF-IDF: 0.308)}

Ninguna persona puede ser arrestada o detenida sino por orden judicial, salvo que fuera sorprendida en delito flagrante. 
\newline {\color{gray} \textbf{1º:} 244-4-Iniciativa-Convencional-de-la-cc-Elsa-Labrana-sobre-Derecho-a-la-Personalidad-1147-hrs.pdf}
\newline {\color{gray} (Emb: 0.620, TF-IDF: 0.499)}
\newline {\color{gray} \textbf{2º:} 303-4-Iniciativa-Convencional-del-cc-Cesar-Valenzuela-sobre-Derecho-a-Vivir-en-un-entorno-seguro-1301-hrs.pdf}
\newline {\color{gray} (Emb: 0.603, TF-IDF: 0.291)}

Ninguna persona puede ser privado de su libertad arbitrariamente ni ésta ser restringida sino en los casos y en la forma determinados por la Constitución y las leyes. 
\newline {\color{gray} \textbf{1º:} 548-Iniciativa-Convencional-Constituyente-de-la-cc-Patricia-Politzer-sobre-libertad-de-expresion-1653-01-02.pdf}
\newline {\color{gray} (Emb: 0.719, TF-IDF: 0.749)}
\newline {\color{gray} \textbf{2º:} 128-4-c-Iniciativa-de-la-cc-Rocio-Cantuarias-Incorpora-una-Garantia-Procesal-en-los-procesos-judiciales.pdf}
\newline {\color{gray} (Emb: 0.635, TF-IDF: 0.468)}

Se le deberán informar de manera inmediata y comprensible sus derechos y los motivos de la privación de su libertad. 
\newline {\color{gray} \textbf{1º:} 271-4-Iniciativa-Convencional-de-la-cc-Natalia-Henriquez-sobre-Libertad-Personal-17-01-1154-hrs.pdf}
\newline {\color{gray} (Emb: 0.986, TF-IDF: 0.925)}
\newline {\color{gray} \textbf{2º:} 131-4-c-Iniciativa-de-la-cc-Rocio-Cantuarias-Establece-la-Libertad-Personal-y-la-Seguridad-Individual.pdf}
\newline {\color{gray} (Emb: 0.969, TF-IDF: 0.771)}


\item \textbf{Artículo} \newline
Toda persona tiene derecho a trasladarse, residir y permanecer en cualquier lugar del territorio nacional, así como a entrar y salir de éste. 
\newline {\color{gray} \textbf{1º:} 271-4-Iniciativa-Convencional-de-la-cc-Natalia-Henriquez-sobre-Libertad-Personal-17-01-1154-hrs.pdf}
\newline {\color{gray} (Emb: 0.914, TF-IDF: 0.565)}
\newline {\color{gray} \textbf{2º:} 514-4-Iniciativa-Convencional-Constituyente-del-cc-Felipe-Harboe-sobre-Derecho-a-la-Privacidad-1245-01-02.pdf}
\newline {\color{gray} (Emb: 0.882, TF-IDF: 0.464)}

La ley regulará el ejercicio de este derecho. 
\newline {\color{gray} \textbf{1º:} 636-6-Iniciativa-Convencional-Constituyente-de-la-cc-Manuela-Royo-sobre-funcion-notarial-y-registral.pdf}
\newline {\color{gray} (Emb: 0.556, TF-IDF: 0.345)}
\newline {\color{gray} \textbf{2º:} 780-Iniciativa-Convencional-Csontituyente-del-cc-Helmuth-Martinez-sobre-Deberes-Constitucionales.pdf}
\newline {\color{gray} (Emb: 0.551, TF-IDF: 0.320)}


\item \textbf{Artículo} \newline
Ninguna persona será sometida a desplazamiento forzado dentro del territorio nacional, salvo en las excepciones que establezca la ley. 
\newline {\color{gray} \textbf{1º:} 1027-Iniciativa-Convencional-Consituyente-del-cc-Felipe-Harboe-sobre-Derecho-a-la-Vida.pdf}
\newline {\color{gray} (Emb: 0.611, TF-IDF: 0.347)}
\newline {\color{gray} \textbf{2º:} 271-4-Iniciativa-Convencional-de-la-cc-Natalia-Henriquez-sobre-Libertad-Personal-17-01-1154-hrs.pdf}
\newline {\color{gray} (Emb: 0.563, TF-IDF: 0.323)}


\item \textbf{Artículo} \newline
Toda persona tiene derecho al libre desarrollo y pleno reconocimiento de su identidad, en todas sus dimensiones y manifestaciones, incluyendo las características sexuales, identidades y expresiones de género, nombre y orientaciones sexoafectivas. 
\newline {\color{gray} \textbf{1º:} 271-4-Iniciativa-Convencional-de-la-cc-Natalia-Henriquez-sobre-Libertad-Personal-17-01-1154-hrs.pdf}
\newline {\color{gray} (Emb: 0.926, TF-IDF: 0.855)}
\newline {\color{gray} \textbf{2º:} 251-4-Iniciativa-Convencional-del-cc-Patricio-Fernandez-sobre-Derechos-Civiles-y-Politicos1150-hrs.pdf}
\newline {\color{gray} (Emb: 0.885, TF-IDF: 0.769)}

El Estado garantizará el pleno ejercicio de este derecho a través de acciones afirmativas, procedimientos y leyes correspondientes. 
\newline {\color{gray} \textbf{1º:} 515-4-Iniciativa-Convencional-Constituyente-de-la-cc-Giovanna-Grandon-sobre-Derecho-a-Migrar-1245-01-02.pdf}
\newline {\color{gray} (Emb: 0.917, TF-IDF: 0.692)}
\newline {\color{gray} \textbf{2º:} 1003-Iniciativa-Convencional-Constituyente-del-cc-Luis-Mayol-sobre-Derecho-de-las-Personas-Privadas-de-Libertad.pdf}
\newline {\color{gray} (Emb: 0.823, TF-IDF: 0.644)}


\item \textbf{Artículo} \newline
El Estado adoptará las medidas de prevención, sanción y erradicación de la esclavitud, el trabajo forzado, la servidumbre y la trata de personas, y de protección, plena restauración de derechos, remediación y reinserción social de las víctimas. 
\newline {\color{gray} \textbf{1º:} 519-4-Iniciativa-Convencional-Constituyente-de-cc-Carolina-Videla-sobre-DDHH-y-Garantias-de-no-repeticion-1257-hrs.-01-02.pdf}
\newline {\color{gray} (Emb: 0.771, TF-IDF: 0.311)}
\newline {\color{gray} \textbf{2º:} 451-4-Iniciativa-Convencional-Constituyente-de-la-cc-Carolina-Videla-sobre-Tortura-y-desaparicion-1409-31-01.pdf}
\newline {\color{gray} (Emb: 0.771, TF-IDF: 0.294)}

Toda persona tiene derecho a su autonomía personal, al libre desarrollo de la personalidad, identidad y de sus proyectos de vida. 
\newline {\color{gray} \textbf{1º:} 758-Iniciativa-Convencional-Constituyente-Irrenunciabilidad-de-la-Nacionalidad-Saldana.pdf}
\newline {\color{gray} (Emb: 0.724, TF-IDF: 0.474)}
\newline {\color{gray} \textbf{2º:} 443-Iniciativa-Convencional-Constituyente-de-la-cc-Giovanna-Grandon-sobre-Nacionalidad-1405-28-01.pdf}
\newline {\color{gray} (Emb: 0.693, TF-IDF: 0.406)}

Se prohíbe la esclavitud, el trabajo forzado, la servidumbre y la trata de personas en cualquiera de sus formas. 
\newline {\color{gray} \textbf{1º:} 304-4-Iniciativa-Convencional-de-la-cc-Valentina-Miranda-sobre-Derechos-Fundamentales-1301-hrs.pdf}
\newline {\color{gray} (Emb: 0.877, TF-IDF: 0.825)}
\newline {\color{gray} \textbf{2º:} 376-4-Iniciativa-Convencional-Constituyente-de-la-cc-Janis-Meneses-sobre-Educacion-sexual-integral-1040-hrs-24-01.pdf}
\newline {\color{gray} (Emb: 0.761, TF-IDF: 0.365)}


\item \textbf{Artículo} \newline
La ley establecerá las sanciones a los responsables. 
\newline {\color{gray} \textbf{1º:} 293-4-Iniciativa-Convencional-de-la-cc-Tatiana-Urrutia-sobre-Derecho-a-Desarrollar-actividades-economicas-1605-hrs.pdf}
\newline {\color{gray} (Emb: 0.719, TF-IDF: 0.707)}
\newline {\color{gray} \textbf{2º:} 902-Iniciativa-Convencional-Constituyente-del-cc-Daniel-Stingo-Sobre-SERNAC.pdf}
\newline {\color{gray} (Emb: 0.709, TF-IDF: 0.349)}

Las prácticas de colusión entre empresas y abusos de posición monopólica, así como las concentraciones empresariales que afecten el funcionamiento eficiente, justo y leal de los mercados, se entenderán como conductas contrarias al interés social. 
\newline {\color{gray} \textbf{1º:} 293-4-Iniciativa-Convencional-de-la-cc-Tatiana-Urrutia-sobre-Derecho-a-Desarrollar-actividades-economicas-1605-hrs.pdf}
\newline {\color{gray} (Emb: 0.752, TF-IDF: 0.791)}
\newline {\color{gray} \textbf{2º:} 251-4-Iniciativa-Convencional-del-cc-Patricio-Fernandez-sobre-Derechos-Civiles-y-Politicos1150-hrs.pdf}
\newline {\color{gray} (Emb: 0.692, TF-IDF: 0.680)}

Su ejercicio deberá ser compatible con los derechos consagrados en esta Constitución y con la protección de la naturaleza. 
\newline {\color{gray} \textbf{1º:} 380-4-Iniciativa-Convencional-Constituyente-del-cc-Bastian-Labbe-sobre-el-Derecho-al-Trabajo1141-24-01.pdf}
\newline {\color{gray} (Emb: 0.883, TF-IDF: 0.708)}
\newline {\color{gray} \textbf{2º:} 587-Iniciativa-Convencional-Constituyente-de-cc-Marcos-Barraza-sobre-Derecho-al-Trabajo-2351-hrs.-01-02.pdf}
\newline {\color{gray} (Emb: 0.806, TF-IDF: 0.639)}

Toda persona, natural o jurídica, tiene libertad de emprender y desarrollar actividades económicas. 
\newline {\color{gray} \textbf{1º:} 533-Iniciativa-Convencional-Constituyente-del-cc-Felipe-Harboe-sobre-D°-a-la-Libertad-1625-hrs.-01-02-1.pdf}
\newline {\color{gray} (Emb: 0.888, TF-IDF: 0.611)}
\newline {\color{gray} \textbf{2º:} 318-4-Iniciativa-Convencional-del-cc-Cristian-Monckeberg-sobre-Orden-y-Seguridad16-00-hrs.pdf}
\newline {\color{gray} (Emb: 0.719, TF-IDF: 0.504)}

El contenido y los límites de este derecho serán determinados por las leyes que regulen su ejercicio, las que deberán promover el desarrollo de las empresas de menor tamaño y asegurarán la protección de los consumidores. 
\newline {\color{gray} \textbf{1º:} 451-4-Iniciativa-Convencional-Constituyente-de-la-cc-Carolina-Videla-sobre-Tortura-y-desaparicion-1409-31-01.pdf}
\newline {\color{gray} (Emb: 0.659, TF-IDF: 0.441)}
\newline {\color{gray} \textbf{2º:} 519-4-Iniciativa-Convencional-Constituyente-de-cc-Carolina-Videla-sobre-DDHH-y-Garantias-de-no-repeticion-1257-hrs.-01-02.pdf}
\newline {\color{gray} (Emb: 0.659, TF-IDF: 0.398)}


\item \textbf{Artículo} \newline
Toda forma de documentación y comunicación privada es inviolable, incluyendo sus metadatos. 
\newline {\color{gray} \textbf{1º:} 524-4-Iniciativa-Convencional-Constituyente-de-cc-Felipe-Harbor-sobre-DERECHO-A-LA-PRIVACIDAD-1154-hrs.-01-02.pdf}
\newline {\color{gray} (Emb: 1.000, TF-IDF: 1.000)}
\newline {\color{gray} \textbf{2º:} 458-4-Iniciativa-Convencional-Constituyente-del-cc-Felipe-Harboe-sobre-Derecho-a-la-Seguridad-Informatica-1857-31-01.pdf}
\newline {\color{gray} (Emb: 0.835, TF-IDF: 0.739)}

La protección, promoción y respeto del derecho a la privacidad de las personas, sus familias y comunidades. 
\newline {\color{gray} \textbf{1º:} 860-Iniciativa-Convencional-Constituyente-del-cc-Jorge-Abarca-sobre-Regimen-Publico-Economico.pdf}
\newline {\color{gray} (Emb: 0.685, TF-IDF: 0.658)}
\newline {\color{gray} \textbf{2º:} 293-4-Iniciativa-Convencional-de-la-cc-Tatiana-Urrutia-sobre-Derecho-a-Desarrollar-actividades-economicas-1605-hrs.pdf}
\newline {\color{gray} (Emb: 0.621, TF-IDF: 0.497)}

Ninguna persona ni autoridad podrá afectar, restringir o impedir el ejercicio del derecho a la privacidad salvo en los casos y formas que determine la ley. 
\newline {\color{gray} \textbf{1º:} 7-2-Iniciativa-Convencional-Constituyente-del-cc-Luis-Barceló-y-otros.pdf}
\newline {\color{gray} (Emb: 0.492, TF-IDF: 0.177)}
\newline {\color{gray} \textbf{2º:} 84-2-Iniciativa-Convencional-Constituyente-del-cc-Martin-Arrau-y-otros.pdf}
\newline {\color{gray} (Emb: 0.486, TF-IDF: 0.176)}

Los recintos privados son inviolables. 
\newline {\color{gray} \textbf{1º:} 784-niciativa-Convencional-Constituyente-de-la-cc-Damaris-Abarca-sobre-Derecho-a-vivir-en-un-ambiente-sano.pdf}
\newline {\color{gray} (Emb: 0.875, TF-IDF: 0.497)}
\newline {\color{gray} \textbf{2º:} 84-2-Iniciativa-Convencional-Constituyente-del-cc-Martin-Arrau-y-otros.pdf}
\newline {\color{gray} (Emb: 0.795, TF-IDF: 0.367)}

La entrada, registro o allanamiento sólo se podrán realizar con orden judicial previa dictada en los casos específicos y en la forma que determine la ley, salvo las hipótesis de flagrancia. 
\newline {\color{gray} \textbf{1º:} 524-4-Iniciativa-Convencional-Constituyente-de-cc-Felipe-Harbor-sobre-DERECHO-A-LA-PRIVACIDAD-1154-hrs.-01-02.pdf}
\newline {\color{gray} (Emb: 1.000, TF-IDF: 1.000)}
\newline {\color{gray} \textbf{2º:} 622-Iniciativa-Convencional-Constituyente-de-cc-Felipe-Harboe-Reconocimiento-y-proteccion-integral-de-derechos-de-NNA.pdf}
\newline {\color{gray} (Emb: 0.723, TF-IDF: 0.391)}

La interceptación, captura, apertura, registro o revisión sólo se podrá realizar con orden judicial previa dictada en la forma y para los casos específicos que determine la ley. 
\newline {\color{gray} \textbf{1º:} 524-4-Iniciativa-Convencional-Constituyente-de-cc-Felipe-Harbor-sobre-DERECHO-A-LA-PRIVACIDAD-1154-hrs.-01-02.pdf}
\newline {\color{gray} (Emb: 1.000, TF-IDF: 1.000)}
\newline {\color{gray} \textbf{2º:} 152-4-c-Iniciativa-del-cc-Bernardo-Fontaine-sobre-Derecho-de-propiedad.pdf}
\newline {\color{gray} (Emb: 0.641, TF-IDF: 0.515)}


\item \textbf{Artículo} \newline
El Estado reconoce y garantiza el derecho de las personas a beneficiarse del progreso científico para ejercer de manera libre, autónoma y no discriminatoria, sus derechos sexuales y reproductivos. 
\newline {\color{gray} \textbf{1º:} 221-4-Iniciativa-Convencional-de-la-cc-Ramona-Reyes-sobre-Derechos-sexuales-y-Reproductivos-2340-hrs.pdf}
\newline {\color{gray} (Emb: 0.917, TF-IDF: 0.784)}
\newline {\color{gray} \textbf{2º:} 994-Iniciativa-Convencional-Constituyente-de-la-cc-Malucha-Pinto-sobre-Salud-Sexual.pdf}
\newline {\color{gray} (Emb: 0.532, TF-IDF: 0.293)}

La ley regulará el ejercicio de estos derechos. 
\newline {\color{gray} \textbf{1º:} 221-4-Iniciativa-Convencional-de-la-cc-Ramona-Reyes-sobre-Derechos-sexuales-y-Reproductivos-2340-hrs.pdf}
\newline {\color{gray} (Emb: 0.812, TF-IDF: 0.580)}
\newline {\color{gray} \textbf{2º:} 736-Iniciativa-Convencional-Constituyente-del-cc-Eric-Chinga-sobre-Educacion-Plurinacional.pdf}
\newline {\color{gray} (Emb: 0.686, TF-IDF: 0.448)}

Asimismo, garantiza su ejercicio libre de violencias y de interferencias por parte de terceros, ya sean individuos o instituciones. 
\newline {\color{gray} \textbf{1º:} 221-4-Iniciativa-Convencional-de-la-cc-Ramona-Reyes-sobre-Derechos-sexuales-y-Reproductivos-2340-hrs.pdf}
\newline {\color{gray} (Emb: 1.000, TF-IDF: 1.000)}
\newline {\color{gray} \textbf{2º:} 994-Iniciativa-Convencional-Constituyente-de-la-cc-Malucha-Pinto-sobre-Salud-Sexual.pdf}
\newline {\color{gray} (Emb: 0.857, TF-IDF: 0.794)}

El Estado garantiza el ejercicio de los derechos sexuales y reproductivos sin discriminación, con enfoque de género, inclusión y pertinencia cultural, así como el acceso a la información, educación, salud, y a los servicios y prestaciones requeridos para ello, asegurando a todas las mujeres y personas con capacidad de gestar, las condiciones para un embarazo, una interrupción voluntaria del embarazo, parto y maternidad voluntarios y protegidos. 
\newline {\color{gray} \textbf{1º:} 524-4-Iniciativa-Convencional-Constituyente-de-cc-Felipe-Harbor-sobre-DERECHO-A-LA-PRIVACIDAD-1154-hrs.-01-02.pdf}
\newline {\color{gray} (Emb: 0.993, TF-IDF: 0.996)}
\newline {\color{gray} \textbf{2º:} 524-4-Iniciativa-Convencional-Constituyente-de-cc-Felipe-Harbor-sobre-DERECHO-A-LA-PRIVACIDAD-1154-hrs.-01-02.pdf}
\newline {\color{gray} (Emb: 0.907, TF-IDF: 0.617)}

Estos comprenden, entre otros, el derecho a decidir de forma libre, autónoma e informada sobre el propio cuerpo, sobre el ejercicio de la sexualidad, la reproducción, el placer y la anticoncepción. 
\newline {\color{gray} \textbf{1º:} 524-4-Iniciativa-Convencional-Constituyente-de-cc-Felipe-Harbor-sobre-DERECHO-A-LA-PRIVACIDAD-1154-hrs.-01-02.pdf}
\newline {\color{gray} (Emb: 0.864, TF-IDF: 0.555)}
\newline {\color{gray} \textbf{2º:} 125-4-c-Iniciativa-de-la-cc-Rocio-Cantuarias-establece-la-inviolabilidad-del-hogar.pdf}
\newline {\color{gray} (Emb: 0.487, TF-IDF: 0.399)}

Todas las personas son titulares de derechos sexuales y derechos reproductivos. 
\newline {\color{gray} \textbf{1º:} 524-4-Iniciativa-Convencional-Constituyente-de-cc-Felipe-Harbor-sobre-DERECHO-A-LA-PRIVACIDAD-1154-hrs.-01-02.pdf}
\newline {\color{gray} (Emb: 0.976, TF-IDF: 0.894)}
\newline {\color{gray} \textbf{2º:} 524-4-Iniciativa-Convencional-Constituyente-de-cc-Felipe-Harbor-sobre-DERECHO-A-LA-PRIVACIDAD-1154-hrs.-01-02.pdf}
\newline {\color{gray} (Emb: 0.907, TF-IDF: 0.556)}


\item \textbf{Artículo} \newline
Todas las personas tienen derecho a recibir una Educación Sexual Integral, que promueva el disfrute pleno y libre de la sexualidad; la responsabilidad sexo-afectiva; la autonomía, el autocuidado y el consentimiento; el reconocimiento de las diversas identidades y expresiones del género y la sexualidad; que erradique los estereotipos de género y prevenga la violencia de género y sexual. 
\newline {\color{gray} \textbf{1º:} 221-4-Iniciativa-Convencional-de-la-cc-Ramona-Reyes-sobre-Derechos-sexuales-y-Reproductivos-2340-hrs.pdf}
\newline {\color{gray} (Emb: 0.556, TF-IDF: 0.435)}
\newline {\color{gray} \textbf{2º:} 900-Iniciativa-Convencional-Constituyente-del-cc-Christian-Viera-sobre-Christian-Viera.pdf}
\newline {\color{gray} (Emb: 0.539, TF-IDF: 0.275)}


\item \textbf{Artículo} \newline
Toda persona, natural o jurídica, tiene derecho de propiedad en todas sus especies y sobre toda clase de bienes, salvo aquellos que la naturaleza ha hecho comunes a todas las personas y los que la Constitución o la ley declaren inapropiables. 
\newline {\color{gray} \textbf{1º:} 610-Iniciativa-Convencional-Constituyente-de-cc-Valentina-Miranda-Derechos-de-las-Personas-LGBTIQ-y-Derecho-a-la-Igualdad.pdf}
\newline {\color{gray} (Emb: 0.671, TF-IDF: 0.460)}
\newline {\color{gray} \textbf{2º:} 718-Iniciativa-Convencional-Constituyente-de-la-cc-Maria-Elisa-Quinteros-Personas-en-Situacion-de-Discapacidad-01-02.pdf}
\newline {\color{gray} (Emb: 0.627, TF-IDF: 0.398)}

(Inciso cuarto) Corresponderá a la ley determinar el modo de adquirir la propiedad, su contenido, límites y deberes, conforme con su función social y ecológica. 
\newline {\color{gray} \textbf{1º:} 515-4-Iniciativa-Convencional-Constituyente-de-la-cc-Giovanna-Grandon-sobre-Derecho-a-Migrar-1245-01-02.pdf}
\newline {\color{gray} (Emb: 0.887, TF-IDF: 0.679)}
\newline {\color{gray} \textbf{2º:} 1003-Iniciativa-Convencional-Constituyente-del-cc-Luis-Mayol-sobre-Derecho-de-las-Personas-Privadas-de-Libertad.pdf}
\newline {\color{gray} (Emb: 0.840, TF-IDF: 0.478)}


\item \textbf{Artículo} \newline
Ninguna persona puede ser privada de su propiedad, sino en virtud de una ley que autorice la expropiación por causa de utilidad pública o interés general declarado por el legislador. 
\newline {\color{gray} \textbf{1º:} 376-4-Iniciativa-Convencional-Constituyente-de-la-cc-Janis-Meneses-sobre-Educacion-sexual-integral-1040-hrs-24-01.pdf}
\newline {\color{gray} (Emb: 0.973, TF-IDF: 0.904)}
\newline {\color{gray} \textbf{2º:} 304-4-Iniciativa-Convencional-de-la-cc-Valentina-Miranda-sobre-Derechos-Fundamentales-1301-hrs.pdf}
\newline {\color{gray} (Emb: 0.750, TF-IDF: 0.306)}

El propietario siempre tendrá derecho a que se le indemnice por el justo precio del bien expropiado. 
\newline {\color{gray} \textbf{1º:} 264-4-Iniciativa-Convencional-de-la-cc-Janis-Meneses-sobre-Derecho-de-Propiedad-1154-hrs.pdf}
\newline {\color{gray} (Emb: 0.905, TF-IDF: 0.631)}
\newline {\color{gray} \textbf{2º:} 152-4-c-Iniciativa-del-cc-Bernardo-Fontaine-sobre-Derecho-de-propiedad.pdf}
\newline {\color{gray} (Emb: 0.865, TF-IDF: 0.609)}

El pago deberá efectuarse de forma previa a la toma de posesión material del bien expropiado y la persona expropiada siempre podrá reclamar de la legalidad del acto expropiatorio, así como del monto y modalidad de pago ante los tribunales que determine la ley. 
\newline {\color{gray} \textbf{1º:} 293-4-Iniciativa-Convencional-de-la-cc-Tatiana-Urrutia-sobre-Derecho-a-Desarrollar-actividades-economicas-1605-hrs.pdf}
\newline {\color{gray} (Emb: 0.935, TF-IDF: 0.796)}
\newline {\color{gray} \textbf{2º:} 455-5-Iniciativa-Convencional-Constituyente-de-la-cc-Bessy-Gallardo-sobre-Derecho-de-Propiedad-1529-31-01.pdf}
\newline {\color{gray} (Emb: 0.738, TF-IDF: 0.455)}

Cualquiera sea la causa invocada para llevar a cabo la expropiación siempre deberá estar debidamente fundada. 
\newline {\color{gray} \textbf{1º:} 293-4-Iniciativa-Convencional-de-la-cc-Tatiana-Urrutia-sobre-Derecho-a-Desarrollar-actividades-economicas-1605-hrs.pdf}
\newline {\color{gray} (Emb: 0.976, TF-IDF: 0.786)}
\newline {\color{gray} \textbf{2º:} 251-4-Iniciativa-Convencional-del-cc-Patricio-Fernandez-sobre-Derechos-Civiles-y-Politicos1150-hrs.pdf}
\newline {\color{gray} (Emb: 0.901, TF-IDF: 0.639)}


\item \textbf{Artículo} \newline
La restitución constituye un mecanismo preferente de reparación, de utilidad pública e interés general. 
\newline {\color{gray} \textbf{1º:} 256-4-Iniciativa-Convencional-de-la-cc-Elsa-Labrana-sobre-Parte-General-de-los-DDFF.pdf}
\newline {\color{gray} (Emb: 0.740, TF-IDF: 0.426)}
\newline {\color{gray} \textbf{2º:} 237-1-Iniciativa-Convencional-de-la-cc-Tania-Madriaga-sobre-Estado-Plurinacional-y-Libre-Determinacion-1146-hrs.pdf}
\newline {\color{gray} (Emb: 0.699, TF-IDF: 0.413)}

El Estado establecerá instrumentos jurídicos eficaces para su catastro, regularización, demarcación, titulación, reparación y restitución. 
\newline {\color{gray} \textbf{1º:} 41-6-Iniciativa-Convencional-Constituyente-del-cc-Mauricio-Daza-y-otros-1.pdf}
\newline {\color{gray} (Emb: 0.713, TF-IDF: 0.318)}
\newline {\color{gray} \textbf{2º:} 432-1-Iniciativa-Convencional-de-la-cc-Vanessa-Hoppe-sobre-Rol-de-las-Fuerzas-Armadas-1145-27-01.pdf}
\newline {\color{gray} (Emb: 0.627, TF-IDF: 0.263)}

Conforme a la constitución y la ley, los pueblos y naciones indígenas tienen derecho a utilizar los recursos que tradicionalmente han usado u ocupado, que se encuentran en sus territorios y sean indispensables para su existencia colectiva. 
\newline {\color{gray} \textbf{1º:} 732-Iniciativa-Convencional-Constituyente-del-cc-Bastian-Labbe-crea-el-Servicio-de-proteccion-de-bienes-comunes.pdf}
\newline {\color{gray} (Emb: 0.725, TF-IDF: 0.359)}
\newline {\color{gray} \textbf{2º:} 737-Iniciativa-Convencional-Constituyente-del-cc-Francisco-Caamano-sobre-Economia-en-Territorios-Rurales.pdf}
\newline {\color{gray} (Emb: 0.712, TF-IDF: 0.309)}

El Estado reconoce y garantiza conforme a la Constitución, el derecho de los pueblos y naciones indígenas a sus tierras, territorios y recursos. 
\newline {\color{gray} \textbf{1º:} 251-4-Iniciativa-Convencional-del-cc-Patricio-Fernandez-sobre-Derechos-Civiles-y-Politicos1150-hrs.pdf}
\newline {\color{gray} (Emb: 0.700, TF-IDF: 0.360)}
\newline {\color{gray} \textbf{2º:} 239-1-Iniciativa-Convencional-de-la-cc-Tania-Madriaga-sobre-Poder-Ejecutivo-1146-hrs.pdf}
\newline {\color{gray} (Emb: 0.666, TF-IDF: 0.303)}

La propiedad de las tierras indígenas goza de especial protección. 
\newline {\color{gray} \textbf{1º:} 1008-Iniciativa-Convencional-Constituyente-de-la-cc-Claudia-Castro-sobre-Propiedad.pdf}
\newline {\color{gray} (Emb: 0.704, TF-IDF: 0.403)}
\newline {\color{gray} \textbf{2º:} 455-5-Iniciativa-Convencional-Constituyente-de-la-cc-Bessy-Gallardo-sobre-Derecho-de-Propiedad-1529-31-01.pdf}
\newline {\color{gray} (Emb: 0.695, TF-IDF: 0.396)}


\item \textbf{Artículo} \newline
Toda persona tiene derecho a la vida. 
\newline {\color{gray} \textbf{1º:} 903-Iniciativa-Convencional-Constituyente-del-cc-Fernando-Salinas-Sobre-Buen-Vivir.pdf}
\newline {\color{gray} (Emb: 0.627, TF-IDF: 0.438)}
\newline {\color{gray} \textbf{2º:} 853-Iniciativa-Convencional-Constituyente-del-cc-Adolfo-Millabur-sobre-Reintegro-de-Tierras-a-PPOO.pdf}
\newline {\color{gray} (Emb: 0.612, TF-IDF: 0.412)}

Ninguna persona podrá ser condenada a muerte ni ejecutada. 
\newline {\color{gray} \textbf{1º:} 527-5-Iniciativa-Convencional-Constituyente-de-cc-Marcos-Barraza-sobre-Renacionalizacion-del-Cobre-1516-hrs.-01-02.pdf}
\newline {\color{gray} (Emb: 0.518, TF-IDF: 0.292)}
\newline {\color{gray} \textbf{2º:} 606-Iniciativa-Convencional-Constituyente-de-cc-Marcos-Barraza-Renacionalziacion-Cobre-y-otro-Bienes-Publicos-Estrategicos.pdf}
\newline {\color{gray} (Emb: 0.518, TF-IDF: 0.285)}


\item \textbf{Artículo} \newline
Toda persona tiene derecho a la integridad física, psicosocial, sexual y afectiva. 
\newline {\color{gray} \textbf{1º:} 315-5-Iniciativa-Convencional-de-la-cc-Isabel-Godoy-sobre-Derechos-de-la-Naturaleza-10-35-hrs.pdf}
\newline {\color{gray} (Emb: 0.663, TF-IDF: 0.431)}
\newline {\color{gray} \textbf{2º:} 74-4-Iniciativa-Convencional-Constituyente-de-la-cc-Natividad-Llanquileo-y-otros.pdf}
\newline {\color{gray} (Emb: 0.662, TF-IDF: 0.393)}

Ninguna persona podrá ser sometida a torturas, ni penas o tratos crueles, inhumanos o degradantes. 
\newline {\color{gray} \textbf{1º:} 273-4-Iniciativa-Convencional-de-la-cc-Natalia-Henriquez-sobre-Buen-Morir-17-01-1154-hrs.pdf}
\newline {\color{gray} (Emb: 1.000, TF-IDF: 1.000)}
\newline {\color{gray} \textbf{2º:} 148-4-c-Iniciativa-del-cc-Manuel-Jose-Ossandon-Derecho-a-la-Vida.pdf}
\newline {\color{gray} (Emb: 1.000, TF-IDF: 1.000)}


\item \textbf{Artículo} \newline
Ninguna persona será sometida a desaparición forzada. 
\newline {\color{gray} \textbf{1º:} 273-4-Iniciativa-Convencional-de-la-cc-Natalia-Henriquez-sobre-Buen-Morir-17-01-1154-hrs.pdf}
\newline {\color{gray} (Emb: 1.000, TF-IDF: 1.000)}
\newline {\color{gray} \textbf{2º:} 128-4-c-Iniciativa-de-la-cc-Rocio-Cantuarias-Incorpora-una-Garantia-Procesal-en-los-procesos-judiciales.pdf}
\newline {\color{gray} (Emb: 0.716, TF-IDF: 0.296)}

Toda persona víctima de desaparición forzada tiene derecho a ser buscada. 
\newline {\color{gray} \textbf{1º:} 273-4-Iniciativa-Convencional-de-la-cc-Natalia-Henriquez-sobre-Buen-Morir-17-01-1154-hrs.pdf}
\newline {\color{gray} (Emb: 1.000, TF-IDF: 1.000)}
\newline {\color{gray} \textbf{2º:} 162-4-c-Iniciativa-de-la-cc-Rocio-Cantuarias-Derecho-a-la-Vida.pdf}
\newline {\color{gray} (Emb: 0.841, TF-IDF: 0.473)}

El Estado garantizará el ejercicio de este derecho, disponiendo de todos los medios necesarios. 
\newline {\color{gray} \textbf{1º:} 273-4-Iniciativa-Convencional-de-la-cc-Natalia-Henriquez-sobre-Buen-Morir-17-01-1154-hrs.pdf}
\newline {\color{gray} (Emb: 1.000, TF-IDF: 1.000)}
\newline {\color{gray} \textbf{2º:} 1027-Iniciativa-Convencional-Consituyente-del-cc-Felipe-Harboe-sobre-Derecho-a-la-Vida.pdf}
\newline {\color{gray} (Emb: 0.965, TF-IDF: 0.969)}


\item \textbf{Artículo} \newline
Los crímenes de guerra, los delitos de lesa humanidad, la desaparición forzada y la tortura, el genocidio y el crimen de agresión y otras penas o tratos crueles, inhumanos o degradantes son imprescriptibles, inamnistiables y no serán susceptibles de ningún impedimento a la investigación. 
\newline {\color{gray} \textbf{1º:} 451-4-Iniciativa-Convencional-Constituyente-de-la-cc-Carolina-Videla-sobre-Tortura-y-desaparicion-1409-31-01.pdf}
\newline {\color{gray} (Emb: 1.000, TF-IDF: 1.000)}
\newline {\color{gray} \textbf{2º:} 519-4-Iniciativa-Convencional-Constituyente-de-cc-Carolina-Videla-sobre-DDHH-y-Garantias-de-no-repeticion-1257-hrs.-01-02.pdf}
\newline {\color{gray} (Emb: 1.000, TF-IDF: 1.000)}


\item \textbf{Artículo} \newline
Son obligaciones del Estado prevenir, investigar, sancionar e impedir la impunidad de los hechos establecidos en el artículo 26. 
\newline {\color{gray} \textbf{1º:} 451-4-Iniciativa-Convencional-Constituyente-de-la-cc-Carolina-Videla-sobre-Tortura-y-desaparicion-1409-31-01.pdf}
\newline {\color{gray} (Emb: 1.000, TF-IDF: 1.000)}
\newline {\color{gray} \textbf{2º:} 519-4-Iniciativa-Convencional-Constituyente-de-cc-Carolina-Videla-sobre-DDHH-y-Garantias-de-no-repeticion-1257-hrs.-01-02.pdf}
\newline {\color{gray} (Emb: 1.000, TF-IDF: 1.000)}

Tales crímenes deberán ser investigados de oficio, con la debida diligencia, seriedad, rapidez, independencia, imparcialidad y en conformidad con los estándares establecidos en los tratados internacionales ratificados y vigentes en Chile. 
\newline {\color{gray} \textbf{1º:} 451-4-Iniciativa-Convencional-Constituyente-de-la-cc-Carolina-Videla-sobre-Tortura-y-desaparicion-1409-31-01.pdf}
\newline {\color{gray} (Emb: 1.000, TF-IDF: 1.000)}
\newline {\color{gray} \textbf{2º:} 519-4-Iniciativa-Convencional-Constituyente-de-cc-Carolina-Videla-sobre-DDHH-y-Garantias-de-no-repeticion-1257-hrs.-01-02.pdf}
\newline {\color{gray} (Emb: 1.000, TF-IDF: 1.000)}


\item \textbf{Artículo} \newline
Todas las personas tienen derecho a reunirse y manifestarse en lugares privados y públicos, sin permiso previo. 
\newline {\color{gray} \textbf{1º:} 666-Iniciativa-Convencional-Constituyente-de-la-cc-Tatiana-Urrutia-sobre-Desaparicion-Forzada-121101-02-1.pdf}
\newline {\color{gray} (Emb: 0.952, TF-IDF: 0.829)}
\newline {\color{gray} \textbf{2º:} 451-4-Iniciativa-Convencional-Constituyente-de-la-cc-Carolina-Videla-sobre-Tortura-y-desaparicion-1409-31-01.pdf}
\newline {\color{gray} (Emb: 0.806, TF-IDF: 0.624)}

Las reuniones en lugares de acceso público sólo podrán restringirse en conformidad a la ley. 
\newline {\color{gray} \textbf{1º:} 909-Iniciativa-Convencional-Constituyente-del-cc-Hugo-Gutierrez-Sobre-Ministerio-Publico.pdf}
\newline {\color{gray} (Emb: 0.536, TF-IDF: 0.513)}
\newline {\color{gray} \textbf{2º:} 519-4-Iniciativa-Convencional-Constituyente-de-cc-Carolina-Videla-sobre-DDHH-y-Garantias-de-no-repeticion-1257-hrs.-01-02.pdf}
\newline {\color{gray} (Emb: 0.502, TF-IDF: 0.417)}


\item \textbf{Artículo} \newline
La ley podrá imponer restricciones específicas al ejercicio de este derecho respecto de las policías y fuerzas armadas. 
\newline {\color{gray} \textbf{1º:} 257-4-Iniciativa-Convencional-de-la-cc-Maria-Magdalena-Rivera-sobre-Derecho-a-Huelga-Solidaria-17-01-1151-hrs.pdf}
\newline {\color{gray} (Emb: 0.744, TF-IDF: 0.621)}
\newline {\color{gray} \textbf{2º:} 146-4-c-Iniciativa-del-cc-Manuel-Jose-Ossandon-Libertad-de-asociacion-y-derecho-a-la-sindicalizacion.pdf}
\newline {\color{gray} (Emb: 0.738, TF-IDF: 0.537)}

Para gozar de personalidad jurídica, las asociaciones deberán constituirse en conformidad a la ley. 
\newline {\color{gray} \textbf{1º:} 256-4-Iniciativa-Convencional-de-la-cc-Elsa-Labrana-sobre-Parte-General-de-los-DDFF.pdf}
\newline {\color{gray} (Emb: 0.608, TF-IDF: 0.466)}
\newline {\color{gray} \textbf{2º:} 133-4-c-Iniciativa-de-la-cc-Rocio-Cantuarias-Incorpora-la-Libertad-de-Reunion.pdf}
\newline {\color{gray} (Emb: 0.602, TF-IDF: 0.395)}

Todas las personas tienen derecho a asociarse, sin permiso previo. 
\newline {\color{gray} \textbf{1º:} 666-Iniciativa-Convencional-Constituyente-de-la-cc-Tatiana-Urrutia-sobre-Desaparicion-Forzada-121101-02-1.pdf}
\newline {\color{gray} (Emb: 0.887, TF-IDF: 0.793)}
\newline {\color{gray} \textbf{2º:} 451-4-Iniciativa-Convencional-Constituyente-de-la-cc-Carolina-Videla-sobre-Tortura-y-desaparicion-1409-31-01.pdf}
\newline {\color{gray} (Emb: 0.743, TF-IDF: 0.641)}

El derecho de asociación comprende la protección de la autonomía de las asociaciones para el cumplimiento de sus fines específicos y el establecimiento de su regulación interna, organización y demás elementos definitorios. 
\newline {\color{gray} \textbf{1º:} 291-4-Iniciativa-Convencional-de-la-cc-Tatiana-Urrutia-sobre-Derecho-de-Reunion-Peticion-y-Asociacion-1605-hrs.pdf}
\newline {\color{gray} (Emb: 0.873, TF-IDF: 0.898)}
\newline {\color{gray} \textbf{2º:} 285-7-Iniciativa-Convencional-de-la-cc-Ivanna-Olivares-sobre-Derecho-de-Prensa-1210-hrs.pdf}
\newline {\color{gray} (Emb: 0.756, TF-IDF: 0.579)}


\item \textbf{Artículo} \newline
El Estado reconoce la función social, económica y productiva de las cooperativas, conforme al principio de ayuda mutua, y fomentará su desarrollo. 
\newline {\color{gray} \textbf{1º:} 291-4-Iniciativa-Convencional-de-la-cc-Tatiana-Urrutia-sobre-Derecho-de-Reunion-Peticion-y-Asociacion-1605-hrs.pdf}
\newline {\color{gray} (Emb: 1.000, TF-IDF: 1.000)}
\newline {\color{gray} \textbf{2º:} 141-4-c-Iniciativa-de-la-cc-Rocio-Cantuarias-Incorpora-la-Libertad-de-asociacion.pdf}
\newline {\color{gray} (Emb: 0.721, TF-IDF: 0.573)}

La ley regulará la creación y funcionamiento de las cooperativas, garantizará su autonomía, y preservará, mediante los instrumentos correspondientes, su naturaleza y finalidades. 
\newline {\color{gray} \textbf{1º:} 141-4-c-Iniciativa-de-la-cc-Rocio-Cantuarias-Incorpora-la-Libertad-de-asociacion.pdf}
\newline {\color{gray} (Emb: 1.000, TF-IDF: 1.000)}
\newline {\color{gray} \textbf{2º:} 146-4-c-Iniciativa-del-cc-Manuel-Jose-Ossandon-Libertad-de-asociacion-y-derecho-a-la-sindicalizacion.pdf}
\newline {\color{gray} (Emb: 1.000, TF-IDF: 1.000)}

Las cooperativas podrán agruparse en federaciones, confederaciones, o en otras formas de organización que determine la ley. 
\newline {\color{gray} \textbf{1º:} 291-4-Iniciativa-Convencional-de-la-cc-Tatiana-Urrutia-sobre-Derecho-de-Reunion-Peticion-y-Asociacion-1605-hrs.pdf}
\newline {\color{gray} (Emb: 0.877, TF-IDF: 0.729)}
\newline {\color{gray} \textbf{2º:} 146-4-c-Iniciativa-del-cc-Manuel-Jose-Ossandon-Libertad-de-asociacion-y-derecho-a-la-sindicalizacion.pdf}
\newline {\color{gray} (Emb: 0.733, TF-IDF: 0.371)}


\item \textbf{Artículo} \newline
Las personas chilenas que se encuentren en el exterior forman parte de la comunidad política del país. 
\newline {\color{gray} \textbf{1º:} 422-4-Iniciativa-Convencional-de-la-cc-Maria-Trinidad-Castillo-sobre-Reconocimiento-de-las-cooperativas-1319-26-01.pdf}
\newline {\color{gray} (Emb: 0.909, TF-IDF: 0.977)}
\newline {\color{gray} \textbf{2º:} 308-4-Iniciativa-Convencional-del-cc-Fuad-Chahin-sobre-Cooperativas-1525-hrs.pdf}
\newline {\color{gray} (Emb: 0.907, TF-IDF: 0.976)}

Se garantiza el derecho a votar en las elecciones de carácter nacional, presidenciales, parlamentarias, plebiscitos y consultas, de conformidad a esta Constitución y las leyes. 
\newline {\color{gray} \textbf{1º:} 308-4-Iniciativa-Convencional-del-cc-Fuad-Chahin-sobre-Cooperativas-1525-hrs.pdf}
\newline {\color{gray} (Emb: 1.000, TF-IDF: 1.000)}
\newline {\color{gray} \textbf{2º:} 422-4-Iniciativa-Convencional-de-la-cc-Maria-Trinidad-Castillo-sobre-Reconocimiento-de-las-cooperativas-1319-26-01.pdf}
\newline {\color{gray} (Emb: 1.000, TF-IDF: 1.000)}

En caso de crisis humanitaria y demás situaciones que determine la ley, el Estado asegurará la reunificación familiar y el retorno voluntario al territorio nacional. 
\newline {\color{gray} \textbf{1º:} 422-4-Iniciativa-Convencional-de-la-cc-Maria-Trinidad-Castillo-sobre-Reconocimiento-de-las-cooperativas-1319-26-01.pdf}
\newline {\color{gray} (Emb: 1.000, TF-IDF: 1.000)}
\newline {\color{gray} \textbf{2º:} 308-4-Iniciativa-Convencional-del-cc-Fuad-Chahin-sobre-Cooperativas-1525-hrs.pdf}
\newline {\color{gray} (Emb: 1.000, TF-IDF: 1.000)}


\item \textbf{Artículo} \newline
Toda persona tiene derecho a presentar peticiones, exposiciones o reclamaciones ante cualquier autoridad del Estado. 
\newline {\color{gray} \textbf{1º:} 494-1-Iniciativa-Convencional-Constituyente-de-la-cc-Elsa-Labrana-sobre-Chilenos-Residentes-en-el-exterior-940-01-02.pdf}
\newline {\color{gray} (Emb: 0.706, TF-IDF: 0.375)}
\newline {\color{gray} \textbf{2º:} 758-Iniciativa-Convencional-Constituyente-Irrenunciabilidad-de-la-Nacionalidad-Saldana.pdf}
\newline {\color{gray} (Emb: 0.693, TF-IDF: 0.342)}

La ley regulará los plazos y la forma en que la autoridad deberá dar respuesta a lo solicitado, además de la manera en que se garantizará el principio de plurilingüismo en el ejercicio de este derecho. 
\newline {\color{gray} \textbf{1º:} 247-4-Iniciativa-Convencional-de-la-cc-Giovanna-Grandon-sobre-Derechos-de-los-chilenos-en-el-extranjero-1148-hrs.pdf}
\newline {\color{gray} (Emb: 0.799, TF-IDF: 0.779)}
\newline {\color{gray} \textbf{2º:} 408-4-Iniciativa-Convencional-Constituyente-de-la-cc-Tatiana-Urrutia-sobre-Sufragio-0041-17-01.pdf}
\newline {\color{gray} (Emb: 0.719, TF-IDF: 0.441)}


\item \textbf{Artículo} \newline
Las víctimas de graves violaciones a los derechos humanos tienen derecho a la reparación integral. 
\newline {\color{gray} \textbf{1º:} 247-4-Iniciativa-Convencional-de-la-cc-Giovanna-Grandon-sobre-Derechos-de-los-chilenos-en-el-extranjero-1148-hrs.pdf}
\newline {\color{gray} (Emb: 0.563, TF-IDF: 0.485)}
\newline {\color{gray} \textbf{2º:} 669-Iniciativa-Convencional-Constituyente-del-cc-Pedro-Munoz-sobre-Desplazamiento-Forzado-151101-02.pdf}
\newline {\color{gray} (Emb: 0.494, TF-IDF: 0.313)}


\item \textbf{Artículo} \newline
Las víctimas y la comunidad tienen el derecho al esclarecimiento y conocimiento de la verdad respecto de graves violaciones a los derechos humanos, especialmente, cuando constituyan crímenes de lesa humanidad, crímenes de guerra, genocidio o despojo territorial. 
\newline {\color{gray} \textbf{1º:} 251-4-Iniciativa-Convencional-del-cc-Patricio-Fernandez-sobre-Derechos-Civiles-y-Politicos1150-hrs.pdf}
\newline {\color{gray} (Emb: 0.723, TF-IDF: 0.537)}
\newline {\color{gray} \textbf{2º:} 291-4-Iniciativa-Convencional-de-la-cc-Tatiana-Urrutia-sobre-Derecho-de-Reunion-Peticion-y-Asociacion-1605-hrs.pdf}
\newline {\color{gray} (Emb: 0.698, TF-IDF: 0.445)}


\item \textbf{Artículo} \newline
El Estado garantiza el derecho a la memoria desde un enfoque que considere su relación con las garantías de no repetición y los derechos a la verdad, justicia y reparación integral. 
\newline {\color{gray} \textbf{1º:} 291-4-Iniciativa-Convencional-de-la-cc-Tatiana-Urrutia-sobre-Derecho-de-Reunion-Peticion-y-Asociacion-1605-hrs.pdf}
\newline {\color{gray} (Emb: 0.758, TF-IDF: 0.359)}
\newline {\color{gray} \textbf{2º:} 263-4-Iniciativa-Convencional-de-la-cc-Janis-Meneses-sobre-Derecho-de-Peticion-1153-hrs.pdf}
\newline {\color{gray} (Emb: 0.654, TF-IDF: 0.313)}


\item \textbf{Artículo} \newline
- Toda persona tiene el derecho a una vivienda digna y adecuada, que permita el libre desarrollo de una vida personal, familiar y comunitaria. 
\newline {\color{gray} \textbf{1º:} 519-4-Iniciativa-Convencional-Constituyente-de-cc-Carolina-Videla-sobre-DDHH-y-Garantias-de-no-repeticion-1257-hrs.-01-02.pdf}
\newline {\color{gray} (Emb: 0.849, TF-IDF: 0.776)}
\newline {\color{gray} \textbf{2º:} 451-4-Iniciativa-Convencional-Constituyente-de-la-cc-Carolina-Videla-sobre-Tortura-y-desaparicion-1409-31-01.pdf}
\newline {\color{gray} (Emb: 0.849, TF-IDF: 0.776)}

- El Estado deberá tomar todas las medidas necesarias para asegurar el goce universal y oportuno de este derecho, contemplando, a lo menos la habitabilidad, el espacio y equipamiento suficiente, doméstico y comunitario, para la producción y reproducción de la vida, la disponibilidad de servicios, la asequibilidad, la accesibilidad, la ubicación apropiada, la seguridad de la tenencia y la pertinencia cultural de las viviendas, de conformidad a la ley. 
\newline {\color{gray} \textbf{1º:} 519-4-Iniciativa-Convencional-Constituyente-de-cc-Carolina-Videla-sobre-DDHH-y-Garantias-de-no-repeticion-1257-hrs.-01-02.pdf}
\newline {\color{gray} (Emb: 0.967, TF-IDF: 0.948)}
\newline {\color{gray} \textbf{2º:} 451-4-Iniciativa-Convencional-Constituyente-de-la-cc-Carolina-Videla-sobre-Tortura-y-desaparicion-1409-31-01.pdf}
\newline {\color{gray} (Emb: 0.967, TF-IDF: 0.948)}

- El Estado podrá participar en el diseño, construcción, rehabilitación, conservación e innovación de la vivienda. 
\newline {\color{gray} \textbf{1º:} 446-Iniciativa-Convencional-Constituyente-de-la-cc-Carolina-Videla-sobre-Negacionismo-1500-28-01.pdf}
\newline {\color{gray} (Emb: 0.654, TF-IDF: 0.646)}
\newline {\color{gray} \textbf{2º:} 519-4-Iniciativa-Convencional-Constituyente-de-cc-Carolina-Videla-sobre-DDHH-y-Garantias-de-no-repeticion-1257-hrs.-01-02.pdf}
\newline {\color{gray} (Emb: 0.653, TF-IDF: 0.646)}

- El Estado considerará particularmente en el diseño de las políticas de vivienda a personas con bajos ingresos económicos o pertenecientes a grupos especialmente vulnerados en sus derechos. 
\newline {\color{gray} \textbf{1º:} 690-Iniciativa-Convencional-Constituyente-del-cc-Patricio-Fernandez-sobre-Derecho-a-una-Vivienda-Digna-121101-02.pdf}
\newline {\color{gray} (Emb: 0.735, TF-IDF: 0.620)}
\newline {\color{gray} \textbf{2º:} 318-4-Iniciativa-Convencional-del-cc-Cristian-Monckeberg-sobre-Orden-y-Seguridad16-00-hrs.pdf}
\newline {\color{gray} (Emb: 0.735, TF-IDF: 0.603)}

El Estado garantizará la creación de viviendas de acogida en casos de violencia de género y otras formas de vulneración de derechos, según determine la ley. 
\newline {\color{gray} \textbf{1º:} 1026-Iniciatva-Convencional-Constituyente-del-cc-Mariela-Serey-sobre-Personas-Discapacitadas.pdf}
\newline {\color{gray} (Emb: 0.631, TF-IDF: 0.345)}
\newline {\color{gray} \textbf{2º:} 986-Iniciativa-Convencional-Constituente-de-la-cc-Loreto-Vidal-sobre-Cuidados-Paliativos.pdf}
\newline {\color{gray} (Emb: 0.571, TF-IDF: 0.259)}

El Estado administrará un Sistema Integrado de Suelos Públicos. 
\newline {\color{gray} \textbf{1º:} 1013-Iniciativa-Convencional-Constituyente-del-cc-felix-galleguillos-sobre-archivo-general-de-asuntos-indigenas.pdf}
\newline {\color{gray} (Emb: 0.527, TF-IDF: 0.261)}
\newline {\color{gray} \textbf{2º:} 58-2-Iniciativa-Convencional-Constituyente-de-la-cc-Loreto-Vallejos-y-otros.pdf}
\newline {\color{gray} (Emb: 0.516, TF-IDF: 0.256)}

Este tendrá las facultades de dar prioridad de uso, gestión y disposición de terrenos fiscales para fines de interés social, así como adquirir terrenos privados, conforme a la ley. 
\newline {\color{gray} \textbf{1º:} 512-4-Iniciativa-Convencional-Constituyente-del-cc-Bernardo-Fontaine-agua-potable-1105-01-02.pdf}
\newline {\color{gray} (Emb: 0.519, TF-IDF: 0.232)}
\newline {\color{gray} \textbf{2º:} 135-4-c-Iniciativa-de-la-cc-Rocio-Cantuarias-Sobre-Seguridad-Social.pdf}
\newline {\color{gray} (Emb: 0.484, TF-IDF: 0.223)}

El Estado garantizará la disponibilidad del suelo necesario para la provisión de vivienda digna y adecuada. 
\newline {\color{gray} \textbf{1º:} 324-6-Iniciativa-Convencional-Constituyente-de-la-cc-Manuela-Royo-sobre-Justicia-Feminista.pdf}
\newline {\color{gray} (Emb: 0.622, TF-IDF: 0.263)}
\newline {\color{gray} \textbf{2º:} 362-2-Iniciativa-Convencional-Constituyente-de-la-cc-Janis-Meneses-sobre-Derechos-Laborales-1836-hrs-21-01.pdf}
\newline {\color{gray} (Emb: 0.602, TF-IDF: 0.236)}

Además, deberá establecer mecanismos para impedir la especulación en materia de suelo y vivienda que vaya en desmedro del interés público, de conformidad a la ley. 
\newline {\color{gray} \textbf{1º:} 570-Iniciativa-Convencional-Constituyente-de-cc-Valentina-Miranda-sobre-Dereco-a-la-Seguridad-Social-2008-hrs.-01-02.pdf}
\newline {\color{gray} (Emb: 0.616, TF-IDF: 0.310)}
\newline {\color{gray} \textbf{2º:} 692-Iniciativa-Convencional-Constituyente-del-cc-Pedro-Munoz-sobre-Agua-y-Buen-Vivir-120001-02.pdf}
\newline {\color{gray} (Emb: 0.610, TF-IDF: 0.294)}


\item \textbf{Artículo} \newline
El Estado garantiza la participación de la comunidad en los procesos de planificación territorial y políticas habitacionales. 
\newline {\color{gray} \textbf{1º:} 688-Iniciativa-Convencional-Constituyente-del-cc-Matias-Orellana-sobre-derecho-a-la-ciudad-120001-02.pdf}
\newline {\color{gray} (Emb: 0.661, TF-IDF: 0.586)}
\newline {\color{gray} \textbf{2º:} 395-4-Iniciativa-Convencional-Constituyente-del-cc-Cesar-Uribe-sobre-Derecho-al-territorio-y-a-la-ciudad155424-01.pdf}
\newline {\color{gray} (Emb: 0.620, TF-IDF: 0.355)}

Asimismo, promueve y apoya la gestión comunitaria del hábitat. 
\newline {\color{gray} \textbf{1º:} 688-Iniciativa-Convencional-Constituyente-del-cc-Matias-Orellana-sobre-derecho-a-la-ciudad-120001-02.pdf}
\newline {\color{gray} (Emb: 0.743, TF-IDF: 0.541)}
\newline {\color{gray} \textbf{2º:} 310-7-Iniciativa-Convencional-de-la-cc-Carolina-Videla-sobre-Derecho-a-la-Comunicacion-2000-hrs.pdf}
\newline {\color{gray} (Emb: 0.577, TF-IDF: 0.340)}

El Estado garantizará la protección y acceso equitativo a servicios básicos, bienes y espacios públicos; movilidad segura y sustentable; conectividad y seguridad vial. 
\newline {\color{gray} \textbf{1º:} 688-Iniciativa-Convencional-Constituyente-del-cc-Matias-Orellana-sobre-derecho-a-la-ciudad-120001-02.pdf}
\newline {\color{gray} (Emb: 0.977, TF-IDF: 0.928)}
\newline {\color{gray} \textbf{2º:} 328-4-Iniciativa-Convencional-Constituyente-de-la-cc-Tania-Madriaga-sobre-Derecho-a-la-Ciudad.pdf}
\newline {\color{gray} (Emb: 0.750, TF-IDF: 0.567)}

Asimismo, promoverá la integración socioespacial y participará en la plusvalía que genere su acción urbanística o regulatoria. 
\newline {\color{gray} \textbf{1º:} 688-Iniciativa-Convencional-Constituyente-del-cc-Matias-Orellana-sobre-derecho-a-la-ciudad-120001-02.pdf}
\newline {\color{gray} (Emb: 0.987, TF-IDF: 0.950)}
\newline {\color{gray} \textbf{2º:} 328-4-Iniciativa-Convencional-Constituyente-de-la-cc-Tania-Madriaga-sobre-Derecho-a-la-Ciudad.pdf}
\newline {\color{gray} (Emb: 0.895, TF-IDF: 0.618)}

El derecho a la ciudad es un derecho colectivo orientado al bien común y se basa en el ejercicio pleno de los derechos humanos en el territorio, en su gestión democrática y en la función social y ecológica de la propiedad. 
\newline {\color{gray} \textbf{1º:} 340-4-Iniciativa-Convencional-Constituyente-de-la-cc-Tania-Madriaga-sobre-Derecho-a-la-Vivienda.pdf}
\newline {\color{gray} (Emb: 0.655, TF-IDF: 0.561)}
\newline {\color{gray} \textbf{2º:} 889-Iniciativa-Convencional-Constituyente-de-la-cc-Natividad-Llanquileo-crea-el-Consejo-de-Pueblos-Indigenas.pdf}
\newline {\color{gray} (Emb: 0.576, TF-IDF: 0.489)}

Todas las personas tienen derecho a habitar, producir, gozar y participar en ciudades y asentamientos humanos libres de violencia y en condiciones apropiadas para una vida digna. 
\newline {\color{gray} \textbf{1º:} 527-5-Iniciativa-Convencional-Constituyente-de-cc-Marcos-Barraza-sobre-Renacionalizacion-del-Cobre-1516-hrs.-01-02.pdf}
\newline {\color{gray} (Emb: 0.575, TF-IDF: 0.624)}
\newline {\color{gray} \textbf{2º:} 606-Iniciativa-Convencional-Constituyente-de-cc-Marcos-Barraza-Renacionalziacion-Cobre-y-otro-Bienes-Publicos-Estrategicos.pdf}
\newline {\color{gray} (Emb: 0.575, TF-IDF: 0.309)}

Es deber del Estado ordenar, planificar y gestionar los territorios, ciudades y asentamientos humanos; así como establecer reglas de uso y transformación del suelo, de acuerdo al interés general, la equidad territorial, sostenibilidad y accesibilidad universal. 
\newline {\color{gray} \textbf{1º:} 60-2-Iniciativa-Convencional-Constituyente-del-cc-Jorge-Baradit-y-otros.pdf}
\newline {\color{gray} (Emb: 0.564, TF-IDF: 0.211)}
\newline {\color{gray} \textbf{2º:} 527-5-Iniciativa-Convencional-Constituyente-de-cc-Marcos-Barraza-sobre-Renacionalizacion-del-Cobre-1516-hrs.-01-02.pdf}
\newline {\color{gray} (Emb: 0.510, TF-IDF: 0.201)}


\item \textbf{Artículo} \newline
En el ámbito rural y agrícola el Estado debe garantizar condiciones justas y dignas en el trabajo de temporada, resguardando el ejercicio de sus derechos laborales y de seguridad social. 
\newline {\color{gray} \textbf{1º:} 682-Iniciativa-Convencional-Constituyente-del-cc-Javier-Fuchslocher-sobre-Trabajo-Decente-131101-02.pdf}
\newline {\color{gray} (Emb: 1.000, TF-IDF: 1.000)}
\newline {\color{gray} \textbf{2º:} 1025-Iniciativa-Convencional-Constituyente-del-cc-Eric-Chinga-sobre-DDFF-de-los-trabajadores.pdf}
\newline {\color{gray} (Emb: 0.820, TF-IDF: 0.531)}

Toda persona tiene derecho al trabajo y su libre elección. 
\newline {\color{gray} \textbf{1º:} 897-Iniciativa-Convencional-Constituyente-del-cc-Cesar-Uribe-sobre-el-Caracter-Publico-de-los-Servicios.pdf}
\newline {\color{gray} (Emb: 0.595, TF-IDF: 0.758)}
\newline {\color{gray} \textbf{2º:} 753-Iniciativa-Convencional-Constituyente-del-cc-Raul-Celis-sobre-Gobiernos-Locales.pdf}
\newline {\color{gray} (Emb: 0.570, TF-IDF: 0.293)}

El Estado garantiza el trabajo decente y su protección. 
\newline {\color{gray} \textbf{1º:} 957-5-Iniciativa-Convencional-Constituyente-de-la-cc-Ivanna-Olivares-sobre-Nuevo-Modelo-Economico.pdf}
\newline {\color{gray} (Emb: 0.646, TF-IDF: 0.375)}
\newline {\color{gray} \textbf{2º:} 67-2-Iniciativa-Convencional-Constituyente-del-cc-Cristián-Monckeberg-y-otros.pdf}
\newline {\color{gray} (Emb: 0.602, TF-IDF: 0.282)}

Este comprende el derecho a condiciones laborales equitativas, a la salud y seguridad en el trabajo, al descanso, al disfrute del tiempo libre, a su desconexión digital, a la garantía de indemnidad, y el pleno respeto de los derechos fundamentales en el contexto del trabajo. 
\newline {\color{gray} \textbf{1º:} 954-5-Iniciativa-Convencional-Constituyente-de-la-cc-Carolina-Vilches-sobre-Estatuto-del-Agua.pdf}
\newline {\color{gray} (Emb: 0.553, TF-IDF: 0.431)}
\newline {\color{gray} \textbf{2º:} 222-7-Iniciativa-Convencional-del-cc-Carlos-Calvo-sobre-Derechos-a-la-Comunicacion-2351-hrs.pdf}
\newline {\color{gray} (Emb: 0.503, TF-IDF: 0.257)}

Los trabajadores y trabajadoras tendrán derecho a una remuneración equitativa, justa y suficiente, que le asegure su sustento y el de su familia. 
\newline {\color{gray} \textbf{1º:} 143-4-c-Iniciativa-de-la-cc-Rocio-Cantuarias-Incorpora-Libertad-de-Trabajo-y-Sindical.pdf}
\newline {\color{gray} (Emb: 0.932, TF-IDF: 0.646)}
\newline {\color{gray} \textbf{2º:} 257-4-Iniciativa-Convencional-de-la-cc-Maria-Magdalena-Rivera-sobre-Derecho-a-Huelga-Solidaria-17-01-1151-hrs.pdf}
\newline {\color{gray} (Emb: 0.882, TF-IDF: 0.636)}

Toda persona tiene derecho a igual remuneración por igual trabajo. 
\newline {\color{gray} \textbf{1º:} 587-Iniciativa-Convencional-Constituyente-de-cc-Marcos-Barraza-sobre-Derecho-al-Trabajo-2351-hrs.-01-02.pdf}
\newline {\color{gray} (Emb: 0.917, TF-IDF: 0.880)}
\newline {\color{gray} \textbf{2º:} 652-Iniciativa-Convencional-Constituyente-de-la-cc-Ericka-Portilla-sobre-Trabajo-Decente-151101-02.pdf}
\newline {\color{gray} (Emb: 0.917, TF-IDF: 0.880)}

Se prohíbe cualquier discriminación entre trabajadoras y trabajadores que no se base en las competencias laborales o idoneidad personal, así como el despido arbitrario. 
\newline {\color{gray} \textbf{1º:} 682-Iniciativa-Convencional-Constituyente-del-cc-Javier-Fuchslocher-sobre-Trabajo-Decente-131101-02.pdf}
\newline {\color{gray} (Emb: 0.865, TF-IDF: 0.697)}
\newline {\color{gray} \textbf{2º:} 380-4-Iniciativa-Convencional-Constituyente-del-cc-Bastian-Labbe-sobre-el-Derecho-al-Trabajo1141-24-01.pdf}
\newline {\color{gray} (Emb: 0.644, TF-IDF: 0.454)}

El Estado generará políticas públicas que permitan la conciliación laboral, la vida familiar y comunitaria, y el trabajo de cuidados. 
\newline {\color{gray} \textbf{1º:} 682-Iniciativa-Convencional-Constituyente-del-cc-Javier-Fuchslocher-sobre-Trabajo-Decente-131101-02.pdf}
\newline {\color{gray} (Emb: 0.970, TF-IDF: 0.908)}
\newline {\color{gray} \textbf{2º:} 380-4-Iniciativa-Convencional-Constituyente-del-cc-Bastian-Labbe-sobre-el-Derecho-al-Trabajo1141-24-01.pdf}
\newline {\color{gray} (Emb: 0.864, TF-IDF: 0.471)}

El Estado garantizará el respeto a los derechos reproductivos de las personas trabajadoras, eliminando riesgos que afecten la salud reproductiva y resguardando los derechos de la paternidad y maternidad. 
\newline {\color{gray} \textbf{1º:} 107-4-c-Iniciativa-de-la-cc-Giovanna-Grandon-Derecho-al-Trabajo.pdf}
\newline {\color{gray} (Emb: 0.988, TF-IDF: 0.789)}
\newline {\color{gray} \textbf{2º:} 652-Iniciativa-Convencional-Constituyente-de-la-cc-Ericka-Portilla-sobre-Trabajo-Decente-151101-02.pdf}
\newline {\color{gray} (Emb: 0.849, TF-IDF: 0.652)}

Se reconoce la función social del trabajo y se deberá asegurar la protección eficaz de los trabajadores, trabajadoras y organizaciones sindicales, mediante un órgano autónomo a cargo de su fiscalización. 
\newline {\color{gray} \textbf{1º:} 555-Iniciativa-Convencional-Constituyente-del-convencional-Bernardo-Fontaine-sobre-libertad-de-trabajo-1737-01-02.pdf}
\newline {\color{gray} (Emb: 0.607, TF-IDF: 0.410)}
\newline {\color{gray} \textbf{2º:} 774-Iniciativa-Convencional-Constituyente-de-la-cc-Barbara-Rebolledo-sobre-Derechos-de-las-Mujeres.pdf}
\newline {\color{gray} (Emb: 0.607, TF-IDF: 0.300)}

Se prohíbe toda forma de precarización laboral, así como el trabajo forzoso, humillante o denigrante. 
\newline {\color{gray} \textbf{1º:} 613-Iniciativa-Convencional-Constituyente-de-cc-Barbara-Sepulveda-sobre-Perspectiva-de-Genero-en-el-Derecho-al-Trabajo.pdf}
\newline {\color{gray} (Emb: 0.896, TF-IDF: 0.664)}
\newline {\color{gray} \textbf{2º:} 380-4-Iniciativa-Convencional-Constituyente-del-cc-Bastian-Labbe-sobre-el-Derecho-al-Trabajo1141-24-01.pdf}
\newline {\color{gray} (Emb: 0.774, TF-IDF: 0.370)}


\item \textbf{Artículo} \newline
La ley regulará los mecanismos por medio de los cuales se ejercerá este derecho. 
\newline {\color{gray} \textbf{1º:} 682-Iniciativa-Convencional-Constituyente-del-cc-Javier-Fuchslocher-sobre-Trabajo-Decente-131101-02.pdf}
\newline {\color{gray} (Emb: 0.862, TF-IDF: 0.814)}
\newline {\color{gray} \textbf{2º:} 587-Iniciativa-Convencional-Constituyente-de-cc-Marcos-Barraza-sobre-Derecho-al-Trabajo-2351-hrs.-01-02.pdf}
\newline {\color{gray} (Emb: 0.779, TF-IDF: 0.508)}

Los trabajadores y trabajadoras, a través de sus organizaciones sindicales, tienen el derecho a participar en las decisiones de la empresa. 
\newline {\color{gray} \textbf{1º:} 652-Iniciativa-Convencional-Constituyente-de-la-cc-Ericka-Portilla-sobre-Trabajo-Decente-151101-02.pdf}
\newline {\color{gray} (Emb: 0.692, TF-IDF: 0.361)}
\newline {\color{gray} \textbf{2º:} 587-Iniciativa-Convencional-Constituyente-de-cc-Marcos-Barraza-sobre-Derecho-al-Trabajo-2351-hrs.-01-02.pdf}
\newline {\color{gray} (Emb: 0.692, TF-IDF: 0.361)}


\item \textbf{Artículo} \newline
Asimismo, velará por el resguardo de los derechos de quienes ejercen trabajos de cuidados. 
\newline {\color{gray} \textbf{1º:} 968-Iniciativa-Convencional-Constituyente-de-la-cc-Valentina-Miranda-sobre-Derecho-a-la-salud.pdf}
\newline {\color{gray} (Emb: 0.738, TF-IDF: 0.486)}
\newline {\color{gray} \textbf{2º:} 387-4-Iniciativa-Convencional-Constituyente-de-la-cc-Francisca-Linconao-sobre-Derecho-a-la-Saud-1228-24-01.pdf}
\newline {\color{gray} (Emb: 0.683, TF-IDF: 0.362)}

Todas las personas tienen derecho a cuidar, a ser cuidadas y a cuidarse desde el nacimiento hasta la muerte. 
\newline {\color{gray} \textbf{1º:} 380-4-Iniciativa-Convencional-Constituyente-del-cc-Bastian-Labbe-sobre-el-Derecho-al-Trabajo1141-24-01.pdf}
\newline {\color{gray} (Emb: 0.788, TF-IDF: 0.534)}
\newline {\color{gray} \textbf{2º:} 587-Iniciativa-Convencional-Constituyente-de-cc-Marcos-Barraza-sobre-Derecho-al-Trabajo-2351-hrs.-01-02.pdf}
\newline {\color{gray} (Emb: 0.725, TF-IDF: 0.517)}

El Estado se obliga a proveer los medios para garantizar que este cuidado sea digno y realizado en condiciones de igualdad y corresponsabilidad. 
\newline {\color{gray} \textbf{1º:} 682-Iniciativa-Convencional-Constituyente-del-cc-Javier-Fuchslocher-sobre-Trabajo-Decente-131101-02.pdf}
\newline {\color{gray} (Emb: 0.913, TF-IDF: 0.842)}
\newline {\color{gray} \textbf{2º:} 555-Iniciativa-Convencional-Constituyente-del-convencional-Bernardo-Fontaine-sobre-libertad-de-trabajo-1737-01-02.pdf}
\newline {\color{gray} (Emb: 0.775, TF-IDF: 0.494)}

El Estado garantizará este derecho a través de un Sistema Integral de Cuidados y otras normativas y políticas públicas que incorporen el enfoque de derechos humanos, de género y la promoción de la autonomía personal. 
\newline {\color{gray} \textbf{1º:} 682-Iniciativa-Convencional-Constituyente-del-cc-Javier-Fuchslocher-sobre-Trabajo-Decente-131101-02.pdf}
\newline {\color{gray} (Emb: 0.998, TF-IDF: 0.873)}
\newline {\color{gray} \textbf{2º:} 995-Iniciativa-Convencional-Constituyente-del-cc-Christian-Viera-sobre-Principio-de-Intervencion-del-Estado.pdf}
\newline {\color{gray} (Emb: 0.840, TF-IDF: 0.450)}

El Sistema tendrá un carácter estatal, paritario, solidario, universal, con pertinencia cultural y perspectiva de género e interseccionalidad. 
\newline {\color{gray} \textbf{1º:} 355-4-Iniciativa-Convencional-Constituyente-de-la-cc-Mariela-Serey-sobre-Derecho-al-Cuidado-1200-21-01.pdf}
\newline {\color{gray} (Emb: 1.000, TF-IDF: 1.000)}
\newline {\color{gray} \textbf{2º:} 478-4-Iniciativa-Convencional-Constituyente-de-la-cc-Barbara-Rebolledo-Derecho-al-cuidado-2029-31-01.pdf}
\newline {\color{gray} (Emb: 0.764, TF-IDF: 0.372)}

Su financiamiento será progresivo, suficiente y permanente. 
\newline {\color{gray} \textbf{1º:} 355-4-Iniciativa-Convencional-Constituyente-de-la-cc-Mariela-Serey-sobre-Derecho-al-Cuidado-1200-21-01.pdf}
\newline {\color{gray} (Emb: 1.000, TF-IDF: 1.000)}
\newline {\color{gray} \textbf{2º:} 391-4-Iniciativa-Convencional-Constituyente-de-la-cc-Valentina-Miranda-sobre-Ninos-Ninas-y-Adolescentes-1440-24-01.pdf}
\newline {\color{gray} (Emb: 0.690, TF-IDF: 0.291)}

El sistema prestará especial atención a lactantes, niños, niñas y adolescentes, personas mayores, personas en situación de discapacidad, personas en situación de dependencia y personas con enfermedades graves o terminales. 
\newline {\color{gray} \textbf{1º:} 613-Iniciativa-Convencional-Constituyente-de-cc-Barbara-Sepulveda-sobre-Perspectiva-de-Genero-en-el-Derecho-al-Trabajo.pdf}
\newline {\color{gray} (Emb: 0.706, TF-IDF: 0.486)}
\newline {\color{gray} \textbf{2º:} 355-4-Iniciativa-Convencional-Constituyente-de-la-cc-Mariela-Serey-sobre-Derecho-al-Cuidado-1200-21-01.pdf}
\newline {\color{gray} (Emb: 0.695, TF-IDF: 0.374)}


\item \textbf{Artículo} \newline
El Estado promoverá la corresponsabilidad social y de género e implementará mecanismos para la redistribución del trabajo doméstico y de cuidados. 
\newline {\color{gray} \textbf{1º:} 355-4-Iniciativa-Convencional-Constituyente-de-la-cc-Mariela-Serey-sobre-Derecho-al-Cuidado-1200-21-01.pdf}
\newline {\color{gray} (Emb: 0.961, TF-IDF: 1.000)}
\newline {\color{gray} \textbf{2º:} 478-4-Iniciativa-Convencional-Constituyente-de-la-cc-Barbara-Rebolledo-Derecho-al-cuidado-2029-31-01.pdf}
\newline {\color{gray} (Emb: 0.601, TF-IDF: 0.414)}

El Estado reconoce que los trabajos domésticos y de cuidados son trabajos socialmente necesarios e indispensables para la sostenibilidad de la vida y el desarrollo de la sociedad, que son una actividad económica que contribuye a las cuentas nacionales y que deben ser considerados en la formulación y ejecución de las políticas públicas. 
\newline {\color{gray} \textbf{1º:} 713-Iniciativa-Convencional-Constituyente-de-la-cc-Lisette-Vergara-sobre-Economia.pdf}
\newline {\color{gray} (Emb: 0.717, TF-IDF: 0.479)}
\newline {\color{gray} \textbf{2º:} 713-Iniciativa-Convencional-Constituyente-de-la-cc-Lisette-Vergara-sobre-Economia.pdf}
\newline {\color{gray} (Emb: 0.717, TF-IDF: 0.381)}


\item \textbf{Artículo} \newline
La Constitución garantiza el derecho a huelga de trabajadores, trabajadoras y organizaciones sindicales. 
\newline {\color{gray} \textbf{1º:} 682-Iniciativa-Convencional-Constituyente-del-cc-Javier-Fuchslocher-sobre-Trabajo-Decente-131101-02.pdf}
\newline {\color{gray} (Emb: 1.000, TF-IDF: 1.000)}
\newline {\color{gray} \textbf{2º:} 362-2-Iniciativa-Convencional-Constituyente-de-la-cc-Janis-Meneses-sobre-Derechos-Laborales-1836-hrs-21-01.pdf}
\newline {\color{gray} (Emb: 0.877, TF-IDF: 0.798)}

La ley sólo podrá establecer limitaciones excepcionales a la huelga para atender servicios esenciales que pudieren afectar la vida, salud o seguridad de la población. 
\newline {\color{gray} \textbf{1º:} 682-Iniciativa-Convencional-Constituyente-del-cc-Javier-Fuchslocher-sobre-Trabajo-Decente-131101-02.pdf}
\newline {\color{gray} (Emb: 0.860, TF-IDF: 0.702)}
\newline {\color{gray} \textbf{2º:} 555-Iniciativa-Convencional-Constituyente-del-convencional-Bernardo-Fontaine-sobre-libertad-de-trabajo-1737-01-02.pdf}
\newline {\color{gray} (Emb: 0.812, TF-IDF: 0.583)}

El legislador no podrá prohibir la huelga. 
\newline {\color{gray} \textbf{1º:} 682-Iniciativa-Convencional-Constituyente-del-cc-Javier-Fuchslocher-sobre-Trabajo-Decente-131101-02.pdf}
\newline {\color{gray} (Emb: 1.000, TF-IDF: 1.000)}
\newline {\color{gray} \textbf{2º:} 682-Iniciativa-Convencional-Constituyente-del-cc-Javier-Fuchslocher-sobre-Trabajo-Decente-131101-02.pdf}
\newline {\color{gray} (Emb: 0.661, TF-IDF: 0.398)}

Las organizaciones sindicales decidirán el ámbito de intereses que se defenderán a través de ella, los que no podrán ser limitados por la ley. 
\newline {\color{gray} \textbf{1º:} 682-Iniciativa-Convencional-Constituyente-del-cc-Javier-Fuchslocher-sobre-Trabajo-Decente-131101-02.pdf}
\newline {\color{gray} (Emb: 1.000, TF-IDF: 1.000)}
\newline {\color{gray} \textbf{2º:} 257-4-Iniciativa-Convencional-de-la-cc-Maria-Magdalena-Rivera-sobre-Derecho-a-Huelga-Solidaria-17-01-1151-hrs.pdf}
\newline {\color{gray} (Emb: 0.572, TF-IDF: 0.404)}

No podrán declararse en huelga los integrantes de las Fuerzas Armadas, de Orden y Seguridad Pública. 
\newline {\color{gray} \textbf{1º:} 682-Iniciativa-Convencional-Constituyente-del-cc-Javier-Fuchslocher-sobre-Trabajo-Decente-131101-02.pdf}
\newline {\color{gray} (Emb: 0.741, TF-IDF: 0.456)}
\newline {\color{gray} \textbf{2º:} 382-4-Iniciativa-Convencional-Constituyente-del-cc-Bastian-Labbe-sobre-el-Derecho-a-la-Libertad-Sindical-1142-24-01.pdf}
\newline {\color{gray} (Emb: 0.725, TF-IDF: 0.342)}

Las únicas limitaciones a las materias susceptibles de negociación serán aquellas concernientes a los mínimos irrenunciables fijados por la ley a favor de los trabajadores y trabajadoras. 
\newline {\color{gray} \textbf{1º:} 682-Iniciativa-Convencional-Constituyente-del-cc-Javier-Fuchslocher-sobre-Trabajo-Decente-131101-02.pdf}
\newline {\color{gray} (Emb: 1.000, TF-IDF: 1.000)}
\newline {\color{gray} \textbf{2º:} 362-2-Iniciativa-Convencional-Constituyente-de-la-cc-Janis-Meneses-sobre-Derechos-Laborales-1836-hrs-21-01.pdf}
\newline {\color{gray} (Emb: 0.915, TF-IDF: 0.710)}

La Constitución asegura a trabajadoras y trabajadores, tanto del sector público como del privado, el derecho a la libertad sindical. 
\newline {\color{gray} \textbf{1º:} 355-4-Iniciativa-Convencional-Constituyente-de-la-cc-Mariela-Serey-sobre-Derecho-al-Cuidado-1200-21-01.pdf}
\newline {\color{gray} (Emb: 1.000, TF-IDF: 1.000)}
\newline {\color{gray} \textbf{2º:} 382-4-Iniciativa-Convencional-Constituyente-del-cc-Bastian-Labbe-sobre-el-Derecho-a-la-Libertad-Sindical-1142-24-01.pdf}
\newline {\color{gray} (Emb: 0.687, TF-IDF: 0.594)}

La Constitución asegura el derecho a la negociación colectiva. 
\newline {\color{gray} \textbf{1º:} 682-Iniciativa-Convencional-Constituyente-del-cc-Javier-Fuchslocher-sobre-Trabajo-Decente-131101-02.pdf}
\newline {\color{gray} (Emb: 0.901, TF-IDF: 0.574)}
\newline {\color{gray} \textbf{2º:} 143-4-c-Iniciativa-de-la-cc-Rocio-Cantuarias-Incorpora-Libertad-de-Trabajo-y-Sindical.pdf}
\newline {\color{gray} (Emb: 0.749, TF-IDF: 0.424)}

Las organizaciones sindicales gozarán de personalidad jurídica por el sólo hecho de registrar sus estatutos en la forma que señale la ley. 
\newline {\color{gray} \textbf{1º:} 682-Iniciativa-Convencional-Constituyente-del-cc-Javier-Fuchslocher-sobre-Trabajo-Decente-131101-02.pdf}
\newline {\color{gray} (Emb: 1.000, TF-IDF: 1.000)}
\newline {\color{gray} \textbf{2º:} 555-Iniciativa-Convencional-Constituyente-del-convencional-Bernardo-Fontaine-sobre-libertad-de-trabajo-1737-01-02.pdf}
\newline {\color{gray} (Emb: 0.769, TF-IDF: 0.579)}

El derecho de sindicalización comprende la facultad de constituir las organizaciones sindicales que estimen conveniente, en cualquier nivel, de carácter nacional e internacional, a afiliarse y desafiliarse de ellas, a darse su propia normativa, trazar sus propios fines y realizar su actividad sin intervención de terceros. 
\newline {\color{gray} \textbf{1º:} 682-Iniciativa-Convencional-Constituyente-del-cc-Javier-Fuchslocher-sobre-Trabajo-Decente-131101-02.pdf}
\newline {\color{gray} (Emb: 1.000, TF-IDF: 1.000)}
\newline {\color{gray} \textbf{2º:} 302-4-Iniciativa-Convencional-del-cc-Fuad-Chahin-sobre-Derecho-al-trabajo-18-01.-1141-hrs.pdf}
\newline {\color{gray} (Emb: 0.865, TF-IDF: 0.565)}

Las organizaciones sindicales son titulares exclusivas del derecho a la negociación colectiva, en tanto únicas representantes de trabajadores y trabajadoras ante el o los empleadores. 
\newline {\color{gray} \textbf{1º:} 355-4-Iniciativa-Convencional-Constituyente-de-la-cc-Mariela-Serey-sobre-Derecho-al-Cuidado-1200-21-01.pdf}
\newline {\color{gray} (Emb: 0.739, TF-IDF: 0.494)}
\newline {\color{gray} \textbf{2º:} 947-Iniciativa-Convencional-constituyente-de-la-cc-Barbara-Rebolledo-reconocimiento-y-valoracion-del-trabajo-domestico.pdf}
\newline {\color{gray} (Emb: 0.656, TF-IDF: 0.387)}

Este derecho comprende el derecho a la sindicalización, a la negociación colectiva y a la huelga. 
\newline {\color{gray} \textbf{1º:} 355-4-Iniciativa-Convencional-Constituyente-de-la-cc-Mariela-Serey-sobre-Derecho-al-Cuidado-1200-21-01.pdf}
\newline {\color{gray} (Emb: 1.000, TF-IDF: 1.000)}
\newline {\color{gray} \textbf{2º:} 719-Iniciativa-Convencional-Constituyente-de-la-cc-Maria-Magdalena-Rivera-sobre-Igualdad-ante-la-Ley.pdf}
\newline {\color{gray} (Emb: 0.605, TF-IDF: 0.354)}

Corresponderá a los trabajadores y trabajadoras elegir el nivel en que se desarrollará dicha negociación, incluyendo la negociación ramal, sectorial y territorial. 
\newline {\color{gray} \textbf{1º:} 682-Iniciativa-Convencional-Constituyente-del-cc-Javier-Fuchslocher-sobre-Trabajo-Decente-131101-02.pdf}
\newline {\color{gray} (Emb: 0.998, TF-IDF: 0.999)}
\newline {\color{gray} \textbf{2º:} 569-Iniciativa-Convencional-Constituyente-de-cc-Roberto-Celedon-sobre-Derecho-al-trabajo-2024-hrs.-01-02.pdf}
\newline {\color{gray} (Emb: 0.959, TF-IDF: 0.892)}


\item \textbf{Artículo} \newline
La Constitución garantiza a toda persona el derecho a la seguridad social, fundada en los principios de universalidad, solidaridad, integralidad, unidad, igualdad, suficiencia, participación, sostenibilidad y oportunidad. 
\newline {\color{gray} \textbf{1º:} 682-Iniciativa-Convencional-Constituyente-del-cc-Javier-Fuchslocher-sobre-Trabajo-Decente-131101-02.pdf}
\newline {\color{gray} (Emb: 1.000, TF-IDF: 1.000)}
\newline {\color{gray} \textbf{2º:} 97-6-Iniciativa-del-cc-Daniel-Bravo-Sistemas-de-Justicia.pdf}
\newline {\color{gray} (Emb: 0.773, TF-IDF: 0.546)}

La ley establecerá un Sistema de Seguridad Social público, que otorgue protección en caso de enfermedad, vejez, discapacidad, supervivencia, maternidad y paternidad, desempleo, accidentes del trabajo y enfermedades profesionales, y en las demás contingencias sociales de falta o disminución de medios de subsistencia o de capacidad para el trabajo. 
\newline {\color{gray} \textbf{1º:} 682-Iniciativa-Convencional-Constituyente-del-cc-Javier-Fuchslocher-sobre-Trabajo-Decente-131101-02.pdf}
\newline {\color{gray} (Emb: 0.986, TF-IDF: 0.942)}
\newline {\color{gray} \textbf{2º:} 587-Iniciativa-Convencional-Constituyente-de-cc-Marcos-Barraza-sobre-Derecho-al-Trabajo-2351-hrs.-01-02.pdf}
\newline {\color{gray} (Emb: 0.877, TF-IDF: 0.411)}

En particular, este sistema asegurará la cobertura de prestaciones a las personas que ejerzan trabajos domésticos y de cuidados. 
\newline {\color{gray} \textbf{1º:} 682-Iniciativa-Convencional-Constituyente-del-cc-Javier-Fuchslocher-sobre-Trabajo-Decente-131101-02.pdf}
\newline {\color{gray} (Emb: 1.000, TF-IDF: 1.000)}
\newline {\color{gray} \textbf{2º:} 587-Iniciativa-Convencional-Constituyente-de-cc-Marcos-Barraza-sobre-Derecho-al-Trabajo-2351-hrs.-01-02.pdf}
\newline {\color{gray} (Emb: 0.819, TF-IDF: 0.623)}

Le corresponderá al Estado definir la política de seguridad social. 
\newline {\color{gray} \textbf{1º:} 674-Iniciativa-Convencional-Constituyente-del-cc-Cesar-Valenzuela-sobre-Seguridad-Social-121101-02.pdf}
\newline {\color{gray} (Emb: 0.939, TF-IDF: 0.808)}
\newline {\color{gray} \textbf{2º:} 655-Iniciativa-Convencional-Constituyente-de-la-cc-Jeniffer-Mella-sobre-Trabajo-y-Seguridad-Social-121101-02.pdf}
\newline {\color{gray} (Emb: 0.756, TF-IDF: 0.497)}

Ésta se financiará por trabajadores y empleadores, a través de cotizaciones obligatorias, y por rentas generales de la nación. 
\newline {\color{gray} \textbf{1º:} 674-Iniciativa-Convencional-Constituyente-del-cc-Cesar-Valenzuela-sobre-Seguridad-Social-121101-02.pdf}
\newline {\color{gray} (Emb: 0.992, TF-IDF: 0.895)}
\newline {\color{gray} \textbf{2º:} 372-4-Iniciativa-Convencional-Constituyente-del-cc-Roberto-Celedon-sobre-Derecho-a-la-Seguridad-Social-0900-hrs-24-01.pdf}
\newline {\color{gray} (Emb: 0.806, TF-IDF: 0.556)}

Los recursos con que se financie la seguridad social no podrán ser destinados a fines distintos que el pago de los beneficios que establezca el sistema. 
\newline {\color{gray} \textbf{1º:} 674-Iniciativa-Convencional-Constituyente-del-cc-Cesar-Valenzuela-sobre-Seguridad-Social-121101-02.pdf}
\newline {\color{gray} (Emb: 0.749, TF-IDF: 0.416)}
\newline {\color{gray} \textbf{2º:} 570-Iniciativa-Convencional-Constituyente-de-cc-Valentina-Miranda-sobre-Dereco-a-la-Seguridad-Social-2008-hrs.-01-02.pdf}
\newline {\color{gray} (Emb: 0.550, TF-IDF: 0.366)}

Las organizaciones sindicales y de empleadores tendrán derecho a participar en la dirección del sistema de seguridad social, en las formas que señale la ley. 
\newline {\color{gray} \textbf{1º:} 674-Iniciativa-Convencional-Constituyente-del-cc-Cesar-Valenzuela-sobre-Seguridad-Social-121101-02.pdf}
\newline {\color{gray} (Emb: 0.839, TF-IDF: 0.881)}
\newline {\color{gray} \textbf{2º:} 436-4-Iniciativa-Convencional-de-la-cc-Elsa-Labrana-sobre-Derecho-al-Trabajo-1203-27-01.pdf}
\newline {\color{gray} (Emb: 0.728, TF-IDF: 0.407)}


\item \textbf{Artículo} \newline
El Sistema Nacional de Salud reconoce, protege e integra estas prácticas y conocimientos como también a quienes las imparten, en conformidad a esta Constitución y la ley. 
\newline {\color{gray} \textbf{1º:} 396-4-Iniciativa-Convencional-Constituyente-de-la-cc-Natalia-Henriquez-sobre-Derecho-a-la-salud1559-24-01.pdf}
\newline {\color{gray} (Emb: 0.883, TF-IDF: 0.941)}
\newline {\color{gray} \textbf{2º:} 387-4-Iniciativa-Convencional-Constituyente-de-la-cc-Francisca-Linconao-sobre-Derecho-a-la-Saud-1228-24-01.pdf}
\newline {\color{gray} (Emb: 0.633, TF-IDF: 0.513)}

Corresponderá exclusivamente al Estado la función de rectoría del sistema de salud, incluyendo la regulación, supervisión y fiscalización de las instituciones públicas y privadas. 
\newline {\color{gray} \textbf{1º:} 454-5-Iniciativa-Convencional-Constituyente-del-cc-Marcos-Barraza-sobre-Economia-Social-1529-31-01.pdf}
\newline {\color{gray} (Emb: 0.588, TF-IDF: 0.283)}
\newline {\color{gray} \textbf{2º:} 968-Iniciativa-Convencional-Constituyente-de-la-cc-Valentina-Miranda-sobre-Derecho-a-la-salud.pdf}
\newline {\color{gray} (Emb: 0.542, TF-IDF: 0.275)}

Los pueblos y naciones indígenas tienen derecho a sus propias medicinas tradicionales, a mantener sus prácticas de salud y a conservar los componentes naturales que las sustentan. 
\newline {\color{gray} \textbf{1º:} 681-Iniciativa-Convencional-Constituyente-del-cc-Gaspar-Dominguez-sobre-el-Derecho-a-la-salud-120001-02.pdf}
\newline {\color{gray} (Emb: 0.653, TF-IDF: 0.422)}
\newline {\color{gray} \textbf{2º:} 396-4-Iniciativa-Convencional-Constituyente-de-la-cc-Natalia-Henriquez-sobre-Derecho-a-la-salud1559-24-01.pdf}
\newline {\color{gray} (Emb: 0.612, TF-IDF: 0.409)}

Adicionalmente, la ley podrá establecer el cobro obligatorio de cotizaciones a empleadoras, empleadores, trabajadoras y trabajadores con el solo objeto de aportar solidariamente al financiamiento de este sistema. 
\newline {\color{gray} \textbf{1º:} 681-Iniciativa-Convencional-Constituyente-del-cc-Gaspar-Dominguez-sobre-el-Derecho-a-la-salud-120001-02.pdf}
\newline {\color{gray} (Emb: 0.641, TF-IDF: 0.315)}
\newline {\color{gray} \textbf{2º:} 202-6-c-Iniciativa-Convencional-de-la-cc-Ingrid-Villena-que-crea-el-Sistema-Nacional-de-Defensa-Jurídica-Integral.pdf}
\newline {\color{gray} (Emb: 0.601, TF-IDF: 0.260)}

La ley determinará el órgano público encargado de la administración del conjunto de los fondos de este sistema. 
\newline {\color{gray} \textbf{1º:} 681-Iniciativa-Convencional-Constituyente-del-cc-Gaspar-Dominguez-sobre-el-Derecho-a-la-salud-120001-02.pdf}
\newline {\color{gray} (Emb: 1.000, TF-IDF: 1.000)}
\newline {\color{gray} \textbf{2º:} 968-Iniciativa-Convencional-Constituyente-de-la-cc-Valentina-Miranda-sobre-Derecho-a-la-salud.pdf}
\newline {\color{gray} (Emb: 0.741, TF-IDF: 0.359)}

El Sistema Nacional de Salud será financiado a través de las rentas generales de la nación. 
\newline {\color{gray} \textbf{1º:} 379-4-Iniciativa-Convencional-Constituyente-del-cc-Bastian-Labbe-sobre-Derecho-a-la-salud-1141-21-01.pdf}
\newline {\color{gray} (Emb: 0.680, TF-IDF: 0.316)}
\newline {\color{gray} \textbf{2º:} 387-4-Iniciativa-Convencional-Constituyente-de-la-cc-Francisca-Linconao-sobre-Derecho-a-la-Saud-1228-24-01.pdf}
\newline {\color{gray} (Emb: 0.647, TF-IDF: 0.285)}

El Estado generará políticas y programas de salud mental destinados a la atención y prevención con enfoque comunitario y aumentará progresivamente su financiamiento. 
\newline {\color{gray} \textbf{1º:} 681-Iniciativa-Convencional-Constituyente-del-cc-Gaspar-Dominguez-sobre-el-Derecho-a-la-salud-120001-02.pdf}
\newline {\color{gray} (Emb: 0.743, TF-IDF: 0.461)}
\newline {\color{gray} \textbf{2º:} 1030-Iniciativa-Convencional-Constituyente-del-cc-Arturo-Zuniga-sobre-Derecho-a-la-Salud.pdf}
\newline {\color{gray} (Emb: 0.740, TF-IDF: 0.447)}

Es deber del Estado velar por el fortalecimiento y desarrollo de las instituciones públicas de salud. 
\newline {\color{gray} \textbf{1º:} 681-Iniciativa-Convencional-Constituyente-del-cc-Gaspar-Dominguez-sobre-el-Derecho-a-la-salud-120001-02.pdf}
\newline {\color{gray} (Emb: 0.818, TF-IDF: 0.736)}
\newline {\color{gray} \textbf{2º:} 396-4-Iniciativa-Convencional-Constituyente-de-la-cc-Natalia-Henriquez-sobre-Derecho-a-la-salud1559-24-01.pdf}
\newline {\color{gray} (Emb: 0.765, TF-IDF: 0.672)}

La ley determinará los requisitos y procedimientos para que prestadores privados puedan integrarse al Sistema Nacional de Salud. 
\newline {\color{gray} \textbf{1º:} 681-Iniciativa-Convencional-Constituyente-del-cc-Gaspar-Dominguez-sobre-el-Derecho-a-la-salud-120001-02.pdf}
\newline {\color{gray} (Emb: 0.863, TF-IDF: 0.849)}
\newline {\color{gray} \textbf{2º:} 379-4-Iniciativa-Convencional-Constituyente-del-cc-Bastian-Labbe-sobre-Derecho-a-la-salud-1141-21-01.pdf}
\newline {\color{gray} (Emb: 0.747, TF-IDF: 0.543)}

El Sistema Nacional de Salud podrá estar integrado por prestadores públicos y privados. 
\newline {\color{gray} \textbf{1º:} 681-Iniciativa-Convencional-Constituyente-del-cc-Gaspar-Dominguez-sobre-el-Derecho-a-la-salud-120001-02.pdf}
\newline {\color{gray} (Emb: 0.887, TF-IDF: 0.747)}
\newline {\color{gray} \textbf{2º:} 396-4-Iniciativa-Convencional-Constituyente-de-la-cc-Natalia-Henriquez-sobre-Derecho-a-la-salud1559-24-01.pdf}
\newline {\color{gray} (Emb: 0.869, TF-IDF: 0.660)}

La atención primaria constituirá la base de este sistema y se promoverá la participación de las comunidades en las políticas de salud y las condiciones para su ejercicio efectivo. 
\newline {\color{gray} \textbf{1º:} 681-Iniciativa-Convencional-Constituyente-del-cc-Gaspar-Dominguez-sobre-el-Derecho-a-la-salud-120001-02.pdf}
\newline {\color{gray} (Emb: 0.864, TF-IDF: 0.800)}
\newline {\color{gray} \textbf{2º:} 396-4-Iniciativa-Convencional-Constituyente-de-la-cc-Natalia-Henriquez-sobre-Derecho-a-la-salud1559-24-01.pdf}
\newline {\color{gray} (Emb: 0.836, TF-IDF: 0.659)}

El Sistema Nacional de Salud incorporará acciones de promoción, prevención, diagnóstico, tratamiento, habilitación, rehabilitación e inclusión. 
\newline {\color{gray} \textbf{1º:} 681-Iniciativa-Convencional-Constituyente-del-cc-Gaspar-Dominguez-sobre-el-Derecho-a-la-salud-120001-02.pdf}
\newline {\color{gray} (Emb: 0.881, TF-IDF: 0.684)}
\newline {\color{gray} \textbf{2º:} 396-4-Iniciativa-Convencional-Constituyente-de-la-cc-Natalia-Henriquez-sobre-Derecho-a-la-salud1559-24-01.pdf}
\newline {\color{gray} (Emb: 0.742, TF-IDF: 0.484)}

Se regirá por los principios de equidad, solidaridad, interculturalidad, pertinencia territorial, desconcentración, eficacia, calidad, oportunidad, enfoque de género, progresividad y no discriminación. 
\newline {\color{gray} \textbf{1º:} 307-4-Iniciativa-Convencional-de-la-cc-Barbara-Rebolledo-sobre-Derecho-a-la-Salud-1517-hrs.pdf}
\newline {\color{gray} (Emb: 0.881, TF-IDF: 0.684)}
\newline {\color{gray} \textbf{2º:} 1027-Iniciativa-Convencional-Consituyente-del-cc-Felipe-Harboe-sobre-Derecho-a-la-Vida.pdf}
\newline {\color{gray} (Emb: 0.832, TF-IDF: 0.620)}

El Sistema Nacional de Salud será de carácter universal, público e integrado. 
\newline {\color{gray} \textbf{1º:} 674-Iniciativa-Convencional-Constituyente-del-cc-Cesar-Valenzuela-sobre-Seguridad-Social-121101-02.pdf}
\newline {\color{gray} (Emb: 0.973, TF-IDF: 0.927)}
\newline {\color{gray} \textbf{2º:} 682-Iniciativa-Convencional-Constituyente-del-cc-Javier-Fuchslocher-sobre-Trabajo-Decente-131101-02.pdf}
\newline {\color{gray} (Emb: 0.702, TF-IDF: 0.392)}

El Estado deberá proveer las condiciones necesarias para alcanzar el más alto nivel posible de la salud, considerando en todas sus decisiones el impacto de las determinantes sociales y ambientales sobre la salud de la población. 
\newline {\color{gray} \textbf{1º:} 674-Iniciativa-Convencional-Constituyente-del-cc-Cesar-Valenzuela-sobre-Seguridad-Social-121101-02.pdf}
\newline {\color{gray} (Emb: 1.000, TF-IDF: 1.000)}
\newline {\color{gray} \textbf{2º:} 570-Iniciativa-Convencional-Constituyente-de-cc-Valentina-Miranda-sobre-Dereco-a-la-Seguridad-Social-2008-hrs.-01-02.pdf}
\newline {\color{gray} (Emb: 0.594, TF-IDF: 0.330)}

Toda persona tiene derecho a la salud y bienestar integral, incluyendo su dimensión física y mental. 
\newline {\color{gray} \textbf{1º:} 674-Iniciativa-Convencional-Constituyente-del-cc-Cesar-Valenzuela-sobre-Seguridad-Social-121101-02.pdf}
\newline {\color{gray} (Emb: 0.882, TF-IDF: 0.804)}
\newline {\color{gray} \textbf{2º:} 381-4-Iniciativa-Convencional-Constituyente-del-cc-Bastian-Labbe-sobre-el-Derecho-a-la-seguridad-social1142-24-01.pdf}
\newline {\color{gray} (Emb: 0.825, TF-IDF: 0.441)}


\item \textbf{Artículo} \newline
El Estado asegura a todas las personas el derecho a la educación. 
\newline {\color{gray} \textbf{1º:} 681-Iniciativa-Convencional-Constituyente-del-cc-Gaspar-Dominguez-sobre-el-Derecho-a-la-salud-120001-02.pdf}
\newline {\color{gray} (Emb: 0.938, TF-IDF: 0.931)}
\newline {\color{gray} \textbf{2º:} 396-4-Iniciativa-Convencional-Constituyente-de-la-cc-Natalia-Henriquez-sobre-Derecho-a-la-salud1559-24-01.pdf}
\newline {\color{gray} (Emb: 0.911, TF-IDF: 0.416)}


\item \textbf{Artículo} \newline
Todas las personas tienen derecho a la educación. 
\newline {\color{gray} \textbf{1º:} 681-Iniciativa-Convencional-Constituyente-del-cc-Gaspar-Dominguez-sobre-el-Derecho-a-la-salud-120001-02.pdf}
\newline {\color{gray} (Emb: 1.000, TF-IDF: 1.000)}
\newline {\color{gray} \textbf{2º:} 968-Iniciativa-Convencional-Constituyente-de-la-cc-Valentina-Miranda-sobre-Derecho-a-la-salud.pdf}
\newline {\color{gray} (Emb: 0.809, TF-IDF: 0.695)}

La educación es un proceso de formación y aprendizaje permanente a lo largo de la vida, indispensable para el ejercicio de los demás derechos y para la actividad científica, tecnológica, económica y cultural del país. 
\newline {\color{gray} \textbf{1º:} 662-Iniciativa-Convencional-Constituyente-de-la-cc-Lorena-Cespedes-sobre-Derecho-a-la-Educacion-121101-02.pdf}
\newline {\color{gray} (Emb: 1.000, TF-IDF: 1.000)}
\newline {\color{gray} \textbf{2º:} 665-Iniciativa-Convencional-Constituyente-de-la-cc-Ramona-Reyes-sobre-Educacion-161101-02.pdf}
\newline {\color{gray} (Emb: 1.000, TF-IDF: 1.000)}

Sus fines son la construcción del bien común, la justicia social, el respeto de los derechos humanos y de la naturaleza, la conciencia ecológica, la convivencia democrática entre los pueblos, la prevención de la violencia y discriminación, así como, la adquisición de conocimientos, el pensamiento crítico y el desarrollo integral de las personas, considerando su dimensión cognitiva, física, social y emocional. 
\newline {\color{gray} \textbf{1º:} 517-7-Iniciativa-Convencional-Constituyente-de-cc-Andres-Cruz-sobre-Educacion-Superior-1525-hrs.-01-02.pdf}
\newline {\color{gray} (Emb: 1.000, TF-IDF: 1.000)}
\newline {\color{gray} \textbf{2º:} 629-Iniciativa-Convencional-Constituyente-de-cc-Valentina-Miranda-sobre-Derecho-a-la-educacion-1632-hrs.-01-02.pdf}
\newline {\color{gray} (Emb: 0.905, TF-IDF: 0.707)}

La educación se regirá́ por los principios de cooperación, no discriminación, inclusión, justicia, participación, solidaridad, interculturalidad, enfoque de género, pluralismo y los demás principios consagrados en esta Constitución. 
\newline {\color{gray} \textbf{1º:} 662-Iniciativa-Convencional-Constituyente-de-la-cc-Lorena-Cespedes-sobre-Derecho-a-la-Educacion-121101-02.pdf}
\newline {\color{gray} (Emb: 0.823, TF-IDF: 0.634)}
\newline {\color{gray} \textbf{2º:} 665-Iniciativa-Convencional-Constituyente-de-la-cc-Ramona-Reyes-sobre-Educacion-161101-02.pdf}
\newline {\color{gray} (Emb: 0.823, TF-IDF: 0.582)}

Tendrá un carácter no sexista y se desarrollará de forma contextualizada, considerando la pertinencia territorial, cultural y lingüística. 
\newline {\color{gray} \textbf{1º:} 662-Iniciativa-Convencional-Constituyente-de-la-cc-Lorena-Cespedes-sobre-Derecho-a-la-Educacion-121101-02.pdf}
\newline {\color{gray} (Emb: 0.519, TF-IDF: 0.526)}
\newline {\color{gray} \textbf{2º:} 665-Iniciativa-Convencional-Constituyente-de-la-cc-Ramona-Reyes-sobre-Educacion-161101-02.pdf}
\newline {\color{gray} (Emb: 0.519, TF-IDF: 0.482)}

La educación deberá orientarse hacia la calidad, entendida como el cumplimiento de los fines y principios establecidos de la educación. 
\newline {\color{gray} \textbf{1º:} 369-4-Iniciativa-Convencional-Constituyente-de-la-cc-Loreto-Vallejos-sobre-Derecho-a-la-Educacion-0900-hrs-24-01.pdf}
\newline {\color{gray} (Emb: 0.634, TF-IDF: 0.326)}
\newline {\color{gray} \textbf{2º:} 788-Iniciativa-Convencional-Constituyente-de-la-cc-Camila-Zarate-sobre-Democracia-Ecologica.pdf}
\newline {\color{gray} (Emb: 0.632, TF-IDF: 0.326)}

La ley establecerá la forma en que estos fines y principios deberán materializarse, en condiciones de equidad en las instituciones educativas y los procesos de enseñanza. 
\newline {\color{gray} \textbf{1º:} 396-4-Iniciativa-Convencional-Constituyente-de-la-cc-Natalia-Henriquez-sobre-Derecho-a-la-salud1559-24-01.pdf}
\newline {\color{gray} (Emb: 0.655, TF-IDF: 0.432)}
\newline {\color{gray} \textbf{2º:} 681-Iniciativa-Convencional-Constituyente-del-cc-Gaspar-Dominguez-sobre-el-Derecho-a-la-salud-120001-02.pdf}
\newline {\color{gray} (Emb: 0.633, TF-IDF: 0.415)}

La educación es un deber primordial e ineludible del Estado. 
\newline {\color{gray} \textbf{1º:} 681-Iniciativa-Convencional-Constituyente-del-cc-Gaspar-Dominguez-sobre-el-Derecho-a-la-salud-120001-02.pdf}
\newline {\color{gray} (Emb: 1.000, TF-IDF: 1.000)}
\newline {\color{gray} \textbf{2º:} 921-Iniciativa-Convencional-Constituyente-de-la-cc-Yarela-Gomez-sobre-Modernizacion-del-Estado.pdf}
\newline {\color{gray} (Emb: 0.664, TF-IDF: 0.356)}


\item \textbf{Artículo} \newline
La educación pública constituye el eje estratégico del Sistema Nacional de Educación; su ampliación y fortalecimiento es un deber primordial del Estado. 
\newline {\color{gray} \textbf{1º:} 327-7-Iniciativa-Convencional-Constituyente-de-la-cc-Carolina-Videla-sobre-Derecho-a-la-Comunicacion.pdf}
\newline {\color{gray} (Emb: 0.557, TF-IDF: 0.319)}
\newline {\color{gray} \textbf{2º:} 310-7-Iniciativa-Convencional-de-la-cc-Carolina-Videla-sobre-Derecho-a-la-Comunicacion-2000-hrs.pdf}
\newline {\color{gray} (Emb: 0.557, TF-IDF: 0.233)}

El Estado deberá financiar este Sistema de forma permanente, directa, pertinente y suficiente, a través de aportes basales, a fin de cumplir plena y equitativamente con los fines y principios de la educación. 
\newline {\color{gray} \textbf{1º:} 468-4-Iniciativa-Convencional-Constituyente-del-cc-Marcos-Barraza-sobre-Personas-Mayores-1950-31-01.pdf}
\newline {\color{gray} (Emb: 0.572, TF-IDF: 0.613)}
\newline {\color{gray} \textbf{2º:} 363-4-Iniciativa-Convencional-Constituyente-de-la-cc-Janis-Meneses-sobre-Derecho-a-la-Educacion-Publica-1844-hrs-21-01.pdf}
\newline {\color{gray} (Emb: 0.561, TF-IDF: 0.270)}

El Estado deberá brindar oportunidades y apoyos adicionales a quienes están en situación de discapacidad y en riesgo de exclusión. 
\newline {\color{gray} \textbf{1º:} 127-4-c-Iniciativa-de-la-cc-Rocio-Cantuarias-Derecho-a-la-eduacion-y-libertad-de-ensenanza.pdf}
\newline {\color{gray} (Emb: 0.847, TF-IDF: 0.794)}
\newline {\color{gray} \textbf{2º:} 1029-Iniciativa-Convencional-Consituyente-de-la-cc-Carolina-Vilches-sobre-Regeneracion-de-la-Vida.pdf}
\newline {\color{gray} (Emb: 0.675, TF-IDF: 0.435)}

Es deber del Estado promover el derecho a la educación permanente a través de oportunidades formativas múltiples, dentro y fuera del Sistema Nacional de Educación, fomentando diversos espacios de desarrollo y aprendizaje integral para todas las personas. 
\newline {\color{gray} \textbf{1º:} 979-Iniciativa-Convencional-Constituyente-de-la-Tania-Madriaga-sobre-Educacion.pdf}
\newline {\color{gray} (Emb: 0.773, TF-IDF: 0.525)}
\newline {\color{gray} \textbf{2º:} 665-Iniciativa-Convencional-Constituyente-de-la-cc-Ramona-Reyes-sobre-Educacion-161101-02.pdf}
\newline {\color{gray} (Emb: 0.747, TF-IDF: 0.448)}

Este Sistema promoverá la diversidad de saberes artísticos, ecológicos, culturales y filosóficos que conviven en el país. 
\newline {\color{gray} \textbf{1º:} 662-Iniciativa-Convencional-Constituyente-de-la-cc-Lorena-Cespedes-sobre-Derecho-a-la-Educacion-121101-02.pdf}
\newline {\color{gray} (Emb: 0.809, TF-IDF: 0.714)}
\newline {\color{gray} \textbf{2º:} 665-Iniciativa-Convencional-Constituyente-de-la-cc-Ramona-Reyes-sobre-Educacion-161101-02.pdf}
\newline {\color{gray} (Emb: 0.809, TF-IDF: 0.714)}

El Estado deberá articular, gestionar y financiar un Sistema de Educación Pública, de carácter laico y gratuito, compuesto por establecimientos e instituciones estatales de todos los niveles y modalidades educativas. 
\newline {\color{gray} \textbf{1º:} 665-Iniciativa-Convencional-Constituyente-de-la-cc-Ramona-Reyes-sobre-Educacion-161101-02.pdf}
\newline {\color{gray} (Emb: 0.806, TF-IDF: 0.534)}
\newline {\color{gray} \textbf{2º:} 662-Iniciativa-Convencional-Constituyente-de-la-cc-Lorena-Cespedes-sobre-Derecho-a-la-Educacion-121101-02.pdf}
\newline {\color{gray} (Emb: 0.806, TF-IDF: 0.534)}

La ley establecerá los requisitos para el reconocimiento oficial de los establecimientos educacionales. 
\newline {\color{gray} \textbf{1º:} 665-Iniciativa-Convencional-Constituyente-de-la-cc-Ramona-Reyes-sobre-Educacion-161101-02.pdf}
\newline {\color{gray} (Emb: 0.745, TF-IDF: 0.518)}
\newline {\color{gray} \textbf{2º:} 662-Iniciativa-Convencional-Constituyente-de-la-cc-Lorena-Cespedes-sobre-Derecho-a-la-Educacion-121101-02.pdf}
\newline {\color{gray} (Emb: 0.737, TF-IDF: 0.507)}

El Estado ejercerá labores de coordinación, regulación, mejoramiento y supervigilancia del Sistema. 
\newline {\color{gray} \textbf{1º:} 665-Iniciativa-Convencional-Constituyente-de-la-cc-Ramona-Reyes-sobre-Educacion-161101-02.pdf}
\newline {\color{gray} (Emb: 0.937, TF-IDF: 0.822)}
\newline {\color{gray} \textbf{2º:} 662-Iniciativa-Convencional-Constituyente-de-la-cc-Lorena-Cespedes-sobre-Derecho-a-la-Educacion-121101-02.pdf}
\newline {\color{gray} (Emb: 0.937, TF-IDF: 0.822)}

Se articulará bajo el principio de colaboración y tendrá como centro la experiencia de aprendizaje de los estudiantes. 
\newline {\color{gray} \textbf{1º:} 164-4-c-Iniciativa-del-cc-Jorge-Arancibia-Reconocimiento-y-Tratamiento-del-Agua.pdf}
\newline {\color{gray} (Emb: 0.668, TF-IDF: 0.342)}
\newline {\color{gray} \textbf{2º:} 9-2-Iniciativa-Convencional-Constituyente-del-cc-Ignacio-Achurra-y-otros-cta.pdf}
\newline {\color{gray} (Emb: 0.644, TF-IDF: 0.320)}

El Sistema Nacional de Educación estará integrado por los establecimientos e instituciones de educación parvularia, básica, media y superior, creadas o reconocidas por el Estado. 
\newline {\color{gray} \textbf{1º:} 662-Iniciativa-Convencional-Constituyente-de-la-cc-Lorena-Cespedes-sobre-Derecho-a-la-Educacion-121101-02.pdf}
\newline {\color{gray} (Emb: 0.820, TF-IDF: 0.379)}
\newline {\color{gray} \textbf{2º:} 665-Iniciativa-Convencional-Constituyente-de-la-cc-Ramona-Reyes-sobre-Educacion-161101-02.pdf}
\newline {\color{gray} (Emb: 0.820, TF-IDF: 0.326)}

La educación será de acceso universal en todos sus niveles y obligatoria desde el nivel básico hasta la educación media. 
\newline {\color{gray} \textbf{1º:} 646-Iniciativa-Convencional-Constituyente-de-la-cc-Maria-Trinidad-Castillo-sobre-Derecho-a-la-Educacion121101-02.pdf}
\newline {\color{gray} (Emb: 0.669, TF-IDF: 0.504)}
\newline {\color{gray} \textbf{2º:} 665-Iniciativa-Convencional-Constituyente-de-la-cc-Ramona-Reyes-sobre-Educacion-161101-02.pdf}
\newline {\color{gray} (Emb: 0.634, TF-IDF: 0.355)}

Las instituciones que lo conforman estarán sujetas al régimen común que fije la ley, serán de carácter democrático, no podrán discriminar en su acceso, se regirán por los fines y principios de este derecho, y tendrán prohibida toda forma de lucro. 
\newline {\color{gray} \textbf{1º:} 662-Iniciativa-Convencional-Constituyente-de-la-cc-Lorena-Cespedes-sobre-Derecho-a-la-Educacion-121101-02.pdf}
\newline {\color{gray} (Emb: 0.585, TF-IDF: 0.469)}
\newline {\color{gray} \textbf{2º:} 665-Iniciativa-Convencional-Constituyente-de-la-cc-Ramona-Reyes-sobre-Educacion-161101-02.pdf}
\newline {\color{gray} (Emb: 0.584, TF-IDF: 0.349)}


\item \textbf{Artículo} \newline
La Constitución reconoce el derecho de las y los integrantes de cada comunidad educativa a participar en las definiciones del proyecto educativo y en las decisiones de cada establecimiento, así como en el diseño, implementación y evaluación de la política educacional local y nacional para el ejercicio del derecho a la educación. 
\newline {\color{gray} \textbf{1º:} 623-Iniciativa-Convencional-Constituyente-de-cc-Lisette-Vergara-Educacion-Plurinacional-Democratica-Pluralista-y-Popular.pdf}
\newline {\color{gray} (Emb: 0.652, TF-IDF: 0.488)}
\newline {\color{gray} \textbf{2º:} 363-4-Iniciativa-Convencional-Constituyente-de-la-cc-Janis-Meneses-sobre-Derecho-a-la-Educacion-Publica-1844-hrs-21-01.pdf}
\newline {\color{gray} (Emb: 0.610, TF-IDF: 0.411)}

La ley especificará las condiciones, órganos y procedimientos que permitan asegurar la participación vinculante de las y los integrantes de la comunidad educativa. 
\newline {\color{gray} \textbf{1º:} 650-Iniciativa-Convencional-Constituyente-de-la-cc-Dayyana-Gonzalez-sobre-Educacion-181101-02.pdf}
\newline {\color{gray} (Emb: 0.745, TF-IDF: 0.404)}
\newline {\color{gray} \textbf{2º:} 363-4-Iniciativa-Convencional-Constituyente-de-la-cc-Janis-Meneses-sobre-Derecho-a-la-Educacion-Publica-1844-hrs-21-01.pdf}
\newline {\color{gray} (Emb: 0.700, TF-IDF: 0.249)}


\item \textbf{Artículo} \newline
La Constitución garantiza la libertad de enseñanza y es deber del Estado respetarla. 
\newline {\color{gray} \textbf{1º:} 665-Iniciativa-Convencional-Constituyente-de-la-cc-Ramona-Reyes-sobre-Educacion-161101-02.pdf}
\newline {\color{gray} (Emb: 0.714, TF-IDF: 0.337)}
\newline {\color{gray} \textbf{2º:} 662-Iniciativa-Convencional-Constituyente-de-la-cc-Lorena-Cespedes-sobre-Derecho-a-la-Educacion-121101-02.pdf}
\newline {\color{gray} (Emb: 0.650, TF-IDF: 0.314)}

Ésta comprende la libertad de padres, madres, apoderados y apoderadas a elegir el tipo de educación de las personas a su cargo, respetando el interés superior y la autonomía progresiva de niños, niñas y adolescentes. 
\newline {\color{gray} \textbf{1º:} 386-4-Iniciativa-Convencional-Constituyente-de-la-cc-Francisca-Linconao-sobre-Educacion-Intercultural-1228-24-01.pdf}
\newline {\color{gray} (Emb: 0.652, TF-IDF: 0.351)}
\newline {\color{gray} \textbf{2º:} 698-Iniciativa-Convencional-Constituyente-de-la-cc-Alejandra-Flores-sobre-Educacion-01-02.pdf}
\newline {\color{gray} (Emb: 0.609, TF-IDF: 0.306)}

Las y los profesores y educadores son titulares de la libertad de cátedra en el ejercicio de sus funciones, en el marco de los fines y principios de la educación. 
\newline {\color{gray} \textbf{1º:} 560-Iniciativa-Convencional-Constituyente-de-cc-Andres-Cruz-sobre-Ministerio-Publico-2016-hrs.-01-02.pdf}
\newline {\color{gray} (Emb: 0.705, TF-IDF: 0.347)}
\newline {\color{gray} \textbf{2º:} 90-6-Iniciativa-Convencional-Constituyente-del-cc-Tomas-Laibe-y-otros.pdf}
\newline {\color{gray} (Emb: 0.705, TF-IDF: 0.269)}


\item \textbf{Artículo} \newline
Las y los trabajadores de educación parvularia, básica y media que se desempeñen en establecimientos que reciban recursos del Estado, gozarán de los mismos derechos que la ley contemple para su respectiva función. 
\newline {\color{gray} \textbf{1º:} 363-4-Iniciativa-Convencional-Constituyente-de-la-cc-Janis-Meneses-sobre-Derecho-a-la-Educacion-Publica-1844-hrs-21-01.pdf}
\newline {\color{gray} (Emb: 0.682, TF-IDF: 0.536)}
\newline {\color{gray} \textbf{2º:} 330-4-Iniciativa-Convencional-Constituyente-de-la-cc-Tania-Madriaga-sobre-Educacion-Inclusiva.pdf}
\newline {\color{gray} (Emb: 0.648, TF-IDF: 0.501)}

Para esto, otorgará estabilidad en el ejercicio de sus funciones; asegurando condiciones laborales óptimas y resguardando su autonomía profesional. 
\newline {\color{gray} \textbf{1º:} 644-Iniciativiva-Convencional-Constituyene-de-la-cc-Isabel-Godoy-sobre-Derechos-Linguisticos-1738-01-02.pdf}
\newline {\color{gray} (Emb: 0.554, TF-IDF: 0.345)}
\newline {\color{gray} \textbf{2º:} 709-Iniciativa-Convencional-Constituyente-de-la-cc-Gloria-Alvarado-sobre-Cooperativismo-01-02.pdf}
\newline {\color{gray} (Emb: 0.500, TF-IDF: 0.344)}

El Estado garantiza el desarrollo del quehacer pedagógico y educativo de quienes trabajen en establecimientos que reciban fondos públicos, incluyendo su formación inicial y continua, su ejercicio reflexivo y colaborativo y la investigación pedagógica, en coherencia con los principios y fines de la educación. 
\newline {\color{gray} \textbf{1º:} 363-4-Iniciativa-Convencional-Constituyente-de-la-cc-Janis-Meneses-sobre-Derecho-a-la-Educacion-Publica-1844-hrs-21-01.pdf}
\newline {\color{gray} (Emb: 0.668, TF-IDF: 0.434)}
\newline {\color{gray} \textbf{2º:} 369-4-Iniciativa-Convencional-Constituyente-de-la-cc-Loreto-Vallejos-sobre-Derecho-a-la-Educacion-0900-hrs-24-01.pdf}
\newline {\color{gray} (Emb: 0.617, TF-IDF: 0.311)}

Asimismo, valora y fomenta la contribución de las y los educadores y asistentes de la educación, incluyendo a las y los educadores tradicionales. 
\newline {\color{gray} \textbf{1º:} 629-Iniciativa-Convencional-Constituyente-de-cc-Valentina-Miranda-sobre-Derecho-a-la-educacion-1632-hrs.-01-02.pdf}
\newline {\color{gray} (Emb: 0.662, TF-IDF: 0.382)}
\newline {\color{gray} \textbf{2º:} 635-4-Iniciativa-Convencional-Constituyente-del-cc-Cristobal-Andrade-sobre-Libertad-de-Conciencia-1730-01-02.pdf}
\newline {\color{gray} (Emb: 0.626, TF-IDF: 0.371)}

La Constitución reconoce el rol fundamental de las profesoras y profesores, como profesionales en el Sistema Nacional de Educación. 
\newline {\color{gray} \textbf{1º:} 252-4-Iniciativa-Convencional-de-la-cc-Tatiana-Urrutia-sobre-Libertad-de-Conciencia-y-Culto-1149-hrs.pdf}
\newline {\color{gray} (Emb: 0.746, TF-IDF: 0.614)}
\newline {\color{gray} \textbf{2º:} 363-4-Iniciativa-Convencional-Constituyente-de-la-cc-Janis-Meneses-sobre-Derecho-a-la-Educacion-Publica-1844-hrs-21-01.pdf}
\newline {\color{gray} (Emb: 0.737, TF-IDF: 0.491)}

Las y los trabajadores de la educación son agentes claves para la garantía del derecho a la educación. 
\newline {\color{gray} \textbf{1º:} 363-4-Iniciativa-Convencional-Constituyente-de-la-cc-Janis-Meneses-sobre-Derecho-a-la-Educacion-Publica-1844-hrs-21-01.pdf}
\newline {\color{gray} (Emb: 0.862, TF-IDF: 0.887)}
\newline {\color{gray} \textbf{2º:} 369-4-Iniciativa-Convencional-Constituyente-de-la-cc-Loreto-Vallejos-sobre-Derecho-a-la-Educacion-0900-hrs-24-01.pdf}
\newline {\color{gray} (Emb: 0.697, TF-IDF: 0.451)}


\item \textbf{Artículo} \newline
El ingreso, permanencia y promoción de quienes estudien en la educación superior se regirá por los principios de equidad e inclusión, con especial atención a los grupos históricamente excluidos, excluyendo cualquier tipo de discriminación arbitraria. 
\newline {\color{gray} \textbf{1º:} 709-Iniciativa-Convencional-Constituyente-de-la-cc-Gloria-Alvarado-sobre-Cooperativismo-01-02.pdf}
\newline {\color{gray} (Emb: 0.559, TF-IDF: 0.325)}
\newline {\color{gray} \textbf{2º:} 363-4-Iniciativa-Convencional-Constituyente-de-la-cc-Janis-Meneses-sobre-Derecho-a-la-Educacion-Publica-1844-hrs-21-01.pdf}
\newline {\color{gray} (Emb: 0.550, TF-IDF: 0.297)}

Los estudios de educación superior, conducentes a títulos y grados académicos iniciales, serán gratuitos en las instituciones públicas y en aquellas privadas que determine la ley. 
\newline {\color{gray} \textbf{1º:} 363-4-Iniciativa-Convencional-Constituyente-de-la-cc-Janis-Meneses-sobre-Derecho-a-la-Educacion-Publica-1844-hrs-21-01.pdf}
\newline {\color{gray} (Emb: 0.572, TF-IDF: 0.344)}
\newline {\color{gray} \textbf{2º:} 113-5-c-Iniciativa-del-cc-Elsa-Labrana-sobre-Soberania-Alimentaria.pdf}
\newline {\color{gray} (Emb: 0.551, TF-IDF: 0.260)}


\item \textbf{Artículo} \newline
El Sistema de Educación Superior estará conformado por las Universidades, Institutos Profesionales, Centros de Formación Técnica, escuelas de formación de las Fuerzas Armadas y Seguridad, además de las Academias creadas o reconocidas por el Estado. 
\newline {\color{gray} \textbf{1º:} 276-4-Iniciativa-Convencional-del-cc-Harry-Jurgensen-sobre-Presencialidad-en-la-Educacion-17-01-1155-hrs.pdf}
\newline {\color{gray} (Emb: 0.636, TF-IDF: 0.402)}
\newline {\color{gray} \textbf{2º:} 593-Iniciativa-Convencional-Constituyente-de-cc-Loreto-Vidal-Practica-y-goce-de-las-artes-como-garantia-constitucional.pdf}
\newline {\color{gray} (Emb: 0.577, TF-IDF: 0.402)}

Estas instituciones se regirán por los principios de la educación y considerarán las necesidades locales, regionales y nacionales. 
\newline {\color{gray} \textbf{1º:} 9-2-Iniciativa-Convencional-Constituyente-del-cc-Ignacio-Achurra-y-otros-cta.pdf}
\newline {\color{gray} (Emb: 0.581, TF-IDF: 0.357)}
\newline {\color{gray} \textbf{2º:} 860-Iniciativa-Convencional-Constituyente-del-cc-Jorge-Abarca-sobre-Regimen-Publico-Economico.pdf}
\newline {\color{gray} (Emb: 0.579, TF-IDF: 0.311)}

Tendrán prohibida toda forma de lucro. 
\newline {\color{gray} \textbf{1º:} 369-4-Iniciativa-Convencional-Constituyente-de-la-cc-Loreto-Vallejos-sobre-Derecho-a-la-Educacion-0900-hrs-24-01.pdf}
\newline {\color{gray} (Emb: 0.707, TF-IDF: 0.250)}
\newline {\color{gray} \textbf{2º:} 665-Iniciativa-Convencional-Constituyente-de-la-cc-Ramona-Reyes-sobre-Educacion-161101-02.pdf}
\newline {\color{gray} (Emb: 0.704, TF-IDF: 0.208)}

Las instituciones de educación superior tienen la misión de enseñar, producir y socializar el conocimiento. 
\newline {\color{gray} \textbf{1º:} 979-Iniciativa-Convencional-Constituyente-de-la-Tania-Madriaga-sobre-Educacion.pdf}
\newline {\color{gray} (Emb: 0.574, TF-IDF: 0.373)}
\newline {\color{gray} \textbf{2º:} 662-Iniciativa-Convencional-Constituyente-de-la-cc-Lorena-Cespedes-sobre-Derecho-a-la-Educacion-121101-02.pdf}
\newline {\color{gray} (Emb: 0.521, TF-IDF: 0.271)}

La Constitución protege la libertad de cátedra, la investigación y la libre discusión de las ideas de los académicos y las académicas de las universidades creadas o reconocidas por ley. 
\newline {\color{gray} \textbf{1º:} 876-Iniciativa-Convencional-Constituyente-de-la-cc-Elisa-Giustinianovich-sobre-Transicion-Productiva.pdf}
\newline {\color{gray} (Emb: 0.620, TF-IDF: 0.353)}
\newline {\color{gray} \textbf{2º:} 732-Iniciativa-Convencional-Constituyente-del-cc-Bastian-Labbe-crea-el-Servicio-de-proteccion-de-bienes-comunes.pdf}
\newline {\color{gray} (Emb: 0.581, TF-IDF: 0.348)}

La formación tendrá un enfoque coherente con los fines y principios de la Educación. 
\newline {\color{gray} \textbf{1º:} 646-Iniciativa-Convencional-Constituyente-de-la-cc-Maria-Trinidad-Castillo-sobre-Derecho-a-la-Educacion121101-02.pdf}
\newline {\color{gray} (Emb: 0.574, TF-IDF: 0.503)}
\newline {\color{gray} \textbf{2º:} 414-5-Iniciativa-Convencional-del-cc-Bernardo-Fontaine-sobre-Estatuto-de-los-Minerales-1500-25-01.pdf}
\newline {\color{gray} (Emb: 0.562, TF-IDF: 0.458)}

Las instituciones de educación superior del Estado forman parte del Sistema de Educación Pública y su financiamiento se sujetará a lo dispuesto por esta Constitución, debiendo garantizar el cumplimiento íntegro de sus funciones de docencia, investigación y colaboración con la sociedad. 
\newline {\color{gray} \textbf{1º:} 330-4-Iniciativa-Convencional-Constituyente-de-la-cc-Tania-Madriaga-sobre-Educacion-Inclusiva.pdf}
\newline {\color{gray} (Emb: 0.593, TF-IDF: 0.358)}
\newline {\color{gray} \textbf{2º:} 368-7-Iniciativa-Convencional-Constituyente-del-cc-Francisco-Caamano-sobre-los-conocimientos-0900-hrs-24-01.pdf}
\newline {\color{gray} (Emb: 0.532, TF-IDF: 0.358)}

El Estado velará por el acceso a la educación superior de todas las personas que cumplan los requisitos establecidos por la ley. 
\newline {\color{gray} \textbf{1º:} 145-4-c-Iniciativa-del-cc-Manuel-Jose-Ossandon-libertad-de-Conciencia-Expresion-y-de-Ensenanza.pdf}
\newline {\color{gray} (Emb: 0.622, TF-IDF: 0.322)}
\newline {\color{gray} \textbf{2º:} 290-4-Iniciativa-Convencional-de-la-cc-Tatiana-Urrutia-sobre-Libertad-de-Expresion-1604-hrs.pdf}
\newline {\color{gray} (Emb: 0.599, TF-IDF: 0.302)}


\item \textbf{Artículo} \newline
La Constitución reconoce la autonomía de los pueblos originarios para desarrollar sus propios establecimientos e instituciones de conformidad a sus costumbres y cultura, respetando los fines y principios de la educación, y dentro de los marcos del Sistema Nacional de Educación establecidos por la ley. 
\newline {\color{gray} \textbf{1º:} 665-Iniciativa-Convencional-Constituyente-de-la-cc-Ramona-Reyes-sobre-Educacion-161101-02.pdf}
\newline {\color{gray} (Emb: 0.638, TF-IDF: 0.500)}
\newline {\color{gray} \textbf{2º:} 662-Iniciativa-Convencional-Constituyente-de-la-cc-Lorena-Cespedes-sobre-Derecho-a-la-Educacion-121101-02.pdf}
\newline {\color{gray} (Emb: 0.638, TF-IDF: 0.284)}


\item \textbf{Artículo} \newline
Este derecho comprende la garantía de alimentos especiales para quienes lo requieran por motivos de salud. 
\newline {\color{gray} \textbf{1º:} 629-Iniciativa-Convencional-Constituyente-de-cc-Valentina-Miranda-sobre-Derecho-a-la-educacion-1632-hrs.-01-02.pdf}
\newline {\color{gray} (Emb: 0.771, TF-IDF: 0.402)}
\newline {\color{gray} \textbf{2º:} 665-Iniciativa-Convencional-Constituyente-de-la-cc-Ramona-Reyes-sobre-Educacion-161101-02.pdf}
\newline {\color{gray} (Emb: 0.733, TF-IDF: 0.360)}

El Estado garantizará en forma continua y permanente la disponibilidad y el acceso a los alimentos que satisfagan este derecho, especialmente en zonas aisladas geográficamente. 
\newline {\color{gray} \textbf{1º:} 698-Iniciativa-Convencional-Constituyente-de-la-cc-Alejandra-Flores-sobre-Educacion-01-02.pdf}
\newline {\color{gray} (Emb: 0.679, TF-IDF: 0.300)}
\newline {\color{gray} \textbf{2º:} 842-Iniciativa-Convencional-Constituyente-de-la-cc-Francisca-Linconao-sobre-Derecho-a-la-Justicia-Intercultural.pdf}
\newline {\color{gray} (Emb: 0.668, TF-IDF: 0.295)}

Toda persona tiene derecho a una alimentación saludable, suficiente, nutricionalmente completa, pertinente culturalmente y adecuada. 
\newline {\color{gray} \textbf{1º:} 665-Iniciativa-Convencional-Constituyente-de-la-cc-Ramona-Reyes-sobre-Educacion-161101-02.pdf}
\newline {\color{gray} (Emb: 0.638, TF-IDF: 0.353)}
\newline {\color{gray} \textbf{2º:} 369-4-Iniciativa-Convencional-Constituyente-de-la-cc-Loreto-Vallejos-sobre-Derecho-a-la-Educacion-0900-hrs-24-01.pdf}
\newline {\color{gray} (Emb: 0.636, TF-IDF: 0.311)}

Adicionalmente, fomentará una producción agropecuaria ecológicamente sustentable, la agricultura campesina, la pesca artesanal, y promoverá el patrimonio culinario y gastronómico del país. 
\newline {\color{gray} \textbf{1º:} 773-Iniciativa-Convencional-Constituyente-de-la-cc-Adriana-Cancino-sobre-Derecho-a-la-Alimentacion-Adecuada.pdf}
\newline {\color{gray} (Emb: 0.800, TF-IDF: 0.564)}
\newline {\color{gray} \textbf{2º:} 379-4-Iniciativa-Convencional-Constituyente-del-cc-Bastian-Labbe-sobre-Derecho-a-la-salud-1141-21-01.pdf}
\newline {\color{gray} (Emb: 0.640, TF-IDF: 0.321)}


\item \textbf{Artículo} \newline
La ley asegurará el involucramiento de las personas y comunidades con la práctica del deporte, incluido el de niños, niñas y adolescentes en los establecimientos educacionales, así como la participación en la dirección de las diferentes formas de instituciones deportivas. 
\newline {\color{gray} \textbf{1º:} 336-4-Iniciativa-Convencional-Constituyente-de-la-cc-Damaris-Abarca-sobre-Derecho-al-Deporte.pdf}
\newline {\color{gray} (Emb: 0.672, TF-IDF: 0.463)}
\newline {\color{gray} \textbf{2º:} 517-7-Iniciativa-Convencional-Constituyente-de-cc-Andres-Cruz-sobre-Educacion-Superior-1525-hrs.-01-02.pdf}
\newline {\color{gray} (Emb: 0.523, TF-IDF: 0.236)}

La ley regulará y establecerá los principios aplicables a las instituciones públicas o privadas que tengan por objeto la gestión del deporte profesional como actividad social, cultural y económica, debiendo garantizar siempre la democracia y participación vinculante de sus organizaciones. 
\newline {\color{gray} \textbf{1º:} 336-4-Iniciativa-Convencional-Constituyente-de-la-cc-Damaris-Abarca-sobre-Derecho-al-Deporte.pdf}
\newline {\color{gray} (Emb: 1.000, TF-IDF: 1.000)}
\newline {\color{gray} \textbf{2º:} 385-3-Iniciativa-Convencional-Constituyente-del-cc-Felipe-Mena-sobre-Municipalidades-1200-24-01.pdf}
\newline {\color{gray} (Emb: 0.587, TF-IDF: 0.301)}

El Estado reconoce la función social del deporte, en tanto permite la participación colectiva, la asociatividad, la integración e inserción social, así como el mantenimiento y mejora de la salud. 
\newline {\color{gray} \textbf{1º:} 424-4-Iniciatva-Convencional-Constituyente-del-cc-Cristian-Monckeberg-1458-26-01.pdf}
\newline {\color{gray} (Emb: 0.966, TF-IDF: 0.781)}
\newline {\color{gray} \textbf{2º:} 685-Iniciativa-Convencional-Constituyente-del-cc-Mario-Vargas-sobre-Derecho-a-la-Cultural-Corporal-141101-02.pdf}
\newline {\color{gray} (Emb: 0.886, TF-IDF: 0.592)}

Para lograr estos objetivos, se podrán considerar políticas diferenciadas según lo disponga la ley. 
\newline {\color{gray} \textbf{1º:} 345-4-Iniciativa-Convencional-Constituyente-de-la-cc-Alejandra-Flores-sobre-Derecho-a-la-Alimentacion.pdf}
\newline {\color{gray} (Emb: 0.605, TF-IDF: 0.352)}
\newline {\color{gray} \textbf{2º:} 105-7-c-Iniciativa-del-cc-Miguel-Angel-Botto-cultura-y-patrimonio.pdf}
\newline {\color{gray} (Emb: 0.529, TF-IDF: 0.347)}

El Estado garantizará el ejercicio de este derecho en sus distintas dimensiones y disciplinas, ya sean recreacionales, educativas, competitivas o de alto rendimiento. 
\newline {\color{gray} \textbf{1º:} 773-Iniciativa-Convencional-Constituyente-de-la-cc-Adriana-Cancino-sobre-Derecho-a-la-Alimentacion-Adecuada.pdf}
\newline {\color{gray} (Emb: 0.655, TF-IDF: 0.475)}
\newline {\color{gray} \textbf{2º:} 773-Iniciativa-Convencional-Constituyente-de-la-cc-Adriana-Cancino-sobre-Derecho-a-la-Alimentacion-Adecuada.pdf}
\newline {\color{gray} (Emb: 0.625, TF-IDF: 0.280)}

Todas las personas tienen derecho al deporte, a la actividad física y a las prácticas corporales. 
\newline {\color{gray} \textbf{1º:} 773-Iniciativa-Convencional-Constituyente-de-la-cc-Adriana-Cancino-sobre-Derecho-a-la-Alimentacion-Adecuada.pdf}
\newline {\color{gray} (Emb: 0.617, TF-IDF: 0.299)}
\newline {\color{gray} \textbf{2º:} 968-Iniciativa-Convencional-Constituyente-de-la-cc-Valentina-Miranda-sobre-Derecho-a-la-salud.pdf}
\newline {\color{gray} (Emb: 0.560, TF-IDF: 0.280)}


\item \textbf{Artículo} \newline
La Constitución asegura el derecho a la igualdad. 
\newline {\color{gray} \textbf{1º:} 336-4-Iniciativa-Convencional-Constituyente-de-la-cc-Damaris-Abarca-sobre-Derecho-al-Deporte.pdf}
\newline {\color{gray} (Emb: 0.968, TF-IDF: 0.803)}
\newline {\color{gray} \textbf{2º:} 396-4-Iniciativa-Convencional-Constituyente-de-la-cc-Natalia-Henriquez-sobre-Derecho-a-la-salud1559-24-01.pdf}
\newline {\color{gray} (Emb: 0.585, TF-IDF: 0.307)}

En Chile no hay persona ni grupo privilegiado. 
\newline {\color{gray} \textbf{1º:} 336-4-Iniciativa-Convencional-Constituyente-de-la-cc-Damaris-Abarca-sobre-Derecho-al-Deporte.pdf}
\newline {\color{gray} (Emb: 0.921, TF-IDF: 0.967)}
\newline {\color{gray} \textbf{2º:} 383-4-Iniciativa-Convencional-Constituyente-de-la-cc-Lorena-Cespedes-sobre-el-Derecho-a-la-Actividad-Fisica-1142-24-01.pdf}
\newline {\color{gray} (Emb: 0.598, TF-IDF: 0.297)}

Queda prohibida toda forma de esclavitud. 
\newline {\color{gray} \textbf{1º:} 263-4-Iniciativa-Convencional-de-la-cc-Janis-Meneses-sobre-Derecho-de-Peticion-1153-hrs.pdf}
\newline {\color{gray} (Emb: 0.571, TF-IDF: 0.268)}
\newline {\color{gray} \textbf{2º:} 383-4-Iniciativa-Convencional-Constituyente-de-la-cc-Lorena-Cespedes-sobre-el-Derecho-a-la-Actividad-Fisica-1142-24-01.pdf}
\newline {\color{gray} (Emb: 0.566, TF-IDF: 0.246)}

Se asegura el derecho a la protección contra toda forma de discriminación, en especial cuando se funde en uno o más motivos tales como nacionalidad o apatridia, edad, sexo, orientación sexual o afectiva, identidad y expresión de género, diversidad corporal, religión o creencia, etnia, pertenencia a un pueblo y nación indígena o tribal, opiniones políticas o de cualquier otra naturaleza, clase social, ruralidad, situación migratoria o de refugio, discapacidad, condición de salud mental o física, estado civil, filiación o cualquier otra condición social. 
\newline {\color{gray} \textbf{1º:} 278-4-Iniciativa-Convencional-del-cc-Pedro-Munoz-sobre-Derecho-a-la-Igualdad-y-no-Discriminacion-1159-hrs.pdf}
\newline {\color{gray} (Emb: 0.977, TF-IDF: 0.861)}
\newline {\color{gray} \textbf{2º:} 682-Iniciativa-Convencional-Constituyente-del-cc-Javier-Fuchslocher-sobre-Trabajo-Decente-131101-02.pdf}
\newline {\color{gray} (Emb: 0.840, TF-IDF: 0.762)}

Se prohíbe y sanciona toda forma de discriminación especialmente aquella basada en alguna de las categorías mencionadas anteriormente u otras que tengan por objeto o resultado anular o menoscabar la dignidad humana, el goce y ejercicio de los derechos de toda persona. 
\newline {\color{gray} \textbf{1º:} 278-4-Iniciativa-Convencional-del-cc-Pedro-Munoz-sobre-Derecho-a-la-Igualdad-y-no-Discriminacion-1159-hrs.pdf}
\newline {\color{gray} (Emb: 1.000, TF-IDF: 1.000)}
\newline {\color{gray} \textbf{2º:} 537-Iniciativa-Convencional-Constituyente-del-cc-Manuel-Ossandon-sobre-Derecho-a-la-salud-1709-01-02-1.pdf}
\newline {\color{gray} (Emb: 0.990, TF-IDF: 0.750)}

El Estado deberá respetar, proteger, promover y garantizar los derechos fundamentales, sin discriminación. 
\newline {\color{gray} \textbf{1º:} 380-4-Iniciativa-Convencional-Constituyente-del-cc-Bastian-Labbe-sobre-el-Derecho-al-Trabajo1141-24-01.pdf}
\newline {\color{gray} (Emb: 0.676, TF-IDF: 0.727)}
\newline {\color{gray} \textbf{2º:} 611-Iniciativa-Convencional-Constituyente-de-cc-Valentina-Miranda-sobre-Derechos-de-las-Personas-Privadas-de-Libertad.pdf}
\newline {\color{gray} (Emb: 0.663, TF-IDF: 0.478)}

La ley determinará las medidas de prevención, prohibición, sanción y reparación de todas las formas de discriminación, en los ámbitos público y privado, así como los mecanismos para garantizar la igualdad material y sustantiva entre todas las personas. 
\newline {\color{gray} \textbf{1º:} 278-4-Iniciativa-Convencional-del-cc-Pedro-Munoz-sobre-Derecho-a-la-Igualdad-y-no-Discriminacion-1159-hrs.pdf}
\newline {\color{gray} (Emb: 0.911, TF-IDF: 0.745)}
\newline {\color{gray} \textbf{2º:} 726-Iniciativa-Convencional-Constituyente-de-la-cc-Valentina-Miranda-sobre-No-discriminacion-01-02.pdf}
\newline {\color{gray} (Emb: 0.784, TF-IDF: 0.458)}

El Estado deberá adoptar todas las medidas necesarias, incluidos los ajustes razonables, para corregir y superar la desventaja o el sometimiento de una persona o grupo. 
\newline {\color{gray} \textbf{1º:} 278-4-Iniciativa-Convencional-del-cc-Pedro-Munoz-sobre-Derecho-a-la-Igualdad-y-no-Discriminacion-1159-hrs.pdf}
\newline {\color{gray} (Emb: 0.953, TF-IDF: 0.944)}
\newline {\color{gray} \textbf{2º:} 108-4-c-Iniciativa-de-la-cc-Giovanna-Grandon-Derecho-a-la-Sindicalizacion.pdf}
\newline {\color{gray} (Emb: 0.625, TF-IDF: 0.279)}

Los órganos del Estado deberán tener especialmente en consideración los casos en que confluyan, respecto de una persona, más de una categoría, condición o criterio de los señalados en el inciso segundo. 
\newline {\color{gray} \textbf{1º:} 213-1-c-Iniciativa-Convencional-del-cc-Jaime-Bassa-sobre-Congreso-plurinacional-2058-hrs.pdf}
\newline {\color{gray} (Emb: 0.853, TF-IDF: 0.650)}
\newline {\color{gray} \textbf{2º:} 12-4-Iniciativa-Convencional-Constituyente-de-la-cc-Tammy-Pustilnick-y-otros.pdf}
\newline {\color{gray} (Emb: 0.762, TF-IDF: 0.521)}

La Constitución asegura a todas las personas la igualdad ante la ley. 
\newline {\color{gray} \textbf{1º:} 278-4-Iniciativa-Convencional-del-cc-Pedro-Munoz-sobre-Derecho-a-la-Igualdad-y-no-Discriminacion-1159-hrs.pdf}
\newline {\color{gray} (Emb: 0.863, TF-IDF: 0.808)}
\newline {\color{gray} \textbf{2º:} 304-4-Iniciativa-Convencional-de-la-cc-Valentina-Miranda-sobre-Derechos-Fundamentales-1301-hrs.pdf}
\newline {\color{gray} (Emb: 0.601, TF-IDF: 0.342)}


\item \textbf{Artículo} \newline
Todos los habitantes del territorio nacional tienen derecho a usar las lenguas. 
\newline {\color{gray} \textbf{1º:} 278-4-Iniciativa-Convencional-del-cc-Pedro-Munoz-sobre-Derecho-a-la-Igualdad-y-no-Discriminacion-1159-hrs.pdf}
\newline {\color{gray} (Emb: 0.997, TF-IDF: 0.945)}
\newline {\color{gray} \textbf{2º:} 588-Iniciativa-Convencional-Constituyente-de-cc-Marcos-Barraza-sobre-Instituto-Nacional-de-DDHH-2351-hrs.-01-02.pdf}
\newline {\color{gray} (Emb: 0.580, TF-IDF: 0.226)}

Ninguna persona o grupo podrá ser discriminado por razones lingüísticas. 
\newline {\color{gray} \textbf{1º:} 441-Iniciativa-Convencional-Constituyente-de-la-cc-Geoconda-Navarrete-sobre-Derecho-a-una-Vejez-Digna-1403-28-01.pdf}
\newline {\color{gray} (Emb: 0.896, TF-IDF: 0.861)}
\newline {\color{gray} \textbf{2º:} 278-4-Iniciativa-Convencional-del-cc-Pedro-Munoz-sobre-Derecho-a-la-Igualdad-y-no-Discriminacion-1159-hrs.pdf}
\newline {\color{gray} (Emb: 0.892, TF-IDF: 0.754)}

Toda persona y pueblo tiene el derecho a comunicarse en su propia lengua en todo espacio. 
\newline {\color{gray} \textbf{1º:} 726-Iniciativa-Convencional-Constituyente-de-la-cc-Valentina-Miranda-sobre-No-discriminacion-01-02.pdf}
\newline {\color{gray} (Emb: 0.668, TF-IDF: 0.361)}
\newline {\color{gray} \textbf{2º:} 304-4-Iniciativa-Convencional-de-la-cc-Valentina-Miranda-sobre-Derechos-Fundamentales-1301-hrs.pdf}
\newline {\color{gray} (Emb: 0.638, TF-IDF: 0.289)}


\item \textbf{Artículo} \newline
Los pueblos y naciones indígenas tienen el derecho a ser consultados previamente a la adopción de medidas administrativas y legislativas que les afectasen. 
\newline {\color{gray} \textbf{1º:} 704-Iniciativa-Convencional-Constituyente-de-la-cc-Elsa-Labrana-sobre-Derecho-a-la-Comunicacion-01-02.pdf}
\newline {\color{gray} (Emb: 0.729, TF-IDF: 0.487)}
\newline {\color{gray} \textbf{2º:} 404-4-Iniciativa-Convencional-Constituyente-de-la-cc-Lidia-Gonzalez-sobre-Derechos-Linguisticos-1940-24-01.pdf}
\newline {\color{gray} (Emb: 0.724, TF-IDF: 0.346)}

El Estado garantiza los medios para la efectiva participación de éstos, a través de sus instituciones representativas, de forma previa y libre, mediante procedimientos apropiados, informados y de buena fe. 
\newline {\color{gray} \textbf{1º:} 404-4-Iniciativa-Convencional-Constituyente-de-la-cc-Lidia-Gonzalez-sobre-Derechos-Linguisticos-1940-24-01.pdf}
\newline {\color{gray} (Emb: 0.808, TF-IDF: 0.660)}
\newline {\color{gray} \textbf{2º:} 443-Iniciativa-Convencional-Constituyente-de-la-cc-Giovanna-Grandon-sobre-Nacionalidad-1405-28-01.pdf}
\newline {\color{gray} (Emb: 0.724, TF-IDF: 0.322)}


\item \textbf{Artículo} \newline
La Constitución garantiza a todas las personas el derecho al agua y al saneamiento suficiente, saludable, aceptable, asequible y accesible. 
\newline {\color{gray} \textbf{1º:} 404-4-Iniciativa-Convencional-Constituyente-de-la-cc-Lidia-Gonzalez-sobre-Derechos-Linguisticos-1940-24-01.pdf}
\newline {\color{gray} (Emb: 1.000, TF-IDF: 1.000)}
\newline {\color{gray} \textbf{2º:} 84-2-Iniciativa-Convencional-Constituyente-del-cc-Martin-Arrau-y-otros.pdf}
\newline {\color{gray} (Emb: 0.692, TF-IDF: 0.263)}

Es deber del Estado garantizar estos derechos para las actuales y futuras generaciones. 
\newline {\color{gray} \textbf{1º:} 654-Iniciativa-Convencional-Constituyente-de-la-cc-Isabella-Mamani-sobre-Derecho-a-la-Consulta-Indigena-121101-02.pdf}
\newline {\color{gray} (Emb: 0.788, TF-IDF: 0.338)}
\newline {\color{gray} \textbf{2º:} 913-Iniciativa-Convencional-Constituyente-del-cc-Luis-Jimenez-que-crea-la-Defensoría-de-los-Pueblos-Indígenas.pdf}
\newline {\color{gray} (Emb: 0.677, TF-IDF: 0.317)}

El Estado velará por la satisfacción de este derecho atendiendo las necesidades de las personas en sus distintos contextos. 
\newline {\color{gray} \textbf{1º:} 123-2-c-Iniciativa-de-la-cc-Carolina-Vilches-Principio-Precautorio-1.pdf}
\newline {\color{gray} (Emb: 0.571, TF-IDF: 0.394)}
\newline {\color{gray} \textbf{2º:} 282-2-Iniciativa-Convencional-de-la-cc-Carolina-Vilches-sobre-Principio-Precautorio-1205-hrs.pdf}
\newline {\color{gray} (Emb: 0.571, TF-IDF: 0.394)}


\item \textbf{Artículo} \newline
Toda persona tiene derecho a la protección de los datos personales. 
\newline {\color{gray} \textbf{1º:} 164-4-c-Iniciativa-del-cc-Jorge-Arancibia-Reconocimiento-y-Tratamiento-del-Agua.pdf}
\newline {\color{gray} (Emb: 0.849, TF-IDF: 0.508)}
\newline {\color{gray} \textbf{2º:} 525-5-Iniciativa-Convencional-Constituyente-de-cc-Isabel-Godoy-sobre-Derecho-Humano-al-Agua-1218-hrs.-01-02.pdf}
\newline {\color{gray} (Emb: 0.786, TF-IDF: 0.472)}

Este derecho comprende la facultad de acceder a los datos recogidos que le conciernen, ser informada y oponerse al tratamiento de sus datos y a obtener su rectificación, cancelación y portabilidad, sin perjuicio de otros que establezca la ley. 
\newline {\color{gray} \textbf{1º:} 8-4-Iniciativa-Convencional-Constituyente-de-la-cc-Cristina-Dorador-y-otros.pdf}
\newline {\color{gray} (Emb: 1.000, TF-IDF: 1.000)}
\newline {\color{gray} \textbf{2º:} 954-5-Iniciativa-Convencional-Constituyente-de-la-cc-Carolina-Vilches-sobre-Estatuto-del-Agua.pdf}
\newline {\color{gray} (Emb: 1.000, TF-IDF: 1.000)}

El tratamiento de datos personales sólo podrá efectuarse en los casos que establezca la ley, sujetándose a los principios de licitud, lealtad, calidad, transparencia, seguridad, limitación de la finalidad y minimización de datos. 
\newline {\color{gray} \textbf{1º:} 8-4-Iniciativa-Convencional-Constituyente-de-la-cc-Cristina-Dorador-y-otros.pdf}
\newline {\color{gray} (Emb: 0.718, TF-IDF: 0.650)}
\newline {\color{gray} \textbf{2º:} 306-5-Iniciativa-Convencional-de-la-cc-Gloria-Alvarado-sobre-Agua-1324-hrs.pdf}
\newline {\color{gray} (Emb: 0.688, TF-IDF: 0.611)}


\item \textbf{Artículo} \newline
Es deber del Estado adoptar acciones de prevención, adaptación, y mitigación de los riesgos, vulnerabilidades y efectos provocados por la crisis climática y ecológica. 
\newline {\color{gray} \textbf{1º:} 508-7-Iniciativa-Convencional-Constituyente-del-cc-Martin-Arrau-sobre-proteccion-de-datos-1140-01-02.pdf}
\newline {\color{gray} (Emb: 0.823, TF-IDF: 0.816)}
\newline {\color{gray} \textbf{2º:} 274-4-Iniciativa-Convencional-de-la-cc-Natalia-Henriquez-sobre-Privacidad-de-Datos-Personales-17-01-1154-hrs.pdf}
\newline {\color{gray} (Emb: 0.796, TF-IDF: 0.816)}

El Estado promoverá el diálogo, cooperación y solidaridad internacional para adaptarse, mitigar y afrontar la crisis climática y ecológica y proteger la Naturaleza. 
\newline {\color{gray} \textbf{1º:} 274-4-Iniciativa-Convencional-de-la-cc-Natalia-Henriquez-sobre-Privacidad-de-Datos-Personales-17-01-1154-hrs.pdf}
\newline {\color{gray} (Emb: 0.933, TF-IDF: 0.803)}
\newline {\color{gray} \textbf{2º:} 161-4-c-Iniciativa-de-la-cc-Rocio-Cantuarias-Proteccion-de-Datos-Personales.pdf}
\newline {\color{gray} (Emb: 0.667, TF-IDF: 0.407)}


\item \textbf{Artículo} \newline
La Naturaleza tiene derecho a que se respete y proteja su existencia, a la regeneración, a la mantención y a la restauración de sus funciones y equilibrios dinámicos, que comprenden los ciclos naturales, los ecosistemas y la biodiversidad. 
\newline {\color{gray} \textbf{1º:} 416-7-Iniciativa-Convencional-del-cc-Francisco-Caamano-sobre-Proteccion-de-datos-de-caracter-personal.pdf}
\newline {\color{gray} (Emb: 0.778, TF-IDF: 0.603)}
\newline {\color{gray} \textbf{2º:} 524-4-Iniciativa-Convencional-Constituyente-de-cc-Felipe-Harbor-sobre-DERECHO-A-LA-PRIVACIDAD-1154-hrs.-01-02.pdf}
\newline {\color{gray} (Emb: 0.705, TF-IDF: 0.468)}

El Estado a través de sus instituciones debe garantizar y promover los derechos de la Naturaleza según lo determine la Constitución y las Leyes. 
\newline {\color{gray} \textbf{1º:} 672-Iniciativa-Convencional-Constituyente-del-cc-Carolina-Sepulveda-sobre-Principios-Ambientales-121101-02.pdf}
\newline {\color{gray} (Emb: 0.658, TF-IDF: 0.376)}
\newline {\color{gray} \textbf{2º:} 965-Iniciativa-Convencional-Constituyente-de-la-cc.pdf}
\newline {\color{gray} (Emb: 0.649, TF-IDF: 0.369)}


\item \textbf{Artículo} \newline
La Ley podrá establecer restricciones al ejercicio de determinados derechos o libertades para proteger el medio ambiente y la Naturaleza. 
\newline {\color{gray} \textbf{1º:} 857-Iniciativa-Convencional-Constituyente-del-cc-Jorge-Abarca-sobre-Derecho-a-un-Ambiente-Sano.pdf}
\newline {\color{gray} (Emb: 0.629, TF-IDF: 0.482)}
\newline {\color{gray} \textbf{2º:} 858-Iniciativa-Convencional-Constituyente-del-cc-Jorge-Abarca-sobre-Estado-Ecologico-de-Derecho.pdf}
\newline {\color{gray} (Emb: 0.623, TF-IDF: 0.482)}


\item \textbf{Artículo} \newline
Son bienes comunes naturales el mar territorial y su fondo marino; las playas; las aguas, glaciares y humedales; los campos geotérmicos; el aire y la atmósfera; la alta montaña, las áreas protegidas y los bosques nativos; el subsuelo, y los demás que declaren la Constitución y la ley. 
\newline {\color{gray} \textbf{1º:} 903-Iniciativa-Convencional-Constituyente-del-cc-Fernando-Salinas-Sobre-Buen-Vivir.pdf}
\newline {\color{gray} (Emb: 0.718, TF-IDF: 0.452)}
\newline {\color{gray} \textbf{2º:} 434-5-Iniciativa-Convencional-de-la-cc-Carolina-Vilches-sobre-Derechos-de-la-Naturaleza-1155-27-01.pdf}
\newline {\color{gray} (Emb: 0.704, TF-IDF: 0.355)}

Entre estos bienes son inapropiables el agua en todos sus estados y el aire, los reconocidos por el derecho internacional y los que la Constitución o las leyes declaren como tales. 
\newline {\color{gray} \textbf{1º:} 772-Iniciativa-Convencional-Constituyente-de-la-cc-Angelica-Tepper-sobre-Fuentes-del-Derecho-Internacional.pdf}
\newline {\color{gray} (Emb: 0.699, TF-IDF: 0.334)}
\newline {\color{gray} \textbf{2º:} 811-Iniciativa-Convencional-Constituyente-del-cc-Juan-Jose-Martin-sobre-Estatuto-de-los-Ecosistemas.pdf}
\newline {\color{gray} (Emb: 0.689, TF-IDF: 0.333)}


\item \textbf{Artículo} \newline
Tratándose de los bienes comunes naturales que sean inapropiables, el Estado deberá preservarlos, conservarlos y, en su caso, restaurarlos. 
\newline {\color{gray} \textbf{1º:} 84-2-Iniciativa-Convencional-Constituyente-del-cc-Martin-Arrau-y-otros.pdf}
\newline {\color{gray} (Emb: 0.944, TF-IDF: 0.896)}
\newline {\color{gray} \textbf{2º:} 784-niciativa-Convencional-Constituyente-de-la-cc-Damaris-Abarca-sobre-Derecho-a-vivir-en-un-ambiente-sano.pdf}
\newline {\color{gray} (Emb: 0.858, TF-IDF: 0.771)}

Deberá, asimismo, administrarlos de forma democrática, solidaria, participativa y equitativa. 
\newline {\color{gray} \textbf{1º:} 964-Iniciativa-Convencional-Constituyente-del-cc-Fernando-Salinas-sobre-Bienes-Comunes.pdf}
\newline {\color{gray} (Emb: 0.705, TF-IDF: 0.520)}
\newline {\color{gray} \textbf{2º:} 71-2-Iniciativa-Convencional-Constitutente-de-Maria-Jose-Oyarzun-y-otros.pdf}
\newline {\color{gray} (Emb: 0.647, TF-IDF: 0.488)}

Respecto de aquellos bienes comunes naturales que se encuentren en el dominio privado, el deber de custodia del Estado implica la facultad de regular su uso y goce, con las finalidades establecidas en el artículo primero. 
\newline {\color{gray} \textbf{1º:} 164-4-c-Iniciativa-del-cc-Jorge-Arancibia-Reconocimiento-y-Tratamiento-del-Agua.pdf}
\newline {\color{gray} (Emb: 0.590, TF-IDF: 0.321)}
\newline {\color{gray} \textbf{2º:} 710-Iniciativa-Convencional-Constituyente-de-la-cc-Gloria-Alvarado-sobre-Maritorio-01-02.pdf}
\newline {\color{gray} (Emb: 0.580, TF-IDF: 0.297)}


\item \textbf{Artículo} \newline
Cualquier persona podrá exigir el cumplimiento de los deberes constitucionales de custodia de los bienes comunes naturales. 
\newline {\color{gray} \textbf{1º:} 826-Iniciativa-Convencional-Constituyente-del-cc-Victorino-Antilef-sobre-Proteccion-del-Medio-Ambiente.pdf}
\newline {\color{gray} (Emb: 0.675, TF-IDF: 0.398)}
\newline {\color{gray} \textbf{2º:} 946-Iniciativa-Convencional-Constituyente-del-cc-Victorino-Antilef-sobre-Proteccion-del-Bosque-Nativo.pdf}
\newline {\color{gray} (Emb: 0.675, TF-IDF: 0.385)}

La ley determinará el procedimiento y los requisitos de esta acción. 
\newline {\color{gray} \textbf{1º:} 864-Iniciativa-Convencional-Constituyente-del-cc-Marcos-Barraza-sobre-RREE.pdf}
\newline {\color{gray} (Emb: 0.623, TF-IDF: 0.361)}
\newline {\color{gray} \textbf{2º:} 377-2-Iniciativa-Convencional-Constituyente-del-cc-Alvin-Saldana-sobre-Mecanismos-de-Democracia-1045-hrs-24-01.pdf}
\newline {\color{gray} (Emb: 0.623, TF-IDF: 0.321)}


\item \textbf{Artículo} \newline
El Estado podrá otorgar autorizaciones administrativas para el uso de los bienes comunes naturales inapropiables, conforme a la ley, de manera temporal, sujeto a causales de caducidad, extinción y revocación, con obligaciones específicas de conservación, justificadas en el interés público, la protección de la naturaleza y el beneficio colectivo. 
\newline {\color{gray} \textbf{1º:} 964-Iniciativa-Convencional-Constituyente-del-cc-Fernando-Salinas-sobre-Bienes-Comunes.pdf}
\newline {\color{gray} (Emb: 0.568, TF-IDF: 0.255)}
\newline {\color{gray} \textbf{2º:} 594-Iniciativa-Convencional-Constituyente-de-cc-Tammy-Pustilnick-sobre-Custodia-Publica-de-la-Naturaleza.pdf}
\newline {\color{gray} (Emb: 0.559, TF-IDF: 0.248)}

Estas autorizaciones, ya sean individuales o colectivas, no generan derechos de propiedad. 
\newline {\color{gray} \textbf{1º:} 964-Iniciativa-Convencional-Constituyente-del-cc-Fernando-Salinas-sobre-Bienes-Comunes.pdf}
\newline {\color{gray} (Emb: 0.650, TF-IDF: 0.657)}
\newline {\color{gray} \textbf{2º:} 594-Iniciativa-Convencional-Constituyente-de-cc-Tammy-Pustilnick-sobre-Custodia-Publica-de-la-Naturaleza.pdf}
\newline {\color{gray} (Emb: 0.649, TF-IDF: 0.548)}


\item \textbf{Artículo} \newline
Se reconoce a todas las personas el derecho de acceso responsable y universal a las montañas, riberas de ríos, mar, playas, lagos, lagunas y humedales, entre otros que defina la ley. 
\newline {\color{gray} \textbf{1º:} 594-Iniciativa-Convencional-Constituyente-de-cc-Tammy-Pustilnick-sobre-Custodia-Publica-de-la-Naturaleza.pdf}
\newline {\color{gray} (Emb: 0.973, TF-IDF: 0.633)}
\newline {\color{gray} \textbf{2º:} 964-Iniciativa-Convencional-Constituyente-del-cc-Fernando-Salinas-sobre-Bienes-Comunes.pdf}
\newline {\color{gray} (Emb: 0.973, TF-IDF: 0.633)}

(Inciso tercero) La ley regulará el ejercicio de este derecho, las obligaciones de los propietarios aledaños y el régimen de responsabilidad aplicable, entre otros. 
\newline {\color{gray} \textbf{1º:} 594-Iniciativa-Convencional-Constituyente-de-cc-Tammy-Pustilnick-sobre-Custodia-Publica-de-la-Naturaleza.pdf}
\newline {\color{gray} (Emb: 0.859, TF-IDF: 0.595)}
\newline {\color{gray} \textbf{2º:} 964-Iniciativa-Convencional-Constituyente-del-cc-Fernando-Salinas-sobre-Bienes-Comunes.pdf}
\newline {\color{gray} (Emb: 0.731, TF-IDF: 0.473)}


\item \textbf{Artículo} \newline
Es deber del Estado normar y fomentar la gestión, reducción y valorización de residuos, en la forma que determine la Ley. 
\newline {\color{gray} \textbf{1º:} 594-Iniciativa-Convencional-Constituyente-de-cc-Tammy-Pustilnick-sobre-Custodia-Publica-de-la-Naturaleza.pdf}
\newline {\color{gray} (Emb: 0.766, TF-IDF: 0.408)}
\newline {\color{gray} \textbf{2º:} 406-4-Iniciativa-Convencional-Constituyente-del-cc-Juan-Jose-Martin-sobre-Titularidad-de-los-DDFF-1949-24-01.pdf}
\newline {\color{gray} (Emb: 0.749, TF-IDF: 0.299)}


\item \textbf{Artículo} \newline
El Estado y sus organismos promoverán una educación basada en la empatía y en el respeto hacia los animales. 
\newline {\color{gray} \textbf{1º:} 239-1-Iniciativa-Convencional-de-la-cc-Tania-Madriaga-sobre-Poder-Ejecutivo-1146-hrs.pdf}
\newline {\color{gray} (Emb: 0.649, TF-IDF: 0.482)}
\newline {\color{gray} \textbf{2º:} 201-6-c-Iniciativa-Convencional-del-cc-Felipe-Harboe-sobre-la-Contraloria-1658-hrs.pdf}
\newline {\color{gray} (Emb: 0.581, TF-IDF: 0.377)}

Los animales son sujetos de especial protección. 
\newline {\color{gray} \textbf{1º:} 996-Iniciativa-Convencional-Constituyente-del-cc-Francisco-Caamano-sobre-Acceso-a-la-Naturaleza.pdf}
\newline {\color{gray} (Emb: 0.673, TF-IDF: 0.540)}
\newline {\color{gray} \textbf{2º:} 418-5-Iniciativa-Convencional-del-cc-Francisco-Caamano-sobre-Derecho-de-Acceso-a-la-Naturaleza.pdf}
\newline {\color{gray} (Emb: 0.673, TF-IDF: 0.540)}

El Estado los protegerá, reconociendo su sintiencia y el derecho a vivir una vida libre de maltrato. 
\newline {\color{gray} \textbf{1º:} 996-Iniciativa-Convencional-Constituyente-del-cc-Francisco-Caamano-sobre-Acceso-a-la-Naturaleza.pdf}
\newline {\color{gray} (Emb: 0.703, TF-IDF: 0.449)}
\newline {\color{gray} \textbf{2º:} 344-3-Iniciativa-Convencional-Constituyente-del-cc-Hernan-Larrain-sobre-Reforma-Administrativa-y-Modernizacion-del-Estado.pdf}
\newline {\color{gray} (Emb: 0.604, TF-IDF: 0.357)}


\item \textbf{Artículo} \newline
El Estado protege la biodiversidad, debiendo preservar, conservar y restaurar el hábitat de las especies nativas silvestres, en tal cantidad y distribución que sostenga adecuadamente la viabilidad de sus poblaciones y asegure las condiciones para su supervivencia y no extinción. 
\newline {\color{gray} \textbf{1º:} 314-5-Iniciativa-Convencional-del-cc-Miguel-Angel-Botto-sobre-Proteccion-a-los-Animales.pdf}
\newline {\color{gray} (Emb: 0.661, TF-IDF: 0.688)}
\newline {\color{gray} \textbf{2º:} 1003-Iniciativa-Convencional-Constituyente-del-cc-Luis-Mayol-sobre-Derecho-de-las-Personas-Privadas-de-Libertad.pdf}
\newline {\color{gray} (Emb: 0.618, TF-IDF: 0.350)}


\item \textbf{Artículo} \newline
Son principios para la protección de la Naturaleza y el medio ambiente, a lo menos, los principios de progresividad, precautorio, preventivo, justicia ambiental, solidaridad intergeneracional, responsabilidad y acción climática justa. 
\newline {\color{gray} \textbf{1º:} 237-1-Iniciativa-Convencional-de-la-cc-Tania-Madriaga-sobre-Estado-Plurinacional-y-Libre-Determinacion-1146-hrs.pdf}
\newline {\color{gray} (Emb: 0.660, TF-IDF: 0.383)}
\newline {\color{gray} \textbf{2º:} 551-Iniciativa-Convencional-Constituyente-de-la-cc-Janis-Meneses-sobre-derecho-al-ocio-1707-01-02.pdf}
\newline {\color{gray} (Emb: 0.654, TF-IDF: 0.383)}


\item \textbf{Artículo} \newline
Se reconoce el derecho de participación informada en materias ambientales. 
\newline {\color{gray} \textbf{1º:} 910-Iniciativa-Convencional-Constituyente-del-cc-Hugo-Gutierrez-sobre-Proteccion-Animal.pdf}
\newline {\color{gray} (Emb: 0.652, TF-IDF: 0.269)}
\newline {\color{gray} \textbf{2º:} 540-Iniciativa-Convencional-Constituyente-de-cc-Jorge-Abarca-sobre-proteccion-animal-1524-01-02.pdf}
\newline {\color{gray} (Emb: 0.628, TF-IDF: 0.265)}

Los mecanismos de participación serán determinados por ley. 
\newline {\color{gray} \textbf{1º:} 1018-Iniciativa-Convencional-Constituyente-de-la-cc-Ramona-Reyes-sobre-Ruralidad-RESPALDO.pdf}
\newline {\color{gray} (Emb: 0.694, TF-IDF: 0.307)}
\newline {\color{gray} \textbf{2º:} 115-5-c-Iniciativa-del-cc-Francisco-Caamano-Proteccion-del-Bosque-Nativo.pdf}
\newline {\color{gray} (Emb: 0.691, TF-IDF: 0.228)}

Todas las personas tienen derecho a acceder a la información ambiental que conste en poder o custodia del Estado. 
\newline {\color{gray} \textbf{1º:} 787-Iniciativa-Convencional-Constituyente-de-la-cc-Camila-Zarate-sobre-Deberes-y-Principios-Ambientales.pdf}
\newline {\color{gray} (Emb: 0.723, TF-IDF: 0.526)}
\newline {\color{gray} \textbf{2º:} 457-6-Iniciativa-Convencional-Constituyente-de-la-cc-Manuela-Royo-sobre-Justicia-Ambiental.pdf}
\newline {\color{gray} (Emb: 0.662, TF-IDF: 0.357)}

Los particulares deberán entregar la información ambiental relacionada con su actividad, en los términos que establezca la ley. 
\newline {\color{gray} \textbf{1º:} 788-Iniciativa-Convencional-Constituyente-de-la-cc-Camila-Zarate-sobre-Democracia-Ecologica.pdf}
\newline {\color{gray} (Emb: 0.825, TF-IDF: 0.441)}
\newline {\color{gray} \textbf{2º:} 859-Iniciativa-Convencional-Constituyente-del-cc-Jorge-Abarca-sobre-Participacion-y-Acceso-a-la-Informacion-Publica.pdf}
\newline {\color{gray} (Emb: 0.754, TF-IDF: 0.324)}


\item \textbf{Artículo} \newline
La ley determinará los demás usos. 
\newline {\color{gray} \textbf{1º:} 954-5-Iniciativa-Convencional-Constituyente-de-la-cc-Carolina-Vilches-sobre-Estatuto-del-Agua.pdf}
\newline {\color{gray} (Emb: 0.669, TF-IDF: 0.365)}
\newline {\color{gray} \textbf{2º:} 390-5-Iniciativa-Convencional-Constituyente-de-la-cc-Isabel-Godoy-sobre-Estatuto-Constitucional-del-Agua-1431-24-01.pdf}
\newline {\color{gray} (Emb: 0.643, TF-IDF: 0.284)}

El Estado debe proteger las aguas, en todos sus estados y fases, y su ciclo hidrológico. 
\newline {\color{gray} \textbf{1º:} 995-Iniciativa-Convencional-Constituyente-del-cc-Christian-Viera-sobre-Principio-de-Intervencion-del-Estado.pdf}
\newline {\color{gray} (Emb: 0.776, TF-IDF: 0.655)}
\newline {\color{gray} \textbf{2º:} 425-6-Iniciativa-Convencional-del-cc-Christian-Viera-sobre-Reforma-de-la-Constitucion-1554-26-01.pdf}
\newline {\color{gray} (Emb: 0.747, TF-IDF: 0.501)}

El agua es esencial para la vida y el ejercicio de los derechos humanos y de la Naturaleza. 
\newline {\color{gray} \textbf{1º:} 784-niciativa-Convencional-Constituyente-de-la-cc-Damaris-Abarca-sobre-Derecho-a-vivir-en-un-ambiente-sano.pdf}
\newline {\color{gray} (Emb: 0.851, TF-IDF: 0.638)}
\newline {\color{gray} \textbf{2º:} 788-Iniciativa-Convencional-Constituyente-de-la-cc-Camila-Zarate-sobre-Democracia-Ecologica.pdf}
\newline {\color{gray} (Emb: 0.774, TF-IDF: 0.610)}

Siempre prevalecerá el ejercicio del derecho humano al agua, el saneamiento y el equilibrio de los ecosistemas. 
\newline {\color{gray} \textbf{1º:} 766-Iniciativa-Convencional-Constituyente-del-cc-Fernando-Salinas-sobre-Derechos-de-los-Consumidores.pdf}
\newline {\color{gray} (Emb: 0.688, TF-IDF: 0.264)}
\newline {\color{gray} \textbf{2º:} 782-Iniciativa-Convencional-Constituyente-del-cc-Fernando-Salinas-sobre-Derechos-de-usuarios-y-consumidores.pdf}
\newline {\color{gray} (Emb: 0.688, TF-IDF: 0.258)}


\item \textbf{Artículo} \newline
Las autorizaciones de uso de agua serán otorgadas por la Agencia Nacional de Aguas, de carácter incomerciable, concedidas basándose en la disponibilidad efectiva de las aguas, y obligarán al titular al uso que justifica su otorgamiento. 
\newline {\color{gray} \textbf{1º:} 954-5-Iniciativa-Convencional-Constituyente-de-la-cc-Carolina-Vilches-sobre-Estatuto-del-Agua.pdf}
\newline {\color{gray} (Emb: 0.847, TF-IDF: 0.670)}
\newline {\color{gray} \textbf{2º:} 71-2-Iniciativa-Convencional-Constitutente-de-Maria-Jose-Oyarzun-y-otros.pdf}
\newline {\color{gray} (Emb: 0.695, TF-IDF: 0.605)}

El Estado velará por un uso razonable de las aguas. 
\newline {\color{gray} \textbf{1º:} 8-4-Iniciativa-Convencional-Constituyente-de-la-cc-Cristina-Dorador-y-otros.pdf}
\newline {\color{gray} (Emb: 0.808, TF-IDF: 0.503)}
\newline {\color{gray} \textbf{2º:} 525-5-Iniciativa-Convencional-Constituyente-de-cc-Isabel-Godoy-sobre-Derecho-Humano-al-Agua-1218-hrs.-01-02.pdf}
\newline {\color{gray} (Emb: 0.760, TF-IDF: 0.438)}


\item \textbf{Artículo} \newline
El Estado asegurará un sistema de gobernanza de las aguas participativo y descentralizado, a través del manejo integrado de cuencas, y siendo la cuenca hidrográfica la unidad mínima de gestión. 
\newline {\color{gray} \textbf{1º:} 813-Iniciativa-Convencional-Constituyente-del-cc-Manuel-Woldarsky-crea-la-Agencia-de-DDHH.pdf}
\newline {\color{gray} (Emb: 0.841, TF-IDF: 0.486)}
\newline {\color{gray} \textbf{2º:} 914-Iniciativa-Convencional-Constituyente-del-cc-Luis-Jimenez-crea-la-Defensoria-de-la-Naturaleza.pdf}
\newline {\color{gray} (Emb: 0.833, TF-IDF: 0.440)}

Los Consejos de Cuenca serán los responsables de la administración de las aguas, sin perjuicio de la supervigilancia y demás atribuciones de la Agencia Nacional de las Aguas y otras instituciones competentes. 
\newline {\color{gray} \textbf{1º:} 954-5-Iniciativa-Convencional-Constituyente-de-la-cc-Carolina-Vilches-sobre-Estatuto-del-Agua.pdf}
\newline {\color{gray} (Emb: 0.828, TF-IDF: 0.418)}
\newline {\color{gray} \textbf{2º:} 512-4-Iniciativa-Convencional-Constituyente-del-cc-Bernardo-Fontaine-agua-potable-1105-01-02.pdf}
\newline {\color{gray} (Emb: 0.741, TF-IDF: 0.380)}

La ley regulará las atribuciones, funcionamiento y composición de los Consejos. 
\newline {\color{gray} \textbf{1º:} 954-5-Iniciativa-Convencional-Constituyente-de-la-cc-Carolina-Vilches-sobre-Estatuto-del-Agua.pdf}
\newline {\color{gray} (Emb: 0.685, TF-IDF: 0.374)}
\newline {\color{gray} \textbf{2º:} 164-4-c-Iniciativa-del-cc-Jorge-Arancibia-Reconocimiento-y-Tratamiento-del-Agua.pdf}
\newline {\color{gray} (Emb: 0.643, TF-IDF: 0.372)}

Esta deberá considerar, a lo menos, la presencia de los titulares de autorizaciones de aguas, la sociedad civil y las entidades territoriales con presencia en la respectiva cuenca, velando que ningún actor pueda alcanzar el control por sí solo. 
\newline {\color{gray} \textbf{1º:} 692-Iniciativa-Convencional-Constituyente-del-cc-Pedro-Munoz-sobre-Agua-y-Buen-Vivir-120001-02.pdf}
\newline {\color{gray} (Emb: 0.703, TF-IDF: 0.462)}
\newline {\color{gray} \textbf{2º:} 554-Iniciativa-Convencional-Constituyente-del-cc-Pedro-Munoz-sobre-gestion-integrada-de-cuencas-1752-01-02.pdf}
\newline {\color{gray} (Emb: 0.703, TF-IDF: 0.388)}

Los Consejos podrán coordinarse y asociarse cuando sea pertinente. 
\newline {\color{gray} \textbf{1º:} 954-5-Iniciativa-Convencional-Constituyente-de-la-cc-Carolina-Vilches-sobre-Estatuto-del-Agua.pdf}
\newline {\color{gray} (Emb: 0.656, TF-IDF: 0.395)}
\newline {\color{gray} \textbf{2º:} 625-Iniciativa-Convencional-Constituyente-de-cc-Roberto-Vega-Organo-autonomo-sobre-aguas-de-la-Agencia-Nacional-de-Aguas.pdf}
\newline {\color{gray} (Emb: 0.639, TF-IDF: 0.392)}

En aquellos casos en que no se constituya un Consejo, la administración será determinada por la Agencia Nacional de Agua. 
\newline {\color{gray} \textbf{1º:} 924-Iniciativa-Convencional-Constituyente-de-la-cc-Constanza-Schonhaut-sobre-Administracion-del-Estado.pdf}
\newline {\color{gray} (Emb: 0.857, TF-IDF: 0.674)}
\newline {\color{gray} \textbf{2º:} 924-Iniciativa-Convencional-Constituyente-de-la-cc-Constanza-Schonhaut-sobre-Administracion-del-Estado.pdf}
\newline {\color{gray} (Emb: 0.810, TF-IDF: 0.536)}


\item \textbf{Artículo} \newline
La Constitución reconoce a los pueblos y naciones indígenas el uso tradicional de las aguas situadas en autonomías territoriales indígenas o territorios indígenas. 
\newline {\color{gray} \textbf{1º:} 390-5-Iniciativa-Convencional-Constituyente-de-la-cc-Isabel-Godoy-sobre-Estatuto-Constitucional-del-Agua-1431-24-01.pdf}
\newline {\color{gray} (Emb: 0.588, TF-IDF: 0.242)}
\newline {\color{gray} \textbf{2º:} 745-Iniciativa-Convencional-Constituyente-del-cc-Maria-Trinidad-Castillo-sobre-Aguas-01-02.pdf}
\newline {\color{gray} (Emb: 0.570, TF-IDF: 0.222)}

Es deber del Estado garantizar su protección, integridad y abastecimiento, en conformidad a la Constitución y la ley. 
\newline {\color{gray} \textbf{1º:} 370-4-Iniciativa-Convencional-Constituyente-de-la-cc-Constanza-San-Juan-sobre-Justicia-Transicional-0900-hrs-24-01.pdf}
\newline {\color{gray} (Emb: 0.572, TF-IDF: 0.288)}
\newline {\color{gray} \textbf{2º:} 560-Iniciativa-Convencional-Constituyente-de-cc-Andres-Cruz-sobre-Ministerio-Publico-2016-hrs.-01-02.pdf}
\newline {\color{gray} (Emb: 0.544, TF-IDF: 0.269)}


\item \textbf{Artículo} \newline
El Estado deberá promover y proteger la gestión comunitaria de agua potable y saneamiento, especialmente en áreas y territorios rurales y extremos, en conformidad a la ley. 
\newline {\color{gray} \textbf{1º:} 557-Iniciativa-Convencional-Constituyente-de-cc-Andres-Cruz-sobre-Agencia-Nacional-del-agua-2013-hrs.-01-02.pdf}
\newline {\color{gray} (Emb: 0.606, TF-IDF: 0.420)}
\newline {\color{gray} \textbf{2º:} 954-5-Iniciativa-Convencional-Constituyente-de-la-cc-Carolina-Vilches-sobre-Estatuto-del-Agua.pdf}
\newline {\color{gray} (Emb: 0.586, TF-IDF: 0.269)}


\item \textbf{Artículo} \newline
El mar territorial y las playas son bienes comunes naturales inapropiables. 
\newline {\color{gray} \textbf{1º:} 954-5-Iniciativa-Convencional-Constituyente-de-la-cc-Carolina-Vilches-sobre-Estatuto-del-Agua.pdf}
\newline {\color{gray} (Emb: 0.723, TF-IDF: 0.412)}
\newline {\color{gray} \textbf{2º:} 954-5-Iniciativa-Convencional-Constituyente-de-la-cc-Carolina-Vilches-sobre-Estatuto-del-Agua.pdf}
\newline {\color{gray} (Emb: 0.685, TF-IDF: 0.406)}


\item \textbf{Artículo} \newline
El Estado garantiza la protección de los glaciares y del entorno glaciar, incluyendo los suelos congelados y sus funciones ecosistémicas. 
\newline {\color{gray} \textbf{1º:} 12-4-Iniciativa-Convencional-Constituyente-de-la-cc-Tammy-Pustilnick-y-otros.pdf}
\newline {\color{gray} (Emb: 0.749, TF-IDF: 0.394)}
\newline {\color{gray} \textbf{2º:} 39-2-Iniciativa-Convencional-Constituyente-de-la-cc-Lisette-Vergara-y-otros.pdf}
\newline {\color{gray} (Emb: 0.692, TF-IDF: 0.327)}


\item \textbf{Artículo} \newline
El Estado deberá conservar, proteger y cuidar la Antártica, mediante una política fundada en el conocimiento y orientada a la investigación científica, la colaboración internacional y la paz. 
\newline {\color{gray} \textbf{1º:} 315-5-Iniciativa-Convencional-de-la-cc-Isabel-Godoy-sobre-Derechos-de-la-Naturaleza-10-35-hrs.pdf}
\newline {\color{gray} (Emb: 0.674, TF-IDF: 0.443)}
\newline {\color{gray} \textbf{2º:} 594-Iniciativa-Convencional-Constituyente-de-cc-Tammy-Pustilnick-sobre-Custodia-Publica-de-la-Naturaleza.pdf}
\newline {\color{gray} (Emb: 0.623, TF-IDF: 0.429)}

El territorio chileno antártico, incluyendo sus espacios marítimos, es un territorio especial y zona fronteriza en el cual Chile ejerce respectivamente soberanía y derechos soberanos, con pleno respeto a los tratados ratificados y vigentes. 
\newline {\color{gray} \textbf{1º:} 954-5-Iniciativa-Convencional-Constituyente-de-la-cc-Carolina-Vilches-sobre-Estatuto-del-Agua.pdf}
\newline {\color{gray} (Emb: 0.892, TF-IDF: 0.687)}
\newline {\color{gray} \textbf{2º:} 306-5-Iniciativa-Convencional-de-la-cc-Gloria-Alvarado-sobre-Agua-1324-hrs.pdf}
\newline {\color{gray} (Emb: 0.846, TF-IDF: 0.548)}


\item \textbf{Artículo} \newline
Los bienes comunes naturales son elementos o componentes de la Naturaleza sobre los cuales el Estado tiene un deber especial de custodia con el fin de asegurar los derechos de la Naturaleza y el interés de las generaciones presentes y futuras. 
\newline {\color{gray} \textbf{1º:} 270-5-Iniciativa-Convencional-de-la-cc-Ivanna-Olivares-sobre-Bienes-Naturales-Estrategicos-17-01-1154-hrs.pdf}
\newline {\color{gray} (Emb: 0.662, TF-IDF: 0.385)}
\newline {\color{gray} \textbf{2º:} 875-Iniciativa-Convencional-Constituyente-de-la-cc-Constanza-San-Juan-Sobre-Criosfera.pdf}
\newline {\color{gray} (Emb: 0.607, TF-IDF: 0.353)}


\item \textbf{Artículo} \newline
El Estado, como custodio de los humedales, bosques nativos y suelos, asegurará la integridad de estos ecosistemas, sus funciones, procesos y conectividad hídrica. 
\newline {\color{gray} \textbf{1º:} 819-Iniciativa-Convencional-Constituyente-del-cc-Mauricio-Daza-sobre-Territorio-Antartico.pdf}
\newline {\color{gray} (Emb: 0.788, TF-IDF: 0.494)}
\newline {\color{gray} \textbf{2º:} 609-Iniciativa-Convencional-Constituyente-de-cc-Rodrigo-Alvarez-sobre-Terreno-Chileno-Antartico-2158-hrs.-01-02.pdf}
\newline {\color{gray} (Emb: 0.708, TF-IDF: 0.493)}


\item \textbf{Artículo} \newline
El Estado, a través de un sistema nacional de áreas protegidas, único, integral y de carácter técnico deberá garantizar la preservación, restauración y la conservación de espacios naturales. 
\newline {\color{gray} \textbf{1º:} 812-Iniciativa-Convencional-Constituyente-del-cc-Juan-Jose-Martin-sobre-Estatuto-del-Espacio.pdf}
\newline {\color{gray} (Emb: 0.660, TF-IDF: 0.336)}
\newline {\color{gray} \textbf{2º:} 810-Iniciativa-Convencional-Constituyente-del-cc-Juan-Jose-Martin-sobre-Estatuto-Antartico.pdf}
\newline {\color{gray} (Emb: 0.605, TF-IDF: 0.307)}

Asimismo, deberá monitorear y mantener información actualizada relativa a los atributos de dichas áreas, y garantizar la participación de las comunidades locales y entidades territoriales. 
\newline {\color{gray} \textbf{1º:} 964-Iniciativa-Convencional-Constituyente-del-cc-Fernando-Salinas-sobre-Bienes-Comunes.pdf}
\newline {\color{gray} (Emb: 0.671, TF-IDF: 0.532)}
\newline {\color{gray} \textbf{2º:} 315-5-Iniciativa-Convencional-de-la-cc-Isabel-Godoy-sobre-Derechos-de-la-Naturaleza-10-35-hrs.pdf}
\newline {\color{gray} (Emb: 0.637, TF-IDF: 0.360)}


\item \textbf{Artículo} \newline
Los planes de ordenamiento y planificación ecológica del territorio priorizarán la protección de las partes altas de las cuencas, glaciares, zonas de recarga natural de acuíferos y ecosistemas. 
\newline {\color{gray} \textbf{1º:} 961-3-Iniciativa-Convencional-Constituyente-de-la-cc-Tania-Madriaga-sobre-Soberania-Territorial.pdf}
\newline {\color{gray} (Emb: 0.624, TF-IDF: 0.484)}
\newline {\color{gray} \textbf{2º:} 597-Iniciativa-Convencional-Constituyente-de-cc-Jorge-Abarca-sobre-Humedales-2356hrs.-01-02.pdf}
\newline {\color{gray} (Emb: 0.610, TF-IDF: 0.320)}

Estos podrán crear zonas de amortiguamiento para las áreas de protección ambiental. 
\newline {\color{gray} \textbf{1º:} 680-Iniciativa-Convencional-Constituyente-del-cc-Francisco-Caamano-sobre-Sistema-Nacional-de-Areas-Protegidas-121101-02.pdf}
\newline {\color{gray} (Emb: 0.679, TF-IDF: 0.497)}
\newline {\color{gray} \textbf{2º:} 60-2-Iniciativa-Convencional-Constituyente-del-cc-Jorge-Baradit-y-otros.pdf}
\newline {\color{gray} (Emb: 0.661, TF-IDF: 0.400)}


\item \textbf{Artículo} \newline
Es deber del Estado asegurar la soberanía y seguridad alimentaria. 
\newline {\color{gray} \textbf{1º:} 525-5-Iniciativa-Convencional-Constituyente-de-cc-Isabel-Godoy-sobre-Derecho-Humano-al-Agua-1218-hrs.-01-02.pdf}
\newline {\color{gray} (Emb: 0.556, TF-IDF: 0.221)}
\newline {\color{gray} \textbf{2º:} 968-Iniciativa-Convencional-Constituyente-de-la-cc-Valentina-Miranda-sobre-Derecho-a-la-salud.pdf}
\newline {\color{gray} (Emb: 0.527, TF-IDF: 0.214)}

Para esto promoverá la producción, distribución y consumo de alimentos que garanticen el derecho a la alimentación sana y adecuada, el comercio justo y sistemas alimentarios ecológicamente responsables. 
\newline {\color{gray} \textbf{1º:} 270-5-Iniciativa-Convencional-de-la-cc-Ivanna-Olivares-sobre-Bienes-Naturales-Estrategicos-17-01-1154-hrs.pdf}
\newline {\color{gray} (Emb: 0.605, TF-IDF: 0.304)}
\newline {\color{gray} \textbf{2º:} 377-2-Iniciativa-Convencional-Constituyente-del-cc-Alvin-Saldana-sobre-Mecanismos-de-Democracia-1045-hrs-24-01.pdf}
\newline {\color{gray} (Emb: 0.603, TF-IDF: 0.303)}


\item \textbf{Artículo} \newline
El Estado garantiza el derecho de campesinas, campesinos y pueblos originarios al libre uso e intercambio de semillas tradicionales. 
\newline {\color{gray} \textbf{1º:} 971-Iniciativa-Convencional-Constituyente-de-la-cc-Ivanna-Olivares-sobre-Relaves.pdf}
\newline {\color{gray} (Emb: 0.576, TF-IDF: 0.360)}
\newline {\color{gray} \textbf{2º:} 614-Iniciativa-Convencional-Constituyente-de-cc-Cesar-Uribe-sobre-Dano-Ambiental-y-Zonas-de-Sacrificio-2100-hrs.-01-02.pdf}
\newline {\color{gray} (Emb: 0.569, TF-IDF: 0.332)}


\item \textbf{Artículo} \newline
El Estado fomentará y protegerá las empresas cooperativas de energía y el autoconsumo. 
\newline {\color{gray} \textbf{1º:} 645-Iniciativa-Convencional-Constituyente-de-la-cc-Carolina-Sepulveda-sobre-Estatuto-de-la-Energia-121101-02.pdf}
\newline {\color{gray} (Emb: 0.811, TF-IDF: 0.573)}
\newline {\color{gray} \textbf{2º:} 516-4-Iniciativa-Convencional-Constituyente-de-la-cc-Manuela-Royo-sobre-Derecho-a-la-Energia-1538-01-02.pdf}
\newline {\color{gray} (Emb: 0.667, TF-IDF: 0.401)}

El Estado deberá regular y fomentar una matriz energética distribuida, descentralizada y diversificada, basada en energías renovables y de bajo impacto ambiental. 
\newline {\color{gray} \textbf{1º:} 776-Iniciativa-Convencional-Constituyente-de-la-cc-Francisca-Arauna-sobre-Agricultura.pdf}
\newline {\color{gray} (Emb: 0.647, TF-IDF: 0.460)}
\newline {\color{gray} \textbf{2º:} 113-5-c-Iniciativa-del-cc-Elsa-Labrana-sobre-Soberania-Alimentaria.pdf}
\newline {\color{gray} (Emb: 0.584, TF-IDF: 0.333)}

La infraestructura energética es de interés público. 
\newline {\color{gray} \textbf{1º:} 478-4-Iniciativa-Convencional-Constituyente-de-la-cc-Barbara-Rebolledo-Derecho-al-cuidado-2029-31-01.pdf}
\newline {\color{gray} (Emb: 0.660, TF-IDF: 0.449)}
\newline {\color{gray} \textbf{2º:} 1027-Iniciativa-Convencional-Consituyente-del-cc-Felipe-Harboe-sobre-Derecho-a-la-Vida.pdf}
\newline {\color{gray} (Emb: 0.657, TF-IDF: 0.337)}

Toda persona tiene derecho a un mínimo vital de energía asequible y segura. 
\newline {\color{gray} \textbf{1º:} 957-5-Iniciativa-Convencional-Constituyente-de-la-cc-Ivanna-Olivares-sobre-Nuevo-Modelo-Economico.pdf}
\newline {\color{gray} (Emb: 0.690, TF-IDF: 0.632)}
\newline {\color{gray} \textbf{2º:} 1022-Iniciativa-Convencional-Constituyente-de-la-cc-Elsa-Labrana-sobre-Principio-de-Politica-Exterior-RESPALDO.pdf}
\newline {\color{gray} (Emb: 0.665, TF-IDF: 0.623)}

Es deber del Estado garantizar el acceso equitativo y no discriminatorio a la energía que permita a las personas satisfacer sus necesidades, velando por la continuidad de los servicios energéticos. 
\newline {\color{gray} \textbf{1º:} 345-4-Iniciativa-Convencional-Constituyente-de-la-cc-Alejandra-Flores-sobre-Derecho-a-la-Alimentacion.pdf}
\newline {\color{gray} (Emb: 0.579, TF-IDF: 0.283)}
\newline {\color{gray} \textbf{2º:} 773-Iniciativa-Convencional-Constituyente-de-la-cc-Adriana-Cancino-sobre-Derecho-a-la-Alimentacion-Adecuada.pdf}
\newline {\color{gray} (Emb: 0.561, TF-IDF: 0.269)}


\item \textbf{Artículo} \newline
El Estado tiene el dominio absoluto, exclusivo, inalienable e imprescriptible de todas las minas y las sustancias minerales, metálicas, no metálicas, y los depósitos de sustancias fósiles e hidrocarburos existentes en el territorio nacional, sin perjuicio de la propiedad sobre los terrenos en que estuvieren situadas. 
\newline {\color{gray} \textbf{1º:} 516-4-Iniciativa-Convencional-Constituyente-de-la-cc-Manuela-Royo-sobre-Derecho-a-la-Energia-1538-01-02.pdf}
\newline {\color{gray} (Emb: 0.738, TF-IDF: 0.625)}
\newline {\color{gray} \textbf{2º:} 645-Iniciativa-Convencional-Constituyente-de-la-cc-Carolina-Sepulveda-sobre-Estatuto-de-la-Energia-121101-02.pdf}
\newline {\color{gray} (Emb: 0.672, TF-IDF: 0.500)}

La exploración, explotación y aprovechamiento de estas sustancias se sujetará a una regulación que considere su carácter finito, no renovable, de interés público intergeneracional y la protección ambiental. 
\newline {\color{gray} \textbf{1º:} 516-4-Iniciativa-Convencional-Constituyente-de-la-cc-Manuela-Royo-sobre-Derecho-a-la-Energia-1538-01-02.pdf}
\newline {\color{gray} (Emb: 0.885, TF-IDF: 0.742)}
\newline {\color{gray} \textbf{2º:} 645-Iniciativa-Convencional-Constituyente-de-la-cc-Carolina-Sepulveda-sobre-Estatuto-de-la-Energia-121101-02.pdf}
\newline {\color{gray} (Emb: 0.833, TF-IDF: 0.699)}


\item \textbf{Artículo} \newline
El Estado establecerá una política para la actividad minera y su encadenamiento productivo, la que considerará, a lo menos, la protección ambiental y social, la innovación, la generación de valor agregado, el acceso y uso de tecnología y la protección de la pequeña minería y pirquineros. 
\newline {\color{gray} \textbf{1º:} 888-Iniciativa-Convencional-Constituyente-de-la-cc-Natalia-Henriquez-sobre-Derechos-de-los-Consumidores.pdf}
\newline {\color{gray} (Emb: 0.570, TF-IDF: 0.536)}
\newline {\color{gray} \textbf{2º:} 99-3-c-Iniciativa-de-la-cc-Tammy-Pustilnick-Disposiciones-del-Estado-Regional.pdf}
\newline {\color{gray} (Emb: 0.563, TF-IDF: 0.337)}


\item \textbf{Artículo} \newline
Quedarán excluidos de toda actividad minera los glaciares, las áreas protegidas, las que por razones de protección hidrográfica establezca la ley, y las demás que ella declare. 
\newline {\color{gray} \textbf{1º:} 873-Iniciativa-Convencional-Constituyente-de-la-cc-Camila-Zarate-Sobre-Mineria.pdf}
\newline {\color{gray} (Emb: 0.864, TF-IDF: 0.753)}
\newline {\color{gray} \textbf{2º:} 414-5-Iniciativa-Convencional-del-cc-Bernardo-Fontaine-sobre-Estatuto-de-los-Minerales-1500-25-01.pdf}
\newline {\color{gray} (Emb: 0.833, TF-IDF: 0.596)}


\item \textbf{Artículo} \newline
Lo dispuesto en el inciso primero del artículo 22 no se aplicará a las arcillas superficiales. 
\newline {\color{gray} \textbf{1º:} 882-Iniciativa-Convencional-Constituyente-de-la-cc-Isabel-Godoy-Sobre-Mineria.pdf}
\newline {\color{gray} (Emb: 0.563, TF-IDF: 0.314)}
\newline {\color{gray} \textbf{2º:} 325-6-Iniciativa-Convencional-del-cc-Tomas-Laibe-sobre-Jurisdiccion-Constitucional.pdf}
\newline {\color{gray} (Emb: 0.556, TF-IDF: 0.285)}


\item \textbf{Artículo} \newline
El Estado deberá regular los impactos y efectos sinérgicos generados en las distintas etapas de la actividad minera, incluyendo su encadenamiento productivo, cierre o paralización, en la forma que establezca la ley. 
\newline {\color{gray} \textbf{1º:} 270-5-Iniciativa-Convencional-de-la-cc-Ivanna-Olivares-sobre-Bienes-Naturales-Estrategicos-17-01-1154-hrs.pdf}
\newline {\color{gray} (Emb: 0.565, TF-IDF: 0.413)}
\newline {\color{gray} \textbf{2º:} 992-Iniciativa-Convencional-Constituyente-de-la-cc-Ivanna-Olivares-sobre-Mineria.pdf}
\newline {\color{gray} (Emb: 0.562, TF-IDF: 0.268)}

Será obligación de quien realice la actividad minera destinar recursos para reparar los daños causados, los pasivos ambientales y mitigar sus efectos nocivos en los territorios en que ésta se desarrolla, de acuerdo a la ley. 
\newline {\color{gray} \textbf{1º:} 875-Iniciativa-Convencional-Constituyente-de-la-cc-Constanza-San-Juan-Sobre-Criosfera.pdf}
\newline {\color{gray} (Emb: 0.593, TF-IDF: 0.299)}
\newline {\color{gray} \textbf{2º:} 254-7-Iniciativa-Convencional-de-la-cc-Cristina-Dorador-sobre-Reservas-Patrimoniales-1150-hrs.pdf}
\newline {\color{gray} (Emb: 0.569, TF-IDF: 0.282)}

La ley especificará el modo en que esta obligación se aplicará a la pequeña minería y pirquineros. 
\newline {\color{gray} \textbf{1º:} 319-6-Iniciativa-Convencional-del-cc-Mauricio-Daza-sobre-el-Sistema-Nacional-de-Justicia17-09-hrs.pdf}
\newline {\color{gray} (Emb: 0.346, TF-IDF: 0.417)}
\newline {\color{gray} \textbf{2º:} 880-Iniciativa-Convencional-Constituyente-de-la-cc-Ingrid-Villena-sobre-Acciones-Constitucionales.pdf}
\newline {\color{gray} (Emb: 0.346, TF-IDF: 0.360)}


\item \textbf{Artículo} \newline
El Estado adoptará las medidas necesarias para proteger la pequeña minería y pirquineros, las fomentará y facilitará el acceso y uso de las herramientas, tecnologías y recursos para el ejercicio tradicional y sustentable de la actividad. 
\newline {\color{gray} \textbf{1º:} 120-3-c-Iniciativa-de-la-cc-Tammy-Pustilnick-atribuciones-exclusivas-de-la-Asamblea-Regional.pdf}
\newline {\color{gray} (Emb: 0.493, TF-IDF: 0.475)}
\newline {\color{gray} \textbf{2º:} 712-Iniciativa-Convencional-Constituyente-de-la-cc-Lisette-Vergara-sobre-Administracion-Comunal.pdf}
\newline {\color{gray} (Emb: 0.478, TF-IDF: 0.302)}


\item \textbf{Artículo} \newline
El Estado impulsará medidas para conservar la atmósfera y el cielo nocturno, según las necesidades territoriales. 
\newline {\color{gray} \textbf{1º:} 628-Iniciativa-Convencional-Constituyente-de-cc-Roberto-Vega-sobre-Mineria-general-y-fomento-de-la-pequena-mineria.pdf}
\newline {\color{gray} (Emb: 0.727, TF-IDF: 0.395)}
\newline {\color{gray} \textbf{2º:} 996-Iniciativa-Convencional-Constituyente-del-cc-Francisco-Caamano-sobre-Acceso-a-la-Naturaleza.pdf}
\newline {\color{gray} (Emb: 0.648, TF-IDF: 0.357)}

Es deber del Estado contribuir y cooperar internacionalmente en la investigación del espacio con fines pacíficos y científicos. 
\newline {\color{gray} \textbf{1º:} 614-Iniciativa-Convencional-Constituyente-de-cc-Cesar-Uribe-sobre-Dano-Ambiental-y-Zonas-de-Sacrificio-2100-hrs.-01-02.pdf}
\newline {\color{gray} (Emb: 0.663, TF-IDF: 0.235)}
\newline {\color{gray} \textbf{2º:} 44-2-Iniciativa-Convencional-Constituyente-de-la-cc-Claudia-Castro-y-otros.pdf}
\newline {\color{gray} (Emb: 0.621, TF-IDF: 0.221)}


\item \textbf{Artículo} \newline
El Estado participa en la economía para cumplir con los objetivos establecidos en esta Constitución. 
\newline {\color{gray} \textbf{1º:} 992-Iniciativa-Convencional-Constituyente-de-la-cc-Ivanna-Olivares-sobre-Mineria.pdf}
\newline {\color{gray} (Emb: 0.533, TF-IDF: 0.434)}
\newline {\color{gray} \textbf{2º:} 113-5-c-Iniciativa-del-cc-Elsa-Labrana-sobre-Soberania-Alimentaria.pdf}
\newline {\color{gray} (Emb: 0.527, TF-IDF: 0.282)}

El rol económico del Estado se fundará, de manera coordinada y coherente, en los principios y objetivos económicos de solidaridad, diversificación productiva, economía social y solidaria y pluralismo económico. 
\newline {\color{gray} \textbf{1º:} 812-Iniciativa-Convencional-Constituyente-del-cc-Juan-Jose-Martin-sobre-Estatuto-del-Espacio.pdf}
\newline {\color{gray} (Emb: 0.811, TF-IDF: 0.696)}
\newline {\color{gray} \textbf{2º:} 812-Iniciativa-Convencional-Constituyente-del-cc-Juan-Jose-Martin-sobre-Estatuto-del-Espacio.pdf}
\newline {\color{gray} (Emb: 0.666, TF-IDF: 0.511)}

El Estado regula, fiscaliza, fomenta y desarrolla actividades económicas, disponiendo de sus potestades públicas, en el marco de sus atribuciones y competencias, en conformidad a lo establecido en esta Constitución y la ley. 
\newline {\color{gray} \textbf{1º:} 356-5-Iniciativa-Convencional-Constituyente-del-cc-Carlos-Calvo-sobre-Proteccion-de-los-Cielos-1201-21-01.pdf}
\newline {\color{gray} (Emb: 0.610, TF-IDF: 0.382)}
\newline {\color{gray} \textbf{2º:} 356-5-Iniciativa-Convencional-Constituyente-del-cc-Carlos-Calvo-sobre-Proteccion-de-los-Cielos-1201-21-01.pdf}
\newline {\color{gray} (Emb: 0.511, TF-IDF: 0.366)}

El Estado fomentará la innovación, los mercados locales, los circuitos cortos y la economía circular. 
\newline {\color{gray} \textbf{1º:} 995-Iniciativa-Convencional-Constituyente-del-cc-Christian-Viera-sobre-Principio-de-Intervencion-del-Estado.pdf}
\newline {\color{gray} (Emb: 0.632, TF-IDF: 0.464)}
\newline {\color{gray} \textbf{2º:} 903-Iniciativa-Convencional-Constituyente-del-cc-Fernando-Salinas-Sobre-Buen-Vivir.pdf}
\newline {\color{gray} (Emb: 0.617, TF-IDF: 0.340)}


\item \textbf{Artículo} \newline
El Estado tendrá iniciativa pública en la actividad económica. 
\newline {\color{gray} \textbf{1º:} 60-2-Iniciativa-Convencional-Constituyente-del-cc-Jorge-Baradit-y-otros.pdf}
\newline {\color{gray} (Emb: 0.604, TF-IDF: 0.258)}
\newline {\color{gray} \textbf{2º:} 422-4-Iniciativa-Convencional-de-la-cc-Maria-Trinidad-Castillo-sobre-Reconocimiento-de-las-cooperativas-1319-26-01.pdf}
\newline {\color{gray} (Emb: 0.584, TF-IDF: 0.236)}

Para ello, podrá desarrollar actividades empresariales, las que podrán adoptar diversas formas de propiedad, gestión y organización según determine la normativa respectiva. 
\newline {\color{gray} \textbf{1º:} 234-1-Iniciativa-Convencional-del-cc-Jaime-Bassa-sobre-Justicia-Complementaria-1144-hrs.pdf}
\newline {\color{gray} (Emb: 0.665, TF-IDF: 0.234)}
\newline {\color{gray} \textbf{2º:} 713-Iniciativa-Convencional-Constituyente-de-la-cc-Lisette-Vergara-sobre-Economia.pdf}
\newline {\color{gray} (Emb: 0.653, TF-IDF: 0.229)}

Las empresas públicas se deberán crear por ley y se regirán por el régimen jurídico que ésta determine. 
\newline {\color{gray} \textbf{1º:} 283-5-Iniciativa-Convencional-de-la-cc-Ivanna-Olivares-sobre-Derecho-de-Desarollar-Actividades-Economicas-1209-hrs.pdf}
\newline {\color{gray} (Emb: 0.556, TF-IDF: 0.424)}
\newline {\color{gray} \textbf{2º:} 708-Iniciativa-Convencional-Constituyente-de-la-cc-Francisca-Arauna-sobre-Gestion-de-Residuos.pdf}
\newline {\color{gray} (Emb: 0.549, TF-IDF: 0.411)}

Sin perjuicio de esto, en lo pertinente, serán aplicables las normas de derecho público sobre probidad y rendición de cuentas. 
\newline {\color{gray} \textbf{1º:} 450-5-Iniciativa-Convencional-Constituyente-de-la-cc-Bessy-Gallardo-sobre-Pp.-Iniciativa-Economica-del-E.-1415-31-01.pdf}
\newline {\color{gray} (Emb: 0.701, TF-IDF: 0.563)}
\newline {\color{gray} \textbf{2º:} 995-Iniciativa-Convencional-Constituyente-del-cc-Christian-Viera-sobre-Principio-de-Intervencion-del-Estado.pdf}
\newline {\color{gray} (Emb: 0.700, TF-IDF: 0.369)}


\item \textbf{Artículo} \newline
El Estado tendrá el deber de fijar una Política Nacional Portuaria, la cual se organizará en torno a los principios de eficiencia en el uso del borde costero; responsabilidad ambiental, poniendo especial énfasis en el cuidado de la naturaleza y bienes comunes naturales; la participación pública de los recursos que genere la actividad; la vinculación con territorios y comunidades donde se emplacen los recintos portuarios; establecimiento de la carrera profesional portuaria, reconociéndose como trabajo de alto riesgo; y la colaboración entre recintos e infraestructura portuaria para asegurar el oportuno abastecimiento de las comunidades. 
\newline {\color{gray} \textbf{1º:} 744-Iniciativa-Convencional-Constituyente-del-cc-Maria-Trinidad-Castillo-sobre-Emprendimiento-01-02.pdf}
\newline {\color{gray} (Emb: 0.631, TF-IDF: 0.594)}
\newline {\color{gray} \textbf{2º:} 860-Iniciativa-Convencional-Constituyente-del-cc-Jorge-Abarca-sobre-Regimen-Publico-Economico.pdf}
\newline {\color{gray} (Emb: 0.578, TF-IDF: 0.344)}


\item \textbf{Artículo} \newline
El Estado debe prevenir y sancionar los abusos en los mercados. 
\newline {\color{gray} \textbf{1º:} 713-Iniciativa-Convencional-Constituyente-de-la-cc-Lisette-Vergara-sobre-Economia.pdf}
\newline {\color{gray} (Emb: 0.675, TF-IDF: 0.344)}
\newline {\color{gray} \textbf{2º:} 635-4-Iniciativa-Convencional-Constituyente-del-cc-Cristobal-Andrade-sobre-Libertad-de-Conciencia-1730-01-02.pdf}
\newline {\color{gray} (Emb: 0.653, TF-IDF: 0.313)}


\item \textbf{Artículo} \newline
El Estado debe garantizar este derecho. 
\newline {\color{gray} \textbf{1º:} 976-Iniciativa-Convencional-Constituye-de-la-cc-Tania-Madriaga-sobre-Puertos.pdf}
\newline {\color{gray} (Emb: 0.697, TF-IDF: 0.782)}
\newline {\color{gray} \textbf{2º:} 415-5-Iniciativa-Convencional-del-cc-Bernardo-Fontaine-sobre-Medio-Ambiente-y-Desarrollo-sostenible.pdf}
\newline {\color{gray} (Emb: 0.513, TF-IDF: 0.314)}

Todas las personas tienen el derecho a un ambiente sano y ecológicamente equilibrado. 
\newline {\color{gray} \textbf{1º:} 70-2-Iniciativa-Convencional-Constituyente-de-la-cc-Paulina-Veloso-y-otros-2.pdf}
\newline {\color{gray} (Emb: 0.670, TF-IDF: 0.450)}
\newline {\color{gray} \textbf{2º:} 179-6-c-Iniciativa-Convencional-DEL-CC-Rodrigo-Álvarez-Ministerio-Público-1044-hrs.pdf}
\newline {\color{gray} (Emb: 0.632, TF-IDF: 0.366)}


\item \textbf{Artículo} \newline
El Estado garantiza el acceso a la justicia ambiental. 
\newline {\color{gray} \textbf{1º:} 443-Iniciativa-Convencional-Constituyente-de-la-cc-Giovanna-Grandon-sobre-Nacionalidad-1405-28-01.pdf}
\newline {\color{gray} (Emb: 0.666, TF-IDF: 0.369)}
\newline {\color{gray} \textbf{2º:} 507-2-Iniciativa-Convencional-Constituyente-de-la-cc-Giovanna-Grandon-sobre-Nacionalidad-1213-01-02.pdf}
\newline {\color{gray} (Emb: 0.666, TF-IDF: 0.366)}


\item \textbf{Artículo} \newline
Todas las personas tienen el derecho al aire limpio durante todo el ciclo de vida, en la forma que determine la ley. 
\newline {\color{gray} \textbf{1º:} 784-niciativa-Convencional-Constituyente-de-la-cc-Damaris-Abarca-sobre-Derecho-a-vivir-en-un-ambiente-sano.pdf}
\newline {\color{gray} (Emb: 0.908, TF-IDF: 0.913)}
\newline {\color{gray} \textbf{2º:} 415-5-Iniciativa-Convencional-del-cc-Bernardo-Fontaine-sobre-Medio-Ambiente-y-Desarrollo-sostenible.pdf}
\newline {\color{gray} (Emb: 0.880, TF-IDF: 0.910)}


\item \textbf{Artículo} \newline
Es deber del Estado garantizar una educación ambiental que fortalezca la preservación, conservación y cuidados requeridos respecto al medio ambiente y la Naturaleza, y que permita formar conciencia ecológica. 
\newline {\color{gray} \textbf{1º:} 291-4-Iniciativa-Convencional-de-la-cc-Tatiana-Urrutia-sobre-Derecho-de-Reunion-Peticion-y-Asociacion-1605-hrs.pdf}
\newline {\color{gray} (Emb: 0.927, TF-IDF: 0.647)}
\newline {\color{gray} \textbf{2º:} 135-4-c-Iniciativa-de-la-cc-Rocio-Cantuarias-Sobre-Seguridad-Social.pdf}
\newline {\color{gray} (Emb: 0.738, TF-IDF: 0.607)}


\item \textbf{Artículo} \newline
La jurisdicción es una función pública que se ejerce en nombre de los pueblos y que consiste en conocer y juzgar, por medio de un debido proceso los conflictos de relevancia jurídica y hacer ejecutar lo resuelto, de conformidad a la Constitución y las leyes, así como los tratados e instrumentos internacionales sobre derechos humanos de los que Chile es parte. 
\newline {\color{gray} \textbf{1º:} 788-Iniciativa-Convencional-Constituyente-de-la-cc-Camila-Zarate-sobre-Democracia-Ecologica.pdf}
\newline {\color{gray} (Emb: 0.808, TF-IDF: 0.700)}
\newline {\color{gray} \textbf{2º:} 672-Iniciativa-Convencional-Constituyente-del-cc-Carolina-Sepulveda-sobre-Principios-Ambientales-121101-02.pdf}
\newline {\color{gray} (Emb: 0.757, TF-IDF: 0.604)}

Se ejerce exclusivamente por los tribunales de justicia y las autoridades de los pueblos indígenas reconocidos por la Constitución o las leyes dictadas conforme a ella. 
\newline {\color{gray} \textbf{1º:} 691-Iniciativa-Convencional-Constituyente-del-cc-Patricio-Fernandez-sobre-Derecho-a-Respirar-Aire-Puro121101-02.pdf}
\newline {\color{gray} (Emb: 0.819, TF-IDF: 0.647)}
\newline {\color{gray} \textbf{2º:} 415-5-Iniciativa-Convencional-del-cc-Bernardo-Fontaine-sobre-Medio-Ambiente-y-Desarrollo-sostenible.pdf}
\newline {\color{gray} (Emb: 0.726, TF-IDF: 0.499)}

Al ejercer la jurisdicción se debe velar por la tutela y promoción de los derechos humanos y de la naturaleza, del sistema democrático y el principio de juridicidad. 
\newline {\color{gray} \textbf{1º:} 858-Iniciativa-Convencional-Constituyente-del-cc-Jorge-Abarca-sobre-Estado-Ecologico-de-Derecho.pdf}
\newline {\color{gray} (Emb: 0.676, TF-IDF: 0.326)}
\newline {\color{gray} \textbf{2º:} 857-Iniciativa-Convencional-Constituyente-del-cc-Jorge-Abarca-sobre-Derecho-a-un-Ambiente-Sano.pdf}
\newline {\color{gray} (Emb: 0.663, TF-IDF: 0.323)}


\item \textbf{Artículo} \newline
El Estado reconoce los sistemas jurídicos de los Pueblos Indígenas, los que en virtud de su derecho a la libre determinación coexisten coordinados en un plano de igualdad con el Sistema Nacional de Justicia. 
\newline {\color{gray} \textbf{1º:} 90-6-Iniciativa-Convencional-Constituyente-del-cc-Tomas-Laibe-y-otros.pdf}
\newline {\color{gray} (Emb: 0.703, TF-IDF: 0.482)}
\newline {\color{gray} \textbf{2º:} 210-1-c-Iniciativa-Convencional-del-cc-Cristián-Monckeberg-sobre-Estado-Intercultural-1953-hrs.pdf}
\newline {\color{gray} (Emb: 0.701, TF-IDF: 0.431)}

Estos deberán respetar los derechos fundamentales que establece esta Constitución y los tratados e instrumentos internacionales sobre derechos humanos de los que Chile es parte. 
\newline {\color{gray} \textbf{1º:} 160-4-c-Iniciativa-de-la-cc-Valentina-Miranda-sobre-contenido-de-los-Derechos-Fundamentales.pdf}
\newline {\color{gray} (Emb: 0.703, TF-IDF: 0.538)}
\newline {\color{gray} \textbf{2º:} 97-6-Iniciativa-del-cc-Daniel-Bravo-Sistemas-de-Justicia.pdf}
\newline {\color{gray} (Emb: 0.681, TF-IDF: 0.400)}

La ley determinará los mecanismos de coordinación, cooperación y de resolución de conflictos de competencia entre los sistemas jurídicos indígenas y las entidades estatales. 
\newline {\color{gray} \textbf{1º:} 41-6-Iniciativa-Convencional-Constituyente-del-cc-Mauricio-Daza-y-otros-1.pdf}
\newline {\color{gray} (Emb: 0.790, TF-IDF: 0.616)}
\newline {\color{gray} \textbf{2º:} 11-4-Iniciativa-Convencional-Constituyente-de-la-cc-María-Elisa-Quinteros-y-otras.pdf}
\newline {\color{gray} (Emb: 0.706, TF-IDF: 0.316)}


\item \textbf{Artículo} \newline
Las juezas y jueces no podrán militar en partidos políticos. 
\newline {\color{gray} \textbf{1º:} 41-6-Iniciativa-Convencional-Constituyente-del-cc-Mauricio-Daza-y-otros-1.pdf}
\newline {\color{gray} (Emb: 0.889, TF-IDF: 0.788)}
\newline {\color{gray} \textbf{2º:} 90-6-Iniciativa-Convencional-Constituyente-del-cc-Tomas-Laibe-y-otros.pdf}
\newline {\color{gray} (Emb: 0.811, TF-IDF: 0.430)}

Las juezas y jueces no podrán desempeñar ninguna otra función o empleo, salvo actividades académicas en los términos que establezca la ley. 
\newline {\color{gray} \textbf{1º:} 98-6-Iniciativa-del-cc-Ruggero-Cozzi-Funcion-y-Principios-de-la-Jurisdiccion.pdf}
\newline {\color{gray} (Emb: 1.000, TF-IDF: 1.000)}
\newline {\color{gray} \textbf{2º:} 226-6-Iniciativa-Convencional-de-la-cc-Manuela-Royo-sobre-Justicia-Local-1140-hrs.pdf}
\newline {\color{gray} (Emb: 0.779, TF-IDF: 0.463)}

Ningún otro órgano del Estado, persona o grupo de personas, podrán ejercer la función jurisdiccional, conocer causas pendientes, modificar los fundamentos o el contenido de las resoluciones judiciales o reabrir procesos concluidos. 
\newline {\color{gray} \textbf{1º:} 88-6-Iniciativa-Convencional-Constituyente-del-cc-Christian-Viera-y-otros.pdf}
\newline {\color{gray} (Emb: 0.874, TF-IDF: 0.603)}
\newline {\color{gray} \textbf{2º:} 98-6-Iniciativa-del-cc-Ruggero-Cozzi-Funcion-y-Principios-de-la-Jurisdiccion.pdf}
\newline {\color{gray} (Emb: 0.865, TF-IDF: 0.575)}

Las juezas y jueces sólo ejercerán la función jurisdiccional, no pudiendo desempeñar función administrativa ni legislativa alguna. 
\newline {\color{gray} \textbf{1º:} 90-6-Iniciativa-Convencional-Constituyente-del-cc-Tomas-Laibe-y-otros.pdf}
\newline {\color{gray} (Emb: 0.959, TF-IDF: 0.694)}
\newline {\color{gray} \textbf{2º:} 88-6-Iniciativa-Convencional-Constituyente-del-cc-Christian-Viera-y-otros.pdf}
\newline {\color{gray} (Emb: 0.786, TF-IDF: 0.546)}

En sus providencias, sólo están sometidos al imperio de la ley. 
\newline {\color{gray} \textbf{1º:} 11-4-Iniciativa-Convencional-Constituyente-de-la-cc-María-Elisa-Quinteros-y-otras.pdf}
\newline {\color{gray} (Emb: 0.843, TF-IDF: 0.451)}
\newline {\color{gray} \textbf{2º:} 841-Iniciativa-Convencional-Constituyente-de-la-cc-Francisca-Arauna-sobre-Seguridad-Publica.pdf}
\newline {\color{gray} (Emb: 0.801, TF-IDF: 0.440)}

Las juezas y jueces que ejercen jurisdicción son independientes entre sí y de todo otro poder o autoridad, debiendo actuar y resolver de forma imparcial. 
\newline {\color{gray} \textbf{1º:} 190-6-c-Iniciativa-Convencional-Constituyente-de-la-cc-Natividad-LLanquileo-sobre-Pluralismo-Jurídico-1240-hrs.pdf}
\newline {\color{gray} (Emb: 0.703, TF-IDF: 0.543)}
\newline {\color{gray} \textbf{2º:} 41-6-Iniciativa-Convencional-Constituyente-del-cc-Mauricio-Daza-y-otros-1.pdf}
\newline {\color{gray} (Emb: 0.680, TF-IDF: 0.337)}

La función jurisdiccional la ejercen exclusivamente los tribunales establecidos por ley. 
\newline {\color{gray} \textbf{1º:} 210-1-c-Iniciativa-Convencional-del-cc-Cristián-Monckeberg-sobre-Estado-Intercultural-1953-hrs.pdf}
\newline {\color{gray} (Emb: 0.692, TF-IDF: 0.381)}
\newline {\color{gray} \textbf{2º:} 700-Iniciativa-Convencional-Constituyente-de-la-cc-Alejandra-Perez-sobre-Resolucion-de-Conflictos.pdf}
\newline {\color{gray} (Emb: 0.679, TF-IDF: 0.381)}


\item \textbf{Artículo} \newline
Las juezas y jueces son inamovibles. 
\newline {\color{gray} \textbf{1º:} 41-6-Iniciativa-Convencional-Constituyente-del-cc-Mauricio-Daza-y-otros-1.pdf}
\newline {\color{gray} (Emb: 0.846, TF-IDF: 0.761)}
\newline {\color{gray} \textbf{2º:} 160-4-c-Iniciativa-de-la-cc-Valentina-Miranda-sobre-contenido-de-los-Derechos-Fundamentales.pdf}
\newline {\color{gray} (Emb: 0.656, TF-IDF: 0.316)}

No pueden ser suspendidos, trasladados o removidos sino conforme a las causales y procedimientos establecidos por la Constitución y las leyes. 
\newline {\color{gray} \textbf{1º:} 97-6-Iniciativa-del-cc-Daniel-Bravo-Sistemas-de-Justicia.pdf}
\newline {\color{gray} (Emb: 0.757, TF-IDF: 0.459)}
\newline {\color{gray} \textbf{2º:} 41-6-Iniciativa-Convencional-Constituyente-del-cc-Mauricio-Daza-y-otros-1.pdf}
\newline {\color{gray} (Emb: 0.750, TF-IDF: 0.421)}


\item \textbf{Artículo} \newline
La Constitución garantiza el pleno acceso a la justicia a todas las personas y colectivos. 
\newline {\color{gray} \textbf{1º:} 41-6-Iniciativa-Convencional-Constituyente-del-cc-Mauricio-Daza-y-otros-1.pdf}
\newline {\color{gray} (Emb: 0.844, TF-IDF: 0.632)}
\newline {\color{gray} \textbf{2º:} 325-6-Iniciativa-Convencional-del-cc-Tomas-Laibe-sobre-Jurisdiccion-Constitucional.pdf}
\newline {\color{gray} (Emb: 0.698, TF-IDF: 0.556)}

Es deber del Estado remover los obstáculos sociales, culturales y económicos que impidan o limiten la posibilidad de acudir a los órganos jurisdiccionales para la tutela y el ejercicio de sus derechos. 
\newline {\color{gray} \textbf{1º:} 90-6-Iniciativa-Convencional-Constituyente-del-cc-Tomas-Laibe-y-otros.pdf}
\newline {\color{gray} (Emb: 0.884, TF-IDF: 0.944)}
\newline {\color{gray} \textbf{2º:} 472-6-Iniciativa-Convencional-Constituyente-del-cc-Daniel-Bravo-sobre-Corte-Constitucional-2003-31-01.pdf}
\newline {\color{gray} (Emb: 0.754, TF-IDF: 0.644)}

Los tribunales deben brindar una atención adecuada a quienes presenten peticiones o consultas ante ellos, otorgando siempre un trato digno y respetuoso. 
\newline {\color{gray} \textbf{1º:} 88-6-Iniciativa-Convencional-Constituyente-del-cc-Christian-Viera-y-otros.pdf}
\newline {\color{gray} (Emb: 0.700, TF-IDF: 0.401)}
\newline {\color{gray} \textbf{2º:} 41-6-Iniciativa-Convencional-Constituyente-del-cc-Mauricio-Daza-y-otros-1.pdf}
\newline {\color{gray} (Emb: 0.694, TF-IDF: 0.383)}

Una ley establecerá sus derechos y deberes. 
\newline {\color{gray} \textbf{1º:} 396-4-Iniciativa-Convencional-Constituyente-de-la-cc-Natalia-Henriquez-sobre-Derecho-a-la-salud1559-24-01.pdf}
\newline {\color{gray} (Emb: 0.800, TF-IDF: 0.502)}
\newline {\color{gray} \textbf{2º:} 441-Iniciativa-Convencional-Constituyente-de-la-cc-Geoconda-Navarrete-sobre-Derecho-a-una-Vejez-Digna-1403-28-01.pdf}
\newline {\color{gray} (Emb: 0.796, TF-IDF: 0.496)}


\item \textbf{Artículo} \newline
Todas las personas tienen derecho a requerir de los tribunales de justicia la tutela efectiva de sus derechos e intereses legítimos, de manera oportuna y eficaz conforme a los principios y estándares reconocidos en la Constitución y las leyes. 
\newline {\color{gray} \textbf{1º:} 317-6-Iniciativa-Convencional-del-cc-Cristian-Monckeberg-sobre-Derecho-de-Acceso-a-la-Justicia-16-00-hrs.pdf}
\newline {\color{gray} (Emb: 0.818, TF-IDF: 0.731)}
\newline {\color{gray} \textbf{2º:} 454-5-Iniciativa-Convencional-Constituyente-del-cc-Marcos-Barraza-sobre-Economia-Social-1529-31-01.pdf}
\newline {\color{gray} (Emb: 0.683, TF-IDF: 0.393)}


\item \textbf{Artículo} \newline
Reclamada su intervención en la forma legal y sobre materias de su competencia, los tribunales no podrán excusarse de ejercer su función en un tiempo razonable ni aún a falta de norma jurídica expresa que resuelva el asunto sometido a su decisión. 
\newline {\color{gray} \textbf{1º:} 232-6-Iniciativa-Convencional-del-cc-Marco-Arellano-que-Crea-el-Consejo-Nacional-de-Justicia-1144-hrs.pdf}
\newline {\color{gray} (Emb: 0.684, TF-IDF: 0.515)}
\newline {\color{gray} \textbf{2º:} 220-6-c-Iniciativa-Convencional-del-cc-Daniel-Bravo-sobre-organizacion-de-tribunales-2315-hrs.pdf}
\newline {\color{gray} (Emb: 0.606, TF-IDF: 0.256)}

El ejercicio de la jurisdicción es indelegable. 
\newline {\color{gray} \textbf{1º:} 648-Iniciativa-Convencional-Constituyente-de-la-cc-Carolina-Sepulveda-sobre-Funcion-Publica-121101-02.pdf}
\newline {\color{gray} (Emb: 0.817, TF-IDF: 0.484)}
\newline {\color{gray} \textbf{2º:} 924-Iniciativa-Convencional-Constituyente-de-la-cc-Constanza-Schonhaut-sobre-Administracion-del-Estado.pdf}
\newline {\color{gray} (Emb: 0.777, TF-IDF: 0.472)}


\item \textbf{Artículo} \newline
Para hacer ejecutar las resoluciones y practicar o hacer practicar las actuaciones que determine la ley, los tribunales de justicia podrán impartir órdenes o instrucciones directas a la fuerza pública, debiendo cumplir lo mandatado de forma rápida y expedita, sin poder calificar su fundamento, oportunidad o legalidad. 
\newline {\color{gray} \textbf{1º:} 700-Iniciativa-Convencional-Constituyente-de-la-cc-Alejandra-Perez-sobre-Resolucion-de-Conflictos.pdf}
\newline {\color{gray} (Emb: 0.803, TF-IDF: 0.387)}
\newline {\color{gray} \textbf{2º:} 97-6-Iniciativa-del-cc-Daniel-Bravo-Sistemas-de-Justicia.pdf}
\newline {\color{gray} (Emb: 0.743, TF-IDF: 0.339)}

Las sentencias dictadas contra el Estado de Chile por tribunales internacionales de derechos humanos, cuya jurisdicción ha sido reconocida por éste, serán cumplidas por los tribunales de justicia conforme al procedimiento establecido por la ley, aun si contraviniere una sentencia firme pronunciada por estos. 
\newline {\color{gray} \textbf{1º:} 180-6-c-Iniciativa-Convencional-del-cc-Rodrigo-Álvarez-que-regula-el-Poder-Judicial-1044-hrs.pdf}
\newline {\color{gray} (Emb: 0.885, TF-IDF: 0.893)}
\newline {\color{gray} \textbf{2º:} 88-6-Iniciativa-Convencional-Constituyente-del-cc-Christian-Viera-y-otros.pdf}
\newline {\color{gray} (Emb: 0.878, TF-IDF: 0.662)}


\item \textbf{Artículo} \newline
La ley podrá establecer excepciones al deber de fundamentación de las resoluciones judiciales. 
\newline {\color{gray} \textbf{1º:} 88-6-Iniciativa-Convencional-Constituyente-del-cc-Christian-Viera-y-otros.pdf}
\newline {\color{gray} (Emb: 0.874, TF-IDF: 0.656)}
\newline {\color{gray} \textbf{2º:} 90-6-Iniciativa-Convencional-Constituyente-del-cc-Tomas-Laibe-y-otros.pdf}
\newline {\color{gray} (Emb: 0.870, TF-IDF: 0.656)}

Las sentencias deberán ser siempre fundadas y redactadas en un lenguaje claro e inclusivo. 
\newline {\color{gray} \textbf{1º:} 232-6-Iniciativa-Convencional-del-cc-Marco-Arellano-que-Crea-el-Consejo-Nacional-de-Justicia-1144-hrs.pdf}
\newline {\color{gray} (Emb: 0.752, TF-IDF: 0.412)}
\newline {\color{gray} \textbf{2º:} 90-6-Iniciativa-Convencional-Constituyente-del-cc-Tomas-Laibe-y-otros.pdf}
\newline {\color{gray} (Emb: 0.719, TF-IDF: 0.405)}


\item \textbf{Artículo} \newline
El acceso a la función jurisdiccional será gratuito, sin perjuicio de las actuaciones judiciales y sanciones procesales establecidas por la ley. 
\newline {\color{gray} \textbf{1º:} 90-6-Iniciativa-Convencional-Constituyente-del-cc-Tomas-Laibe-y-otros.pdf}
\newline {\color{gray} (Emb: 0.794, TF-IDF: 0.339)}
\newline {\color{gray} \textbf{2º:} 90-6-Iniciativa-Convencional-Constituyente-del-cc-Tomas-Laibe-y-otros.pdf}
\newline {\color{gray} (Emb: 0.740, TF-IDF: 0.286)}

La justicia arbitral será siempre voluntaria. 
\newline {\color{gray} \textbf{1º:} 90-6-Iniciativa-Convencional-Constituyente-del-cc-Tomas-Laibe-y-otros.pdf}
\newline {\color{gray} (Emb: 0.845, TF-IDF: 0.688)}
\newline {\color{gray} \textbf{2º:} 232-6-Iniciativa-Convencional-del-cc-Marco-Arellano-que-Crea-el-Consejo-Nacional-de-Justicia-1144-hrs.pdf}
\newline {\color{gray} (Emb: 0.750, TF-IDF: 0.422)}

La ley no podrá establecer arbitrajes forzosos. 
\newline {\color{gray} \textbf{1º:} 97-6-Iniciativa-del-cc-Daniel-Bravo-Sistemas-de-Justicia.pdf}
\newline {\color{gray} (Emb: 0.749, TF-IDF: 0.573)}
\newline {\color{gray} \textbf{2º:} 143-4-c-Iniciativa-de-la-cc-Rocio-Cantuarias-Incorpora-Libertad-de-Trabajo-y-Sindical.pdf}
\newline {\color{gray} (Emb: 0.717, TF-IDF: 0.392)}


\item \textbf{Artículo} \newline
Las juezas y jueces son personalmente responsables por los delitos de cohecho, falta de observancia en materia sustancial de las leyes que reglan el procedimiento, y, en general, por toda prevaricación, denegación o torcida administración de justicia. 
\newline {\color{gray} \textbf{1º:} 97-6-Iniciativa-del-cc-Daniel-Bravo-Sistemas-de-Justicia.pdf}
\newline {\color{gray} (Emb: 0.715, TF-IDF: 0.377)}
\newline {\color{gray} \textbf{2º:} 367-2-Iniciativa-Convencional-Constituyente-del-cc-Eduardo-Castillo-sobre-Ciudadania-y-Migracion-0900-hrs-24-01.pdf}
\newline {\color{gray} (Emb: 0.676, TF-IDF: 0.364)}

La ley determinará los casos y el modo de hacer efectiva esta responsabilidad. 
\newline {\color{gray} \textbf{1º:} 41-6-Iniciativa-Convencional-Constituyente-del-cc-Mauricio-Daza-y-otros-1.pdf}
\newline {\color{gray} (Emb: 1.000, TF-IDF: 1.000)}
\newline {\color{gray} \textbf{2º:} 41-6-Iniciativa-Convencional-Constituyente-del-cc-Mauricio-Daza-y-otros-1.pdf}
\newline {\color{gray} (Emb: 0.691, TF-IDF: 0.517)}

Los perjuicios por error judicial otorgarán el derecho a una indemnización por el Estado, conforme al procedimiento establecido en la Constitución y las leyes. 
\newline {\color{gray} \textbf{1º:} 97-6-Iniciativa-del-cc-Daniel-Bravo-Sistemas-de-Justicia.pdf}
\newline {\color{gray} (Emb: 0.964, TF-IDF: 0.245)}
\newline {\color{gray} \textbf{2º:} 682-Iniciativa-Convencional-Constituyente-del-cc-Javier-Fuchslocher-sobre-Trabajo-Decente-131101-02.pdf}
\newline {\color{gray} (Emb: 0.811, TF-IDF: 0.232)}


\item \textbf{Artículo} \newline
En los procesos en que intervengan niñas, niños y adolescentes, se deberá procurar el resguardo de su identidad. 
\newline {\color{gray} \textbf{1º:} 97-6-Iniciativa-del-cc-Daniel-Bravo-Sistemas-de-Justicia.pdf}
\newline {\color{gray} (Emb: 0.923, TF-IDF: 0.922)}
\newline {\color{gray} \textbf{2º:} 180-6-c-Iniciativa-Convencional-del-cc-Rodrigo-Álvarez-que-regula-el-Poder-Judicial-1044-hrs.pdf}
\newline {\color{gray} (Emb: 0.920, TF-IDF: 0.915)}

Los principios de probidad y de transparencia serán aplicables a todas las personas que ejercen jurisdicción en el país. 
\newline {\color{gray} \textbf{1º:} 32-2-Iniciativa-Convencional-Constituyente-del-cc-Martín-Arrau-y-otros.pdf}
\newline {\color{gray} (Emb: 0.733, TF-IDF: 0.434)}
\newline {\color{gray} \textbf{2º:} 179-6-c-Iniciativa-Convencional-DEL-CC-Rodrigo-Álvarez-Ministerio-Público-1044-hrs.pdf}
\newline {\color{gray} (Emb: 0.733, TF-IDF: 0.381)}

La ley establecerá las responsabilidades correspondientes en caso de infracción a esta disposición. 
\newline {\color{gray} \textbf{1º:} 234-1-Iniciativa-Convencional-del-cc-Jaime-Bassa-sobre-Justicia-Complementaria-1144-hrs.pdf}
\newline {\color{gray} (Emb: 0.750, TF-IDF: 0.601)}
\newline {\color{gray} \textbf{2º:} 414-5-Iniciativa-Convencional-del-cc-Bernardo-Fontaine-sobre-Estatuto-de-los-Minerales-1500-25-01.pdf}
\newline {\color{gray} (Emb: 0.744, TF-IDF: 0.601)}


\item \textbf{Artículo} \newline
La función jurisdiccional se basa en los principios rectores de la Justicia Abierta, que se manifiesta en la transparencia, participación y colaboración, con el fin de garantizar el Estado de Derecho, promover la paz social y fortalecer la democracia. 
\newline {\color{gray} \textbf{1º:} 369-4-Iniciativa-Convencional-Constituyente-de-la-cc-Loreto-Vallejos-sobre-Derecho-a-la-Educacion-0900-hrs-24-01.pdf}
\newline {\color{gray} (Emb: 0.562, TF-IDF: 0.399)}
\newline {\color{gray} \textbf{2º:} 622-Iniciativa-Convencional-Constituyente-de-cc-Felipe-Harboe-Reconocimiento-y-proteccion-integral-de-derechos-de-NNA.pdf}
\newline {\color{gray} (Emb: 0.545, TF-IDF: 0.377)}


\item \textbf{Artículo} \newline
El Estado garantiza que los nombramientos en el Sistema Nacional de Justicia respeten el principio de paridad en todos los órganos de la jurisdicción, incluyendo la designación de las presidencias. 
\newline {\color{gray} \textbf{1º:} 90-6-Iniciativa-Convencional-Constituyente-del-cc-Tomas-Laibe-y-otros.pdf}
\newline {\color{gray} (Emb: 0.987, TF-IDF: 0.951)}
\newline {\color{gray} \textbf{2º:} 198-6-c-Iniciativa-Convencional-de-la-cc-Ingrid-Villena-que-crea-el-Consejo-de-la-Justicia-1644-hrs.pdf}
\newline {\color{gray} (Emb: 0.966, TF-IDF: 0.788)}

Los tribunales, cualquiera sea su competencia, deben resolver con enfoque de género. 
\newline {\color{gray} \textbf{1º:} 655-Iniciativa-Convencional-Constituyente-de-la-cc-Jeniffer-Mella-sobre-Trabajo-y-Seguridad-Social-121101-02.pdf}
\newline {\color{gray} (Emb: 0.678, TF-IDF: 0.493)}
\newline {\color{gray} \textbf{2º:} 864-Iniciativa-Convencional-Constituyente-del-cc-Marcos-Barraza-sobre-RREE.pdf}
\newline {\color{gray} (Emb: 0.636, TF-IDF: 0.452)}

Todos los órganos y personas que intervienen en la función jurisdiccional deben garantizar la igualdad sustantiva. 
\newline {\color{gray} \textbf{1º:} 41-6-Iniciativa-Convencional-Constituyente-del-cc-Mauricio-Daza-y-otros-1.pdf}
\newline {\color{gray} (Emb: 1.000, TF-IDF: 1.000)}
\newline {\color{gray} \textbf{2º:} 232-6-Iniciativa-Convencional-del-cc-Marco-Arellano-que-Crea-el-Consejo-Nacional-de-Justicia-1144-hrs.pdf}
\newline {\color{gray} (Emb: 1.000, TF-IDF: 1.000)}

La función jurisdiccional se regirá por los principios de paridad y perspectiva de género. 
\newline {\color{gray} \textbf{1º:} 232-6-Iniciativa-Convencional-del-cc-Marco-Arellano-que-Crea-el-Consejo-Nacional-de-Justicia-1144-hrs.pdf}
\newline {\color{gray} (Emb: 1.000, TF-IDF: 1.000)}
\newline {\color{gray} \textbf{2º:} 41-6-Iniciativa-Convencional-Constituyente-del-cc-Mauricio-Daza-y-otros-1.pdf}
\newline {\color{gray} (Emb: 1.000, TF-IDF: 1.000)}


\item \textbf{Artículo} \newline
La función jurisdiccional se define en su estructura, integración y procedimientos conforme a los principios de plurinacionalidad, pluralismo jurídico e interculturalidad. 
\newline {\color{gray} \textbf{1º:} 610-Iniciativa-Convencional-Constituyente-de-cc-Valentina-Miranda-Derechos-de-las-Personas-LGBTIQ-y-Derecho-a-la-Igualdad.pdf}
\newline {\color{gray} (Emb: 0.704, TF-IDF: 0.422)}
\newline {\color{gray} \textbf{2º:} 147-4-c-Iniciativa-del-cc-Manuel-Jose-Ossandon-Garantias-Procesales-y-Seguridad-Individual.pdf}
\newline {\color{gray} (Emb: 0.653, TF-IDF: 0.417)}

Cuando se trate de personas indígenas, los tribunales y sus funcionarios deberán adoptar una perspectiva intercultural en el tratamiento y resolución de las materias de su competencia, tomando debidamente en consideración las costumbres, tradiciones, protocolos y los sistemas normativos de los pueblos indígenas, conforme a los tratados e instrumentos internacionales de derechos humanos de los que Chile es parte. 
\newline {\color{gray} \textbf{1º:} 319-6-Iniciativa-Convencional-del-cc-Mauricio-Daza-sobre-el-Sistema-Nacional-de-Justicia17-09-hrs.pdf}
\newline {\color{gray} (Emb: 0.653, TF-IDF: 0.366)}
\newline {\color{gray} \textbf{2º:} 319-6-Iniciativa-Convencional-del-cc-Mauricio-Daza-sobre-el-Sistema-Nacional-de-Justicia17-09-hrs.pdf}
\newline {\color{gray} (Emb: 0.650, TF-IDF: 0.362)}


\item \textbf{Artículo} \newline
Es deber del Estado promover e implementar mecanismos colaborativos de resolución de conflictos que garanticen la participación activa y el diálogo. 
\newline {\color{gray} \textbf{1º:} 41-6-Iniciativa-Convencional-Constituyente-del-cc-Mauricio-Daza-y-otros-1.pdf}
\newline {\color{gray} (Emb: 0.680, TF-IDF: 0.384)}
\newline {\color{gray} \textbf{2º:} 440-Iniciativa-Convencional-Constituyente-del-cc-Felipe-Harboe-sobre-Principios-del-Debido-Proceso-1401-28-01.pdf}
\newline {\color{gray} (Emb: 0.632, TF-IDF: 0.298)}

Sólo la ley podrá determinar los requisitos y efectos de los mecanismos alternativos de resolución de conflictos. 
\newline {\color{gray} \textbf{1º:} 41-6-Iniciativa-Convencional-Constituyente-del-cc-Mauricio-Daza-y-otros-1.pdf}
\newline {\color{gray} (Emb: 0.915, TF-IDF: 0.815)}
\newline {\color{gray} \textbf{2º:} 190-6-c-Iniciativa-Convencional-Constituyente-de-la-cc-Natividad-LLanquileo-sobre-Pluralismo-Jurídico-1240-hrs.pdf}
\newline {\color{gray} (Emb: 0.759, TF-IDF: 0.551)}


\item \textbf{Artículo} \newline
Los tribunales de justicia se estructuran conforme al principio de unidad jurisdiccional como base de su organización y funcionamiento, estando sujetos al mismo estatuto jurídico y principios. 
\newline {\color{gray} \textbf{1º:} 190-6-c-Iniciativa-Convencional-Constituyente-de-la-cc-Natividad-LLanquileo-sobre-Pluralismo-Jurídico-1240-hrs.pdf}
\newline {\color{gray} (Emb: 0.725, TF-IDF: 0.456)}
\newline {\color{gray} \textbf{2º:} 97-6-Iniciativa-del-cc-Daniel-Bravo-Sistemas-de-Justicia.pdf}
\newline {\color{gray} (Emb: 0.645, TF-IDF: 0.298)}


\item \textbf{Artículo} \newline
Las y los integrantes de los órganos jurisdiccionales, unipersonales o colegiados, se denominarán juezas o jueces. 
\newline {\color{gray} \textbf{1º:} 226-6-Iniciativa-Convencional-de-la-cc-Manuela-Royo-sobre-Justicia-Local-1140-hrs.pdf}
\newline {\color{gray} (Emb: 0.918, TF-IDF: 0.620)}
\newline {\color{gray} \textbf{2º:} 226-6-Iniciativa-Convencional-de-la-cc-Manuela-Royo-sobre-Justicia-Local-1140-hrs.pdf}
\newline {\color{gray} (Emb: 0.601, TF-IDF: 0.599)}

No existirá jerarquía entre quienes ejercen jurisdicción y sólo se diferenciarán por la función que desempeñen. 
\newline {\color{gray} \textbf{1º:} 41-6-Iniciativa-Convencional-Constituyente-del-cc-Mauricio-Daza-y-otros-1.pdf}
\newline {\color{gray} (Emb: 0.764, TF-IDF: 0.414)}
\newline {\color{gray} \textbf{2º:} 916-Iniciativa-Convencional-Constituyente-del-cc-Luis-Jimenez-sobre-Pueblo-Tribal-Afrodescendiente.pdf}
\newline {\color{gray} (Emb: 0.675, TF-IDF: 0.400)}

Las juezas o jueces no recibirán tratamiento honorífico alguno. 
\newline {\color{gray} \textbf{1º:} 90-6-Iniciativa-Convencional-Constituyente-del-cc-Tomas-Laibe-y-otros.pdf}
\newline {\color{gray} (Emb: 0.859, TF-IDF: 0.495)}
\newline {\color{gray} \textbf{2º:} 97-6-Iniciativa-del-cc-Daniel-Bravo-Sistemas-de-Justicia.pdf}
\newline {\color{gray} (Emb: 0.727, TF-IDF: 0.419)}

Sólo la ley podrá establecer cargos de jueces y juezas. 
\newline {\color{gray} \textbf{1º:} 220-6-c-Iniciativa-Convencional-del-cc-Daniel-Bravo-sobre-organizacion-de-tribunales-2315-hrs.pdf}
\newline {\color{gray} (Emb: 0.742, TF-IDF: 0.608)}
\newline {\color{gray} \textbf{2º:} 88-6-Iniciativa-Convencional-Constituyente-del-cc-Christian-Viera-y-otros.pdf}
\newline {\color{gray} (Emb: 0.718, TF-IDF: 0.402)}

La Corte Suprema y las Cortes de Apelaciones no podrán ser integradas por personas que no tengan la calidad de juezas o jueces titulares, interinos, suplentes o subrogantes. 
\newline {\color{gray} \textbf{1º:} 90-6-Iniciativa-Convencional-Constituyente-del-cc-Tomas-Laibe-y-otros.pdf}
\newline {\color{gray} (Emb: 0.672, TF-IDF: 0.287)}
\newline {\color{gray} \textbf{2º:} 97-6-Iniciativa-del-cc-Daniel-Bravo-Sistemas-de-Justicia.pdf}
\newline {\color{gray} (Emb: 0.663, TF-IDF: 0.282)}

La planta de personal y organización administrativa interna de los tribunales será establecida por la ley. 
\newline {\color{gray} \textbf{1º:} 90-6-Iniciativa-Convencional-Constituyente-del-cc-Tomas-Laibe-y-otros.pdf}
\newline {\color{gray} (Emb: 0.723, TF-IDF: 0.504)}
\newline {\color{gray} \textbf{2º:} 128-4-c-Iniciativa-de-la-cc-Rocio-Cantuarias-Incorpora-una-Garantia-Procesal-en-los-procesos-judiciales.pdf}
\newline {\color{gray} (Emb: 0.696, TF-IDF: 0.428)}


\item \textbf{Artículo} \newline
Las juezas y jueces cesan en sus cargos por alcanzar los setenta años de edad, por renuncia, por constatarse una incapacidad legal sobreviniente o por remoción. 
\newline {\color{gray} \textbf{1º:} 97-6-Iniciativa-del-cc-Daniel-Bravo-Sistemas-de-Justicia.pdf}
\newline {\color{gray} (Emb: 0.926, TF-IDF: 0.700)}
\newline {\color{gray} \textbf{2º:} 90-6-Iniciativa-Convencional-Constituyente-del-cc-Tomas-Laibe-y-otros.pdf}
\newline {\color{gray} (Emb: 0.748, TF-IDF: 0.536)}


\item \textbf{Artículo} \newline
Encontrándose firme la resolución que acoge la querella, el procedimiento penal continuará de acuerdo a las reglas generales y la jueza o el juez quedará suspendido del ejercicio de sus funciones. 
\newline {\color{gray} \textbf{1º:} 41-6-Iniciativa-Convencional-Constituyente-del-cc-Mauricio-Daza-y-otros-1.pdf}
\newline {\color{gray} (Emb: 0.842, TF-IDF: 0.710)}
\newline {\color{gray} \textbf{2º:} 88-6-Iniciativa-Convencional-Constituyente-del-cc-Christian-Viera-y-otros.pdf}
\newline {\color{gray} (Emb: 0.793, TF-IDF: 0.512)}

Las juezas y los jueces no podrán ser acusados o privados de libertad, salvo el caso de delito flagrante, si la Corte de Apelaciones correspondiente no declara admisible uno o más capítulos de la acusación respectiva. 
\newline {\color{gray} \textbf{1º:} 97-6-Iniciativa-del-cc-Daniel-Bravo-Sistemas-de-Justicia.pdf}
\newline {\color{gray} (Emb: 0.716, TF-IDF: 0.499)}
\newline {\color{gray} \textbf{2º:} 180-6-c-Iniciativa-Convencional-del-cc-Rodrigo-Álvarez-que-regula-el-Poder-Judicial-1044-hrs.pdf}
\newline {\color{gray} (Emb: 0.665, TF-IDF: 0.460)}

La resolución que se pronuncie sobre la querella de capítulos será apelable para ante la Corte Suprema. 
\newline {\color{gray} \textbf{1º:} 220-6-c-Iniciativa-Convencional-del-cc-Daniel-Bravo-sobre-organizacion-de-tribunales-2315-hrs.pdf}
\newline {\color{gray} (Emb: 1.000, TF-IDF: 1.000)}
\newline {\color{gray} \textbf{2º:} 319-6-Iniciativa-Convencional-del-cc-Mauricio-Daza-sobre-el-Sistema-Nacional-de-Justicia17-09-hrs.pdf}
\newline {\color{gray} (Emb: 0.729, TF-IDF: 0.434)}


\item \textbf{Artículo} \newline
El Sistema Nacional de Justicia gozará de autonomía financiera. 
\newline {\color{gray} \textbf{1º:} 440-Iniciativa-Convencional-Constituyente-del-cc-Felipe-Harboe-sobre-Principios-del-Debido-Proceso-1401-28-01.pdf}
\newline {\color{gray} (Emb: 0.737, TF-IDF: 0.295)}
\newline {\color{gray} \textbf{2º:} 514-4-Iniciativa-Convencional-Constituyente-del-cc-Felipe-Harboe-sobre-Derecho-a-la-Privacidad-1245-01-02.pdf}
\newline {\color{gray} (Emb: 0.737, TF-IDF: 0.265)}

Anualmente, se destinarán en la Ley de Presupuestos del Estado los fondos necesarios para su adecuado funcionamiento. 
\newline {\color{gray} \textbf{1º:} 143-4-c-Iniciativa-de-la-cc-Rocio-Cantuarias-Incorpora-Libertad-de-Trabajo-y-Sindical.pdf}
\newline {\color{gray} (Emb: 0.769, TF-IDF: 0.404)}
\newline {\color{gray} \textbf{2º:} 579-Iniciativa-Convencional-Constituyente-de-cc-Christian-Viera-sobre-Sistema-Electoral-y-Justicia-Electoral-2330-hrs.pdf}
\newline {\color{gray} (Emb: 0.759, TF-IDF: 0.404)}


\item \textbf{Artículo} \newline
Todas las etapas de los procedimientos y las resoluciones judiciales son públicas. 
\newline {\color{gray} \textbf{1º:} 41-6-Iniciativa-Convencional-Constituyente-del-cc-Mauricio-Daza-y-otros-1.pdf}
\newline {\color{gray} (Emb: 0.722, TF-IDF: 0.258)}
\newline {\color{gray} \textbf{2º:} 98-6-Iniciativa-del-cc-Ruggero-Cozzi-Funcion-y-Principios-de-la-Jurisdiccion.pdf}
\newline {\color{gray} (Emb: 0.707, TF-IDF: 0.236)}

Excepcionalmente, la ley podrá establecer su reserva o secreto en casos calificados. 
\newline {\color{gray} \textbf{1º:} 909-Iniciativa-Convencional-Constituyente-del-cc-Hugo-Gutierrez-Sobre-Ministerio-Publico.pdf}
\newline {\color{gray} (Emb: 0.665, TF-IDF: 0.533)}
\newline {\color{gray} \textbf{2º:} 41-6-Iniciativa-Convencional-Constituyente-del-cc-Mauricio-Daza-y-otros-1.pdf}
\newline {\color{gray} (Emb: 0.630, TF-IDF: 0.392)}


\item \textbf{Artículo} \newline
Los tribunales, con la finalidad de garantizar el acceso a la justicia y a la tutela judicial efectiva, podrán funcionar en localidades situadas fuera de su lugar de asiento, siempre dentro de su territorio jurisdiccional. 
\newline {\color{gray} \textbf{1º:} 98-6-Iniciativa-del-cc-Ruggero-Cozzi-Funcion-y-Principios-de-la-Jurisdiccion.pdf}
\newline {\color{gray} (Emb: 0.780, TF-IDF: 0.771)}
\newline {\color{gray} \textbf{2º:} 353-1-Iniciativa-Convencional-Constituyente-del-cc-Jaime-Bassa-sobre-Sistema-de-Gobierno-1158-21-01.pdf}
\newline {\color{gray} (Emb: 0.703, TF-IDF: 0.707)}


\item \textbf{Artículo} \newline
El Sistema Nacional de Justicia está integrado por la justicia vecinal, los tribunales de instancia, las Cortes de Apelaciones y la Corte Suprema. 
\newline {\color{gray} \textbf{1º:} 232-6-Iniciativa-Convencional-del-cc-Marco-Arellano-que-Crea-el-Consejo-Nacional-de-Justicia-1144-hrs.pdf}
\newline {\color{gray} (Emb: 0.967, TF-IDF: 0.900)}
\newline {\color{gray} \textbf{2º:} 41-6-Iniciativa-Convencional-Constituyente-del-cc-Mauricio-Daza-y-otros-1.pdf}
\newline {\color{gray} (Emb: 0.966, TF-IDF: 0.885)}

Todos los tribunales estarán sometidos, a lo menos cada cinco años, a una revisión integral de la gestión por el Consejo de la Justicia, que incluirá audiencias públicas, para determinar su correcto funcionamiento, en conformidad a lo establecido en la Constitución y la ley. 
\newline {\color{gray} \textbf{1º:} 238-1-Iniciativa-Convencional-de-la-cc-Tania-Madriaga-sobre-Buen-Gobierno-1146-hrs.pdf}
\newline {\color{gray} (Emb: 0.914, TF-IDF: 0.796)}
\newline {\color{gray} \textbf{2º:} 319-6-Iniciativa-Convencional-del-cc-Mauricio-Daza-sobre-el-Sistema-Nacional-de-Justicia17-09-hrs.pdf}
\newline {\color{gray} (Emb: 0.659, TF-IDF: 0.496)}

Esta revisión, en ningún caso, incluirá las resoluciones judiciales. 
\newline {\color{gray} \textbf{1º:} 220-6-c-Iniciativa-Convencional-del-cc-Daniel-Bravo-sobre-organizacion-de-tribunales-2315-hrs.pdf}
\newline {\color{gray} (Emb: 0.695, TF-IDF: 0.557)}
\newline {\color{gray} \textbf{2º:} 220-6-c-Iniciativa-Convencional-del-cc-Daniel-Bravo-sobre-organizacion-de-tribunales-2315-hrs.pdf}
\newline {\color{gray} (Emb: 0.643, TF-IDF: 0.368)}


\item \textbf{Artículo} \newline
Es deber del Estado garantizar que los órganos que intervienen en el proceso respeten y promuevan el derecho a acceder a una justicia con perspectiva intercultural. 
\newline {\color{gray} \textbf{1º:} 41-6-Iniciativa-Convencional-Constituyente-del-cc-Mauricio-Daza-y-otros-1.pdf}
\newline {\color{gray} (Emb: 0.828, TF-IDF: 0.502)}
\newline {\color{gray} \textbf{2º:} 41-6-Iniciativa-Convencional-Constituyente-del-cc-Mauricio-Daza-y-otros-1.pdf}
\newline {\color{gray} (Emb: 0.771, TF-IDF: 0.479)}

Las personas tienen derecho a una asistencia jurídica especializada, intérpretes, facilitadores interculturales y peritajes consultivos, cuando así lo requieran y no puedan proveérselas por sí mismas. 
\newline {\color{gray} \textbf{1º:} 90-6-Iniciativa-Convencional-Constituyente-del-cc-Tomas-Laibe-y-otros.pdf}
\newline {\color{gray} (Emb: 0.710, TF-IDF: 0.315)}
\newline {\color{gray} \textbf{2º:} 465-6-Iniciativa-Convencional-Constituyente-de-la-cc-Adriana-Cancino-sobre-Justicia-Electoral-1925-31-01.pdf}
\newline {\color{gray} (Emb: 0.710, TF-IDF: 0.291)}


\item \textbf{Artículo} \newline
Quien ejerza la Presidencia no podrá integrar alguna de las salas. 
\newline {\color{gray} \textbf{1º:} 730-Iniciativa-Convencional-Constituyente-del-cc-Adolfo-Millabur-sobre-Gobiernos-Locales.pdf}
\newline {\color{gray} (Emb: 0.693, TF-IDF: 0.553)}
\newline {\color{gray} \textbf{2º:} 513-3-Iniciativa-Convencional-Constituyente-del-cc-Adolfo-Millabur-sobre-gobiernos-locales-1115-01-02.pdf}
\newline {\color{gray} (Emb: 0.693, TF-IDF: 0.546)}

Durará en sus funciones dos años sin posibilidad de ejercer nuevamente el cargo. 
\newline {\color{gray} \textbf{1º:} 220-6-c-Iniciativa-Convencional-del-cc-Daniel-Bravo-sobre-organizacion-de-tribunales-2315-hrs.pdf}
\newline {\color{gray} (Emb: 0.710, TF-IDF: 0.579)}
\newline {\color{gray} \textbf{2º:} 319-6-Iniciativa-Convencional-del-cc-Mauricio-Daza-sobre-el-Sistema-Nacional-de-Justicia17-09-hrs.pdf}
\newline {\color{gray} (Emb: 0.707, TF-IDF: 0.577)}

La presidencia de la Corte Suprema será ejercida por una persona elegida por sus pares. 
\newline {\color{gray} \textbf{1º:} 90-6-Iniciativa-Convencional-Constituyente-del-cc-Tomas-Laibe-y-otros.pdf}
\newline {\color{gray} (Emb: 0.788, TF-IDF: 0.644)}
\newline {\color{gray} \textbf{2º:} 319-6-Iniciativa-Convencional-del-cc-Mauricio-Daza-sobre-el-Sistema-Nacional-de-Justicia17-09-hrs.pdf}
\newline {\color{gray} (Emb: 0.786, TF-IDF: 0.480)}

Se compondrá de veintiún juezas y jueces y funcionará en pleno o salas especializadas. 
\newline {\color{gray} \textbf{1º:} 190-6-c-Iniciativa-Convencional-Constituyente-de-la-cc-Natividad-LLanquileo-sobre-Pluralismo-Jurídico-1240-hrs.pdf}
\newline {\color{gray} (Emb: 0.938, TF-IDF: 0.916)}
\newline {\color{gray} \textbf{2º:} 842-Iniciativa-Convencional-Constituyente-de-la-cc-Francisca-Linconao-sobre-Derecho-a-la-Justicia-Intercultural.pdf}
\newline {\color{gray} (Emb: 0.899, TF-IDF: 0.746)}

La Corte Suprema es un órgano colegiado con jurisdicción en todo el país, que tiene como función velar por la correcta aplicación del derecho y uniformar su interpretación, así como las demás atribuciones que establezca esta Constitución y la ley. 
\newline {\color{gray} \textbf{1º:} 325-6-Iniciativa-Convencional-del-cc-Tomas-Laibe-sobre-Jurisdiccion-Constitucional.pdf}
\newline {\color{gray} (Emb: 0.807, TF-IDF: 0.398)}
\newline {\color{gray} \textbf{2º:} 759-Iniciativa-Convencional-Constituyente-de-la-cc-Paulina-Veloso-sobre-Democracia-Directa.pdf}
\newline {\color{gray} (Emb: 0.654, TF-IDF: 0.368)}

Sus juezas y jueces durarán en sus cargos un máximo de catorce años, sin posibilidad de reelección. 
\newline {\color{gray} \textbf{1º:} 842-Iniciativa-Convencional-Constituyente-de-la-cc-Francisca-Linconao-sobre-Derecho-a-la-Justicia-Intercultural.pdf}
\newline {\color{gray} (Emb: 0.812, TF-IDF: 0.797)}
\newline {\color{gray} \textbf{2º:} 190-6-c-Iniciativa-Convencional-Constituyente-de-la-cc-Natividad-LLanquileo-sobre-Pluralismo-Jurídico-1240-hrs.pdf}
\newline {\color{gray} (Emb: 0.812, TF-IDF: 0.797)}


\item \textbf{Artículo} \newline
Las Cortes de Apelaciones son órganos colegiados con jurisdicción sobre una región o parte de ella, cuya función principal es resolver las impugnaciones que se interpongan contra resoluciones de los tribunales de instancia, así como las demás competencias que establezca la Constitución y la ley. 
\newline {\color{gray} \textbf{1º:} 95-6-Iniciativa-Convencional-Constituyente-de-Cc-Mauricio-Daza-y-otros-2.pdf}
\newline {\color{gray} (Emb: 0.723, TF-IDF: 0.373)}
\newline {\color{gray} \textbf{2º:} 353-1-Iniciativa-Convencional-Constituyente-del-cc-Jaime-Bassa-sobre-Sistema-de-Gobierno-1158-21-01.pdf}
\newline {\color{gray} (Emb: 0.708, TF-IDF: 0.372)}

Funcionarán en pleno o en salas preferentemente especializadas. 
\newline {\color{gray} \textbf{1º:} 918-Iniciativa-Convencional-Constituyente-del-cc-Mauricio-Daza-sobre-Disposiciones-Transitorias.pdf}
\newline {\color{gray} (Emb: 0.740, TF-IDF: 0.477)}
\newline {\color{gray} \textbf{2º:} 785-Iniciativa-Convencional-Constituyente-de-la-cc-Patricia-Labra-sobre-Defensor-del-Usuario.pdf}
\newline {\color{gray} (Emb: 0.732, TF-IDF: 0.421)}

La presidencia de cada Corte de Apelaciones será ejercida por una persona elegida por sus pares. 
\newline {\color{gray} \textbf{1º:} 319-6-Iniciativa-Convencional-del-cc-Mauricio-Daza-sobre-el-Sistema-Nacional-de-Justicia17-09-hrs.pdf}
\newline {\color{gray} (Emb: 1.000, TF-IDF: 1.000)}
\newline {\color{gray} \textbf{2º:} 319-6-Iniciativa-Convencional-del-cc-Mauricio-Daza-sobre-el-Sistema-Nacional-de-Justicia17-09-hrs.pdf}
\newline {\color{gray} (Emb: 0.962, TF-IDF: 0.833)}

Durará en sus funciones dos años. 
\newline {\color{gray} \textbf{1º:} 319-6-Iniciativa-Convencional-del-cc-Mauricio-Daza-sobre-el-Sistema-Nacional-de-Justicia17-09-hrs.pdf}
\newline {\color{gray} (Emb: 0.850, TF-IDF: 0.525)}
\newline {\color{gray} \textbf{2º:} 41-6-Iniciativa-Convencional-Constituyente-del-cc-Mauricio-Daza-y-otros-1.pdf}
\newline {\color{gray} (Emb: 0.682, TF-IDF: 0.368)}


\item \textbf{Artículo} \newline
Son tribunales de instancia los civiles, penales, de ejecución de penas, de familia, laborales, administrativos, ambientales, de competencia común o mixtos, vecinales y demás que establezca la ley. 
\newline {\color{gray} \textbf{1º:} 319-6-Iniciativa-Convencional-del-cc-Mauricio-Daza-sobre-el-Sistema-Nacional-de-Justicia17-09-hrs.pdf}
\newline {\color{gray} (Emb: 0.651, TF-IDF: 0.568)}
\newline {\color{gray} \textbf{2º:} 319-6-Iniciativa-Convencional-del-cc-Mauricio-Daza-sobre-el-Sistema-Nacional-de-Justicia17-09-hrs.pdf}
\newline {\color{gray} (Emb: 0.560, TF-IDF: 0.531)}

La competencia de estos tribunales y el número de juezas o jueces que los integrarán serán determinados por la ley. 
\newline {\color{gray} \textbf{1º:} 319-6-Iniciativa-Convencional-del-cc-Mauricio-Daza-sobre-el-Sistema-Nacional-de-Justicia17-09-hrs.pdf}
\newline {\color{gray} (Emb: 0.703, TF-IDF: 0.347)}
\newline {\color{gray} \textbf{2º:} 95-6-Iniciativa-Convencional-Constituyente-de-Cc-Mauricio-Daza-y-otros-2.pdf}
\newline {\color{gray} (Emb: 0.684, TF-IDF: 0.308)}


\item \textbf{Artículo} \newline
La ley establecerá un procedimiento unificado, simple y expedito para conocer y resolver tales asuntos. 
\newline {\color{gray} \textbf{1º:} 88-6-Iniciativa-Convencional-Constituyente-del-cc-Christian-Viera-y-otros.pdf}
\newline {\color{gray} (Emb: 0.749, TF-IDF: 0.451)}
\newline {\color{gray} \textbf{2º:} 90-6-Iniciativa-Convencional-Constituyente-del-cc-Tomas-Laibe-y-otros.pdf}
\newline {\color{gray} (Emb: 0.725, TF-IDF: 0.429)}

Los asuntos de competencia de estos tribunales no podrán ser sometidos a arbitraje. 
\newline {\color{gray} \textbf{1º:} 220-6-c-Iniciativa-Convencional-del-cc-Daniel-Bravo-sobre-organizacion-de-tribunales-2315-hrs.pdf}
\newline {\color{gray} (Emb: 1.000, TF-IDF: 0.811)}
\newline {\color{gray} \textbf{2º:} 90-6-Iniciativa-Convencional-Constituyente-del-cc-Tomas-Laibe-y-otros.pdf}
\newline {\color{gray} (Emb: 0.733, TF-IDF: 0.393)}

Habrá al menos un Tribunal Administrativo en cada región del país, los que podrán funcionar en salas especializadas. 
\newline {\color{gray} \textbf{1º:} 220-6-c-Iniciativa-Convencional-del-cc-Daniel-Bravo-sobre-organizacion-de-tribunales-2315-hrs.pdf}
\newline {\color{gray} (Emb: 0.840, TF-IDF: 0.717)}
\newline {\color{gray} \textbf{2º:} 319-6-Iniciativa-Convencional-del-cc-Mauricio-Daza-sobre-el-Sistema-Nacional-de-Justicia17-09-hrs.pdf}
\newline {\color{gray} (Emb: 0.656, TF-IDF: 0.396)}

Los Tribunales Administrativos conocen y resuelven las acciones dirigidas en contra de la Administración del Estado o promovidas por ésta y de las demás materias que establezca la ley. 
\newline {\color{gray} \textbf{1º:} 918-Iniciativa-Convencional-Constituyente-del-cc-Mauricio-Daza-sobre-Disposiciones-Transitorias.pdf}
\newline {\color{gray} (Emb: 0.793, TF-IDF: 0.461)}
\newline {\color{gray} \textbf{2º:} 390-5-Iniciativa-Convencional-Constituyente-de-la-cc-Isabel-Godoy-sobre-Estatuto-Constitucional-del-Agua-1431-24-01.pdf}
\newline {\color{gray} (Emb: 0.621, TF-IDF: 0.430)}


\item \textbf{Artículo} \newline
Sólo el Estado puede ejecutar el cumplimiento de penas y medidas privativas de libertad, a través de instituciones públicas especialmente establecidas para estos fines. 
\newline {\color{gray} \textbf{1º:} 88-6-Iniciativa-Convencional-Constituyente-del-cc-Christian-Viera-y-otros.pdf}
\newline {\color{gray} (Emb: 0.900, TF-IDF: 0.721)}
\newline {\color{gray} \textbf{2º:} 220-6-c-Iniciativa-Convencional-del-cc-Daniel-Bravo-sobre-organizacion-de-tribunales-2315-hrs.pdf}
\newline {\color{gray} (Emb: 0.756, TF-IDF: 0.584)}

La función establecida en este artículo no podrá ser ejercida por privados. 
\newline {\color{gray} \textbf{1º:} 88-6-Iniciativa-Convencional-Constituyente-del-cc-Christian-Viera-y-otros.pdf}
\newline {\color{gray} (Emb: 0.846, TF-IDF: 0.667)}
\newline {\color{gray} \textbf{2º:} 759-Iniciativa-Convencional-Constituyente-de-la-cc-Paulina-Veloso-sobre-Democracia-Directa.pdf}
\newline {\color{gray} (Emb: 0.775, TF-IDF: 0.483)}

Para la inserción, integración y reparación de las personas privadas de libertad, los establecimientos penitenciarios deben contar con espacios para el estudio, trabajo, deporte, las artes y culturas. 
\newline {\color{gray} \textbf{1º:} 803-Iniciativa-Convencional-Constituyente-del-cc-Christian-Viera-sobre-Accion-de-Proteccion.pdf}
\newline {\color{gray} (Emb: 0.714, TF-IDF: 0.445)}
\newline {\color{gray} \textbf{2º:} 179-6-c-Iniciativa-Convencional-DEL-CC-Rodrigo-Álvarez-Ministerio-Público-1044-hrs.pdf}
\newline {\color{gray} (Emb: 0.680, TF-IDF: 0.324)}

En el caso de mujeres embarazadas y madres de lactantes, el Estado adoptará las medidas necesarias tales como infraestructura y equipamiento tanto en el régimen de control cerrado, abierto y post penitenciario. 
\newline {\color{gray} \textbf{1º:} 1031-Iniciativa-Convencional-Constituyente-del-cc-Tomas-Laibe-sobre-Personas-Privadas-de-Libertad.pdf}
\newline {\color{gray} (Emb: 0.939, TF-IDF: 0.865)}
\newline {\color{gray} \textbf{2º:} 533-Iniciativa-Convencional-Constituyente-del-cc-Felipe-Harboe-sobre-D°-a-la-Libertad-1625-hrs.-01-02-1.pdf}
\newline {\color{gray} (Emb: 0.663, TF-IDF: 0.346)}


\item \textbf{Artículo} \newline
El sistema de cumplimiento de las sanciones penales y de las medidas de seguridad se organizará sobre la base del respeto a los derechos humanos y tendrá como objetivos el cumplimiento de la pena y la integración e inserción social de la persona que cumpla una condena judicial. 
\newline {\color{gray} \textbf{1º:} 1031-Iniciativa-Convencional-Constituyente-del-cc-Tomas-Laibe-sobre-Personas-Privadas-de-Libertad.pdf}
\newline {\color{gray} (Emb: 0.893, TF-IDF: 0.801)}
\newline {\color{gray} \textbf{2º:} 594-Iniciativa-Convencional-Constituyente-de-cc-Tammy-Pustilnick-sobre-Custodia-Publica-de-la-Naturaleza.pdf}
\newline {\color{gray} (Emb: 0.582, TF-IDF: 0.317)}

Es deber del Estado, en su especial posición de garante frente a las personas privadas de libertad, velar por la protección y ejercicio efectivo de sus derechos fundamentales consagrados en esta Constitución y en los tratados e instrumentos internacionales sobre derechos humanos. 
\newline {\color{gray} \textbf{1º:} 611-Iniciativa-Convencional-Constituyente-de-cc-Valentina-Miranda-sobre-Derechos-de-las-Personas-Privadas-de-Libertad.pdf}
\newline {\color{gray} (Emb: 0.608, TF-IDF: 0.322)}
\newline {\color{gray} \textbf{2º:} 380-4-Iniciativa-Convencional-Constituyente-del-cc-Bastian-Labbe-sobre-el-Derecho-al-Trabajo1141-24-01.pdf}
\newline {\color{gray} (Emb: 0.522, TF-IDF: 0.268)}


\item \textbf{Artículo} \newline
Habrá tribunales de ejecución de penas que velarán por los derechos fundamentales de las personas condenadas o sujetas a medidas de seguridad, conforme a lo reconocido en esta Constitución y los tratados e instrumentos internacionales de derechos humanos, procurando su integración e inserción social. 
\newline {\color{gray} \textbf{1º:} 611-Iniciativa-Convencional-Constituyente-de-cc-Valentina-Miranda-sobre-Derechos-de-las-Personas-Privadas-de-Libertad.pdf}
\newline {\color{gray} (Emb: 0.491, TF-IDF: 0.239)}
\newline {\color{gray} \textbf{2º:} 354-4-Iniciativa-Convencional-Constituyente-del-cc-Miguel-Angel-Botto-sobre-Derecho-a-la-Maternidad-1120-21-01.pdf}
\newline {\color{gray} (Emb: 0.486, TF-IDF: 0.236)}

Ejercerán funciones jurisdiccionales en materia de ejecución de penas y medidas de seguridad, control jurisdiccional de la potestad disciplinaria de las autoridades penitenciarias, protección de los derechos y beneficios de los internos en los establecimientos penitenciarios y demás que señale la ley. 
\newline {\color{gray} \textbf{1º:} 1003-Iniciativa-Convencional-Constituyente-del-cc-Luis-Mayol-sobre-Derecho-de-las-Personas-Privadas-de-Libertad.pdf}
\newline {\color{gray} (Emb: 0.894, TF-IDF: 0.774)}
\newline {\color{gray} \textbf{2º:} 303-4-Iniciativa-Convencional-del-cc-Cesar-Valenzuela-sobre-Derecho-a-Vivir-en-un-entorno-seguro-1301-hrs.pdf}
\newline {\color{gray} (Emb: 0.668, TF-IDF: 0.309)}


\item \textbf{Artículo} \newline
La justicia vecinal se compone por los juzgados vecinales y los centros de justicia vecinal. 
\newline {\color{gray} \textbf{1º:} 1031-Iniciativa-Convencional-Constituyente-del-cc-Tomas-Laibe-sobre-Personas-Privadas-de-Libertad.pdf}
\newline {\color{gray} (Emb: 0.940, TF-IDF: 0.917)}
\newline {\color{gray} \textbf{2º:} 104-4-c-Iniciativa-del-cc-Alfredo-Moreno-Clausula-de-obligaciones-generales.pdf}
\newline {\color{gray} (Emb: 0.727, TF-IDF: 0.413)}

En cada comuna del país que sea asiento de una municipalidad habrá, a lo menos, un juzgado vecinal que ejerce la función jurisdiccional respecto de todas aquellas controversias jurídicas que se susciten a nivel comunal que no sean competencia de otro tribunal y de los demás asuntos que la ley les encomiende, conforme a un procedimiento breve, oral, simple y expedito. 
\newline {\color{gray} \textbf{1º:} 378-2-Iniciativa-Convencional-Constituyente-del-cc-Alvin-Saldana-sobre-Integracion-de-DDHH-1046-hrs-24-01.pdf}
\newline {\color{gray} (Emb: 0.684, TF-IDF: 0.351)}
\newline {\color{gray} \textbf{2º:} 11-4-Iniciativa-Convencional-Constituyente-de-la-cc-María-Elisa-Quinteros-y-otras.pdf}
\newline {\color{gray} (Emb: 0.672, TF-IDF: 0.346)}


\item \textbf{Artículo} \newline
La organización, atribuciones, materias y procedimientos que correspondan a los centros de justicia vecinal se regirán por la ley respectiva. 
\newline {\color{gray} \textbf{1º:} 90-6-Iniciativa-Convencional-Constituyente-del-cc-Tomas-Laibe-y-otros.pdf}
\newline {\color{gray} (Emb: 0.772, TF-IDF: 0.503)}
\newline {\color{gray} \textbf{2º:} 220-6-c-Iniciativa-Convencional-del-cc-Daniel-Bravo-sobre-organizacion-de-tribunales-2315-hrs.pdf}
\newline {\color{gray} (Emb: 0.708, TF-IDF: 0.480)}

Los centros de justicia vecinal son órganos encargados de promover la solución de conflictos vecinales y de pequeña cuantía dentro de una comunidad determinada por ley, en base al diálogo social, la paz y la participación de las partes involucradas, debiendo priorizar su instalación en zonas rurales y lugares alejados de áreas urbanas. 
\newline {\color{gray} \textbf{1º:} 220-6-c-Iniciativa-Convencional-del-cc-Daniel-Bravo-sobre-organizacion-de-tribunales-2315-hrs.pdf}
\newline {\color{gray} (Emb: 0.789, TF-IDF: 0.715)}
\newline {\color{gray} \textbf{2º:} 959-1-Iniciativa-Convencional-Constituyente-de-la-cc-Rosa-Catrileo-sobre-Defensa-Plurinacional-1.pdf}
\newline {\color{gray} (Emb: 0.649, TF-IDF: 0.302)}

Los centros de justicia vecinal deberán orientar e informar al público en materias jurídicas, haciendo las derivaciones que fuesen necesarias, así como ejercer las demás funciones que la ley les encomiende. 
\newline {\color{gray} \textbf{1º:} 472-6-Iniciativa-Convencional-Constituyente-del-cc-Daniel-Bravo-sobre-Corte-Constitucional-2003-31-01.pdf}
\newline {\color{gray} (Emb: 0.634, TF-IDF: 0.437)}
\newline {\color{gray} \textbf{2º:} 220-6-c-Iniciativa-Convencional-del-cc-Daniel-Bravo-sobre-organizacion-de-tribunales-2315-hrs.pdf}
\newline {\color{gray} (Emb: 0.626, TF-IDF: 0.400)}


\item \textbf{Artículo} \newline
El Sistema de Justicia deberá adoptar todas las medidas para prevenir, sancionar y erradicar la violencia contra mujeres, disidencias y diversidades sexo genéricas, en todas sus manifestaciones y ámbitos. 
\newline {\color{gray} \textbf{1º:} 220-6-c-Iniciativa-Convencional-del-cc-Daniel-Bravo-sobre-organizacion-de-tribunales-2315-hrs.pdf}
\newline {\color{gray} (Emb: 0.657, TF-IDF: 0.502)}
\newline {\color{gray} \textbf{2º:} 157-3-c-Iniciativa-de-la-cc-Tammy-Pustilnick-Disposiciones-Generales-del-Estado.pdf}
\newline {\color{gray} (Emb: 0.553, TF-IDF: 0.209)}

El Consejo de la Justicia deberá asegurar la formación inicial y capacitación constante de la totalidad de funcionarias y funcionarios y auxiliares de la administración de justicia, con el fin de eliminar estereotipos de género y garantizar la incorporación de la perspectiva de género, el enfoque interseccional y de derechos humanos, sin discriminación en la administración de justicia. 
\newline {\color{gray} \textbf{1º:} 319-6-Iniciativa-Convencional-del-cc-Mauricio-Daza-sobre-el-Sistema-Nacional-de-Justicia17-09-hrs.pdf}
\newline {\color{gray} (Emb: 0.594, TF-IDF: 0.397)}
\newline {\color{gray} \textbf{2º:} 559-Iniciativa-Convencional-Constituyente-de-cc-Andres-Cruz-sobre-Defensoria-Penal-Publica-2017-hrs.-01-02.pdf}
\newline {\color{gray} (Emb: 0.574, TF-IDF: 0.338)}


\item \textbf{Artículo} \newline
La función jurisdiccional debe ejercerse bajo un enfoque interseccional, debiendo garantizar la igualdad sustantiva y el cumplimiento de las obligaciones internacionales de derechos humanos en la materia. 
\newline {\color{gray} \textbf{1º:} 220-6-c-Iniciativa-Convencional-del-cc-Daniel-Bravo-sobre-organizacion-de-tribunales-2315-hrs.pdf}
\newline {\color{gray} (Emb: 0.834, TF-IDF: 0.662)}
\newline {\color{gray} \textbf{2º:} 400-1-Iniciativa-Convencional-Constituyente-de-la-cc-Constanza-Hube-sobre-Servicio-y-Registro-Electoral-1905-24-01.pdf}
\newline {\color{gray} (Emb: 0.688, TF-IDF: 0.378)}

Este deber resulta extensivo a todo órgano jurisdiccional y auxiliar, funcionarias y funcionarios del Sistema de Justicia, durante todo el curso del proceso y en todas las actuaciones que realicen. 
\newline {\color{gray} \textbf{1º:} 774-Iniciativa-Convencional-Constituyente-de-la-cc-Barbara-Rebolledo-sobre-Derechos-de-las-Mujeres.pdf}
\newline {\color{gray} (Emb: 0.681, TF-IDF: 0.483)}
\newline {\color{gray} \textbf{2º:} 774-Iniciativa-Convencional-Constituyente-de-la-cc-Barbara-Rebolledo-sobre-Derechos-de-las-Mujeres.pdf}
\newline {\color{gray} (Emb: 0.638, TF-IDF: 0.448)}

Asimismo, los tribunales, cualquiera sea su competencia. 
\newline {\color{gray} \textbf{1º:} 539-Iniciativa-Convencional-Constituyente-de-la-cc-Barbara-Sepulveda-sobre-administracion-de-justicia-1510-01-02-1.pdf}
\newline {\color{gray} (Emb: 0.834, TF-IDF: 0.777)}
\newline {\color{gray} \textbf{2º:} 539-Iniciativa-Convencional-Constituyente-de-la-cc-Barbara-Sepulveda-sobre-administracion-de-justicia-1510-01-02-1.pdf}
\newline {\color{gray} (Emb: 0.744, TF-IDF: 0.409)}


\item \textbf{Artículo} \newline
La Corte Suprema conocerá y resolverá de las impugnaciones deducidas en contra de la decisiones de la jurisdicción indígena, en sala especializada y asistida por una consejería técnica integrada por expertos en su cultura y derecho propio, en la forma que establezca la ley. 
\newline {\color{gray} \textbf{1º:} 242-6-Iniciativa-Convencional-de-la-cc-Vanessa-Hoppe-sobre-Justicia-Feminista-1146-hrs.pdf}
\newline {\color{gray} (Emb: 0.643, TF-IDF: 0.371)}
\newline {\color{gray} \textbf{2º:} 933-Iniciativa-Convencional-Constituyente-del-cc-Ricardo-Montero-sobre-Fuerzas-Armadas-y-de-Orden.pdf}
\newline {\color{gray} (Emb: 0.631, TF-IDF: 0.341)}


\item \textbf{Artículo} \newline
Está encargado del nombramiento, gobierno, gestión, formación y disciplina en el Sistema Nacional de Justicia. 
\newline {\color{gray} \textbf{1º:} 319-6-Iniciativa-Convencional-del-cc-Mauricio-Daza-sobre-el-Sistema-Nacional-de-Justicia17-09-hrs.pdf}
\newline {\color{gray} (Emb: 0.658, TF-IDF: 0.318)}
\newline {\color{gray} \textbf{2º:} 90-6-Iniciativa-Convencional-Constituyente-del-cc-Tomas-Laibe-y-otros.pdf}
\newline {\color{gray} (Emb: 0.658, TF-IDF: 0.291)}

El Consejo de la Justicia es un órgano autónomo, técnico, paritario y plurinacional, con personalidad jurídica y patrimonio propio, cuya finalidad es fortalecer la independencia judicial. 
\newline {\color{gray} \textbf{1º:} 539-Iniciativa-Convencional-Constituyente-de-la-cc-Barbara-Sepulveda-sobre-administracion-de-justicia-1510-01-02-1.pdf}
\newline {\color{gray} (Emb: 0.816, TF-IDF: 0.549)}
\newline {\color{gray} \textbf{2º:} 539-Iniciativa-Convencional-Constituyente-de-la-cc-Barbara-Sepulveda-sobre-administracion-de-justicia-1510-01-02-1.pdf}
\newline {\color{gray} (Emb: 0.756, TF-IDF: 0.442)}

En el ejercicio de sus atribuciones debe considerar el principio de no discriminación, la inclusión, paridad de género, equidad territorial y plurinacionalidad. 
\newline {\color{gray} \textbf{1º:} 190-6-c-Iniciativa-Convencional-Constituyente-de-la-cc-Natividad-LLanquileo-sobre-Pluralismo-Jurídico-1240-hrs.pdf}
\newline {\color{gray} (Emb: 0.612, TF-IDF: 0.267)}
\newline {\color{gray} \textbf{2º:} 431-6-Iniciativa-Convencional-de-la-cc-Bessy-Gallardo-sobre-Defensoria-Penal-Publica-1145-27-01.pdf}
\newline {\color{gray} (Emb: 0.591, TF-IDF: 0.240)}


\item \textbf{Artículo} \newline
i) Velar por la habilitación, formación y continuo perfeccionamiento de quienes integran el sistema nacional de justicia. 
\newline {\color{gray} \textbf{1º:} 41-6-Iniciativa-Convencional-Constituyente-del-cc-Mauricio-Daza-y-otros-1.pdf}
\newline {\color{gray} (Emb: 0.632, TF-IDF: 0.316)}
\newline {\color{gray} \textbf{2º:} 202-6-c-Iniciativa-Convencional-de-la-cc-Ingrid-Villena-que-crea-el-Sistema-Nacional-de-Defensa-Jurídica-Integral.pdf}
\newline {\color{gray} (Emb: 0.625, TF-IDF: 0.287)}

Para estos efectos, la Academia Judicial estará sometida a la dirección del Consejo. 
\newline {\color{gray} \textbf{1º:} 184-6-c-Iniciativa-Convencional-del-cc-Rodrigo-Álvarez-que-crea-la-Corte-Constitucional-1044hrs.pdf}
\newline {\color{gray} (Emb: 0.667, TF-IDF: 0.310)}
\newline {\color{gray} \textbf{2º:} 172-6-c-Iniciativa-Convencional-del-cc-Rodrigo-Álvarez-sobre-Banco-central1044-hrs.pdf}
\newline {\color{gray} (Emb: 0.665, TF-IDF: 0.302)}

j) Dictar instrucciones relativas a la organización y gestión administrativa de los tribunales. 
\newline {\color{gray} \textbf{1º:} 198-6-c-Iniciativa-Convencional-de-la-cc-Ingrid-Villena-que-crea-el-Consejo-de-la-Justicia-1644-hrs.pdf}
\newline {\color{gray} (Emb: 0.656, TF-IDF: 0.461)}
\newline {\color{gray} \textbf{2º:} 574-Iniciativa-Convencional-Constituyente-de-cc-Vanessa-Hoppe-sobre-Defensoria-de-los-Pueblos-2351-hrs.-01-02.pdf}
\newline {\color{gray} (Emb: 0.578, TF-IDF: 0.386)}

c) Efectuar una revisión integral de la gestión de todos los tribunales del sistema nacional de justicia. 
\newline {\color{gray} \textbf{1º:} 88-6-Iniciativa-Convencional-Constituyente-del-cc-Christian-Viera-y-otros.pdf}
\newline {\color{gray} (Emb: 0.674, TF-IDF: 0.470)}
\newline {\color{gray} \textbf{2º:} 864-Iniciativa-Convencional-Constituyente-del-cc-Marcos-Barraza-sobre-RREE.pdf}
\newline {\color{gray} (Emb: 0.670, TF-IDF: 0.407)}

k) Las demás que encomiende esta Constitución y las leyes. 
\newline {\color{gray} \textbf{1º:} 198-6-c-Iniciativa-Convencional-de-la-cc-Ingrid-Villena-que-crea-el-Consejo-de-la-Justicia-1644-hrs.pdf}
\newline {\color{gray} (Emb: 0.918, TF-IDF: 0.835)}
\newline {\color{gray} \textbf{2º:} 41-6-Iniciativa-Convencional-Constituyente-del-cc-Mauricio-Daza-y-otros-1.pdf}
\newline {\color{gray} (Emb: 0.608, TF-IDF: 0.395)}

h) Proponer la creación, modificación o supresión de tribunales a la autoridad competente. 
\newline {\color{gray} \textbf{1º:} 753-Iniciativa-Convencional-Constituyente-del-cc-Raul-Celis-sobre-Gobiernos-Locales.pdf}
\newline {\color{gray} (Emb: 0.603, TF-IDF: 0.257)}
\newline {\color{gray} \textbf{2º:} 98-6-Iniciativa-del-cc-Ruggero-Cozzi-Funcion-y-Principios-de-la-Jurisdiccion.pdf}
\newline {\color{gray} (Emb: 0.596, TF-IDF: 0.254)}

Estas instrucciones podrán tener un alcance nacional, regional o local. 
\newline {\color{gray} \textbf{1º:} 232-6-Iniciativa-Convencional-del-cc-Marco-Arellano-que-Crea-el-Consejo-Nacional-de-Justicia-1144-hrs.pdf}
\newline {\color{gray} (Emb: 0.765, TF-IDF: 0.660)}
\newline {\color{gray} \textbf{2º:} 41-6-Iniciativa-Convencional-Constituyente-del-cc-Mauricio-Daza-y-otros-1.pdf}
\newline {\color{gray} (Emb: 0.679, TF-IDF: 0.634)}

El Congreso deberá oficiar al Consejo, el que deberá responder dentro treinta días contados desde su recepción. 
\newline {\color{gray} \textbf{1º:} 319-6-Iniciativa-Convencional-del-cc-Mauricio-Daza-sobre-el-Sistema-Nacional-de-Justicia17-09-hrs.pdf}
\newline {\color{gray} (Emb: 0.629, TF-IDF: 0.474)}
\newline {\color{gray} \textbf{2º:} 41-6-Iniciativa-Convencional-Constituyente-del-cc-Mauricio-Daza-y-otros-1.pdf}
\newline {\color{gray} (Emb: 0.629, TF-IDF: 0.455)}

Son atribuciones del Consejo de la Justicia: a) Nombrar, previo concurso público y por resolución motivada, todas las juezas, jueces, funcionarias y funcionarios del Sistema Nacional de Justicia. 
\newline {\color{gray} \textbf{1º:} 574-Iniciativa-Convencional-Constituyente-de-cc-Vanessa-Hoppe-sobre-Defensoria-de-los-Pueblos-2351-hrs.-01-02.pdf}
\newline {\color{gray} (Emb: 0.702, TF-IDF: 0.507)}
\newline {\color{gray} \textbf{2º:} 409-6-Iniciativa-Convencional-Constituyente-de-la-cc-Ingrid-Villena-sobre-Defensoria-del-Pueblo-2239-24-01.pdf}
\newline {\color{gray} (Emb: 0.689, TF-IDF: 0.435)}

f) Definir las necesidades presupuestarias, ejecutar y gestionar los recursos para el adecuado funcionamiento del Sistema Nacional de Justicia. 
\newline {\color{gray} \textbf{1º:} 325-6-Iniciativa-Convencional-del-cc-Tomas-Laibe-sobre-Jurisdiccion-Constitucional.pdf}
\newline {\color{gray} (Emb: 0.718, TF-IDF: 0.457)}
\newline {\color{gray} \textbf{2º:} 759-Iniciativa-Convencional-Constituyente-de-la-cc-Paulina-Veloso-sobre-Democracia-Directa.pdf}
\newline {\color{gray} (Emb: 0.713, TF-IDF: 0.387)}

b) Adoptar las medidas disciplinarias de juezas, jueces, funcionarias y funcionarios del Sistema Nacional de Justicia, incluida su remoción, conforme a lo dispuesto en esta Constitución y la ley. 
\newline {\color{gray} \textbf{1º:} 41-6-Iniciativa-Convencional-Constituyente-del-cc-Mauricio-Daza-y-otros-1.pdf}
\newline {\color{gray} (Emb: 0.646, TF-IDF: 0.341)}
\newline {\color{gray} \textbf{2º:} 41-6-Iniciativa-Convencional-Constituyente-del-cc-Mauricio-Daza-y-otros-1.pdf}
\newline {\color{gray} (Emb: 0.642, TF-IDF: 0.341)}

En ningún caso incluirá las resoluciones judiciales. 
\newline {\color{gray} \textbf{1º:} 95-6-Iniciativa-Convencional-Constituyente-de-Cc-Mauricio-Daza-y-otros-2.pdf}
\newline {\color{gray} (Emb: 0.740, TF-IDF: 0.414)}
\newline {\color{gray} \textbf{2º:} 95-6-Iniciativa-Convencional-Constituyente-de-Cc-Mauricio-Daza-y-otros-2.pdf}
\newline {\color{gray} (Emb: 0.716, TF-IDF: 0.372)}

g) Pronunciarse sobre cualquier modificación legal en la organización y atribuciones del sistema nacional de justicia. 
\newline {\color{gray} \textbf{1º:} 95-6-Iniciativa-Convencional-Constituyente-de-Cc-Mauricio-Daza-y-otros-2.pdf}
\newline {\color{gray} (Emb: 0.704, TF-IDF: 0.370)}
\newline {\color{gray} \textbf{2º:} 41-6-Iniciativa-Convencional-Constituyente-del-cc-Mauricio-Daza-y-otros-1.pdf}
\newline {\color{gray} (Emb: 0.652, TF-IDF: 0.313)}

e) Decidir sobre promociones, traslados, permutas y cese de funciones de integrantes del sistema nacional de justicia. 
\newline {\color{gray} \textbf{1º:} 41-6-Iniciativa-Convencional-Constituyente-del-cc-Mauricio-Daza-y-otros-1.pdf}
\newline {\color{gray} (Emb: 0.695, TF-IDF: 0.327)}
\newline {\color{gray} \textbf{2º:} 235-6-Iniciativa-Convencional-de-la-cc-Vanessa-Hoppe-sobre-Juzgados-Comunitarios-1145-hrs.pdf}
\newline {\color{gray} (Emb: 0.681, TF-IDF: 0.326)}

d) Evaluar y calificar, periódicamente, el desempeño de juezas, jueces, funcionarias y funcionarios del Sistema Nacional de Justicia. 
\newline {\color{gray} \textbf{1º:} 41-6-Iniciativa-Convencional-Constituyente-del-cc-Mauricio-Daza-y-otros-1.pdf}
\newline {\color{gray} (Emb: 0.706, TF-IDF: 0.348)}
\newline {\color{gray} \textbf{2º:} 95-6-Iniciativa-Convencional-Constituyente-de-Cc-Mauricio-Daza-y-otros-2.pdf}
\newline {\color{gray} (Emb: 0.678, TF-IDF: 0.346)}


\item \textbf{Artículo} \newline
El Consejo de la Justicia se compone por diecisiete integrantes, conforme a la siguiente integración: a) Ocho integrantes serán juezas o jueces titulares elegidos por sus pares. 
\newline {\color{gray} \textbf{1º:} 232-6-Iniciativa-Convencional-del-cc-Marco-Arellano-que-Crea-el-Consejo-Nacional-de-Justicia-1144-hrs.pdf}
\newline {\color{gray} (Emb: 0.693, TF-IDF: 0.264)}
\newline {\color{gray} \textbf{2º:} 90-6-Iniciativa-Convencional-Constituyente-del-cc-Tomas-Laibe-y-otros.pdf}
\newline {\color{gray} (Emb: 0.684, TF-IDF: 0.249)}

b) Dos integrantes serán funcionarios o profesionales del Sistema Nacional de Justicia elegidos por sus pares. 
\newline {\color{gray} \textbf{1º:} 732-Iniciativa-Convencional-Constituyente-del-cc-Bastian-Labbe-crea-el-Servicio-de-proteccion-de-bienes-comunes.pdf}
\newline {\color{gray} (Emb: 0.619, TF-IDF: 0.366)}
\newline {\color{gray} \textbf{2º:} 715-Iniciativa-Convencional-Constituyente-de-la-cc-Lisette-Vergara-sobre-Empresas-Publicas.pdf}
\newline {\color{gray} (Emb: 0.612, TF-IDF: 0.364)}

c) Dos integrantes elegidos por los pueblos indígenas en la forma que determine la Constitución y la ley. 
\newline {\color{gray} \textbf{1º:} 95-6-Iniciativa-Convencional-Constituyente-de-Cc-Mauricio-Daza-y-otros-2.pdf}
\newline {\color{gray} (Emb: 0.956, TF-IDF: 0.593)}
\newline {\color{gray} \textbf{2º:} 319-6-Iniciativa-Convencional-del-cc-Mauricio-Daza-sobre-el-Sistema-Nacional-de-Justicia17-09-hrs.pdf}
\newline {\color{gray} (Emb: 0.952, TF-IDF: 0.517)}

d) Cinco integrantes elegidos por el Congreso, previa determinación de las ternas correspondientes por concurso público, a cargo del Consejo de Alta Dirección Pública. 
\newline {\color{gray} \textbf{1º:} 88-6-Iniciativa-Convencional-Constituyente-del-cc-Christian-Viera-y-otros.pdf}
\newline {\color{gray} (Emb: 0.757, TF-IDF: 0.566)}
\newline {\color{gray} \textbf{2º:} 90-6-Iniciativa-Convencional-Constituyente-del-cc-Tomas-Laibe-y-otros.pdf}
\newline {\color{gray} (Emb: 0.706, TF-IDF: 0.465)}

Las y los integrantes señalados en la letra c) deberán ser personas de comprobada idoneidad para el ejercicio del cargo y que se hayan destacado en la función pública o social. 
\newline {\color{gray} \textbf{1º:} 95-6-Iniciativa-Convencional-Constituyente-de-Cc-Mauricio-Daza-y-otros-2.pdf}
\newline {\color{gray} (Emb: 0.817, TF-IDF: 0.818)}
\newline {\color{gray} \textbf{2º:} 95-6-Iniciativa-Convencional-Constituyente-de-Cc-Mauricio-Daza-y-otros-2.pdf}
\newline {\color{gray} (Emb: 0.671, TF-IDF: 0.580)}

En el caso de la letra d) deberán ser profesionales con a lo menos diez años del título correspondiente, que se hubieren destacado en la actividad profesional, académica o en la función pública. 
\newline {\color{gray} \textbf{1º:} 95-6-Iniciativa-Convencional-Constituyente-de-Cc-Mauricio-Daza-y-otros-2.pdf}
\newline {\color{gray} (Emb: 0.744, TF-IDF: 0.442)}
\newline {\color{gray} \textbf{2º:} 234-1-Iniciativa-Convencional-del-cc-Jaime-Bassa-sobre-Justicia-Complementaria-1144-hrs.pdf}
\newline {\color{gray} (Emb: 0.744, TF-IDF: 0.413)}

Las y los integrantes durarán seis años en sus cargos y no podrán ser reelegidos, debiendo renovarse por parcialidades cada tres años de conformidad a lo establecido por la ley. 
\newline {\color{gray} \textbf{1º:} 579-Iniciativa-Convencional-Constituyente-de-cc-Christian-Viera-sobre-Sistema-Electoral-y-Justicia-Electoral-2330-hrs.pdf}
\newline {\color{gray} (Emb: 0.663, TF-IDF: 0.401)}
\newline {\color{gray} \textbf{2º:} 98-6-Iniciativa-del-cc-Ruggero-Cozzi-Funcion-y-Principios-de-la-Jurisdiccion.pdf}
\newline {\color{gray} (Emb: 0.638, TF-IDF: 0.389)}

Sus integrantes serán elegidos de acuerdo a criterios de paridad de género, plurinacionalidad y equidad territorial. 
\newline {\color{gray} \textbf{1º:} 7-2-Iniciativa-Convencional-Constituyente-del-cc-Luis-Barceló-y-otros.pdf}
\newline {\color{gray} (Emb: 0.539, TF-IDF: 0.406)}
\newline {\color{gray} \textbf{2º:} 574-Iniciativa-Convencional-Constituyente-de-cc-Vanessa-Hoppe-sobre-Defensoria-de-los-Pueblos-2351-hrs.-01-02.pdf}
\newline {\color{gray} (Emb: 0.502, TF-IDF: 0.347)}


\item \textbf{Artículo} \newline
La ley determinará la organización, funcionamiento, procedimientos de elección de integrantes del Consejo y fijará la planta, régimen de remuneraciones y estatuto de su personal. 
\newline {\color{gray} \textbf{1º:} 319-6-Iniciativa-Convencional-del-cc-Mauricio-Daza-sobre-el-Sistema-Nacional-de-Justicia17-09-hrs.pdf}
\newline {\color{gray} (Emb: 0.700, TF-IDF: 0.421)}
\newline {\color{gray} \textbf{2º:} 319-6-Iniciativa-Convencional-del-cc-Mauricio-Daza-sobre-el-Sistema-Nacional-de-Justicia17-09-hrs.pdf}
\newline {\color{gray} (Emb: 0.689, TF-IDF: 0.420)}

El Consejo se organizará desconcentradamente, en conformidad a lo que establezca la ley. 
\newline {\color{gray} \textbf{1º:} 466-6-Iniciativa-Convencional-Constituyente-de-la-cc-Adriana-Cancino-sobre-Defensoria-de-los-DDHH-1933-31-01.pdf}
\newline {\color{gray} (Emb: 0.793, TF-IDF: 0.482)}
\newline {\color{gray} \textbf{2º:} 866-Iniciativa-Convencional-Constituyente-del-cc-Renato-Garin-sobre-Banco-Central-Autonomo.pdf}
\newline {\color{gray} (Emb: 0.687, TF-IDF: 0.435)}

En ambos casos, tomará sus decisiones por la mayoría de sus integrantes en ejercicio, con las excepciones que establezca esta Constitución. 
\newline {\color{gray} \textbf{1º:} 88-6-Iniciativa-Convencional-Constituyente-del-cc-Christian-Viera-y-otros.pdf}
\newline {\color{gray} (Emb: 0.815, TF-IDF: 0.746)}
\newline {\color{gray} \textbf{2º:} 801-Iniciativa-Convencional-Constituyente-del-cc-Christian-Viera-sobre-Consejo-de-Contiendas-de-Trabajo.pdf}
\newline {\color{gray} (Emb: 0.745, TF-IDF: 0.696)}

El Consejo de la Justicia podrá funcionar en pleno o en comisiones. 
\newline {\color{gray} \textbf{1º:} 98-6-Iniciativa-del-cc-Ruggero-Cozzi-Funcion-y-Principios-de-la-Jurisdiccion.pdf}
\newline {\color{gray} (Emb: 0.763, TF-IDF: 0.725)}
\newline {\color{gray} \textbf{2º:} 785-Iniciativa-Convencional-Constituyente-de-la-cc-Patricia-Labra-sobre-Defensor-del-Usuario.pdf}
\newline {\color{gray} (Emb: 0.762, TF-IDF: 0.656)}


\item \textbf{Artículo} \newline
Las y los consejeros no podrán ejercer otra función o empleo, sea o no remunerado, con exclusión de las actividades académicas. 
\newline {\color{gray} \textbf{1º:} 936-IniciativaConvencional-Constituyente-del-cc-Mauricio-Daza-sobre-Banco-Central.pdf}
\newline {\color{gray} (Emb: 0.601, TF-IDF: 0.442)}
\newline {\color{gray} \textbf{2º:} 184-6-c-Iniciativa-Convencional-del-cc-Rodrigo-Álvarez-que-crea-la-Corte-Constitucional-1044hrs.pdf}
\newline {\color{gray} (Emb: 0.582, TF-IDF: 0.361)}

Tampoco podrán concursar para ser designados en cargos judiciales hasta transcurrido un año desde que cesen en sus funciones. 
\newline {\color{gray} \textbf{1º:} 801-Iniciativa-Convencional-Constituyente-del-cc-Christian-Viera-sobre-Consejo-de-Contiendas-de-Trabajo.pdf}
\newline {\color{gray} (Emb: 0.706, TF-IDF: 0.335)}
\newline {\color{gray} \textbf{2º:} 504-1-Iniciativa-Convencional-Constituyente-de-la-cc-Pollyana-Rivera-Crea-el-Consejo-de-Seguridad-Nacional-1054-01-02.pdf}
\newline {\color{gray} (Emb: 0.698, TF-IDF: 0.312)}

La ley podrá establecer otras incompatibilidades en el ejercicio del cargo. 
\newline {\color{gray} \textbf{1º:} 198-6-c-Iniciativa-Convencional-de-la-cc-Ingrid-Villena-que-crea-el-Consejo-de-la-Justicia-1644-hrs.pdf}
\newline {\color{gray} (Emb: 0.937, TF-IDF: 0.938)}
\newline {\color{gray} \textbf{2º:} 801-Iniciativa-Convencional-Constituyente-del-cc-Christian-Viera-sobre-Consejo-de-Contiendas-de-Trabajo.pdf}
\newline {\color{gray} (Emb: 0.872, TF-IDF: 0.906)}

Las y los consejeros indicados en las letras a y b del artículo sobre Composición del Consejo de la Justicia se entenderán suspendidos del ejercicio de su función mientras dure su cometido en el Consejo. 
\newline {\color{gray} \textbf{1º:} 98-6-Iniciativa-del-cc-Ruggero-Cozzi-Funcion-y-Principios-de-la-Jurisdiccion.pdf}
\newline {\color{gray} (Emb: 0.769, TF-IDF: 0.503)}
\newline {\color{gray} \textbf{2º:} 90-6-Iniciativa-Convencional-Constituyente-del-cc-Tomas-Laibe-y-otros.pdf}
\newline {\color{gray} (Emb: 0.741, TF-IDF: 0.474)}


\item \textbf{Artículo} \newline
Las y los integrantes del Consejo cesarán en su cargo al término de su período, por cumplir setenta años de edad, por remoción, renuncia, incapacidad física o mental sobreviniente o condena por delito que merezca pena aflictiva. 
\newline {\color{gray} \textbf{1º:} 98-6-Iniciativa-del-cc-Ruggero-Cozzi-Funcion-y-Principios-de-la-Jurisdiccion.pdf}
\newline {\color{gray} (Emb: 0.863, TF-IDF: 0.596)}
\newline {\color{gray} \textbf{2º:} 90-6-Iniciativa-Convencional-Constituyente-del-cc-Tomas-Laibe-y-otros.pdf}
\newline {\color{gray} (Emb: 0.821, TF-IDF: 0.583)}

Tanto la renuncia como la incapacidad sobreviniente deberá ser aceptada por el Consejo. 
\newline {\color{gray} \textbf{1º:} 325-6-Iniciativa-Convencional-del-cc-Tomas-Laibe-sobre-Jurisdiccion-Constitucional.pdf}
\newline {\color{gray} (Emb: 0.755, TF-IDF: 0.476)}
\newline {\color{gray} \textbf{2º:} 472-6-Iniciativa-Convencional-Constituyente-del-cc-Daniel-Bravo-sobre-Corte-Constitucional-2003-31-01.pdf}
\newline {\color{gray} (Emb: 0.752, TF-IDF: 0.399)}

El proceso de remoción será determinado por la ley, respetando todas las garantías de un debido proceso. 
\newline {\color{gray} \textbf{1º:} 711-Iniciativa-Convencional-Constituyente-de-la-cc-Ingrid-Villena-sobre-Tricel.pdf}
\newline {\color{gray} (Emb: 0.576, TF-IDF: 0.316)}
\newline {\color{gray} \textbf{2º:} 711-Iniciativa-Convencional-Constituyente-de-la-cc-Ingrid-Villena-sobre-Tricel.pdf}
\newline {\color{gray} (Emb: 0.576, TF-IDF: 0.302)}


\item \textbf{Artículo} \newline
El Consejo efectuará los nombramientos mediante concursos públicos regulados por la ley, los que incluirán audiencias públicas. 
\newline {\color{gray} \textbf{1º:} 95-6-Iniciativa-Convencional-Constituyente-de-Cc-Mauricio-Daza-y-otros-2.pdf}
\newline {\color{gray} (Emb: 0.885, TF-IDF: 0.745)}
\newline {\color{gray} \textbf{2º:} 319-6-Iniciativa-Convencional-del-cc-Mauricio-Daza-sobre-el-Sistema-Nacional-de-Justicia17-09-hrs.pdf}
\newline {\color{gray} (Emb: 0.720, TF-IDF: 0.459)}

Para acceder a un cargo de juez o jueza dentro del Sistema Nacional de Justicia se requerirá haber aprobado el curso de habilitación de la Academia Judicial para el ejercicio de la función jurisdiccional, contar con tres años de ejercicio de la profesión de abogado o abogada para el caso de tribunales de instancia, cinco años para el caso de las Cortes de Apelaciones y veinte años para el caso de la Corte Suprema y los demás requisitos que establezca la Constitución y la ley. 
\newline {\color{gray} \textbf{1º:} 95-6-Iniciativa-Convencional-Constituyente-de-Cc-Mauricio-Daza-y-otros-2.pdf}
\newline {\color{gray} (Emb: 0.669, TF-IDF: 0.891)}
\newline {\color{gray} \textbf{2º:} 98-6-Iniciativa-del-cc-Ruggero-Cozzi-Funcion-y-Principios-de-la-Jurisdiccion.pdf}
\newline {\color{gray} (Emb: 0.602, TF-IDF: 0.574)}


\item \textbf{Artículo} \newline
Las decisiones adoptadas conforme a los incisos anteriores, no podrán ser revisadas ni impugnadas ante otros órganos del Sistema Nacional de Justicia. 
\newline {\color{gray} \textbf{1º:} 88-6-Iniciativa-Convencional-Constituyente-del-cc-Christian-Viera-y-otros.pdf}
\newline {\color{gray} (Emb: 0.706, TF-IDF: 0.414)}
\newline {\color{gray} \textbf{2º:} 472-6-Iniciativa-Convencional-Constituyente-del-cc-Daniel-Bravo-sobre-Corte-Constitucional-2003-31-01.pdf}
\newline {\color{gray} (Emb: 0.670, TF-IDF: 0.401)}

Los procedimientos disciplinarios serán conocidos y resueltos por una comisión compuesta por cinco integrantes del Consejo elegidos por sorteo, decisión que será revisable por su Pleno a petición del afectado. 
\newline {\color{gray} \textbf{1º:} 325-6-Iniciativa-Convencional-del-cc-Tomas-Laibe-sobre-Jurisdiccion-Constitucional.pdf}
\newline {\color{gray} (Emb: 0.737, TF-IDF: 0.483)}
\newline {\color{gray} \textbf{2º:} 198-6-c-Iniciativa-Convencional-de-la-cc-Ingrid-Villena-que-crea-el-Consejo-de-la-Justicia-1644-hrs.pdf}
\newline {\color{gray} (Emb: 0.731, TF-IDF: 0.430)}

La resolución del Consejo que ponga término al procedimiento será impugnable ante el órgano que establezca la Constitución. 
\newline {\color{gray} \textbf{1º:} 504-1-Iniciativa-Convencional-Constituyente-de-la-cc-Pollyana-Rivera-Crea-el-Consejo-de-Seguridad-Nacional-1054-01-02.pdf}
\newline {\color{gray} (Emb: 0.742, TF-IDF: 0.387)}
\newline {\color{gray} \textbf{2º:} 374-2-Iniciativa-Convencional-Constituyente-de-la-cc-Loreto-Vallejos-sobre-Democracia-Directa-0900-hrs-24-01.pdf}
\newline {\color{gray} (Emb: 0.717, TF-IDF: 0.365)}


\item \textbf{Artículo} \newline
Los Tribunales Ambientales conocerán y resolverán sobre la legalidad de los actos administrativos en materia ambiental, de la acción de tutela de derechos fundamentales ambientales y de los derechos de la Naturaleza, la reparación por daño ambiental y las demás que señale la Constitución y la ley. 
\newline {\color{gray} \textbf{1º:} 631-Iniciativa-Convencional-Constituyente-de-cc-Ingrid-Villena-sobre-Contraloria-General-de-la-Republica.pdf}
\newline {\color{gray} (Emb: 0.654, TF-IDF: 0.452)}
\newline {\color{gray} \textbf{2º:} 98-6-Iniciativa-del-cc-Ruggero-Cozzi-Funcion-y-Principios-de-la-Jurisdiccion.pdf}
\newline {\color{gray} (Emb: 0.548, TF-IDF: 0.301)}

Las acciones para impugnar la legalidad de los actos administrativos que se pronuncien sobre materia ambiental, podrán interponerse directamente a los Tribunales Ambientales, sin que pueda exigirse el agotamiento previo de la vía administrativa. 
\newline {\color{gray} \textbf{1º:} 184-6-c-Iniciativa-Convencional-del-cc-Rodrigo-Álvarez-que-crea-la-Corte-Constitucional-1044hrs.pdf}
\newline {\color{gray} (Emb: 0.690, TF-IDF: 0.425)}
\newline {\color{gray} \textbf{2º:} 151-3-c-Iniciativa-de-la-cc-Angelica-Tepper-Competencias-de-los-Gobiernos-Regionales.pdf}
\newline {\color{gray} (Emb: 0.671, TF-IDF: 0.279)}

Habrá al menos un Tribunal Ambiental en cada región del país. 
\newline {\color{gray} \textbf{1º:} 95-6-Iniciativa-Convencional-Constituyente-de-Cc-Mauricio-Daza-y-otros-2.pdf}
\newline {\color{gray} (Emb: 0.770, TF-IDF: 0.777)}
\newline {\color{gray} \textbf{2º:} 95-6-Iniciativa-Convencional-Constituyente-de-Cc-Mauricio-Daza-y-otros-2.pdf}
\newline {\color{gray} (Emb: 0.695, TF-IDF: 0.264)}

La ley regulará la integración, competencia y demás aspectos que sean necesarios para su adecuado funcionamiento. 
\newline {\color{gray} \textbf{1º:} 700-Iniciativa-Convencional-Constituyente-de-la-cc-Alejandra-Perez-sobre-Resolucion-de-Conflictos.pdf}
\newline {\color{gray} (Emb: 0.657, TF-IDF: 0.395)}
\newline {\color{gray} \textbf{2º:} 971-Iniciativa-Convencional-Constituyente-de-la-cc-Ivanna-Olivares-sobre-Relaves.pdf}
\newline {\color{gray} (Emb: 0.653, TF-IDF: 0.383)}

En el caso de actos de la administración que decidan un proceso de evaluación ambiental y de la solicitud de medidas cautelares no se podrá exigir el agotamiento de la vía administrativa para acceder a la justicia ambiental. 
\newline {\color{gray} \textbf{1º:} 457-6-Iniciativa-Convencional-Constituyente-de-la-cc-Manuela-Royo-sobre-Justicia-Ambiental.pdf}
\newline {\color{gray} (Emb: 1.000, TF-IDF: 1.000)}
\newline {\color{gray} \textbf{2º:} 472-6-Iniciativa-Convencional-Constituyente-del-cc-Daniel-Bravo-sobre-Corte-Constitucional-2003-31-01.pdf}
\newline {\color{gray} (Emb: 0.604, TF-IDF: 0.316)}


\item \textbf{Artículo} \newline
Todos los órganos autónomos se rigen por el principio de paridad. 
\newline {\color{gray} \textbf{1º:} 88-6-Iniciativa-Convencional-Constituyente-del-cc-Christian-Viera-y-otros.pdf}
\newline {\color{gray} (Emb: 0.832, TF-IDF: 0.827)}
\newline {\color{gray} \textbf{2º:} 11-4-Iniciativa-Convencional-Constituyente-de-la-cc-María-Elisa-Quinteros-y-otras.pdf}
\newline {\color{gray} (Emb: 0.648, TF-IDF: 0.628)}

Se promueve la implementación de medidas de acción afirmativa, asegurando que, al menos, el cincuenta por ciento de su integración sean mujeres. 
\newline {\color{gray} \textbf{1º:} 681-Iniciativa-Convencional-Constituyente-del-cc-Gaspar-Dominguez-sobre-el-Derecho-a-la-salud-120001-02.pdf}
\newline {\color{gray} (Emb: 0.724, TF-IDF: 0.468)}
\newline {\color{gray} \textbf{2º:} 749-Iniciativa-Convencional-Constituyente-del-cc-Pedro-Munoz-sobre-Acceso-a-la-Salud.pdf}
\newline {\color{gray} (Emb: 0.707, TF-IDF: 0.386)}


\item \textbf{Artículo} \newline
En dichas funciones deberá velar por el respeto y promoción de los derechos humanos, considerando también los intereses de la víctima, respecto de quienes deberá adoptar todas aquellas medidas que sean necesarias para protegerlas, al igual que a los testigos. 
\newline {\color{gray} \textbf{1º:} 613-Iniciativa-Convencional-Constituyente-de-cc-Barbara-Sepulveda-sobre-Perspectiva-de-Genero-en-el-Derecho-al-Trabajo.pdf}
\newline {\color{gray} (Emb: 0.605, TF-IDF: 0.478)}
\newline {\color{gray} \textbf{2º:} 116-1-c-Iniciativa-de-la-cc-Alondra-Carrillo-Democracia-Paritaria.pdf}
\newline {\color{gray} (Emb: 0.596, TF-IDF: 0.424)}

La autoridad policial requerida deberá cumplir sin más trámite dichas órdenes y no podrá calificar su fundamento, oportunidad, justicia o legalidad, salvo requerir la exhibición, a menos que ésta sea verbal, de la autorización judicial. 
\newline {\color{gray} \textbf{1º:} 560-Iniciativa-Convencional-Constituyente-de-cc-Andres-Cruz-sobre-Ministerio-Publico-2016-hrs.-01-02.pdf}
\newline {\color{gray} (Emb: 1.000, TF-IDF: 1.000)}
\newline {\color{gray} \textbf{2º:} 608-Iniciativa-Convencional-Constituyente-de-cc-Miguel-Angel-Botto-sobre-Ministerio-Publico-2119-hrs.-01-02.pdf}
\newline {\color{gray} (Emb: 1.000, TF-IDF: 1.000)}

El Ministerio Público podrá impartir órdenes directas a las Fuerzas de Orden y Seguridad Pública para el ejercicio de sus funciones, en cuyo caso podrá además participar, tanto en la fijación de metas y objetivos, como en la evaluación del cumplimiento de estas órdenes, metas y objetivos. 
\newline {\color{gray} \textbf{1º:} 909-Iniciativa-Convencional-Constituyente-del-cc-Hugo-Gutierrez-Sobre-Ministerio-Publico.pdf}
\newline {\color{gray} (Emb: 0.750, TF-IDF: 0.446)}
\newline {\color{gray} \textbf{2º:} 695-Iniciativa-Convencional-del-cc-Felipe-Harboe-sobre-Ministerio-Publico-120001-02.pdf}
\newline {\color{gray} (Emb: 0.744, TF-IDF: 0.375)}

La víctima del delito y las demás personas que determine la ley podrán ejercer igualmente la acción penal. 
\newline {\color{gray} \textbf{1º:} 560-Iniciativa-Convencional-Constituyente-de-cc-Andres-Cruz-sobre-Ministerio-Publico-2016-hrs.-01-02.pdf}
\newline {\color{gray} (Emb: 1.000, TF-IDF: 1.000)}
\newline {\color{gray} \textbf{2º:} 695-Iniciativa-Convencional-del-cc-Felipe-Harboe-sobre-Ministerio-Publico-120001-02.pdf}
\newline {\color{gray} (Emb: 0.888, TF-IDF: 0.718)}

El Ministerio Público es un organismo autónomo y jerarquizado, con personalidad jurídica y patrimonio propio que tiene como función dirigir en forma exclusiva la investigación de los hechos que pudiesen ser constitutivos de delito, los que determinen la participación punible y los que acrediten la inocencia del imputado. 
\newline {\color{gray} \textbf{1º:} 457-6-Iniciativa-Convencional-Constituyente-de-la-cc-Manuela-Royo-sobre-Justicia-Ambiental.pdf}
\newline {\color{gray} (Emb: 0.674, TF-IDF: 0.333)}
\newline {\color{gray} \textbf{2º:} 975-Iniciativa-Convencional-Constituyente-del-cc-Bastian-Labbe-sobre-Derechos-Humanos.pdf}
\newline {\color{gray} (Emb: 0.572, TF-IDF: 0.327)}

La facultad exclusiva de ciertos órganos de la administración para presentar denuncias y querellas, no impedirá que el Ministerio Público investigue y ejerza la acción penal pública, en el caso de delitos que atenten en contra de la probidad, el patrimonio público o lesionen bienes jurídicos colectivos. 
\newline {\color{gray} \textbf{1º:} 560-Iniciativa-Convencional-Constituyente-de-cc-Andres-Cruz-sobre-Ministerio-Publico-2016-hrs.-01-02.pdf}
\newline {\color{gray} (Emb: 0.904, TF-IDF: 0.776)}
\newline {\color{gray} \textbf{2º:} 695-Iniciativa-Convencional-del-cc-Felipe-Harboe-sobre-Ministerio-Publico-120001-02.pdf}
\newline {\color{gray} (Emb: 0.848, TF-IDF: 0.776)}

Ejercerá la acción penal pública en representación exclusiva de la sociedad, en la forma prevista por la ley. 
\newline {\color{gray} \textbf{1º:} 344-3-Iniciativa-Convencional-Constituyente-del-cc-Hernan-Larrain-sobre-Reforma-Administrativa-y-Modernizacion-del-Estado.pdf}
\newline {\color{gray} (Emb: 0.534, TF-IDF: 0.380)}
\newline {\color{gray} \textbf{2º:} 698-Iniciativa-Convencional-Constituyente-de-la-cc-Alejandra-Flores-sobre-Educacion-01-02.pdf}
\newline {\color{gray} (Emb: 0.518, TF-IDF: 0.335)}

Las actuaciones que amenacen, priven o perturben al imputado o a terceros del ejercicio de los derechos que esta Constitución asegura, requerirán siempre de autorización judicial previa y motivada. 
\newline {\color{gray} \textbf{1º:} 608-Iniciativa-Convencional-Constituyente-de-cc-Miguel-Angel-Botto-sobre-Ministerio-Publico-2119-hrs.-01-02.pdf}
\newline {\color{gray} (Emb: 0.962, TF-IDF: 0.744)}
\newline {\color{gray} \textbf{2º:} 578-Iniciativa-Convencional-Constituyente-de-cc-Christian-Viera-sobre-Ministerio-Publico-2308-hrs.-01-02.pdf}
\newline {\color{gray} (Emb: 0.962, TF-IDF: 0.744)}

En caso alguno podrá ejercer funciones jurisdiccionales. 
\newline {\color{gray} \textbf{1º:} 560-Iniciativa-Convencional-Constituyente-de-cc-Andres-Cruz-sobre-Ministerio-Publico-2016-hrs.-01-02.pdf}
\newline {\color{gray} (Emb: 1.000, TF-IDF: 1.000)}
\newline {\color{gray} \textbf{2º:} 706-Iniciativa-Convencional-Constituyente-de-la-cc-Elsa-Labrana-sobre-Ministerio-Publico-01-02.pdf}
\newline {\color{gray} (Emb: 0.757, TF-IDF: 0.833)}


\item \textbf{Artículo} \newline
Las autoridades superiores de la institución deberán siempre fundar aquellas órdenes e instrucciones dirigidas a los fiscales que puedan afectar una investigación o el ejercicio de la acción penal. 
\newline {\color{gray} \textbf{1º:} 560-Iniciativa-Convencional-Constituyente-de-cc-Andres-Cruz-sobre-Ministerio-Publico-2016-hrs.-01-02.pdf}
\newline {\color{gray} (Emb: 0.967, TF-IDF: 0.816)}
\newline {\color{gray} \textbf{2º:} 608-Iniciativa-Convencional-Constituyente-de-cc-Miguel-Angel-Botto-sobre-Ministerio-Publico-2119-hrs.-01-02.pdf}
\newline {\color{gray} (Emb: 0.912, TF-IDF: 0.766)}

Las y los fiscales y funcionarios tendrán un sistema de promoción y ascenso que garantice una carrera que permita fomentar la excelencia técnica y la acumulación de experiencia en las funciones que éstos desempeñan. 
\newline {\color{gray} \textbf{1º:} 560-Iniciativa-Convencional-Constituyente-de-cc-Andres-Cruz-sobre-Ministerio-Publico-2016-hrs.-01-02.pdf}
\newline {\color{gray} (Emb: 1.000, TF-IDF: 1.000)}
\newline {\color{gray} \textbf{2º:} 608-Iniciativa-Convencional-Constituyente-de-cc-Miguel-Angel-Botto-sobre-Ministerio-Publico-2119-hrs.-01-02.pdf}
\newline {\color{gray} (Emb: 0.945, TF-IDF: 0.917)}

Una ley determinará la organización y atribuciones del Ministerio Público, señalará las calidades y requisitos que deberán tener y cumplir las y los fiscales para su nombramiento y las causales de su remoción. 
\newline {\color{gray} \textbf{1º:} 608-Iniciativa-Convencional-Constituyente-de-cc-Miguel-Angel-Botto-sobre-Ministerio-Publico-2119-hrs.-01-02.pdf}
\newline {\color{gray} (Emb: 0.987, TF-IDF: 0.940)}
\newline {\color{gray} \textbf{2º:} 706-Iniciativa-Convencional-Constituyente-de-la-cc-Elsa-Labrana-sobre-Ministerio-Publico-01-02.pdf}
\newline {\color{gray} (Emb: 0.944, TF-IDF: 0.898)}

Los fiscales cesarán en su cargo al cumplir los 70 años. 
\newline {\color{gray} \textbf{1º:} 560-Iniciativa-Convencional-Constituyente-de-cc-Andres-Cruz-sobre-Ministerio-Publico-2016-hrs.-01-02.pdf}
\newline {\color{gray} (Emb: 0.994, TF-IDF: 0.934)}
\newline {\color{gray} \textbf{2º:} 608-Iniciativa-Convencional-Constituyente-de-cc-Miguel-Angel-Botto-sobre-Ministerio-Publico-2119-hrs.-01-02.pdf}
\newline {\color{gray} (Emb: 0.967, TF-IDF: 0.927)}


\item \textbf{Artículo} \newline
Durarán cuatro años y una vez concluida su labor, podrán retornar a la función que ejercían en el Ministerio Público. 
\newline {\color{gray} \textbf{1º:} 608-Iniciativa-Convencional-Constituyente-de-cc-Miguel-Angel-Botto-sobre-Ministerio-Publico-2119-hrs.-01-02.pdf}
\newline {\color{gray} (Emb: 0.949, TF-IDF: 0.695)}
\newline {\color{gray} \textbf{2º:} 560-Iniciativa-Convencional-Constituyente-de-cc-Andres-Cruz-sobre-Ministerio-Publico-2016-hrs.-01-02.pdf}
\newline {\color{gray} (Emb: 0.949, TF-IDF: 0.695)}

No podrán ser reelectos ni postular nuevamente al cargo de fiscal regional. 
\newline {\color{gray} \textbf{1º:} 850-Iniciativa-Convencional-Constituyente-de-la-cc-Patricia-Labra-sobre-Ministerio-Publico.pdf}
\newline {\color{gray} (Emb: 0.786, TF-IDF: 0.365)}
\newline {\color{gray} \textbf{2º:} 909-Iniciativa-Convencional-Constituyente-del-cc-Hugo-Gutierrez-Sobre-Ministerio-Publico.pdf}
\newline {\color{gray} (Emb: 0.759, TF-IDF: 0.362)}

Existirá una Fiscalía Regional en cada región del país, sin perjuicio que la ley podrá establecer más de una por región. 
\newline {\color{gray} \textbf{1º:} 179-6-c-Iniciativa-Convencional-DEL-CC-Rodrigo-Álvarez-Ministerio-Público-1044-hrs.pdf}
\newline {\color{gray} (Emb: 0.829, TF-IDF: 0.861)}
\newline {\color{gray} \textbf{2º:} 850-Iniciativa-Convencional-Constituyente-de-la-cc-Patricia-Labra-sobre-Ministerio-Publico.pdf}
\newline {\color{gray} (Emb: 0.789, TF-IDF: 0.767)}

Las y los fiscales regionales deberán haberse desempeñado como fiscales adjuntos durante cinco o más años, no haber ejercido como fiscal regional, haber aprobado cursos de formación especializada y poseer las demás calidades que establezca la ley. 
\newline {\color{gray} \textbf{1º:} 560-Iniciativa-Convencional-Constituyente-de-cc-Andres-Cruz-sobre-Ministerio-Publico-2016-hrs.-01-02.pdf}
\newline {\color{gray} (Emb: 1.000, TF-IDF: 1.000)}
\newline {\color{gray} \textbf{2º:} 608-Iniciativa-Convencional-Constituyente-de-cc-Miguel-Angel-Botto-sobre-Ministerio-Publico-2119-hrs.-01-02.pdf}
\newline {\color{gray} (Emb: 0.803, TF-IDF: 0.723)}


\item \textbf{Artículo} \newline
La dirección superior del Ministerio Público reside en la o el Fiscal Nacional, quien durará seis años en el cargo, sin reelección. 
\newline {\color{gray} \textbf{1º:} 560-Iniciativa-Convencional-Constituyente-de-cc-Andres-Cruz-sobre-Ministerio-Publico-2016-hrs.-01-02.pdf}
\newline {\color{gray} (Emb: 0.922, TF-IDF: 0.738)}
\newline {\color{gray} \textbf{2º:} 608-Iniciativa-Convencional-Constituyente-de-cc-Miguel-Angel-Botto-sobre-Ministerio-Publico-2119-hrs.-01-02.pdf}
\newline {\color{gray} (Emb: 0.698, TF-IDF: 0.544)}

La o el Fiscal Nacional será nombrado por la mayoría de las y los integrantes del Congreso de Diputadas y Diputados y la Cámara de las Regiones en sesión conjunta, a partir de una terna propuesta por la o el Presidente de la República, quien contará con la asistencia técnica del Consejo de la Alta Dirección Pública, conforme al procedimiento que determine la ley. 
\newline {\color{gray} \textbf{1º:} 560-Iniciativa-Convencional-Constituyente-de-cc-Andres-Cruz-sobre-Ministerio-Publico-2016-hrs.-01-02.pdf}
\newline {\color{gray} (Emb: 0.905, TF-IDF: 0.645)}
\newline {\color{gray} \textbf{2º:} 730-Iniciativa-Convencional-Constituyente-del-cc-Adolfo-Millabur-sobre-Gobiernos-Locales.pdf}
\newline {\color{gray} (Emb: 0.780, TF-IDF: 0.337)}

Corresponderá al Fiscal Nacional: a) Dirigir las sesiones ordinarias y extraordinarias del Comité del Ministerio Público. 
\newline {\color{gray} \textbf{1º:} 560-Iniciativa-Convencional-Constituyente-de-cc-Andres-Cruz-sobre-Ministerio-Publico-2016-hrs.-01-02.pdf}
\newline {\color{gray} (Emb: 0.855, TF-IDF: 0.799)}
\newline {\color{gray} \textbf{2º:} 377-2-Iniciativa-Convencional-Constituyente-del-cc-Alvin-Saldana-sobre-Mecanismos-de-Democracia-1045-hrs-24-01.pdf}
\newline {\color{gray} (Emb: 0.715, TF-IDF: 0.272)}

b) Representar a la institución ante los demás órganos del Estado. 
\newline {\color{gray} \textbf{1º:} 679-Iniciativa-Convencional-Constituyente-del-cc-Francisco-Caamano-sobre-Crisis-Climatica-y-Ecologica-121101-02.pdf}
\newline {\color{gray} (Emb: 0.641, TF-IDF: 0.406)}
\newline {\color{gray} \textbf{2º:} 574-Iniciativa-Convencional-Constituyente-de-cc-Vanessa-Hoppe-sobre-Defensoria-de-los-Pueblos-2351-hrs.-01-02.pdf}
\newline {\color{gray} (Emb: 0.639, TF-IDF: 0.372)}

c) Impulsar la ejecución de la política de persecución penal en el país. 
\newline {\color{gray} \textbf{1º:} 578-Iniciativa-Convencional-Constituyente-de-cc-Christian-Viera-sobre-Ministerio-Publico-2308-hrs.-01-02.pdf}
\newline {\color{gray} (Emb: 0.765, TF-IDF: 0.466)}
\newline {\color{gray} \textbf{2º:} 608-Iniciativa-Convencional-Constituyente-de-cc-Miguel-Angel-Botto-sobre-Ministerio-Publico-2119-hrs.-01-02.pdf}
\newline {\color{gray} (Emb: 0.707, TF-IDF: 0.455)}

d) Determinar la política de gestión profesional de las y los funcionarios del Ministerio Público. 
\newline {\color{gray} \textbf{1º:} 608-Iniciativa-Convencional-Constituyente-de-cc-Miguel-Angel-Botto-sobre-Ministerio-Publico-2119-hrs.-01-02.pdf}
\newline {\color{gray} (Emb: 0.682, TF-IDF: 0.355)}
\newline {\color{gray} \textbf{2º:} 608-Iniciativa-Convencional-Constituyente-de-cc-Miguel-Angel-Botto-sobre-Ministerio-Publico-2119-hrs.-01-02.pdf}
\newline {\color{gray} (Emb: 0.667, TF-IDF: 0.355)}

e) Presidir el Comité del Ministerio Público. 
\newline {\color{gray} \textbf{1º:} 325-6-Iniciativa-Convencional-del-cc-Tomas-Laibe-sobre-Jurisdiccion-Constitucional.pdf}
\newline {\color{gray} (Emb: 0.560, TF-IDF: 0.392)}
\newline {\color{gray} \textbf{2º:} 184-6-c-Iniciativa-Convencional-del-cc-Rodrigo-Álvarez-que-crea-la-Corte-Constitucional-1044hrs.pdf}
\newline {\color{gray} (Emb: 0.532, TF-IDF: 0.337)}

f) Designar a los fiscales regionales, a partir de una terna elaborada por la Asamblea Regional respectiva. 
\newline {\color{gray} \textbf{1º:} 529-4-Iniciativa-Convencional-Constituyente-de-cc-Maria-Magdalena-Rivera-sobre-Fin-a-Prision-Politica-1355-hrs.-01-02.pdf}
\newline {\color{gray} (Emb: 0.650, TF-IDF: 0.382)}
\newline {\color{gray} \textbf{2º:} 452-4-Iniciativa-Convencional-Constituyente-de-la-cc-Maria-Magdalena-Rivera-sobre-Prision-Politica-1520-31-01.pdf}
\newline {\color{gray} (Emb: 0.650, TF-IDF: 0.358)}

g) Designar a los fiscales adjuntos, a partir de una terna elaborada por el Comité del Ministerio Público. 
\newline {\color{gray} \textbf{1º:} 957-5-Iniciativa-Convencional-Constituyente-de-la-cc-Ivanna-Olivares-sobre-Nuevo-Modelo-Economico.pdf}
\newline {\color{gray} (Emb: 0.643, TF-IDF: 0.424)}
\newline {\color{gray} \textbf{2º:} 924-Iniciativa-Convencional-Constituyente-de-la-cc-Constanza-Schonhaut-sobre-Administracion-del-Estado.pdf}
\newline {\color{gray} (Emb: 0.612, TF-IDF: 0.326)}

h) Las demás atribuciones que establezca la Constitución y la ley. 
\newline {\color{gray} \textbf{1º:} 122-3-c-Iniciativa-de-la-cc-Jennifer-Mella-Forma-del-Estado.pdf}
\newline {\color{gray} (Emb: 0.699, TF-IDF: 0.308)}
\newline {\color{gray} \textbf{2º:} 560-Iniciativa-Convencional-Constituyente-de-cc-Andres-Cruz-sobre-Ministerio-Publico-2016-hrs.-01-02.pdf}
\newline {\color{gray} (Emb: 0.636, TF-IDF: 0.272)}


\item \textbf{Artículo} \newline
La o el Fiscal Nacional debe tener a lo menos quince años de título de abogado, tener ciudadanía con derecho a sufragio y contar con comprobadas competencias para el cargo. 
\newline {\color{gray} \textbf{1º:} 560-Iniciativa-Convencional-Constituyente-de-cc-Andres-Cruz-sobre-Ministerio-Publico-2016-hrs.-01-02.pdf}
\newline {\color{gray} (Emb: 0.668, TF-IDF: 0.351)}
\newline {\color{gray} \textbf{2º:} 159-3-c-Iniciativa-de-la-cc-Jennifer-Mella-.pdf}
\newline {\color{gray} (Emb: 0.642, TF-IDF: 0.299)}


\item \textbf{Artículo} \newline
b) Evaluar y calificar permanentemente el desempeño de las y los funcionarios del Ministerio Público. 
\newline {\color{gray} \textbf{1º:} 151-3-c-Iniciativa-de-la-cc-Angelica-Tepper-Competencias-de-los-Gobiernos-Regionales.pdf}
\newline {\color{gray} (Emb: 0.952, TF-IDF: 0.488)}
\newline {\color{gray} \textbf{2º:} 151-3-c-Iniciativa-de-la-cc-Angelica-Tepper-Competencias-de-los-Gobiernos-Regionales.pdf}
\newline {\color{gray} (Emb: 0.952, TF-IDF: 0.488)}

c) Ejercer la potestad disciplinaria respecto de las y los funcionarios del Ministerio Público, en conformidad a la ley. 
\newline {\color{gray} \textbf{1º:} 850-Iniciativa-Convencional-Constituyente-de-la-cc-Patricia-Labra-sobre-Ministerio-Publico.pdf}
\newline {\color{gray} (Emb: 0.802, TF-IDF: 0.630)}
\newline {\color{gray} \textbf{2º:} 173-6-c-Iniciativa-Convencional-del-cc-Rodrigo-Álvarez-Contraloría-1044-hrs.pdf}
\newline {\color{gray} (Emb: 0.746, TF-IDF: 0.568)}

d) Designar al Director Ejecutivo Nacional. 
\newline {\color{gray} \textbf{1º:} 560-Iniciativa-Convencional-Constituyente-de-cc-Andres-Cruz-sobre-Ministerio-Publico-2016-hrs.-01-02.pdf}
\newline {\color{gray} (Emb: 0.602, TF-IDF: 0.420)}
\newline {\color{gray} \textbf{2º:} 560-Iniciativa-Convencional-Constituyente-de-cc-Andres-Cruz-sobre-Ministerio-Publico-2016-hrs.-01-02.pdf}
\newline {\color{gray} (Emb: 0.596, TF-IDF: 0.367)}

e) Proponer al Fiscal Nacional las ternas para el nombramiento de los fiscales adjuntos. 
\newline {\color{gray} \textbf{1º:} 909-Iniciativa-Convencional-Constituyente-del-cc-Hugo-Gutierrez-Sobre-Ministerio-Publico.pdf}
\newline {\color{gray} (Emb: 0.582, TF-IDF: 0.231)}
\newline {\color{gray} \textbf{2º:} 122-3-c-Iniciativa-de-la-cc-Jennifer-Mella-Forma-del-Estado.pdf}
\newline {\color{gray} (Emb: 0.516, TF-IDF: 0.221)}

f) Las demás atribuciones que establezca la Constitución y la ley. 
\newline {\color{gray} \textbf{1º:} 159-3-c-Iniciativa-de-la-cc-Jennifer-Mella-.pdf}
\newline {\color{gray} (Emb: 0.702, TF-IDF: 0.331)}
\newline {\color{gray} \textbf{2º:} 122-3-c-Iniciativa-de-la-cc-Jennifer-Mella-Forma-del-Estado.pdf}
\newline {\color{gray} (Emb: 0.702, TF-IDF: 0.310)}

Son atribuciones del Comité del Ministerio Público las siguientes: a) Asesorar al Fiscal Nacional en la dirección del organismo, velando por el cumplimiento de sus objetivos. 
\newline {\color{gray} \textbf{1º:} 95-6-Iniciativa-Convencional-Constituyente-de-Cc-Mauricio-Daza-y-otros-2.pdf}
\newline {\color{gray} (Emb: 0.615, TF-IDF: 0.415)}
\newline {\color{gray} \textbf{2º:} 560-Iniciativa-Convencional-Constituyente-de-cc-Andres-Cruz-sobre-Ministerio-Publico-2016-hrs.-01-02.pdf}
\newline {\color{gray} (Emb: 0.611, TF-IDF: 0.321)}


\item \textbf{Artículo} \newline
Existirá un Comité del Ministerio Público, integrado por las y los fiscales regionales y la o el Fiscal Nacional, quien lo presidirá. 
\newline {\color{gray} \textbf{1º:} 374-2-Iniciativa-Convencional-Constituyente-de-la-cc-Loreto-Vallejos-sobre-Democracia-Directa-0900-hrs-24-01.pdf}
\newline {\color{gray} (Emb: 0.688, TF-IDF: 0.367)}
\newline {\color{gray} \textbf{2º:} 557-Iniciativa-Convencional-Constituyente-de-cc-Andres-Cruz-sobre-Agencia-Nacional-del-agua-2013-hrs.-01-02.pdf}
\newline {\color{gray} (Emb: 0.616, TF-IDF: 0.355)}

El Comité deberá fijar la política de persecución penal y los criterios de actuación para el cumplimiento de dichos objetivos, debiendo siempre velar por la transparencia y objetividad, resguardando los intereses de la sociedad y el respeto de los derechos humanos. 
\newline {\color{gray} \textbf{1º:} 608-Iniciativa-Convencional-Constituyente-de-cc-Miguel-Angel-Botto-sobre-Ministerio-Publico-2119-hrs.-01-02.pdf}
\newline {\color{gray} (Emb: 0.696, TF-IDF: 0.512)}
\newline {\color{gray} \textbf{2º:} 179-6-c-Iniciativa-Convencional-DEL-CC-Rodrigo-Álvarez-Ministerio-Público-1044-hrs.pdf}
\newline {\color{gray} (Emb: 0.641, TF-IDF: 0.379)}


\item \textbf{Artículo} \newline
Existirán fiscales adjuntos del Ministerio Público quienes ejercerán su labor en los casos específicos que se les asignen, conforme a lo establecido en la Constitución y las leyes. 
\newline {\color{gray} \textbf{1º:} 151-3-c-Iniciativa-de-la-cc-Angelica-Tepper-Competencias-de-los-Gobiernos-Regionales.pdf}
\newline {\color{gray} (Emb: 0.939, TF-IDF: 0.506)}
\newline {\color{gray} \textbf{2º:} 151-3-c-Iniciativa-de-la-cc-Angelica-Tepper-Competencias-de-los-Gobiernos-Regionales.pdf}
\newline {\color{gray} (Emb: 0.939, TF-IDF: 0.506)}


\item \textbf{Artículo} \newline
La o el Fiscal Nacional y las y los fiscales regionales deberán rendir, anualmente, una cuenta pública de su gestión. 
\newline {\color{gray} \textbf{1º:} 560-Iniciativa-Convencional-Constituyente-de-cc-Andres-Cruz-sobre-Ministerio-Publico-2016-hrs.-01-02.pdf}
\newline {\color{gray} (Emb: 0.775, TF-IDF: 0.498)}
\newline {\color{gray} \textbf{2º:} 230-2-Iniciativa-Convencional-de-la-cc-Alondra-Carrillo-sobre-Participacion-en-la-Democracia-1142-hrs.pdf}
\newline {\color{gray} (Emb: 0.773, TF-IDF: 0.472)}

En el caso de la o el Fiscal Nacional se rendirá la cuenta ante el Congreso, y en el caso de las y los fiscales regionales ante la Asamblea Regional respectiva. 
\newline {\color{gray} \textbf{1º:} 560-Iniciativa-Convencional-Constituyente-de-cc-Andres-Cruz-sobre-Ministerio-Publico-2016-hrs.-01-02.pdf}
\newline {\color{gray} (Emb: 0.744, TF-IDF: 0.692)}
\newline {\color{gray} \textbf{2º:} 560-Iniciativa-Convencional-Constituyente-de-cc-Andres-Cruz-sobre-Ministerio-Publico-2016-hrs.-01-02.pdf}
\newline {\color{gray} (Emb: 0.564, TF-IDF: 0.366)}


\item \textbf{Artículo} \newline
La remoción de los fiscales regionales también podrá ser solicitada por el o la Fiscal Nacional. 
\newline {\color{gray} \textbf{1º:} 230-2-Iniciativa-Convencional-de-la-cc-Alondra-Carrillo-sobre-Participacion-en-la-Democracia-1142-hrs.pdf}
\newline {\color{gray} (Emb: 0.682, TF-IDF: 0.602)}
\newline {\color{gray} \textbf{2º:} 120-3-c-Iniciativa-de-la-cc-Tammy-Pustilnick-atribuciones-exclusivas-de-la-Asamblea-Regional.pdf}
\newline {\color{gray} (Emb: 0.672, TF-IDF: 0.550)}

La Corte conocerá del asunto en pleno especialmente convocado al efecto y para acordar la remoción deberá reunir el voto conforme de la mayoría de sus miembros en ejercicio. 
\newline {\color{gray} \textbf{1º:} 608-Iniciativa-Convencional-Constituyente-de-cc-Miguel-Angel-Botto-sobre-Ministerio-Publico-2119-hrs.-01-02.pdf}
\newline {\color{gray} (Emb: 0.943, TF-IDF: 0.836)}
\newline {\color{gray} \textbf{2º:} 560-Iniciativa-Convencional-Constituyente-de-cc-Andres-Cruz-sobre-Ministerio-Publico-2016-hrs.-01-02.pdf}
\newline {\color{gray} (Emb: 0.754, TF-IDF: 0.704)}

El Fiscal Nacional y los fiscales regionales serán removidos por la Corte Suprema, a requerimiento del Presidente de la República, del Congreso de Diputadas y Diputados, o de diez de sus miembros, por incapacidad, falta grave a la probidad o negligencia manifiesta en el ejercicio de sus funciones. 
\newline {\color{gray} \textbf{1º:} 241-1-Iniciativa-Convencional-de-la-cc-Alejandra-Flores-sobre-Buen-Gobierno-1146-hrs.pdf}
\newline {\color{gray} (Emb: 0.697, TF-IDF: 0.440)}
\newline {\color{gray} \textbf{2º:} 608-Iniciativa-Convencional-Constituyente-de-cc-Miguel-Angel-Botto-sobre-Ministerio-Publico-2119-hrs.-01-02.pdf}
\newline {\color{gray} (Emb: 0.695, TF-IDF: 0.383)}


\item \textbf{Artículo} \newline
Toda persona tiene derecho a un proceso razonable y justo, en que se salvaguarden las garantías que se señalan en esta Constitución, sin perjuicio de las que se establezcan en la ley y en los tratados internacionales que se encuentren vigentes y hayan sido ratificados por Chile. 
\newline {\color{gray} \textbf{1º:} 179-6-c-Iniciativa-Convencional-DEL-CC-Rodrigo-Álvarez-Ministerio-Público-1044-hrs.pdf}
\newline {\color{gray} (Emb: 0.848, TF-IDF: 0.670)}
\newline {\color{gray} \textbf{2º:} 578-Iniciativa-Convencional-Constituyente-de-cc-Christian-Viera-sobre-Ministerio-Publico-2308-hrs.-01-02.pdf}
\newline {\color{gray} (Emb: 0.828, TF-IDF: 0.648)}

Dicho proceso se realizará ante el tribunal competente, independiente e imparcial, establecido con anterioridad por la ley. 
\newline {\color{gray} \textbf{1º:} 465-6-Iniciativa-Convencional-Constituyente-de-la-cc-Adriana-Cancino-sobre-Justicia-Electoral-1925-31-01.pdf}
\newline {\color{gray} (Emb: 1.000, TF-IDF: 1.000)}
\newline {\color{gray} \textbf{2º:} 400-1-Iniciativa-Convencional-Constituyente-de-la-cc-Constanza-Hube-sobre-Servicio-y-Registro-Electoral-1905-24-01.pdf}
\newline {\color{gray} (Emb: 1.000, TF-IDF: 1.000)}

Toda persona tiene derecho a ser oída y juzgada en igualdad de condiciones y dentro de un plazo razonable. 
\newline {\color{gray} \textbf{1º:} 179-6-c-Iniciativa-Convencional-DEL-CC-Rodrigo-Álvarez-Ministerio-Público-1044-hrs.pdf}
\newline {\color{gray} (Emb: 0.672, TF-IDF: 0.439)}
\newline {\color{gray} \textbf{2º:} 230-2-Iniciativa-Convencional-de-la-cc-Alondra-Carrillo-sobre-Participacion-en-la-Democracia-1142-hrs.pdf}
\newline {\color{gray} (Emb: 0.658, TF-IDF: 0.328)}

Las sentencias serán fundadas, asegurando la existencia de un recurso adecuado y efectivo ante el tribunal que determine la ley. 
\newline {\color{gray} \textbf{1º:} 444-Iniciativa-Convencional-Constituyente-del-cc-Mauricio-Daza-sobre-Debido-Proceso-1406-28-01.pdf}
\newline {\color{gray} (Emb: 0.889, TF-IDF: 0.662)}
\newline {\color{gray} \textbf{2º:} 515-4-Iniciativa-Convencional-Constituyente-de-la-cc-Giovanna-Grandon-sobre-Derecho-a-Migrar-1245-01-02.pdf}
\newline {\color{gray} (Emb: 0.779, TF-IDF: 0.459)}

Toda persona tiene derecho a defensa jurídica y ninguna autoridad o individuo podrá impedir, restringir o perturbar la debida intervención del letrado. 
\newline {\color{gray} \textbf{1º:} 180-6-c-Iniciativa-Convencional-del-cc-Rodrigo-Álvarez-que-regula-el-Poder-Judicial-1044-hrs.pdf}
\newline {\color{gray} (Emb: 0.751, TF-IDF: 0.463)}
\newline {\color{gray} \textbf{2º:} 147-4-c-Iniciativa-del-cc-Manuel-Jose-Ossandon-Garantias-Procesales-y-Seguridad-Individual.pdf}
\newline {\color{gray} (Emb: 0.692, TF-IDF: 0.428)}


\item \textbf{Artículo} \newline
Los procedimientos judiciales serán establecidos por ley. 
\newline {\color{gray} \textbf{1º:} 128-4-c-Iniciativa-de-la-cc-Rocio-Cantuarias-Incorpora-una-Garantia-Procesal-en-los-procesos-judiciales.pdf}
\newline {\color{gray} (Emb: 0.748, TF-IDF: 0.626)}
\newline {\color{gray} \textbf{2º:} 147-4-c-Iniciativa-del-cc-Manuel-Jose-Ossandon-Garantias-Procesales-y-Seguridad-Individual.pdf}
\newline {\color{gray} (Emb: 0.743, TF-IDF: 0.450)}


\item \textbf{Artículo} \newline
La Constitución asegura la asistencia y ajustes de procedimientos necesarios y adecuados a la edad o discapacidad de las personas, según corresponda, a fin de poder de que ellas puedan intervenir debidamente en el proceso. 
\newline {\color{gray} \textbf{1º:} 400-1-Iniciativa-Convencional-Constituyente-de-la-cc-Constanza-Hube-sobre-Servicio-y-Registro-Electoral-1905-24-01.pdf}
\newline {\color{gray} (Emb: 0.734, TF-IDF: 0.298)}
\newline {\color{gray} \textbf{2º:} 445-Iniciativa-Convencional-Constituyente-de-la-cc-Manuela-Royo-sobre-Tutela-Judicial-Efectiva-1407-28-01.pdf}
\newline {\color{gray} (Emb: 0.727, TF-IDF: 0.280)}


\item \textbf{Artículo} \newline
e) A ser informada, sin demora y en forma detallada, de sus derechos y causa de la investigación seguida en su contra. 
\newline {\color{gray} \textbf{1º:} 1031-Iniciativa-Convencional-Constituyente-del-cc-Tomas-Laibe-sobre-Personas-Privadas-de-Libertad.pdf}
\newline {\color{gray} (Emb: 0.496, TF-IDF: 0.456)}
\newline {\color{gray} \textbf{2º:} 325-6-Iniciativa-Convencional-del-cc-Tomas-Laibe-sobre-Jurisdiccion-Constitucional.pdf}
\newline {\color{gray} (Emb: 0.486, TF-IDF: 0.443)}

k) A que no se le imponga como pena la pérdida de los derechos previsionales. 
\newline {\color{gray} \textbf{1º:} 514-4-Iniciativa-Convencional-Constituyente-del-cc-Felipe-Harboe-sobre-Derecho-a-la-Privacidad-1245-01-02.pdf}
\newline {\color{gray} (Emb: 0.907, TF-IDF: 0.883)}
\newline {\color{gray} \textbf{2º:} 440-Iniciativa-Convencional-Constituyente-del-cc-Felipe-Harboe-sobre-Principios-del-Debido-Proceso-1401-28-01.pdf}
\newline {\color{gray} (Emb: 0.907, TF-IDF: 0.883)}

j) A que no se le imponga la pena de confiscación de bienes, sin perjuicio del comiso en los casos establecidos por las leyes. 
\newline {\color{gray} \textbf{1º:} 271-4-Iniciativa-Convencional-de-la-cc-Natalia-Henriquez-sobre-Libertad-Personal-17-01-1154-hrs.pdf}
\newline {\color{gray} (Emb: 0.723, TF-IDF: 0.410)}
\newline {\color{gray} \textbf{2º:} 130-4-c-Iniciativa-de-la-cc-Rocio-Cantuarias-Incorpora-obligaciones-del-Estado.pdf}
\newline {\color{gray} (Emb: 0.595, TF-IDF: 0.273)}

i) A ser sancionada de forma proporcional a la infracción cometida. 
\newline {\color{gray} \textbf{1º:} 271-4-Iniciativa-Convencional-de-la-cc-Natalia-Henriquez-sobre-Libertad-Personal-17-01-1154-hrs.pdf}
\newline {\color{gray} (Emb: 0.874, TF-IDF: 0.593)}
\newline {\color{gray} \textbf{2º:} 262-4-Iniciativa-Convencional-de-la-cc-Janis-Meneses-sobre-Derecho-a-la-Libertad-de-Opinion-1153-hrs.pdf}
\newline {\color{gray} (Emb: 0.644, TF-IDF: 0.369)}

h) A no ser sometida a un nuevo procedimiento, investigación o persecución penal por el mismo hecho respecto del cual haya sido condenada, absuelta o sobreseída definitivamente por sentencia ejecutoriada. 
\newline {\color{gray} \textbf{1º:} 210-1-c-Iniciativa-Convencional-del-cc-Cristián-Monckeberg-sobre-Estado-Intercultural-1953-hrs.pdf}
\newline {\color{gray} (Emb: 0.587, TF-IDF: 0.361)}
\newline {\color{gray} \textbf{2º:} 758-Iniciativa-Convencional-Constituyente-Irrenunciabilidad-de-la-Nacionalidad-Saldana.pdf}
\newline {\color{gray} (Emb: 0.556, TF-IDF: 0.331)}

Las medidas cautelares personales son excepcionales, temporales y proporcionales, debiendo la ley regular los casos de procedencia y requisitos. 
\newline {\color{gray} \textbf{1º:} 608-Iniciativa-Convencional-Constituyente-de-cc-Miguel-Angel-Botto-sobre-Ministerio-Publico-2119-hrs.-01-02.pdf}
\newline {\color{gray} (Emb: 0.485, TF-IDF: 0.544)}
\newline {\color{gray} \textbf{2º:} 391-4-Iniciativa-Convencional-Constituyente-de-la-cc-Valentina-Miranda-sobre-Ninos-Ninas-y-Adolescentes-1440-24-01.pdf}
\newline {\color{gray} (Emb: 0.471, TF-IDF: 0.544)}

g) A que su libertad sea la regla general. 
\newline {\color{gray} \textbf{1º:} 244-4-Iniciativa-Convencional-de-la-cc-Elsa-Labrana-sobre-Derecho-a-la-Personalidad-1147-hrs.pdf}
\newline {\color{gray} (Emb: 0.544, TF-IDF: 0.526)}
\newline {\color{gray} \textbf{2º:} 440-Iniciativa-Convencional-Constituyente-del-cc-Felipe-Harboe-sobre-Principios-del-Debido-Proceso-1401-28-01.pdf}
\newline {\color{gray} (Emb: 0.530, TF-IDF: 0.526)}

d) A que no se presuma de derecho la responsabilidad penal. 
\newline {\color{gray} \textbf{1º:} 860-Iniciativa-Convencional-Constituyente-del-cc-Jorge-Abarca-sobre-Regimen-Publico-Economico.pdf}
\newline {\color{gray} (Emb: 0.661, TF-IDF: 0.268)}
\newline {\color{gray} \textbf{2º:} 909-Iniciativa-Convencional-Constituyente-del-cc-Hugo-Gutierrez-Sobre-Ministerio-Publico.pdf}
\newline {\color{gray} (Emb: 0.653, TF-IDF: 0.256)}

f) A guardar silencio ni a ser obligada a declarar contra sí misma o reconocer su responsabilidad. 
\newline {\color{gray} \textbf{1º:} 319-6-Iniciativa-Convencional-del-cc-Mauricio-Daza-sobre-el-Sistema-Nacional-de-Justicia17-09-hrs.pdf}
\newline {\color{gray} (Emb: 0.742, TF-IDF: 0.409)}
\newline {\color{gray} \textbf{2º:} 319-6-Iniciativa-Convencional-del-cc-Mauricio-Daza-sobre-el-Sistema-Nacional-de-Justicia17-09-hrs.pdf}
\newline {\color{gray} (Emb: 0.742, TF-IDF: 0.409)}

l) A que la detención o la internación de una o un adolescente se utilice sólo de forma excepcional, durante el período más breve que proceda y conforme a lo establecido en esta Constitución, la ley y los tratados internacionales de derechos humanos. 
\newline {\color{gray} \textbf{1º:} 784-niciativa-Convencional-Constituyente-de-la-cc-Damaris-Abarca-sobre-Derecho-a-vivir-en-un-ambiente-sano.pdf}
\newline {\color{gray} (Emb: 0.633, TF-IDF: 0.368)}
\newline {\color{gray} \textbf{2º:} 272-4-Iniciativa-Convencional-de-la-cc-Natalia-Henriquez-sobre-Garantias-Procesales-17-01-1155-hrs.pdf}
\newline {\color{gray} (Emb: 0.620, TF-IDF: 0.296)}

c) A que se presuma su inocencia mientras no exista una sentencia condenatoria firme dictada en su contra. 
\newline {\color{gray} \textbf{1º:} 27-4-Iniciativa-Convencional-Constituyente-de-la-cc-Adriana-Cancino-y-otros.pdf}
\newline {\color{gray} (Emb: 0.544, TF-IDF: 0.288)}
\newline {\color{gray} \textbf{2º:} 603-Iniciativa-Convencional-Constituyente-de-cc-Jorge-Baradit-sobre-Nacionalidad-y-Ciudadania-2105-hrs.-01-02.pdf}
\newline {\color{gray} (Emb: 0.502, TF-IDF: 0.253)}

b) A conocer los antecedentes de la investigación seguida en su contra, salvo las excepciones que la ley señale. 
\newline {\color{gray} \textbf{1º:} 322-1-Iniciativa-Convencional-Constituyente-de-la-cc-Barbara-Sepulveda-sobre-Formacion-de-la-Ley.pdf}
\newline {\color{gray} (Emb: 0.915, TF-IDF: 0.505)}
\newline {\color{gray} \textbf{2º:} 234-1-Iniciativa-Convencional-del-cc-Jaime-Bassa-sobre-Justicia-Complementaria-1144-hrs.pdf}
\newline {\color{gray} (Emb: 0.817, TF-IDF: 0.503)}

Toda persona tiene derecho a las siguientes garantías procesales penales mínimas: a) A que toda actuación de la investigación o procedimiento que le prive, restrinja o perturbe el ejercicio de los derechos que asegura la Constitución, requiere previa autorización judicial. 
\newline {\color{gray} \textbf{1º:} 317-6-Iniciativa-Convencional-del-cc-Cristian-Monckeberg-sobre-Derecho-de-Acceso-a-la-Justicia-16-00-hrs.pdf}
\newline {\color{gray} (Emb: 0.919, TF-IDF: 0.855)}
\newline {\color{gray} \textbf{2º:} 128-4-c-Iniciativa-de-la-cc-Rocio-Cantuarias-Incorpora-una-Garantia-Procesal-en-los-procesos-judiciales.pdf}
\newline {\color{gray} (Emb: 0.919, TF-IDF: 0.855)}

No podrán ser obligados a declarar en contra del imputado sus ascendientes, descendientes, cónyuge, conviviente civil y demás personas que señale la ley. 
\newline {\color{gray} \textbf{1º:} 147-4-c-Iniciativa-del-cc-Manuel-Jose-Ossandon-Garantias-Procesales-y-Seguridad-Individual.pdf}
\newline {\color{gray} (Emb: 0.938, TF-IDF: 0.588)}
\newline {\color{gray} \textbf{2º:} 128-4-c-Iniciativa-de-la-cc-Rocio-Cantuarias-Incorpora-una-Garantia-Procesal-en-los-procesos-judiciales.pdf}
\newline {\color{gray} (Emb: 0.876, TF-IDF: 0.375)}


\item \textbf{Artículo} \newline
Ningún delito se castigará con otra pena que la señalada por una ley que haya entrado en vigencia con anterioridad a su perpetración, a menos que una nueva ley favorezca al imputado. 
\newline {\color{gray} \textbf{1º:} 272-4-Iniciativa-Convencional-de-la-cc-Natalia-Henriquez-sobre-Garantias-Procesales-17-01-1155-hrs.pdf}
\newline {\color{gray} (Emb: 0.677, TF-IDF: 0.441)}
\newline {\color{gray} \textbf{2º:} 514-4-Iniciativa-Convencional-Constituyente-del-cc-Felipe-Harboe-sobre-Derecho-a-la-Privacidad-1245-01-02.pdf}
\newline {\color{gray} (Emb: 0.588, TF-IDF: 0.306)}

Ninguna ley podrá establecer penas sin que la conducta que se sanciona esté descrita de manera clara y precisa en ella. 
\newline {\color{gray} \textbf{1º:} 881-Iniciativa-Convencional-Constituyente-de-la-cc-Ingrid-Villena-sobre-Garantias-Procesales-para-NNA.pdf}
\newline {\color{gray} (Emb: 0.646, TF-IDF: 0.427)}
\newline {\color{gray} \textbf{2º:} 11-4-Iniciativa-Convencional-Constituyente-de-la-cc-María-Elisa-Quinteros-y-otras.pdf}
\newline {\color{gray} (Emb: 0.602, TF-IDF: 0.267)}

Ninguna persona podrá ser condenada por acciones u omisiones que al producirse no constituyan delito, según la legislación vigente en aquel momento. 
\newline {\color{gray} \textbf{1º:} 131-4-c-Iniciativa-de-la-cc-Rocio-Cantuarias-Establece-la-Libertad-Personal-y-la-Seguridad-Individual.pdf}
\newline {\color{gray} (Emb: 0.684, TF-IDF: 0.404)}
\newline {\color{gray} \textbf{2º:} 89-6-Iniciativa-Convencional-Constituyente-del-cc-Christian-Viera-y-otros.pdf}
\newline {\color{gray} (Emb: 0.631, TF-IDF: 0.327)}

Lo establecido en este artículo también será aplicable a las medidas de seguridad. 
\newline {\color{gray} \textbf{1º:} 440-Iniciativa-Convencional-Constituyente-del-cc-Felipe-Harboe-sobre-Principios-del-Debido-Proceso-1401-28-01.pdf}
\newline {\color{gray} (Emb: 0.753, TF-IDF: 0.278)}
\newline {\color{gray} \textbf{2º:} 514-4-Iniciativa-Convencional-Constituyente-del-cc-Felipe-Harboe-sobre-Derecho-a-la-Privacidad-1245-01-02.pdf}
\newline {\color{gray} (Emb: 0.753, TF-IDF: 0.273)}


\item \textbf{Artículo} \newline
El Estado asegura la asesoría jurídica gratuita e integra por parte de abogadas y abogados habilitados para el ejercicio de la profesión, a toda persona que no pueda obtenerla por sí misma. 
\newline {\color{gray} \textbf{1º:} 440-Iniciativa-Convencional-Constituyente-del-cc-Felipe-Harboe-sobre-Principios-del-Debido-Proceso-1401-28-01.pdf}
\newline {\color{gray} (Emb: 0.922, TF-IDF: 0.867)}
\newline {\color{gray} \textbf{2º:} 514-4-Iniciativa-Convencional-Constituyente-del-cc-Felipe-Harboe-sobre-Derecho-a-la-Privacidad-1245-01-02.pdf}
\newline {\color{gray} (Emb: 0.922, TF-IDF: 0.867)}

Toda persona tiene derecho a la asesoría jurídica gratuita en los casos y en la forma que establezca la Constitución y la ley. 
\newline {\color{gray} \textbf{1º:} 128-4-c-Iniciativa-de-la-cc-Rocio-Cantuarias-Incorpora-una-Garantia-Procesal-en-los-procesos-judiciales.pdf}
\newline {\color{gray} (Emb: 0.984, TF-IDF: 0.774)}
\newline {\color{gray} \textbf{2º:} 440-Iniciativa-Convencional-Constituyente-del-cc-Felipe-Harboe-sobre-Principios-del-Debido-Proceso-1401-28-01.pdf}
\newline {\color{gray} (Emb: 0.984, TF-IDF: 0.774)}


\item \textbf{Artículo} \newline
Es deber del Estado otorgar asistencia jurídica especializada para la protección del interés superior de los niños, niñas y adolescentes, especialmente cuando estos han sido sujetos de medidas de protección, procurando crear todas las condiciones necesarias para el resguardo de sus derechos. 
\newline {\color{gray} \textbf{1º:} 610-Iniciativa-Convencional-Constituyente-de-cc-Valentina-Miranda-Derechos-de-las-Personas-LGBTIQ-y-Derecho-a-la-Igualdad.pdf}
\newline {\color{gray} (Emb: 0.523, TF-IDF: 0.534)}
\newline {\color{gray} \textbf{2º:} 942-Iniciativa-Convencional-Constituyente-de-la-cc-Malucha-Pinto-sobre-Trabajadores-del-Conocimiento.pdf}
\newline {\color{gray} (Emb: 0.491, TF-IDF: 0.358)}


\item \textbf{Artículo} \newline
Un organismo desconcentrado de carácter técnico, denominado Servicio Integral de Acceso a la Justicia, tendrá por función prestar asesoría, defensa y representación letrada de calidad a las personas, así como también brindar apoyo profesional de tipo psicológico y social en los casos que corresponda. 
\newline {\color{gray} \textbf{1º:} 222-7-Iniciativa-Convencional-del-cc-Carlos-Calvo-sobre-Derechos-a-la-Comunicacion-2351-hrs.pdf}
\newline {\color{gray} (Emb: 0.745, TF-IDF: 0.436)}
\newline {\color{gray} \textbf{2º:} 445-Iniciativa-Convencional-Constituyente-de-la-cc-Manuela-Royo-sobre-Tutela-Judicial-Efectiva-1407-28-01.pdf}
\newline {\color{gray} (Emb: 0.736, TF-IDF: 0.386)}

La ley determinará la organización, áreas de atención, composición y planta de personal del Servicio Integral de Acceso a la Justicia, considerando un despliegue territorialmente desconcentrado. 
\newline {\color{gray} \textbf{1º:} 445-Iniciativa-Convencional-Constituyente-de-la-cc-Manuela-Royo-sobre-Tutela-Judicial-Efectiva-1407-28-01.pdf}
\newline {\color{gray} (Emb: 0.989, TF-IDF: 0.907)}
\newline {\color{gray} \textbf{2º:} 317-6-Iniciativa-Convencional-del-cc-Cristian-Monckeberg-sobre-Derecho-de-Acceso-a-la-Justicia-16-00-hrs.pdf}
\newline {\color{gray} (Emb: 0.642, TF-IDF: 0.314)}


\item \textbf{Artículo} \newline
La Defensoría Penal Pública es un organismo autónomo, con personalidad jurídica y patrimonio propio, que tiene por función proporcionar defensa penal a los imputados por hechos que pudiesen ser constitutivos de delito, que deban ser conocidos por los tribunales con competencia en lo penal, desde la primera actuación de la investigación dirigida en su contra y hasta la completa ejecución de la pena que le haya sido impuesta, y que carezcan de defensa letrada. 
\newline {\color{gray} \textbf{1º:} 848-Iniciativa-Convencional-Constituyente-de-la-cc-Janis-Meneses-sobre-Derechos-de-los-NNA.pdf}
\newline {\color{gray} (Emb: 0.696, TF-IDF: 0.393)}
\newline {\color{gray} \textbf{2º:} 848-Iniciativa-Convencional-Constituyente-de-la-cc-Janis-Meneses-sobre-Derechos-de-los-NNA.pdf}
\newline {\color{gray} (Emb: 0.676, TF-IDF: 0.350)}

En las causas en que intervenga la Defensoría Penal Pública, podrá concurrir ante los organismos internacionales de derechos humanos. 
\newline {\color{gray} \textbf{1º:} 202-6-c-Iniciativa-Convencional-de-la-cc-Ingrid-Villena-que-crea-el-Sistema-Nacional-de-Defensa-Jurídica-Integral.pdf}
\newline {\color{gray} (Emb: 0.834, TF-IDF: 0.695)}
\newline {\color{gray} \textbf{2º:} 788-Iniciativa-Convencional-Constituyente-de-la-cc-Camila-Zarate-sobre-Democracia-Ecologica.pdf}
\newline {\color{gray} (Emb: 0.547, TF-IDF: 0.283)}

La ley determinará la organización y atribuciones de la Defensoría Penal Pública, debiendo garantizarse su independencia externa. 
\newline {\color{gray} \textbf{1º:} 97-6-Iniciativa-del-cc-Daniel-Bravo-Sistemas-de-Justicia.pdf}
\newline {\color{gray} (Emb: 0.662, TF-IDF: 0.319)}
\newline {\color{gray} \textbf{2º:} 711-Iniciativa-Convencional-Constituyente-de-la-cc-Ingrid-Villena-sobre-Tricel.pdf}
\newline {\color{gray} (Emb: 0.647, TF-IDF: 0.315)}


\item \textbf{Artículo} \newline
La función de defensa penal pública será ejercida por defensoras y defensores penales públicos. 
\newline {\color{gray} \textbf{1º:} 559-Iniciativa-Convencional-Constituyente-de-cc-Andres-Cruz-sobre-Defensoria-Penal-Publica-2017-hrs.-01-02.pdf}
\newline {\color{gray} (Emb: 0.865, TF-IDF: 0.690)}
\newline {\color{gray} \textbf{2º:} 805-Iniciativa-Convencional-Constituyente-del-cc-Hugo-Gutierrez-sobre-Defensoria-Penal-Publica.pdf}
\newline {\color{gray} (Emb: 0.728, TF-IDF: 0.464)}

Los servicios de defensa jurídica que preste la Defensoría Penal Pública no podrán ser licitados o delegados en abogados particulares, sin perjuicio de la contratación excepcional que pueda realizar en los casos y forma que establezca la ley. 
\newline {\color{gray} \textbf{1º:} 451-4-Iniciativa-Convencional-Constituyente-de-la-cc-Carolina-Videla-sobre-Tortura-y-desaparicion-1409-31-01.pdf}
\newline {\color{gray} (Emb: 0.657, TF-IDF: 0.254)}
\newline {\color{gray} \textbf{2º:} 519-4-Iniciativa-Convencional-Constituyente-de-cc-Carolina-Videla-sobre-DDHH-y-Garantias-de-no-repeticion-1257-hrs.-01-02.pdf}
\newline {\color{gray} (Emb: 0.657, TF-IDF: 0.251)}


\item \textbf{Artículo} \newline
La Defensora o Defensor Nacional será nombrado por la mayoría de las y los integrantes del Congreso de Diputadas y Diputados y la Cámara de las Regiones en sesión conjunta, a partir de una terna propuesta por la o el Presidente de la República, conforme al procedimiento que determine la ley. 
\newline {\color{gray} \textbf{1º:} 302-4-Iniciativa-Convencional-del-cc-Fuad-Chahin-sobre-Derecho-al-trabajo-18-01.-1141-hrs.pdf}
\newline {\color{gray} (Emb: 0.695, TF-IDF: 0.385)}
\newline {\color{gray} \textbf{2º:} 959-1-Iniciativa-Convencional-Constituyente-de-la-cc-Rosa-Catrileo-sobre-Defensa-Plurinacional-1.pdf}
\newline {\color{gray} (Emb: 0.674, TF-IDF: 0.385)}

La dirección superior de la Defensoría Penal Pública será ejercida por la o el Defensor Nacional, quien durará seis años en su cargo, sin reelección. 
\newline {\color{gray} \textbf{1º:} 431-6-Iniciativa-Convencional-de-la-cc-Bessy-Gallardo-sobre-Defensoria-Penal-Publica-1145-27-01.pdf}
\newline {\color{gray} (Emb: 0.646, TF-IDF: 0.412)}
\newline {\color{gray} \textbf{2º:} 431-6-Iniciativa-Convencional-de-la-cc-Bessy-Gallardo-sobre-Defensoria-Penal-Publica-1145-27-01.pdf}
\newline {\color{gray} (Emb: 0.604, TF-IDF: 0.412)}


\item \textbf{Artículo} \newline
Un organismo autónomo, con personalidad jurídica y patrimonio propio, denominada Defensoría del Pueblo, tendrá por finalidad la promoción y protección de los derechos humanos asegurados en esta Constitución, en los tratados internacionales de derechos humanos ratificados por Chile, así como los emanados de los principios generales del derecho y de las normas imperativas reconocidas por el derecho internacional, ante los actos u omisiones de los órganos de la administración del Estado y de entidades privadas que ejerzan actividades de servicio o utilidad pública, en la forma que establezca la ley. 
\newline {\color{gray} \textbf{1º:} 431-6-Iniciativa-Convencional-de-la-cc-Bessy-Gallardo-sobre-Defensoria-Penal-Publica-1145-27-01.pdf}
\newline {\color{gray} (Emb: 0.689, TF-IDF: 0.266)}
\newline {\color{gray} \textbf{2º:} 238-1-Iniciativa-Convencional-de-la-cc-Tania-Madriaga-sobre-Buen-Gobierno-1146-hrs.pdf}
\newline {\color{gray} (Emb: 0.667, TF-IDF: 0.266)}

La Defensoría del Pueblo tendrá defensorías regionales, que funcionarán en forma desconcentrada, en conformidad a lo que establezca su ley. 
\newline {\color{gray} \textbf{1º:} 574-Iniciativa-Convencional-Constituyente-de-cc-Vanessa-Hoppe-sobre-Defensoria-de-los-Pueblos-2351-hrs.-01-02.pdf}
\newline {\color{gray} (Emb: 0.737, TF-IDF: 0.425)}
\newline {\color{gray} \textbf{2º:} 88-6-Iniciativa-Convencional-Constituyente-del-cc-Christian-Viera-y-otros.pdf}
\newline {\color{gray} (Emb: 0.658, TF-IDF: 0.398)}

La ley determinará las atribuciones, organización, funcionamiento y procedimientos de la Defensoría del Pueblo. 
\newline {\color{gray} \textbf{1º:} 578-Iniciativa-Convencional-Constituyente-de-cc-Christian-Viera-sobre-Ministerio-Publico-2308-hrs.-01-02.pdf}
\newline {\color{gray} (Emb: 0.757, TF-IDF: 0.407)}
\newline {\color{gray} \textbf{2º:} 631-Iniciativa-Convencional-Constituyente-de-cc-Ingrid-Villena-sobre-Contraloria-General-de-la-Republica.pdf}
\newline {\color{gray} (Emb: 0.703, TF-IDF: 0.400)}


\item \textbf{Artículo} \newline
Durante los estados de excepción constitucional la Defensoría del Pueblo ejercerá plenamente sus atribuciones. 
\newline {\color{gray} \textbf{1º:} 466-6-Iniciativa-Convencional-Constituyente-de-la-cc-Adriana-Cancino-sobre-Defensoria-de-los-DDHH-1933-31-01.pdf}
\newline {\color{gray} (Emb: 0.705, TF-IDF: 0.419)}
\newline {\color{gray} \textbf{2º:} 409-6-Iniciativa-Convencional-Constituyente-de-la-cc-Ingrid-Villena-sobre-Defensoria-del-Pueblo-2239-24-01.pdf}
\newline {\color{gray} (Emb: 0.679, TF-IDF: 0.338)}

Todo órgano deberá colaborar con los requerimientos de la Defensoría del Pueblo, pudiendo acceder a la información necesaria, y constituirse en dependencias de los órganos objeto de fiscalización, en conformidad a la ley. 
\newline {\color{gray} \textbf{1º:} 409-6-Iniciativa-Convencional-Constituyente-de-la-cc-Ingrid-Villena-sobre-Defensoria-del-Pueblo-2239-24-01.pdf}
\newline {\color{gray} (Emb: 0.999, TF-IDF: 0.934)}
\newline {\color{gray} \textbf{2º:} 106-7-c-Iniciativa-del-cc-Alexis-Caiguan-Herencia-cultural-Memorias-e-historicidades.pdf}
\newline {\color{gray} (Emb: 0.627, TF-IDF: 0.464)}

Las demás que le encomiende la Constitución y la ley. 
\newline {\color{gray} \textbf{1º:} 409-6-Iniciativa-Convencional-Constituyente-de-la-cc-Ingrid-Villena-sobre-Defensoria-del-Pueblo-2239-24-01.pdf}
\newline {\color{gray} (Emb: 0.954, TF-IDF: 0.947)}
\newline {\color{gray} \textbf{2º:} 813-Iniciativa-Convencional-Constituyente-del-cc-Manuel-Woldarsky-crea-la-Agencia-de-DDHH.pdf}
\newline {\color{gray} (Emb: 0.828, TF-IDF: 0.693)}

Recomendar la presentación de proyectos de ley en materias de su competencia. 
\newline {\color{gray} \textbf{1º:} 409-6-Iniciativa-Convencional-Constituyente-de-la-cc-Ingrid-Villena-sobre-Defensoria-del-Pueblo-2239-24-01.pdf}
\newline {\color{gray} (Emb: 0.807, TF-IDF: 0.679)}
\newline {\color{gray} \textbf{2º:} 601-Iniciativa-Convencional-Constituyente-de-cc-Jorge-Abarca-Deber-de-Proteccion-Ambiental-de-los-Gobiernos-Locales.pdf}
\newline {\color{gray} (Emb: 0.683, TF-IDF: 0.321)}

Custodiar y preservar los antecedentes reunidos por comisiones de verdad, justicia, reparación y garantías de no repetición. 
\newline {\color{gray} \textbf{1º:} 409-6-Iniciativa-Convencional-Constituyente-de-la-cc-Ingrid-Villena-sobre-Defensoria-del-Pueblo-2239-24-01.pdf}
\newline {\color{gray} (Emb: 0.884, TF-IDF: 0.472)}
\newline {\color{gray} \textbf{2º:} 256-4-Iniciativa-Convencional-de-la-cc-Elsa-Labrana-sobre-Parte-General-de-los-DDFF.pdf}
\newline {\color{gray} (Emb: 0.712, TF-IDF: 0.382)}

Interponer acciones constitucionales y legales ante los tribunales de justicia respecto de hechos que revistan carácter de crímenes de genocidio, de lesa humanidad o de guerra, tortura, desaparición forzada de personas, trata de personas y demás que establezca la ley. 
\newline {\color{gray} \textbf{1º:} 409-6-Iniciativa-Convencional-Constituyente-de-la-cc-Ingrid-Villena-sobre-Defensoria-del-Pueblo-2239-24-01.pdf}
\newline {\color{gray} (Emb: 0.939, TF-IDF: 0.860)}
\newline {\color{gray} \textbf{2º:} 813-Iniciativa-Convencional-Constituyente-del-cc-Manuel-Woldarsky-crea-la-Agencia-de-DDHH.pdf}
\newline {\color{gray} (Emb: 0.842, TF-IDF: 0.722)}

La Defensoría del Pueblo tendrá las siguientes atribuciones: 1. 
\newline {\color{gray} \textbf{1º:} 588-Iniciativa-Convencional-Constituyente-de-cc-Marcos-Barraza-sobre-Instituto-Nacional-de-DDHH-2351-hrs.-01-02.pdf}
\newline {\color{gray} (Emb: 0.825, TF-IDF: 0.563)}
\newline {\color{gray} \textbf{2º:} 409-6-Iniciativa-Convencional-Constituyente-de-la-cc-Ingrid-Villena-sobre-Defensoria-del-Pueblo-2239-24-01.pdf}
\newline {\color{gray} (Emb: 0.824, TF-IDF: 0.492)}

Tramitar y hacer seguimiento de los reclamos sobre vulneraciones de derechos humanos, y derivar en su caso. 
\newline {\color{gray} \textbf{1º:} 409-6-Iniciativa-Convencional-Constituyente-de-la-cc-Ingrid-Villena-sobre-Defensoria-del-Pueblo-2239-24-01.pdf}
\newline {\color{gray} (Emb: 0.803, TF-IDF: 0.537)}
\newline {\color{gray} \textbf{2º:} 304-4-Iniciativa-Convencional-de-la-cc-Valentina-Miranda-sobre-Derechos-Fundamentales-1301-hrs.pdf}
\newline {\color{gray} (Emb: 0.715, TF-IDF: 0.519)}

Realizar acciones de seguimiento y monitoreo respecto de las recomendaciones formuladas por los organismos internacionales en materia de derechos humanos y de las sentencias dictadas contra el Estado de Chile por tribunales internacionales de derechos humanos. 
\newline {\color{gray} \textbf{1º:} 409-6-Iniciativa-Convencional-Constituyente-de-la-cc-Ingrid-Villena-sobre-Defensoria-del-Pueblo-2239-24-01.pdf}
\newline {\color{gray} (Emb: 0.895, TF-IDF: 0.893)}
\newline {\color{gray} \textbf{2º:} 466-6-Iniciativa-Convencional-Constituyente-de-la-cc-Adriana-Cancino-sobre-Defensoria-de-los-DDHH-1933-31-01.pdf}
\newline {\color{gray} (Emb: 0.834, TF-IDF: 0.791)}

Formular recomendaciones en las materias de su competencia. 
\newline {\color{gray} \textbf{1º:} 88-6-Iniciativa-Convencional-Constituyente-del-cc-Christian-Viera-y-otros.pdf}
\newline {\color{gray} (Emb: 0.797, TF-IDF: 0.611)}
\newline {\color{gray} \textbf{2º:} 349-6-Iniciativa-Convencional-Constituyente-del-cc-Luis-Mayol-sobre-Banco-Central-1826-20-01.pdf}
\newline {\color{gray} (Emb: 0.781, TF-IDF: 0.553)}

Fiscalizar a los órganos del Estado en el cumplimiento de sus obligaciones en materia de derechos humanos. 
\newline {\color{gray} \textbf{1º:} 466-6-Iniciativa-Convencional-Constituyente-de-la-cc-Adriana-Cancino-sobre-Defensoria-de-los-DDHH-1933-31-01.pdf}
\newline {\color{gray} (Emb: 0.790, TF-IDF: 0.634)}
\newline {\color{gray} \textbf{2º:} 574-Iniciativa-Convencional-Constituyente-de-cc-Vanessa-Hoppe-sobre-Defensoria-de-los-Pueblos-2351-hrs.-01-02.pdf}
\newline {\color{gray} (Emb: 0.702, TF-IDF: 0.415)}

Deducir acciones y recursos que esta Constitución y las leyes establecen, cuando se identifiquen patrones de violación de derechos humanos. 
\newline {\color{gray} \textbf{1º:} 573-Iniciativa-Convencional-Constituyente-de-cc-Vanessa-Hoppe-sobre-Defensoria-de-la-Naturaleza-2351-hrs.-01-02.pdf}
\newline {\color{gray} (Emb: 0.578, TF-IDF: 0.611)}
\newline {\color{gray} \textbf{2º:} 914-Iniciativa-Convencional-Constituyente-del-cc-Luis-Jimenez-crea-la-Defensoria-de-la-Naturaleza.pdf}
\newline {\color{gray} (Emb: 0.578, TF-IDF: 0.576)}


\item \textbf{Artículo} \newline
Gozará de inamovilidad en su cargo y será inviolable en el ejercicio de sus atribuciones. 
\newline {\color{gray} \textbf{1º:} 409-6-Iniciativa-Convencional-Constituyente-de-la-cc-Ingrid-Villena-sobre-Defensoria-del-Pueblo-2239-24-01.pdf}
\newline {\color{gray} (Emb: 0.715, TF-IDF: 0.449)}
\newline {\color{gray} \textbf{2º:} 409-6-Iniciativa-Convencional-Constituyente-de-la-cc-Ingrid-Villena-sobre-Defensoria-del-Pueblo-2239-24-01.pdf}
\newline {\color{gray} (Emb: 0.635, TF-IDF: 0.402)}

Podrá ser removido por la Corte Suprema, por notable abandono de deberes, en la forma que establezca la ley. 
\newline {\color{gray} \textbf{1º:} 409-6-Iniciativa-Convencional-Constituyente-de-la-cc-Ingrid-Villena-sobre-Defensoria-del-Pueblo-2239-24-01.pdf}
\newline {\color{gray} (Emb: 0.781, TF-IDF: 0.876)}
\newline {\color{gray} \textbf{2º:} 472-6-Iniciativa-Convencional-Constituyente-del-cc-Daniel-Bravo-sobre-Corte-Constitucional-2003-31-01.pdf}
\newline {\color{gray} (Emb: 0.768, TF-IDF: 0.655)}

Cesará en su cargo por cumplimento de su periodo, por condena por crimen o simple delito, renuncia, enfermedad incompatible con el ejercicio de la función y por remoción. 
\newline {\color{gray} \textbf{1º:} 574-Iniciativa-Convencional-Constituyente-de-cc-Vanessa-Hoppe-sobre-Defensoria-de-los-Pueblos-2351-hrs.-01-02.pdf}
\newline {\color{gray} (Emb: 0.888, TF-IDF: 0.673)}
\newline {\color{gray} \textbf{2º:} 679-Iniciativa-Convencional-Constituyente-del-cc-Francisco-Caamano-sobre-Crisis-Climatica-y-Ecologica-121101-02.pdf}
\newline {\color{gray} (Emb: 0.791, TF-IDF: 0.588)}

Al cesar su mandato y durante los dieciocho meses siguientes, no podrá optar a ningún cargo de elección popular ni de exclusiva confianza de alguna autoridad. 
\newline {\color{gray} \textbf{1º:} 128-4-c-Iniciativa-de-la-cc-Rocio-Cantuarias-Incorpora-una-Garantia-Procesal-en-los-procesos-judiciales.pdf}
\newline {\color{gray} (Emb: 0.704, TF-IDF: 0.418)}
\newline {\color{gray} \textbf{2º:} 440-Iniciativa-Convencional-Constituyente-del-cc-Felipe-Harboe-sobre-Principios-del-Debido-Proceso-1401-28-01.pdf}
\newline {\color{gray} (Emb: 0.704, TF-IDF: 0.408)}

La dirección de la Defensoría del Pueblo estará a cargo de una Defensora o Defensor del Pueblo, quien será designado por la mayoría de las y los integrantes del Congreso de Diputadas y Diputados y de la Cámara de las Regiones, en sesión conjunta, a partir de una terna elaborada por las organizaciones sociales y de derechos humanos, en la forma que determine la ley. 
\newline {\color{gray} \textbf{1º:} 151-3-c-Iniciativa-de-la-cc-Angelica-Tepper-Competencias-de-los-Gobiernos-Regionales.pdf}
\newline {\color{gray} (Emb: 0.971, TF-IDF: 0.899)}
\newline {\color{gray} \textbf{2º:} 151-3-c-Iniciativa-de-la-cc-Angelica-Tepper-Competencias-de-los-Gobiernos-Regionales.pdf}
\newline {\color{gray} (Emb: 0.971, TF-IDF: 0.565)}

Las personas propuestas por las organizaciones deberán cumplir los requisitos de comprobada idoneidad y trayectoria en la defensa de los derechos humanos. 
\newline {\color{gray} \textbf{1º:} 441-Iniciativa-Convencional-Constituyente-de-la-cc-Geoconda-Navarrete-sobre-Derecho-a-una-Vejez-Digna-1403-28-01.pdf}
\newline {\color{gray} (Emb: 0.693, TF-IDF: 0.544)}
\newline {\color{gray} \textbf{2º:} 326-4-Iniciativa-Convencional-de-la-cc-Pollyana-Rivera-sobre-Proteccion-de-la-Vejez.pdf}
\newline {\color{gray} (Emb: 0.688, TF-IDF: 0.366)}

Existirá un Consejo de la Defensoría del Pueblo, cuya composición, funcionamiento y atribuciones será determinado por la ley. 
\newline {\color{gray} \textbf{1º:} 409-6-Iniciativa-Convencional-Constituyente-de-la-cc-Ingrid-Villena-sobre-Defensoria-del-Pueblo-2239-24-01.pdf}
\newline {\color{gray} (Emb: 0.764, TF-IDF: 0.622)}
\newline {\color{gray} \textbf{2º:} 466-6-Iniciativa-Convencional-Constituyente-de-la-cc-Adriana-Cancino-sobre-Defensoria-de-los-DDHH-1933-31-01.pdf}
\newline {\color{gray} (Emb: 0.640, TF-IDF: 0.280)}

La Defensora o el Defensor del Pueblo durará un período de seis años en el ejercicio del cargo, sin posibilidad de reelección. 
\newline {\color{gray} \textbf{1º:} 409-6-Iniciativa-Convencional-Constituyente-de-la-cc-Ingrid-Villena-sobre-Defensoria-del-Pueblo-2239-24-01.pdf}
\newline {\color{gray} (Emb: 0.852, TF-IDF: 0.595)}
\newline {\color{gray} \textbf{2º:} 11-4-Iniciativa-Convencional-Constituyente-de-la-cc-María-Elisa-Quinteros-y-otras.pdf}
\newline {\color{gray} (Emb: 0.639, TF-IDF: 0.460)}


\item \textbf{Artículo} \newline
Existirá un organismo autónomo, con personalidad jurídica y patrimonio propio, denominado Defensoría de los Derechos de la Niñez, que tendrá por objeto la difusión, promoción y protección de los derechos de que son titulares los niños, en conformidad a la Constitución Política de la República, a la Convención sobre los Derechos del Niño y a los demás tratados internacionales ratificados por Chile que se encuentren vigentes, así como a la legislación nacional, velando por su interés superior. 
\newline {\color{gray} \textbf{1º:} 409-6-Iniciativa-Convencional-Constituyente-de-la-cc-Ingrid-Villena-sobre-Defensoria-del-Pueblo-2239-24-01.pdf}
\newline {\color{gray} (Emb: 1.000, TF-IDF: 1.000)}
\newline {\color{gray} \textbf{2º:} 466-6-Iniciativa-Convencional-Constituyente-de-la-cc-Adriana-Cancino-sobre-Defensoria-de-los-DDHH-1933-31-01.pdf}
\newline {\color{gray} (Emb: 0.614, TF-IDF: 0.341)}

La ley determinará la organización, funciones, financiamiento y atribuciones de la Defensoría de los Derechos de la Niñez. 
\newline {\color{gray} \textbf{1º:} 409-6-Iniciativa-Convencional-Constituyente-de-la-cc-Ingrid-Villena-sobre-Defensoria-del-Pueblo-2239-24-01.pdf}
\newline {\color{gray} (Emb: 0.945, TF-IDF: 0.919)}
\newline {\color{gray} \textbf{2º:} 472-6-Iniciativa-Convencional-Constituyente-del-cc-Daniel-Bravo-sobre-Corte-Constitucional-2003-31-01.pdf}
\newline {\color{gray} (Emb: 0.915, TF-IDF: 0.765)}


\item \textbf{Artículo} \newline
Un organismo autónomo, con personalidad jurídica y patrimonio propio, denominada Defensoría de la Naturaleza, tendrá por finalidad la promoción y protección de los derechos de la naturaleza y de los derechos ambientales asegurados en esta Constitución, en los tratados internacionales ambientales ratificados por Chile, frente los actos u omisiones de los órganos de la administración del Estado y de entidades privadas. 
\newline {\color{gray} \textbf{1º:} 817-Iniciativa-Convencional-Constituyente-del-cc-Mauricio-daza-sobre-Acciones-Constitucionales.pdf}
\newline {\color{gray} (Emb: 0.708, TF-IDF: 0.731)}
\newline {\color{gray} \textbf{2º:} 940-Iniciativa-Convencional-Constituyente-del-cc-Christian-Viera-sobre-Proteccion-de-Derechos-Fundamentales.pdf}
\newline {\color{gray} (Emb: 0.708, TF-IDF: 0.408)}

La Defensoría de la Naturaleza tendrá defensorías regionales, que funcionarán en forma desconcentrada, en conformidad a lo que establezca su ley. 
\newline {\color{gray} \textbf{1º:} 622-Iniciativa-Convencional-Constituyente-de-cc-Felipe-Harboe-Reconocimiento-y-proteccion-integral-de-derechos-de-NNA.pdf}
\newline {\color{gray} (Emb: 0.756, TF-IDF: 0.424)}
\newline {\color{gray} \textbf{2º:} 466-6-Iniciativa-Convencional-Constituyente-de-la-cc-Adriana-Cancino-sobre-Defensoria-de-los-DDHH-1933-31-01.pdf}
\newline {\color{gray} (Emb: 0.641, TF-IDF: 0.412)}

La ley determinará las atribuciones, organización, funcionamiento y procedimientos de la Defensoría de la Naturaleza. 
\newline {\color{gray} \textbf{1º:} 409-6-Iniciativa-Convencional-Constituyente-de-la-cc-Ingrid-Villena-sobre-Defensoria-del-Pueblo-2239-24-01.pdf}
\newline {\color{gray} (Emb: 0.772, TF-IDF: 0.639)}
\newline {\color{gray} \textbf{2º:} 349-6-Iniciativa-Convencional-Constituyente-del-cc-Luis-Mayol-sobre-Banco-Central-1826-20-01.pdf}
\newline {\color{gray} (Emb: 0.702, TF-IDF: 0.490)}


\item \textbf{Artículo} \newline
La Defensoría de la Naturaleza tendrá las siguientes atribuciones: Fiscalizar a los órganos del Estado en el cumplimiento de sus obligaciones en materia de derechos ambientales y derechos de la Naturaleza; formular recomendaciones en las materias de su competencia; tramitar y hacer seguimiento de los reclamos sobre vulneraciones de derechos ambientales y derivar en su caso; deducir acciones constitucionales y legales, cuando se vulneren derechos ambientales y de la naturaleza, y las demás que le encomiende la Constitución y la ley. 
\newline {\color{gray} \textbf{1º:} 564-Iniciativa-Convencional-Constituyente-de-cc-Jaime-Bassa-sobre-Defensoria-de-los-Pueblos-2050-hrs.-01-02.pdf}
\newline {\color{gray} (Emb: 0.709, TF-IDF: 0.453)}
\newline {\color{gray} \textbf{2º:} 573-Iniciativa-Convencional-Constituyente-de-cc-Vanessa-Hoppe-sobre-Defensoria-de-la-Naturaleza-2351-hrs.-01-02.pdf}
\newline {\color{gray} (Emb: 0.689, TF-IDF: 0.434)}


\item \textbf{Artículo} \newline
La dirección de la Defensoría de la Naturaleza estará a cargo de una Defensora o Defensor de la Naturaleza, quien será designado por la mayoría de las y los integrantes del Congreso de Diputadas y Diputados y de la Cámara de las Regiones, en sesión conjunta, a partir de una terna elaborada por las organizaciones ambientales de la sociedad civil, en la forma que determine la ley. 
\newline {\color{gray} \textbf{1º:} 466-6-Iniciativa-Convencional-Constituyente-de-la-cc-Adriana-Cancino-sobre-Defensoria-de-los-DDHH-1933-31-01.pdf}
\newline {\color{gray} (Emb: 0.764, TF-IDF: 0.605)}
\newline {\color{gray} \textbf{2º:} 325-6-Iniciativa-Convencional-del-cc-Tomas-Laibe-sobre-Jurisdiccion-Constitucional.pdf}
\newline {\color{gray} (Emb: 0.684, TF-IDF: 0.427)}


\item \textbf{Artículo} \newline
Las sanciones impuestas por la agencia podrán ser reclamadas ante los tribunales de justicia. 
\newline {\color{gray} \textbf{1º:} 557-Iniciativa-Convencional-Constituyente-de-cc-Andres-Cruz-sobre-Agencia-Nacional-del-agua-2013-hrs.-01-02.pdf}
\newline {\color{gray} (Emb: 0.705, TF-IDF: 0.371)}
\newline {\color{gray} \textbf{2º:} 954-5-Iniciativa-Convencional-Constituyente-de-la-cc-Carolina-Vilches-sobre-Estatuto-del-Agua.pdf}
\newline {\color{gray} (Emb: 0.657, TF-IDF: 0.301)}

Podrá determinar la calidad de los servicios sanitarios, así como las demás que señale la ley. 
\newline {\color{gray} \textbf{1º:} 793-Iniciativa-Convencional-Constituyente-de-la-cc-Gloria-Alvarado-sobre-Agua.pdf}
\newline {\color{gray} (Emb: 0.620, TF-IDF: 0.331)}
\newline {\color{gray} \textbf{2º:} 745-Iniciativa-Convencional-Constituyente-del-cc-Maria-Trinidad-Castillo-sobre-Aguas-01-02.pdf}
\newline {\color{gray} (Emb: 0.609, TF-IDF: 0.278)}

Deberá fiscalizar el uso responsable y sostenible del agua y aplicar las sanciones administrativas que correspondan. 
\newline {\color{gray} \textbf{1º:} 557-Iniciativa-Convencional-Constituyente-de-cc-Andres-Cruz-sobre-Agencia-Nacional-del-agua-2013-hrs.-01-02.pdf}
\newline {\color{gray} (Emb: 0.878, TF-IDF: 0.637)}
\newline {\color{gray} \textbf{2º:} 954-5-Iniciativa-Convencional-Constituyente-de-la-cc-Carolina-Vilches-sobre-Estatuto-del-Agua.pdf}
\newline {\color{gray} (Emb: 0.755, TF-IDF: 0.601)}

La Agencia Nacional del Agua es un órgano autónomo, con personalidad jurídica y patrimonio propio, que se organizará desconcentradamente, cuya finalidad es asegurar el uso sostenible del agua, para las generaciones presentes y futuras, el acceso al derecho humano al agua y al saneamiento y la conservación y preservación de sus ecosistemas asociados. 
\newline {\color{gray} \textbf{1º:} 213-1-c-Iniciativa-Convencional-del-cc-Jaime-Bassa-sobre-Congreso-plurinacional-2058-hrs.pdf}
\newline {\color{gray} (Emb: 0.796, TF-IDF: 0.591)}
\newline {\color{gray} \textbf{2º:} 96-6-Iniciativa-del-cc-Felipe-Harboe-Defensoria-Penal-Publica.pdf}
\newline {\color{gray} (Emb: 0.796, TF-IDF: 0.573)}

Para ello, se encargará de recopilar información, coordinar, dirigir y fiscalizar la actuación de los órganos del Estado con competencias en materia hídrica y de los particulares en su caso. 
\newline {\color{gray} \textbf{1º:} 700-Iniciativa-Convencional-Constituyente-de-la-cc-Alejandra-Perez-sobre-Resolucion-de-Conflictos.pdf}
\newline {\color{gray} (Emb: 0.612, TF-IDF: 0.417)}
\newline {\color{gray} \textbf{2º:} 792-Iniciativa-Convencional-Constituyente-de-la-cc-Elsa-Labrana-sobre-Derechos-de-la-Naturaleza.pdf}
\newline {\color{gray} (Emb: 0.605, TF-IDF: 0.391)}

Entre las demás funciones que determine la ley, la Agencia Nacional del Agua deberá liderar y coordinar a los organismos con competencia en materia hídrica; velar por el cumplimiento de la Política Hídrica Nacional que establezca la autoridad respectiva; otorgar, revisar, modificar, caducar o revocar autorizaciones administrativas sobre las aguas; implementar y monitorear los instrumentos de gestión y protección ambiental establecidos en ella; coordinar y elaborar un sistema unificado de información de carácter público; e impulsar la constitución de organismos a nivel de cuencas, a quienes prestará asistencia, para que realicen la gestión integrada, gobernanza participativa y planificación de las intervenciones en los cuerpos de agua y los ecosistemas asociados a la o las respectivas cuencas. 
\newline {\color{gray} \textbf{1º:} 409-6-Iniciativa-Convencional-Constituyente-de-la-cc-Ingrid-Villena-sobre-Defensoria-del-Pueblo-2239-24-01.pdf}
\newline {\color{gray} (Emb: 0.715, TF-IDF: 0.516)}
\newline {\color{gray} \textbf{2º:} 914-Iniciativa-Convencional-Constituyente-del-cc-Luis-Jimenez-crea-la-Defensoria-de-la-Naturaleza.pdf}
\newline {\color{gray} (Emb: 0.669, TF-IDF: 0.485)}


\item \textbf{Artículo} \newline
La ley regulará las instancias de coordinación entre la Autoridad Nacional del Agua y el Ejecutivo, especialmente respecto de la Política Nacional Hídrica, así también la organización, designación, estructura, funcionamiento, y demás funciones y competencias de la Autoridad Nacional, como de los organismos de cuenca. 
\newline {\color{gray} \textbf{1º:} 512-4-Iniciativa-Convencional-Constituyente-del-cc-Bernardo-Fontaine-agua-potable-1105-01-02.pdf}
\newline {\color{gray} (Emb: 0.627, TF-IDF: 0.304)}
\newline {\color{gray} \textbf{2º:} 954-5-Iniciativa-Convencional-Constituyente-de-la-cc-Carolina-Vilches-sobre-Estatuto-del-Agua.pdf}
\newline {\color{gray} (Emb: 0.620, TF-IDF: 0.283)}


\item \textbf{Artículo} \newline
El Banco Central es un órgano autónomo con personalidad jurídica y patrimonio propio, de carácter técnico, encargado de formular y conducir la política monetaria. 
\newline {\color{gray} \textbf{1º:} 180-6-c-Iniciativa-Convencional-del-cc-Rodrigo-Álvarez-que-regula-el-Poder-Judicial-1044-hrs.pdf}
\newline {\color{gray} (Emb: 0.591, TF-IDF: 0.436)}
\newline {\color{gray} \textbf{2º:} 187-7-c-Iniciativa-Convenciona-del-cc-Ignacio-Achurra-sobre-Los-Patrimonios-1126-hrs.pdf}
\newline {\color{gray} (Emb: 0.573, TF-IDF: 0.397)}

La ley regulará su organización, atribuciones y sistemas de control, así como la determinación de instancias de coordinación entre el Banco y el Gobierno. 
\newline {\color{gray} \textbf{1º:} 97-6-Iniciativa-del-cc-Daniel-Bravo-Sistemas-de-Justicia.pdf}
\newline {\color{gray} (Emb: 0.731, TF-IDF: 0.292)}
\newline {\color{gray} \textbf{2º:} 143-4-c-Iniciativa-de-la-cc-Rocio-Cantuarias-Incorpora-Libertad-de-Trabajo-y-Sindical.pdf}
\newline {\color{gray} (Emb: 0.728, TF-IDF: 0.256)}


\item \textbf{Artículo} \newline
Para el cumplimiento de sus objetivos, el Banco Central deberá considerar la estabilidad financiera, la volatilidad cambiaria, la protección del empleo, el cuidado del medioambiente y patrimonio natural y los principios que señale la Constitución y la ley. 
\newline {\color{gray} \textbf{1º:} 937-IniciativaConvencional-Constituyente-del-cc-Tomas-Laibe-sobre-Banco-Central.pdf}
\newline {\color{gray} (Emb: 0.971, TF-IDF: 0.994)}
\newline {\color{gray} \textbf{2º:} 866-Iniciativa-Convencional-Constituyente-del-cc-Renato-Garin-sobre-Banco-Central-Autonomo.pdf}
\newline {\color{gray} (Emb: 0.781, TF-IDF: 0.630)}

Le corresponderá en especial al Banco Central, para contribuir al bienestar de la población, velar por la estabilidad de los precios y el normal funcionamiento de los pagos internos y externos. 
\newline {\color{gray} \textbf{1º:} 584-Iniciativa-Convencional-Constituyente-de-cc-Malucha-Pinto-sobre-Democracia-Cultural-2351-hrs.-01-02.pdf}
\newline {\color{gray} (Emb: 0.681, TF-IDF: 0.386)}
\newline {\color{gray} \textbf{2º:} 557-Iniciativa-Convencional-Constituyente-de-cc-Andres-Cruz-sobre-Agencia-Nacional-del-agua-2013-hrs.-01-02.pdf}
\newline {\color{gray} (Emb: 0.668, TF-IDF: 0.365)}

El Banco, al adoptar sus decisiones, deberá tener presente la orientación general de la política económica del gobierno. 
\newline {\color{gray} \textbf{1º:} 937-IniciativaConvencional-Constituyente-del-cc-Tomas-Laibe-sobre-Banco-Central.pdf}
\newline {\color{gray} (Emb: 0.869, TF-IDF: 0.725)}
\newline {\color{gray} \textbf{2º:} 717-Iniciativa-Convencional-Constituyente-de-la-cc-Manuela-Royo-sobre-Banco-Central.pdf}
\newline {\color{gray} (Emb: 0.843, TF-IDF: 0.665)}


\item \textbf{Artículo} \newline
Son atribuciones del Banco Central la regulación de la cantidad de dinero y de crédito en circulación, la ejecución de operaciones de crédito y cambios internacionales, como, asimismo, la dictación de normas en materia monetaria, crediticia, financiera y de cambios internacionales, y las demás que establezca la ley. 
\newline {\color{gray} \textbf{1º:} 937-IniciativaConvencional-Constituyente-del-cc-Tomas-Laibe-sobre-Banco-Central.pdf}
\newline {\color{gray} (Emb: 0.839, TF-IDF: 0.826)}
\newline {\color{gray} \textbf{2º:} 866-Iniciativa-Convencional-Constituyente-del-cc-Renato-Garin-sobre-Banco-Central-Autonomo.pdf}
\newline {\color{gray} (Emb: 0.731, TF-IDF: 0.657)}


\item \textbf{Artículo} \newline
El Banco Central sólo podrá efectuar operaciones con instituciones financieras, sean éstas públicas o privadas. 
\newline {\color{gray} \textbf{1º:} 412-6-Iniciativa-Convencional-Constituyente-del-cc-Ivanna-Olivares-sobre-Banco-Central-1206-25-01.pdf}
\newline {\color{gray} (Emb: 0.637, TF-IDF: 0.371)}
\newline {\color{gray} \textbf{2º:} 717-Iniciativa-Convencional-Constituyente-de-la-cc-Manuela-Royo-sobre-Banco-Central.pdf}
\newline {\color{gray} (Emb: 0.621, TF-IDF: 0.359)}

De ninguna manera podrá otorgar a ellas su garantía, ni adquirir documentos emitidos por el Estado, sus organismos o empresas. 
\newline {\color{gray} \textbf{1º:} 717-Iniciativa-Convencional-Constituyente-de-la-cc-Manuela-Royo-sobre-Banco-Central.pdf}
\newline {\color{gray} (Emb: 0.619, TF-IDF: 0.284)}
\newline {\color{gray} \textbf{2º:} 866-Iniciativa-Convencional-Constituyente-del-cc-Renato-Garin-sobre-Banco-Central-Autonomo.pdf}
\newline {\color{gray} (Emb: 0.593, TF-IDF: 0.253)}

Ningún gasto público o préstamo podrá financiarse con créditos directos e indirectos del Banco Central. 
\newline {\color{gray} \textbf{1º:} 936-IniciativaConvencional-Constituyente-del-cc-Mauricio-Daza-sobre-Banco-Central.pdf}
\newline {\color{gray} (Emb: 0.965, TF-IDF: 0.966)}
\newline {\color{gray} \textbf{2º:} 937-IniciativaConvencional-Constituyente-del-cc-Tomas-Laibe-sobre-Banco-Central.pdf}
\newline {\color{gray} (Emb: 0.770, TF-IDF: 0.779)}

Sin perjuicio de lo anterior, en situaciones excepcionales y transitorias en las que así lo requiera la preservación del normal funcionamiento de los pagos internos y externos, el Banco Central podrá comprar durante un período determinado y vender en el mercado secundario abierto, instrumentos de deuda emitidos por el Fisco, en conformidad a la ley. 
\newline {\color{gray} \textbf{1º:} 937-IniciativaConvencional-Constituyente-del-cc-Tomas-Laibe-sobre-Banco-Central.pdf}
\newline {\color{gray} (Emb: 1.000, TF-IDF: 1.000)}
\newline {\color{gray} \textbf{2º:} 349-6-Iniciativa-Convencional-Constituyente-del-cc-Luis-Mayol-sobre-Banco-Central-1826-20-01.pdf}
\newline {\color{gray} (Emb: 0.998, TF-IDF: 0.940)}


\item \textbf{Artículo} \newline
El Banco rendirá cuenta periódicamente al Congreso sobre la ejecución de las políticas a su cargo, respecto de las medidas y normas generales que adopte en el ejercicio de sus funciones y atribuciones y sobre los demás asuntos que se le soliciten mediante informes u otros mecanismos que determine la ley. 
\newline {\color{gray} \textbf{1º:} 937-IniciativaConvencional-Constituyente-del-cc-Tomas-Laibe-sobre-Banco-Central.pdf}
\newline {\color{gray} (Emb: 1.000, TF-IDF: 1.000)}
\newline {\color{gray} \textbf{2º:} 172-6-c-Iniciativa-Convencional-del-cc-Rodrigo-Álvarez-sobre-Banco-central1044-hrs.pdf}
\newline {\color{gray} (Emb: 1.000, TF-IDF: 1.000)}


\item \textbf{Artículo} \newline
La dirección y administración superior del Banco estará a cargo de un Consejo, al que le corresponderá cumplir las funciones y ejercer las atribuciones que señale la Constitución y la ley. 
\newline {\color{gray} \textbf{1º:} 937-IniciativaConvencional-Constituyente-del-cc-Tomas-Laibe-sobre-Banco-Central.pdf}
\newline {\color{gray} (Emb: 1.000, TF-IDF: 1.000)}
\newline {\color{gray} \textbf{2º:} 349-6-Iniciativa-Convencional-Constituyente-del-cc-Luis-Mayol-sobre-Banco-Central-1826-20-01.pdf}
\newline {\color{gray} (Emb: 0.973, TF-IDF: 0.986)}

El Consejo estará integrado por siete consejeras y consejeros designados por la o el Presidente de la República, con acuerdo de la mayoría de las y los integrantes del Congreso de las Diputadas y Diputados y la Cámara de las Regiones, en sesión conjunta. 
\newline {\color{gray} \textbf{1º:} 349-6-Iniciativa-Convencional-Constituyente-del-cc-Luis-Mayol-sobre-Banco-Central-1826-20-01.pdf}
\newline {\color{gray} (Emb: 0.999, TF-IDF: 0.987)}
\newline {\color{gray} \textbf{2º:} 937-IniciativaConvencional-Constituyente-del-cc-Tomas-Laibe-sobre-Banco-Central.pdf}
\newline {\color{gray} (Emb: 0.998, TF-IDF: 0.963)}

Durarán en el cargo por un período de diez años, no reelegibles, renovándose por parcialidades en conformidad a la ley. 
\newline {\color{gray} \textbf{1º:} 937-IniciativaConvencional-Constituyente-del-cc-Tomas-Laibe-sobre-Banco-Central.pdf}
\newline {\color{gray} (Emb: 0.872, TF-IDF: 0.963)}
\newline {\color{gray} \textbf{2º:} 717-Iniciativa-Convencional-Constituyente-de-la-cc-Manuela-Royo-sobre-Banco-Central.pdf}
\newline {\color{gray} (Emb: 0.807, TF-IDF: 0.889)}

Las y los consejeros del Banco Central deben ser profesionales de comprobada idoneidad y trayectoria en materias relacionadas con las competencias de la institución. 
\newline {\color{gray} \textbf{1º:} 866-Iniciativa-Convencional-Constituyente-del-cc-Renato-Garin-sobre-Banco-Central-Autonomo.pdf}
\newline {\color{gray} (Emb: 0.883, TF-IDF: 0.823)}
\newline {\color{gray} \textbf{2º:} 937-IniciativaConvencional-Constituyente-del-cc-Tomas-Laibe-sobre-Banco-Central.pdf}
\newline {\color{gray} (Emb: 0.872, TF-IDF: 0.793)}

La o el Presidente del Consejo, que lo será también del Banco, será designado por la o el Presidente de la República de entre las y los integrantes del Consejo, y durará cinco años en este cargo o el tiempo menor que le reste como consejero, pudiendo ser reelegido para un nuevo periodo. 
\newline {\color{gray} \textbf{1º:} 937-IniciativaConvencional-Constituyente-del-cc-Tomas-Laibe-sobre-Banco-Central.pdf}
\newline {\color{gray} (Emb: 0.875, TF-IDF: 0.685)}
\newline {\color{gray} \textbf{2º:} 866-Iniciativa-Convencional-Constituyente-del-cc-Renato-Garin-sobre-Banco-Central-Autonomo.pdf}
\newline {\color{gray} (Emb: 0.769, TF-IDF: 0.522)}

La ley determinará los requisitos, responsabilidades, inhabilidades e incompatibilidades para las y los consejeros del Banco. 
\newline {\color{gray} \textbf{1º:} 201-6-c-Iniciativa-Convencional-del-cc-Felipe-Harboe-sobre-la-Contraloria-1658-hrs.pdf}
\newline {\color{gray} (Emb: 0.798, TF-IDF: 0.635)}
\newline {\color{gray} \textbf{2º:} 400-1-Iniciativa-Convencional-Constituyente-de-la-cc-Constanza-Hube-sobre-Servicio-y-Registro-Electoral-1905-24-01.pdf}
\newline {\color{gray} (Emb: 0.777, TF-IDF: 0.598)}


\item \textbf{Artículo} \newline
La remoción sólo podrá fundarse en la circunstancia de que el consejero afectado hubiere realizado actos graves en contra de la probidad pública, o haber incurrido en alguna de las prohibiciones o incompatibilidades establecidas en la Constitución o la ley, o haya concurrido con su voto a decisiones que afectan gravemente la consecución del objeto del Banco. 
\newline {\color{gray} \textbf{1º:} 172-6-c-Iniciativa-Convencional-del-cc-Rodrigo-Álvarez-sobre-Banco-central1044-hrs.pdf}
\newline {\color{gray} (Emb: 0.911, TF-IDF: 0.803)}
\newline {\color{gray} \textbf{2º:} 349-6-Iniciativa-Convencional-Constituyente-del-cc-Luis-Mayol-sobre-Banco-Central-1826-20-01.pdf}
\newline {\color{gray} (Emb: 0.909, TF-IDF: 0.763)}

La persona destituida no podrá ser designada nuevamente como integrante del Consejo, ni ser funcionaria o funcionario del Banco Central o prestarle servicios, sin perjuicio de las demás sanciones que establezca la ley. 
\newline {\color{gray} \textbf{1º:} 866-Iniciativa-Convencional-Constituyente-del-cc-Renato-Garin-sobre-Banco-Central-Autonomo.pdf}
\newline {\color{gray} (Emb: 0.976, TF-IDF: 0.869)}
\newline {\color{gray} \textbf{2º:} 937-IniciativaConvencional-Constituyente-del-cc-Tomas-Laibe-sobre-Banco-Central.pdf}
\newline {\color{gray} (Emb: 0.976, TF-IDF: 0.869)}

Las y los integrantes del Consejo podrán ser destituidos de sus cargos por resolución de la mayoría de los integrantes del pleno de la Corte Suprema, previo requerimiento de la mayoría de quienes ejerzan como consejeros, de la o el Presidente de la República, de la mayoría de los integrantes del Congreso de Diputadas y Diputados o de la Cámara de las Regiones, conforme al procedimiento que establezca la ley. 
\newline {\color{gray} \textbf{1º:} 937-IniciativaConvencional-Constituyente-del-cc-Tomas-Laibe-sobre-Banco-Central.pdf}
\newline {\color{gray} (Emb: 0.757, TF-IDF: 0.758)}
\newline {\color{gray} \textbf{2º:} 866-Iniciativa-Convencional-Constituyente-del-cc-Renato-Garin-sobre-Banco-Central-Autonomo.pdf}
\newline {\color{gray} (Emb: 0.607, TF-IDF: 0.318)}


\item \textbf{Artículo} \newline
No podrán integrar el Consejo quienes en los doce meses anteriores a su designación hayan participado en la propiedad o ejercido como director, gerente o ejecutivo principal de una empresa bancaria, administradora de fondos, o cualquiera otra que preste servicios de intermediación financiera, sin perjuicio de las demás inhabilidades que establezca la ley. 
\newline {\color{gray} \textbf{1º:} 936-IniciativaConvencional-Constituyente-del-cc-Mauricio-Daza-sobre-Banco-Central.pdf}
\newline {\color{gray} (Emb: 0.774, TF-IDF: 0.598)}
\newline {\color{gray} \textbf{2º:} 560-Iniciativa-Convencional-Constituyente-de-cc-Andres-Cruz-sobre-Ministerio-Publico-2016-hrs.-01-02.pdf}
\newline {\color{gray} (Emb: 0.744, TF-IDF: 0.413)}

Una vez que hayan cesado en sus cargos, los integrantes del Consejo tendrán la misma incompatibilidad por un periodo de doce meses. 
\newline {\color{gray} \textbf{1º:} 936-IniciativaConvencional-Constituyente-del-cc-Mauricio-Daza-sobre-Banco-Central.pdf}
\newline {\color{gray} (Emb: 0.996, TF-IDF: 0.994)}
\newline {\color{gray} \textbf{2º:} 472-6-Iniciativa-Convencional-Constituyente-del-cc-Daniel-Bravo-sobre-Corte-Constitucional-2003-31-01.pdf}
\newline {\color{gray} (Emb: 0.607, TF-IDF: 0.336)}


\item \textbf{Artículo} \newline
La Contraloría General de la República es un órgano técnico, autónomo, con personalidad jurídica y patrimonio propio, encargada de velar por el cumplimiento del principio de probidad en la función pública, ejercer el control de constitucionalidad y legalidad de los actos de la Administración del Estado, incluidos los gobiernos regionales, locales y demás entidades, organismos y servicios que determine la ley. 
\newline {\color{gray} \textbf{1º:} 936-IniciativaConvencional-Constituyente-del-cc-Mauricio-Daza-sobre-Banco-Central.pdf}
\newline {\color{gray} (Emb: 0.990, TF-IDF: 0.962)}
\newline {\color{gray} \textbf{2º:} 717-Iniciativa-Convencional-Constituyente-de-la-cc-Manuela-Royo-sobre-Banco-Central.pdf}
\newline {\color{gray} (Emb: 0.606, TF-IDF: 0.361)}

Estará encargado de fiscalizar y auditar el ingreso, cuentas y gastos de los fondos públicos. 
\newline {\color{gray} \textbf{1º:} 936-IniciativaConvencional-Constituyente-del-cc-Mauricio-Daza-sobre-Banco-Central.pdf}
\newline {\color{gray} (Emb: 0.798, TF-IDF: 0.760)}
\newline {\color{gray} \textbf{2º:} 717-Iniciativa-Convencional-Constituyente-de-la-cc-Manuela-Royo-sobre-Banco-Central.pdf}
\newline {\color{gray} (Emb: 0.798, TF-IDF: 0.760)}

En el ejercicio de sus funciones, no podrá evaluar los aspectos de mérito o conveniencia de las decisiones políticas o administrativas. 
\newline {\color{gray} \textbf{1º:} 717-Iniciativa-Convencional-Constituyente-de-la-cc-Manuela-Royo-sobre-Banco-Central.pdf}
\newline {\color{gray} (Emb: 0.608, TF-IDF: 0.565)}
\newline {\color{gray} \textbf{2º:} 98-6-Iniciativa-del-cc-Ruggero-Cozzi-Funcion-y-Principios-de-la-Jurisdiccion.pdf}
\newline {\color{gray} (Emb: 0.583, TF-IDF: 0.359)}

La ley establecerá la organización, funcionamiento, planta, procedimientos y demás atribuciones de la Contraloría General de la República. 
\newline {\color{gray} \textbf{1º:} 641-Iniciativa-Convencional-Constituyente-del-cc-Mauricio-Daza-sobre-Contraloria-General-1730-01-02.pdf}
\newline {\color{gray} (Emb: 0.907, TF-IDF: 0.691)}
\newline {\color{gray} \textbf{2º:} 631-Iniciativa-Convencional-Constituyente-de-cc-Ingrid-Villena-sobre-Contraloria-General-de-la-Republica.pdf}
\newline {\color{gray} (Emb: 0.778, TF-IDF: 0.508)}


\item \textbf{Artículo} \newline
La dirección de la Contraloría General de la República estará a cargo de una Contralora o Contralor General, quien será designado por la o el Presidente de la República, con acuerdo de la mayoría de las y los integrantes del Congreso de Diputadas y Diputados y la Cámara de las Regiones, en sesión conjunta. 
\newline {\color{gray} \textbf{1º:} 1014-Iniciativa-Convencional-Constituyente-cc-Adriana-Ampuero-Haciendas-territoriales-y-autonomia-financiera.pdf}
\newline {\color{gray} (Emb: 0.741, TF-IDF: 0.488)}
\newline {\color{gray} \textbf{2º:} 641-Iniciativa-Convencional-Constituyente-del-cc-Mauricio-Daza-sobre-Contraloria-General-1730-01-02.pdf}
\newline {\color{gray} (Emb: 0.684, TF-IDF: 0.333)}

La Contralora o Contralor General durará en su cargo por un plazo de ocho años, sin posibilidad de reelección. 
\newline {\color{gray} \textbf{1º:} 173-6-c-Iniciativa-Convencional-del-cc-Rodrigo-Álvarez-Contraloría-1044-hrs.pdf}
\newline {\color{gray} (Emb: 0.807, TF-IDF: 0.745)}
\newline {\color{gray} \textbf{2º:} 631-Iniciativa-Convencional-Constituyente-de-cc-Ingrid-Villena-sobre-Contraloria-General-de-la-Republica.pdf}
\newline {\color{gray} (Emb: 0.711, TF-IDF: 0.724)}

Un Consejo de la Contraloría, cuya conformación y funcionamiento determinará la ley, aprobará anualmente el programa de fiscalización y auditoría de servicios públicos, determinando servicios o programas que, a su juicio, deben necesariamente ser incluidos en el programa referido. 
\newline {\color{gray} \textbf{1º:} 641-Iniciativa-Convencional-Constituyente-del-cc-Mauricio-Daza-sobre-Contraloria-General-1730-01-02.pdf}
\newline {\color{gray} (Emb: 0.918, TF-IDF: 0.840)}
\newline {\color{gray} \textbf{2º:} 631-Iniciativa-Convencional-Constituyente-de-cc-Ingrid-Villena-sobre-Contraloria-General-de-la-Republica.pdf}
\newline {\color{gray} (Emb: 0.870, TF-IDF: 0.688)}

Los dictámenes que modifican la jurisprudencia administrativa de la Contraloría, deberán ser consultados ante el Consejo. 
\newline {\color{gray} \textbf{1º:} 631-Iniciativa-Convencional-Constituyente-de-cc-Ingrid-Villena-sobre-Contraloria-General-de-la-Republica.pdf}
\newline {\color{gray} (Emb: 0.899, TF-IDF: 0.730)}
\newline {\color{gray} \textbf{2º:} 631-Iniciativa-Convencional-Constituyente-de-cc-Ingrid-Villena-sobre-Contraloria-General-de-la-Republica.pdf}
\newline {\color{gray} (Emb: 0.745, TF-IDF: 0.580)}


\item \textbf{Artículo} \newline
Respecto de los decretos, resoluciones y otros actos administrativos de entidades territoriales que, de acuerdo a la ley, deban tramitarse por la Contraloría, la toma de razón corresponderá a la respectiva contraloría regional. 
\newline {\color{gray} \textbf{1º:} 631-Iniciativa-Convencional-Constituyente-de-cc-Ingrid-Villena-sobre-Contraloria-General-de-la-Republica.pdf}
\newline {\color{gray} (Emb: 0.846, TF-IDF: 0.938)}
\newline {\color{gray} \textbf{2º:} 798-Iniciativa-Convencional-Constituyente-de-la-cc-Patricia-Labra-sonbre-Contraloria-General.pdf}
\newline {\color{gray} (Emb: 0.846, TF-IDF: 0.938)}

Los antecedentes que debieran remitirse, en su caso, lo serán a la correspondiente asamblea regional. 
\newline {\color{gray} \textbf{1º:} 325-6-Iniciativa-Convencional-del-cc-Tomas-Laibe-sobre-Jurisdiccion-Constitucional.pdf}
\newline {\color{gray} (Emb: 0.812, TF-IDF: 0.329)}
\newline {\color{gray} \textbf{2º:} 729-Iniciativa-Convencional-Constituyente-del-cc-Ruggero-Cozzi-sobre-Justicia-Constitucional-y-Administrativa.pdf}
\newline {\color{gray} (Emb: 0.742, TF-IDF: 0.329)}

Además, le corresponderá tomar razón de los decretos con fuerza de ley, debiendo representarlos, cuando ellos excedan o contravengan la respectiva ley delegatoria. 
\newline {\color{gray} \textbf{1º:} 173-6-c-Iniciativa-Convencional-del-cc-Rodrigo-Álvarez-Contraloría-1044-hrs.pdf}
\newline {\color{gray} (Emb: 0.727, TF-IDF: 0.599)}
\newline {\color{gray} \textbf{2º:} 885-Iniciativa-Convencional-Constituyente-de-la-cc-Lidia-Gonzalez-Sobre-Estados-de-Excepcion-Constitucional.pdf}
\newline {\color{gray} (Emb: 0.717, TF-IDF: 0.595)}

En ningún caso dará curso a los decretos de gastos que excedan el límite señalado en la Constitución o la ley y remitirá copia íntegra de los antecedentes al Congreso. 
\newline {\color{gray} \textbf{1º:} 654-Iniciativa-Convencional-Constituyente-de-la-cc-Isabella-Mamani-sobre-Derecho-a-la-Consulta-Indigena-121101-02.pdf}
\newline {\color{gray} (Emb: 0.588, TF-IDF: 0.327)}
\newline {\color{gray} \textbf{2º:} 304-4-Iniciativa-Convencional-de-la-cc-Valentina-Miranda-sobre-Derechos-Fundamentales-1301-hrs.pdf}
\newline {\color{gray} (Emb: 0.542, TF-IDF: 0.261)}

Sin embargo, deberá darles curso cuando la o el Presidente de la República insista con la firma de todos sus Ministros, debiendo enviar copia de los respectivos decretos al Congreso. 
\newline {\color{gray} \textbf{1º:} 580-Iniciativa-Convencional-Constituyente-de-cc-Daniel-Stingo-sobre-Contraloria-General-de-la-Republica-2318-hrs.-01-02.pdf}
\newline {\color{gray} (Emb: 0.971, TF-IDF: 0.962)}
\newline {\color{gray} \textbf{2º:} 631-Iniciativa-Convencional-Constituyente-de-cc-Ingrid-Villena-sobre-Contraloria-General-de-la-Republica.pdf}
\newline {\color{gray} (Emb: 0.603, TF-IDF: 0.270)}

En el ejercicio del control de constitucionalidad y legalidad, la Contraloría General tomará razón de los decretos, resoluciones y otros actos administrativos o representará su ilegalidad. 
\newline {\color{gray} \textbf{1º:} 641-Iniciativa-Convencional-Constituyente-del-cc-Mauricio-Daza-sobre-Contraloria-General-1730-01-02.pdf}
\newline {\color{gray} (Emb: 0.850, TF-IDF: 0.574)}
\newline {\color{gray} \textbf{2º:} 631-Iniciativa-Convencional-Constituyente-de-cc-Ingrid-Villena-sobre-Contraloria-General-de-la-Republica.pdf}
\newline {\color{gray} (Emb: 0.718, TF-IDF: 0.511)}

Tratándose de la representación por inconstitucionalidad no procederá insistencia y el pronunciamiento de la Contraloría será reclamable ante la Corte Constitucional. 
\newline {\color{gray} \textbf{1º:} 641-Iniciativa-Convencional-Constituyente-del-cc-Mauricio-Daza-sobre-Contraloria-General-1730-01-02.pdf}
\newline {\color{gray} (Emb: 0.807, TF-IDF: 0.657)}
\newline {\color{gray} \textbf{2º:} 374-2-Iniciativa-Convencional-Constituyente-de-la-cc-Loreto-Vallejos-sobre-Democracia-Directa-0900-hrs-24-01.pdf}
\newline {\color{gray} (Emb: 0.679, TF-IDF: 0.518)}


\item \textbf{Artículo} \newline
La Contraloría General de la República podrá emitir dictámenes obligatorios para toda autoridad, funcionario o trabajador de cualquier órgano integrante de la Administración del Estado, de las regiones y de las comunas, incluyendo los directivos de empresas públicas o sociedades en las que tenga participación una entidad estatal o cualquier otra entidad territorial. 
\newline {\color{gray} \textbf{1º:} 641-Iniciativa-Convencional-Constituyente-del-cc-Mauricio-Daza-sobre-Contraloria-General-1730-01-02.pdf}
\newline {\color{gray} (Emb: 0.921, TF-IDF: 0.882)}
\newline {\color{gray} \textbf{2º:} 558-Iniciativa-Convencional-Constituyente-de-cc-Andres-Cruz-sobre-Contraloria-General-de-la-Republica-1755-hrs.-01-02.pdf}
\newline {\color{gray} (Emb: 0.739, TF-IDF: 0.823)}

Los órganos de la Administración del Estado, los Gobiernos Regionales y locales, órganos autónomos, empresas públicas, sociedades en que el Estado tenga participación, personas jurídicas que dispongan de recursos fiscales o administren bienes públicos, y los demás que defina la ley, estarán sujetos a la fiscalización y auditorías de la Contraloría General de la República. 
\newline {\color{gray} \textbf{1º:} 631-Iniciativa-Convencional-Constituyente-de-cc-Ingrid-Villena-sobre-Contraloria-General-de-la-Republica.pdf}
\newline {\color{gray} (Emb: 0.791, TF-IDF: 0.792)}
\newline {\color{gray} \textbf{2º:} 631-Iniciativa-Convencional-Constituyente-de-cc-Ingrid-Villena-sobre-Contraloria-General-de-la-Republica.pdf}
\newline {\color{gray} (Emb: 0.702, TF-IDF: 0.442)}

La ley regulará el ejercicio de estas potestades fiscalizadoras y auditoras. 
\newline {\color{gray} \textbf{1º:} 631-Iniciativa-Convencional-Constituyente-de-cc-Ingrid-Villena-sobre-Contraloria-General-de-la-Republica.pdf}
\newline {\color{gray} (Emb: 0.884, TF-IDF: 0.951)}
\newline {\color{gray} \textbf{2º:} 120-3-c-Iniciativa-de-la-cc-Tammy-Pustilnick-atribuciones-exclusivas-de-la-Asamblea-Regional.pdf}
\newline {\color{gray} (Emb: 0.663, TF-IDF: 0.272)}


\item \textbf{Artículo} \newline
Respecto de las entidades territoriales, a través de las Contralorías Regionales, controlará la legalidad de su actividad financiera, la gestión y los resultados de la administración de los recursos públicos. 
\newline {\color{gray} \textbf{1º:} 631-Iniciativa-Convencional-Constituyente-de-cc-Ingrid-Villena-sobre-Contraloria-General-de-la-Republica.pdf}
\newline {\color{gray} (Emb: 0.881, TF-IDF: 0.815)}
\newline {\color{gray} \textbf{2º:} 631-Iniciativa-Convencional-Constituyente-de-cc-Ingrid-Villena-sobre-Contraloria-General-de-la-Republica.pdf}
\newline {\color{gray} (Emb: 0.811, TF-IDF: 0.690)}

La Contraloría General de la República funcionará desconcentradamente en cada una de las regiones del país mediante Contralorías Regionales. 
\newline {\color{gray} \textbf{1º:} 641-Iniciativa-Convencional-Constituyente-del-cc-Mauricio-Daza-sobre-Contraloria-General-1730-01-02.pdf}
\newline {\color{gray} (Emb: 0.967, TF-IDF: 0.853)}
\newline {\color{gray} \textbf{2º:} 641-Iniciativa-Convencional-Constituyente-del-cc-Mauricio-Daza-sobre-Contraloria-General-1730-01-02.pdf}
\newline {\color{gray} (Emb: 0.694, TF-IDF: 0.330)}

La dirección de cada contraloría regional estará a cargo de una o un Contralor Regional, designado por la o el Contralor General de la República. 
\newline {\color{gray} \textbf{1º:} 641-Iniciativa-Convencional-Constituyente-del-cc-Mauricio-Daza-sobre-Contraloria-General-1730-01-02.pdf}
\newline {\color{gray} (Emb: 0.935, TF-IDF: 0.943)}
\newline {\color{gray} \textbf{2º:} 631-Iniciativa-Convencional-Constituyente-de-cc-Ingrid-Villena-sobre-Contraloria-General-de-la-Republica.pdf}
\newline {\color{gray} (Emb: 0.771, TF-IDF: 0.431)}

En el ejercicio de sus funciones deberán mantener la unidad de acción, con el fin de aplicar un criterio uniforme en todo el territorio del país. 
\newline {\color{gray} \textbf{1º:} 641-Iniciativa-Convencional-Constituyente-del-cc-Mauricio-Daza-sobre-Contraloria-General-1730-01-02.pdf}
\newline {\color{gray} (Emb: 0.884, TF-IDF: 0.805)}
\newline {\color{gray} \textbf{2º:} 239-1-Iniciativa-Convencional-de-la-cc-Tania-Madriaga-sobre-Poder-Ejecutivo-1146-hrs.pdf}
\newline {\color{gray} (Emb: 0.795, TF-IDF: 0.327)}

La ley determinará las demás atribuciones de las Contralorías Regionales y regulará su organización y funcionamiento. 
\newline {\color{gray} \textbf{1º:} 631-Iniciativa-Convencional-Constituyente-de-cc-Ingrid-Villena-sobre-Contraloria-General-de-la-Republica.pdf}
\newline {\color{gray} (Emb: 0.961, TF-IDF: 0.634)}
\newline {\color{gray} \textbf{2º:} 631-Iniciativa-Convencional-Constituyente-de-cc-Ingrid-Villena-sobre-Contraloria-General-de-la-Republica.pdf}
\newline {\color{gray} (Emb: 0.690, TF-IDF: 0.441)}


\item \textbf{Artículo} \newline
Los pagos se efectuarán considerando, además, el orden cronológico establecido en ella y previa refrendación presupuestaria del documento que ordene el pago. 
\newline {\color{gray} \textbf{1º:} 631-Iniciativa-Convencional-Constituyente-de-cc-Ingrid-Villena-sobre-Contraloria-General-de-la-Republica.pdf}
\newline {\color{gray} (Emb: 1.000, TF-IDF: 1.000)}
\newline {\color{gray} \textbf{2º:} 400-1-Iniciativa-Convencional-Constituyente-de-la-cc-Constanza-Hube-sobre-Servicio-y-Registro-Electoral-1905-24-01.pdf}
\newline {\color{gray} (Emb: 0.771, TF-IDF: 0.712)}

Las Tesorerías del Estado no podrán efectuar ningún pago sino en virtud de un decreto o resolución expedido por autoridad competente, en que se exprese la ley o la parte del presupuesto que autorice aquel gasto. 
\newline {\color{gray} \textbf{1º:} 631-Iniciativa-Convencional-Constituyente-de-cc-Ingrid-Villena-sobre-Contraloria-General-de-la-Republica.pdf}
\newline {\color{gray} (Emb: 0.773, TF-IDF: 0.814)}
\newline {\color{gray} \textbf{2º:} 113-5-c-Iniciativa-del-cc-Elsa-Labrana-sobre-Soberania-Alimentaria.pdf}
\newline {\color{gray} (Emb: 0.557, TF-IDF: 0.252)}


\item \textbf{Artículo} \newline
La Corte conocerá del asunto en pleno especialmente convocado al efecto y, para acordar la remoción deberá reunir el voto conforme de la mayoría de sus juezas y jueces. 
\newline {\color{gray} \textbf{1º:} 711-Iniciativa-Convencional-Constituyente-de-la-cc-Ingrid-Villena-sobre-Tricel.pdf}
\newline {\color{gray} (Emb: 0.803, TF-IDF: 0.600)}
\newline {\color{gray} \textbf{2º:} 937-IniciativaConvencional-Constituyente-del-cc-Tomas-Laibe-sobre-Banco-Central.pdf}
\newline {\color{gray} (Emb: 0.724, TF-IDF: 0.576)}

Un organismo autónomo, con personalidad jurídica y patrimonio propio, denominado Servicio Electoral, ejercerá la administración, supervigilancia y fiscalización de los procesos electorales y plebiscitarios, del cumplimiento de las normas sobre transparencia, límite y control del gasto electoral, de las normas sobre organizaciones políticas, de las normas relativas a mecanismos de democracia directa y participación ciudadana, así como las demás funciones que señale la Constitución y la ley. 
\newline {\color{gray} \textbf{1º:} 159-3-c-Iniciativa-de-la-cc-Jennifer-Mella-.pdf}
\newline {\color{gray} (Emb: 0.783, TF-IDF: 0.570)}
\newline {\color{gray} \textbf{2º:} 122-3-c-Iniciativa-de-la-cc-Jennifer-Mella-Forma-del-Estado.pdf}
\newline {\color{gray} (Emb: 0.783, TF-IDF: 0.300)}

En lo referente a la democracia participativa y los mecanismos consagrados en esta Constitución, será función del Servicio Electoral promover la información, educación y participación ciudadana y/o electoral en relación a tales procesos, en colaboración con otros organismos del Estado y la sociedad civil. 
\newline {\color{gray} \textbf{1º:} 558-Iniciativa-Convencional-Constituyente-de-cc-Andres-Cruz-sobre-Contraloria-General-de-la-Republica-1755-hrs.-01-02.pdf}
\newline {\color{gray} (Emb: 1.000, TF-IDF: 1.000)}
\newline {\color{gray} \textbf{2º:} 173-6-c-Iniciativa-Convencional-del-cc-Rodrigo-Álvarez-Contraloría-1044-hrs.pdf}
\newline {\color{gray} (Emb: 1.000, TF-IDF: 1.000)}

Así también deberá velar por la implementación y la recta ejecución de estos mecanismos. 
\newline {\color{gray} \textbf{1º:} 558-Iniciativa-Convencional-Constituyente-de-cc-Andres-Cruz-sobre-Contraloria-General-de-la-Republica-1755-hrs.-01-02.pdf}
\newline {\color{gray} (Emb: 1.000, TF-IDF: 1.000)}
\newline {\color{gray} \textbf{2º:} 173-6-c-Iniciativa-Convencional-del-cc-Rodrigo-Álvarez-Contraloría-1044-hrs.pdf}
\newline {\color{gray} (Emb: 1.000, TF-IDF: 1.000)}

La dirección superior del Servicio Electoral corresponderá a un Consejo Directivo, el que ejercerá de forma exclusiva las atribuciones que le encomienden la Constitución y las leyes. 
\newline {\color{gray} \textbf{1º:} 465-6-Iniciativa-Convencional-Constituyente-de-la-cc-Adriana-Cancino-sobre-Justicia-Electoral-1925-31-01.pdf}
\newline {\color{gray} (Emb: 0.926, TF-IDF: 0.845)}
\newline {\color{gray} \textbf{2º:} 711-Iniciativa-Convencional-Constituyente-de-la-cc-Ingrid-Villena-sobre-Tricel.pdf}
\newline {\color{gray} (Emb: 0.921, TF-IDF: 0.825)}

Dicho Consejo estará integrado por cinco consejeras y consejeros designados por la o el Presidente de la República, con acuerdo de la mayoría de las y los integrantes del Congreso de Diputadas y Diputados y de la Cámara de las Regiones en sesión conjunta. 
\newline {\color{gray} \textbf{1º:} 592-Iniciativa-Convencional-Constituyente-de-cc-Loreto-Vellejos-sobre-Servicio-Electoral-2352-hrs.-01-02.pdf}
\newline {\color{gray} (Emb: 0.828, TF-IDF: 0.645)}
\newline {\color{gray} \textbf{2º:} 602-Iniciativa-Convencional-Constituyente-de-cc-Jorge-Baradit-sobre-Mecanismos-de-Demoracia-Directa-y-Semidirecta.pdf}
\newline {\color{gray} (Emb: 0.719, TF-IDF: 0.319)}

Las y los consejeros durarán ocho años en sus cargos, no podrán ser reelegidos y se renovarán por parcialidades cada cuatro años. 
\newline {\color{gray} \textbf{1º:} 951-5-Iniciativa-Convencional-Constituyente-de-la-cc-Camila-Zarate.pdf}
\newline {\color{gray} (Emb: 0.683, TF-IDF: 0.275)}
\newline {\color{gray} \textbf{2º:} 267-4-Iniciativa-Convencional-de-la-cc-Janis-Meneses-sobre-Derechos-Politicos-1154-hrs.pdf}
\newline {\color{gray} (Emb: 0.598, TF-IDF: 0.269)}

Las y los consejeros sólo podrán ser removidos por la Corte Suprema a requerimiento de la o el Presidente de la República, de la mayoría absoluta de los miembros en ejercicio del Congreso de Diputadas y Diputados o de la Cámara de las Regiones, por infracción grave a la Constitución o a las leyes, incapacidad legal sobreviniente, mal comportamiento o negligencia manifiesta en el ejercicio de sus funciones. 
\newline {\color{gray} \textbf{1º:} 579-Iniciativa-Convencional-Constituyente-de-cc-Christian-Viera-sobre-Sistema-Electoral-y-Justicia-Electoral-2330-hrs.pdf}
\newline {\color{gray} (Emb: 1.000, TF-IDF: 1.000)}
\newline {\color{gray} \textbf{2º:} 711-Iniciativa-Convencional-Constituyente-de-la-cc-Ingrid-Villena-sobre-Tricel.pdf}
\newline {\color{gray} (Emb: 1.000, TF-IDF: 1.000)}


\item \textbf{Artículo} \newline
Estará constituido por cinco juezas y jueces, designados por el Consejo de la Justicia, los cuales deberán postular en la forma y oportunidad que determine la ley respectiva. 
\newline {\color{gray} \textbf{1º:} 240-1-Iniciativa-Convencional-de-la-cc-Tania-Madriaga-sobre-Poder-Legislativo-1146-hrs.pdf}
\newline {\color{gray} (Emb: 0.535, TF-IDF: 0.257)}
\newline {\color{gray} \textbf{2º:} 240-1-Iniciativa-Convencional-de-la-cc-Tania-Madriaga-sobre-Poder-Legislativo-1146-hrs.pdf}
\newline {\color{gray} (Emb: 0.472, TF-IDF: 0.217)}

Una ley regulará la organización y funcionamiento del Tribunal Calificador de Elecciones, planta, remuneraciones y estatuto del personal. 
\newline {\color{gray} \textbf{1º:} 400-1-Iniciativa-Convencional-Constituyente-de-la-cc-Constanza-Hube-sobre-Servicio-y-Registro-Electoral-1905-24-01.pdf}
\newline {\color{gray} (Emb: 0.612, TF-IDF: 0.361)}
\newline {\color{gray} \textbf{2º:} 400-1-Iniciativa-Convencional-Constituyente-de-la-cc-Constanza-Hube-sobre-Servicio-y-Registro-Electoral-1905-24-01.pdf}
\newline {\color{gray} (Emb: 0.594, TF-IDF: 0.351)}

Las juezas y los jueces del Tribunal Calificador de Elecciones durarán seis años en sus funciones. 
\newline {\color{gray} \textbf{1º:} 711-Iniciativa-Convencional-Constituyente-de-la-cc-Ingrid-Villena-sobre-Tricel.pdf}
\newline {\color{gray} (Emb: 1.000, TF-IDF: 1.000)}
\newline {\color{gray} \textbf{2º:} 400-1-Iniciativa-Convencional-Constituyente-de-la-cc-Constanza-Hube-sobre-Servicio-y-Registro-Electoral-1905-24-01.pdf}
\newline {\color{gray} (Emb: 0.889, TF-IDF: 0.948)}

El Tribunal valorará la prueba conforme a las reglas de la sana crítica. 
\newline {\color{gray} \textbf{1º:} 325-6-Iniciativa-Convencional-del-cc-Tomas-Laibe-sobre-Jurisdiccion-Constitucional.pdf}
\newline {\color{gray} (Emb: 0.665, TF-IDF: 0.577)}
\newline {\color{gray} \textbf{2º:} 384-3-Iniciativa-Convencional-Constituyente-del-cc-Felipe-Mena-sobre-Gobiernos-Regionales-1159-24-01.pdf}
\newline {\color{gray} (Emb: 0.593, TF-IDF: 0.511)}

Dicho Tribunal conocerá, asimismo, de los plebiscitos nacionales, y tendrá las demás atribuciones que determine la ley. 
\newline {\color{gray} \textbf{1º:} 465-6-Iniciativa-Convencional-Constituyente-de-la-cc-Adriana-Cancino-sobre-Justicia-Electoral-1925-31-01.pdf}
\newline {\color{gray} (Emb: 0.674, TF-IDF: 0.407)}
\newline {\color{gray} \textbf{2º:} 579-Iniciativa-Convencional-Constituyente-de-cc-Christian-Viera-sobre-Sistema-Electoral-y-Justicia-Electoral-2330-hrs.pdf}
\newline {\color{gray} (Emb: 0.663, TF-IDF: 0.231)}

También conocerá y resolverá sobre las inhabilidades, incompatibilidades y causales de cesación en el cargo de diputadas y diputados del Congreso o los representantes regionales. 
\newline {\color{gray} \textbf{1º:} 711-Iniciativa-Convencional-Constituyente-de-la-cc-Ingrid-Villena-sobre-Tricel.pdf}
\newline {\color{gray} (Emb: 0.989, TF-IDF: 0.990)}
\newline {\color{gray} \textbf{2º:} 465-6-Iniciativa-Convencional-Constituyente-de-la-cc-Adriana-Cancino-sobre-Justicia-Electoral-1925-31-01.pdf}
\newline {\color{gray} (Emb: 0.936, TF-IDF: 0.923)}

El Tribunal Calificador de Elecciones conocerá del escrutinio general y de la calificación de las elecciones de las autoridades electas por votación popular a nivel nacional, resolverá las reclamaciones que se suscitaren y proclamará a los que resulten elegidas y elegidos. 
\newline {\color{gray} \textbf{1º:} 559-Iniciativa-Convencional-Constituyente-de-cc-Andres-Cruz-sobre-Defensoria-Penal-Publica-2017-hrs.-01-02.pdf}
\newline {\color{gray} (Emb: 0.840, TF-IDF: 0.766)}
\newline {\color{gray} \textbf{2º:} 400-1-Iniciativa-Convencional-Constituyente-de-la-cc-Constanza-Hube-sobre-Servicio-y-Registro-Electoral-1905-24-01.pdf}
\newline {\color{gray} (Emb: 0.837, TF-IDF: 0.674)}

De igual manera, calificará la renuncia de éstos, cuando les afecte una enfermedad grave, debidamente acreditada, que les impida desempeñar el cargo. 
\newline {\color{gray} \textbf{1º:} 711-Iniciativa-Convencional-Constituyente-de-la-cc-Ingrid-Villena-sobre-Tricel.pdf}
\newline {\color{gray} (Emb: 0.880, TF-IDF: 0.889)}
\newline {\color{gray} \textbf{2º:} 400-1-Iniciativa-Convencional-Constituyente-de-la-cc-Constanza-Hube-sobre-Servicio-y-Registro-Electoral-1905-24-01.pdf}
\newline {\color{gray} (Emb: 0.731, TF-IDF: 0.720)}

Además, conocerá y resolverá los reclamos administrativos que se entablen contra actos del Servicio Electoral y las decisiones emanadas de tribunales supremos u órganos equivalentes de las organizaciones políticas. 
\newline {\color{gray} \textbf{1º:} 711-Iniciativa-Convencional-Constituyente-de-la-cc-Ingrid-Villena-sobre-Tricel.pdf}
\newline {\color{gray} (Emb: 0.903, TF-IDF: 0.861)}
\newline {\color{gray} \textbf{2º:} 400-1-Iniciativa-Convencional-Constituyente-de-la-cc-Constanza-Hube-sobre-Servicio-y-Registro-Electoral-1905-24-01.pdf}
\newline {\color{gray} (Emb: 0.888, TF-IDF: 0.844)}


\item \textbf{Artículo} \newline
Una ley regulará la organización y funcionamiento de los tribunales electorales regionales, plantas, remuneraciones y estatuto del personal. 
\newline {\color{gray} \textbf{1º:} 579-Iniciativa-Convencional-Constituyente-de-cc-Christian-Viera-sobre-Sistema-Electoral-y-Justicia-Electoral-2330-hrs.pdf}
\newline {\color{gray} (Emb: 0.802, TF-IDF: 0.822)}
\newline {\color{gray} \textbf{2º:} 711-Iniciativa-Convencional-Constituyente-de-la-cc-Ingrid-Villena-sobre-Tricel.pdf}
\newline {\color{gray} (Emb: 0.787, TF-IDF: 0.632)}

Habrá tribunales electorales regionales encargados de conocer el escrutinio general y la calificación de las elecciones de nivel regional, comunal y de organismos de la sociedad civil y demás organizaciones reconocidas por esta Constitución o por la ley, así como resolver las reclamaciones a que dieren lugar y proclamar las candidaturas que resultaren electas. 
\newline {\color{gray} \textbf{1º:} 711-Iniciativa-Convencional-Constituyente-de-la-cc-Ingrid-Villena-sobre-Tricel.pdf}
\newline {\color{gray} (Emb: 0.779, TF-IDF: 0.805)}
\newline {\color{gray} \textbf{2º:} 465-6-Iniciativa-Convencional-Constituyente-de-la-cc-Adriana-Cancino-sobre-Justicia-Electoral-1925-31-01.pdf}
\newline {\color{gray} (Emb: 0.683, TF-IDF: 0.686)}

Conocerán, asimismo, de los plebiscitos regionales y comunales, sin perjuicio de las demás atribuciones que determine la ley. 
\newline {\color{gray} \textbf{1º:} 711-Iniciativa-Convencional-Constituyente-de-la-cc-Ingrid-Villena-sobre-Tricel.pdf}
\newline {\color{gray} (Emb: 1.000, TF-IDF: 1.000)}
\newline {\color{gray} \textbf{2º:} 711-Iniciativa-Convencional-Constituyente-de-la-cc-Ingrid-Villena-sobre-Tricel.pdf}
\newline {\color{gray} (Emb: 0.832, TF-IDF: 0.530)}

Sus resoluciones serán apelables y su conocimiento corresponderá al Tribunal Calificador de Elecciones en la forma que determine la ley. 
\newline {\color{gray} \textbf{1º:} 711-Iniciativa-Convencional-Constituyente-de-la-cc-Ingrid-Villena-sobre-Tricel.pdf}
\newline {\color{gray} (Emb: 0.998, TF-IDF: 0.850)}
\newline {\color{gray} \textbf{2º:} 711-Iniciativa-Convencional-Constituyente-de-la-cc-Ingrid-Villena-sobre-Tricel.pdf}
\newline {\color{gray} (Emb: 0.846, TF-IDF: 0.615)}

Igualmente, les corresponderá conocer de la calificación de las elecciones de carácter gremial y de las que tengan lugar en aquellas organizaciones que la ley señale. 
\newline {\color{gray} \textbf{1º:} 711-Iniciativa-Convencional-Constituyente-de-la-cc-Ingrid-Villena-sobre-Tricel.pdf}
\newline {\color{gray} (Emb: 0.904, TF-IDF: 0.726)}
\newline {\color{gray} \textbf{2º:} 400-1-Iniciativa-Convencional-Constituyente-de-la-cc-Constanza-Hube-sobre-Servicio-y-Registro-Electoral-1905-24-01.pdf}
\newline {\color{gray} (Emb: 0.826, TF-IDF: 0.686)}

Los tribunales electorales regionales estarán constituidos por tres juezas y jueces, designados por el Consejo de la Justicia, los cuales deberán postular en la forma y oportunidad que determine la ley respectiva. 
\newline {\color{gray} \textbf{1º:} 120-3-c-Iniciativa-de-la-cc-Tammy-Pustilnick-atribuciones-exclusivas-de-la-Asamblea-Regional.pdf}
\newline {\color{gray} (Emb: 0.719, TF-IDF: 0.496)}
\newline {\color{gray} \textbf{2º:} 631-Iniciativa-Convencional-Constituyente-de-cc-Ingrid-Villena-sobre-Contraloria-General-de-la-Republica.pdf}
\newline {\color{gray} (Emb: 0.706, TF-IDF: 0.470)}

Las juezas y los jueces de los tribunales electorales regionales durarán seis años en sus funciones. 
\newline {\color{gray} \textbf{1º:} 711-Iniciativa-Convencional-Constituyente-de-la-cc-Ingrid-Villena-sobre-Tricel.pdf}
\newline {\color{gray} (Emb: 0.896, TF-IDF: 0.862)}
\newline {\color{gray} \textbf{2º:} 400-1-Iniciativa-Convencional-Constituyente-de-la-cc-Constanza-Hube-sobre-Servicio-y-Registro-Electoral-1905-24-01.pdf}
\newline {\color{gray} (Emb: 0.896, TF-IDF: 0.862)}

Estos tribunales valorarán la prueba conforme a las reglas de la sana crítica. 
\newline {\color{gray} \textbf{1º:} 711-Iniciativa-Convencional-Constituyente-de-la-cc-Ingrid-Villena-sobre-Tricel.pdf}
\newline {\color{gray} (Emb: 0.998, TF-IDF: 0.919)}
\newline {\color{gray} \textbf{2º:} 400-1-Iniciativa-Convencional-Constituyente-de-la-cc-Constanza-Hube-sobre-Servicio-y-Registro-Electoral-1905-24-01.pdf}
\newline {\color{gray} (Emb: 0.856, TF-IDF: 0.750)}


\item \textbf{Artículo} \newline
La gestión administrativa y la superintendencia directiva y correccional del Tribunal Calificador de Elecciones y de los tribunales electorales regionales corresponderá al Consejo de la Justicia. 
\newline {\color{gray} \textbf{1º:} 711-Iniciativa-Convencional-Constituyente-de-la-cc-Ingrid-Villena-sobre-Tricel.pdf}
\newline {\color{gray} (Emb: 0.920, TF-IDF: 0.781)}
\newline {\color{gray} \textbf{2º:} 711-Iniciativa-Convencional-Constituyente-de-la-cc-Ingrid-Villena-sobre-Tricel.pdf}
\newline {\color{gray} (Emb: 0.916, TF-IDF: 0.675)}


\item \textbf{Artículo} \newline
Las funciones de la Dirección del Servicio Civil respecto de los procesos de selección de la Administración Pública en los distintos niveles será determinado por ley. 
\newline {\color{gray} \textbf{1º:} 909-Iniciativa-Convencional-Constituyente-del-cc-Hugo-Gutierrez-Sobre-Ministerio-Publico.pdf}
\newline {\color{gray} (Emb: 0.648, TF-IDF: 0.318)}
\newline {\color{gray} \textbf{2º:} 924-Iniciativa-Convencional-Constituyente-de-la-cc-Constanza-Schonhaut-sobre-Administracion-del-Estado.pdf}
\newline {\color{gray} (Emb: 0.599, TF-IDF: 0.302)}

Las atribuciones de la Dirección del Servicio Civil no afectarán las competencias que, en el ámbito de la gestión, correspondan a las autoridades y jefaturas de los servicios públicos. 
\newline {\color{gray} \textbf{1º:} 122-3-c-Iniciativa-de-la-cc-Jennifer-Mella-Forma-del-Estado.pdf}
\newline {\color{gray} (Emb: 0.756, TF-IDF: 0.382)}
\newline {\color{gray} \textbf{2º:} 159-3-c-Iniciativa-de-la-cc-Jennifer-Mella-.pdf}
\newline {\color{gray} (Emb: 0.755, TF-IDF: 0.349)}

La Dirección del Servicio Civil estará encargada de regular los procesos de selección de candidatas y candidatos a cargos del Sistema de Alta Dirección Pública, o de aquellos que deben seleccionarse con su participación y conducir los concursos destinados a proveer cargos de jefaturas superiores de servicios, a través de un Consejo de Alta Dirección Pública. 
\newline {\color{gray} \textbf{1º:} 465-6-Iniciativa-Convencional-Constituyente-de-la-cc-Adriana-Cancino-sobre-Justicia-Electoral-1925-31-01.pdf}
\newline {\color{gray} (Emb: 0.714, TF-IDF: 0.430)}
\newline {\color{gray} \textbf{2º:} 711-Iniciativa-Convencional-Constituyente-de-la-cc-Ingrid-Villena-sobre-Tricel.pdf}
\newline {\color{gray} (Emb: 0.694, TF-IDF: 0.350)}

Se excluyen del Servicio Civil los cargos de exclusiva confianza del gobierno central, regional y municipal. 
\newline {\color{gray} \textbf{1º:} 400-1-Iniciativa-Convencional-Constituyente-de-la-cc-Constanza-Hube-sobre-Servicio-y-Registro-Electoral-1905-24-01.pdf}
\newline {\color{gray} (Emb: 0.799, TF-IDF: 0.821)}
\newline {\color{gray} \textbf{2º:} 465-6-Iniciativa-Convencional-Constituyente-de-la-cc-Adriana-Cancino-sobre-Justicia-Electoral-1925-31-01.pdf}
\newline {\color{gray} (Emb: 0.786, TF-IDF: 0.809)}

La Dirección del Servicio Civil será un organismo autónomo, con personalidad jurídica y patrimonio propio, encargado del fortalecimiento de la función pública y de los procedimientos de selección de cargos en la Administración Pública y demás entidades que establezca la Constitución y la ley, resguardando los principios de transparencia, objetividad, no discriminación y mérito. 
\newline {\color{gray} \textbf{1º:} 711-Iniciativa-Convencional-Constituyente-de-la-cc-Ingrid-Villena-sobre-Tricel.pdf}
\newline {\color{gray} (Emb: 0.620, TF-IDF: 0.373)}
\newline {\color{gray} \textbf{2º:} 400-1-Iniciativa-Convencional-Constituyente-de-la-cc-Constanza-Hube-sobre-Servicio-y-Registro-Electoral-1905-24-01.pdf}
\newline {\color{gray} (Emb: 0.620, TF-IDF: 0.351)}

La ley regulará la organización y demás atribuciones de la Dirección del Servicio Civil. 
\newline {\color{gray} \textbf{1º:} 465-6-Iniciativa-Convencional-Constituyente-de-la-cc-Adriana-Cancino-sobre-Justicia-Electoral-1925-31-01.pdf}
\newline {\color{gray} (Emb: 0.725, TF-IDF: 0.413)}
\newline {\color{gray} \textbf{2º:} 813-Iniciativa-Convencional-Constituyente-del-cc-Manuel-Woldarsky-crea-la-Agencia-de-DDHH.pdf}
\newline {\color{gray} (Emb: 0.721, TF-IDF: 0.402)}

El Servicio Civil está integrado por las funcionarias y funcionarios públicos que, bajo la dirección del Gobierno, los Gobiernos Regionales o las Municipalidades, desarrollan las funciones de la Administración Pública. 
\newline {\color{gray} \textbf{1º:} 711-Iniciativa-Convencional-Constituyente-de-la-cc-Ingrid-Villena-sobre-Tricel.pdf}
\newline {\color{gray} (Emb: 0.926, TF-IDF: 0.895)}
\newline {\color{gray} \textbf{2º:} 711-Iniciativa-Convencional-Constituyente-de-la-cc-Ingrid-Villena-sobre-Tricel.pdf}
\newline {\color{gray} (Emb: 0.916, TF-IDF: 0.736)}


\item \textbf{Artículo} \newline
Existirá un órgano encargado de la protección de las personas en su rol de consumidoras y usuarias de bienes y servicios, el cual contará con facultades interpretativas, fiscalizadoras, sancionadoras y las demás que le otorgue la ley. 
\newline {\color{gray} \textbf{1º:} 134-4-c-Iniciativa-de-la-cc-Rocio-Cantuarias-Establece-la-libertad-de-ejercer-actividades-economicas-1.pdf}
\newline {\color{gray} (Emb: 0.652, TF-IDF: 0.433)}
\newline {\color{gray} \textbf{2º:} 124-4-c-Iniciativa-del-cc-Bernardo-Fontaine-Derecho-a-Emprender.pdf}
\newline {\color{gray} (Emb: 0.652, TF-IDF: 0.310)}


\item \textbf{Artículo} \newline
Sus resoluciones se fundarán únicamente en razones de derecho. 
\newline {\color{gray} \textbf{1º:} 733-Iniciativa-Convencional-Constituyente-del-cc-Bernardo-Fontaine-Consejo-de-Evaluacion-de-Politicas-Publicas-01-02.pdf}
\newline {\color{gray} (Emb: 0.749, TF-IDF: 0.572)}
\newline {\color{gray} \textbf{2º:} 398-5-Iniciativa-Convencional-Constituyente-de-la-cc-Ivanna-Olivares-sobre-Banca-Publica-1823-24-01.pdf}
\newline {\color{gray} (Emb: 0.746, TF-IDF: 0.545)}

La Corte Constitucional es un órgano autónomo, técnico y profesional, encargado de ejercer la justicia constitucional con la finalidad de garantizar la supremacía de la Constitución, de acuerdo a los principios de deferencia al órgano legislativo, presunción de constitucionalidad de la ley y búsqueda de una interpretación conforme con la Constitución. 
\newline {\color{gray} \textbf{1º:} 937-IniciativaConvencional-Constituyente-del-cc-Tomas-Laibe-sobre-Banco-Central.pdf}
\newline {\color{gray} (Emb: 0.607, TF-IDF: 0.423)}
\newline {\color{gray} \textbf{2º:} 95-6-Iniciativa-Convencional-Constituyente-de-Cc-Mauricio-Daza-y-otros-2.pdf}
\newline {\color{gray} (Emb: 0.600, TF-IDF: 0.304)}


\item \textbf{Artículo} \newline
Una ley determinará la organización, funcionamiento, procedimientos y fijará la planta, régimen de remuneraciones y estatuto del personal de la Corte Constitucional. 
\newline {\color{gray} \textbf{1º:} 88-6-Iniciativa-Convencional-Constituyente-del-cc-Christian-Viera-y-otros.pdf}
\newline {\color{gray} (Emb: 0.843, TF-IDF: 0.565)}
\newline {\color{gray} \textbf{2º:} 472-6-Iniciativa-Convencional-Constituyente-del-cc-Daniel-Bravo-sobre-Corte-Constitucional-2003-31-01.pdf}
\newline {\color{gray} (Emb: 0.821, TF-IDF: 0.463)}

De igual manera, las juezas o jueces de la Corte Constitucional no podrán tener impedimento que los inhabilite para desempeñar el cargo de jueza o juez del Sistema Nacional de Justicia. 
\newline {\color{gray} \textbf{1º:} 41-6-Iniciativa-Convencional-Constituyente-del-cc-Mauricio-Daza-y-otros-1.pdf}
\newline {\color{gray} (Emb: 0.772, TF-IDF: 0.751)}
\newline {\color{gray} \textbf{2º:} 95-6-Iniciativa-Convencional-Constituyente-de-Cc-Mauricio-Daza-y-otros-2.pdf}
\newline {\color{gray} (Emb: 0.737, TF-IDF: 0.751)}

No podrán ser juezas o jueces de la Corte Constitucional quienes se hubiesen desempeñado en cargos de elección popular, quienes hayan desempeñado el cargo de Ministra o Ministro de Estado u otros cargos de exclusiva confianza del gobierno, durante los dos años anteriores a la elección. 
\newline {\color{gray} \textbf{1º:} 729-Iniciativa-Convencional-Constituyente-del-cc-Ruggero-Cozzi-sobre-Justicia-Constitucional-y-Administrativa.pdf}
\newline {\color{gray} (Emb: 0.706, TF-IDF: 0.354)}
\newline {\color{gray} \textbf{2º:} 325-6-Iniciativa-Convencional-del-cc-Tomas-Laibe-sobre-Jurisdiccion-Constitucional.pdf}
\newline {\color{gray} (Emb: 0.703, TF-IDF: 0.347)}

Las y los postulantes al cargo de jueza o juez de la Corte Constitucional deberán ser abogadas o abogados, con más de quince años de ejercicio profesional, con reconocida y comprobada competencia e idoneidad profesional o académica y, preferentemente, de distintas especialidades del Derecho. 
\newline {\color{gray} \textbf{1º:} 915-Iniciativa-Convencional-Constituyente-del-cc-Luis-Jimenez-sobre-Plurinacionalidad.pdf}
\newline {\color{gray} (Emb: 0.754, TF-IDF: 0.521)}
\newline {\color{gray} \textbf{2º:} 321-1-Iniciativa-Convencional-Constituyente-de-la-cc-Barbara-Sepulveda-sobre-Estatuto-de-Diputados-y-Diputadas.pdf}
\newline {\color{gray} (Emb: 0.681, TF-IDF: 0.372)}

En caso de ser designados juezas o jueces del Sistema Nacional de Justicia, quedarán suspendidos de sus cargos judiciales de origen en tanto se extienda su función en la Corte Constitucional. 
\newline {\color{gray} \textbf{1º:} 232-6-Iniciativa-Convencional-del-cc-Marco-Arellano-que-Crea-el-Consejo-Nacional-de-Justicia-1144-hrs.pdf}
\newline {\color{gray} (Emb: 0.568, TF-IDF: 0.389)}
\newline {\color{gray} \textbf{2º:} 425-6-Iniciativa-Convencional-del-cc-Christian-Viera-sobre-Reforma-de-la-Constitucion-1554-26-01.pdf}
\newline {\color{gray} (Emb: 0.562, TF-IDF: 0.314)}

b) Tres integrantes elegidos por la o el Presidente de la República. 
\newline {\color{gray} \textbf{1º:} 220-6-c-Iniciativa-Convencional-del-cc-Daniel-Bravo-sobre-organizacion-de-tribunales-2315-hrs.pdf}
\newline {\color{gray} (Emb: 0.852, TF-IDF: 0.759)}
\newline {\color{gray} \textbf{2º:} 472-6-Iniciativa-Convencional-Constituyente-del-cc-Daniel-Bravo-sobre-Corte-Constitucional-2003-31-01.pdf}
\newline {\color{gray} (Emb: 0.727, TF-IDF: 0.660)}

Su designación se efectuará en base a criterios técnicos y de mérito profesional de la siguiente manera: a) Cuatro integrantes elegidos por la mayoría de las y los integrantes del Congreso de Diputadas y Diputados y de la Cámara de las Regiones en sesión conjunta. 
\newline {\color{gray} \textbf{1º:} 98-6-Iniciativa-del-cc-Ruggero-Cozzi-Funcion-y-Principios-de-la-Jurisdiccion.pdf}
\newline {\color{gray} (Emb: 0.717, TF-IDF: 0.371)}
\newline {\color{gray} \textbf{2º:} 180-6-c-Iniciativa-Convencional-del-cc-Rodrigo-Álvarez-que-regula-el-Poder-Judicial-1044-hrs.pdf}
\newline {\color{gray} (Emb: 0.681, TF-IDF: 0.273)}

Las juezas y jueces de la Corte Constitucional durarán nueve años en sus cargos, no reelegibles, y se renovarán por parcialidades cada tres años en la forma que establezca la ley. 
\newline {\color{gray} \textbf{1º:} 472-6-Iniciativa-Convencional-Constituyente-del-cc-Daniel-Bravo-sobre-Corte-Constitucional-2003-31-01.pdf}
\newline {\color{gray} (Emb: 0.814, TF-IDF: 0.648)}
\newline {\color{gray} \textbf{2º:} 472-6-Iniciativa-Convencional-Constituyente-del-cc-Daniel-Bravo-sobre-Corte-Constitucional-2003-31-01.pdf}
\newline {\color{gray} (Emb: 0.777, TF-IDF: 0.436)}

Estará conformada por once integrantes, uno de los cuales será su presidenta o presidente elegido por sus pares y que ejercerá sus funciones durante dos años. 
\newline {\color{gray} \textbf{1º:} 888-Iniciativa-Convencional-Constituyente-de-la-cc-Natalia-Henriquez-sobre-Derechos-de-los-Consumidores.pdf}
\newline {\color{gray} (Emb: 0.745, TF-IDF: 0.388)}
\newline {\color{gray} \textbf{2º:} 839-Iniciativa-Convencional-Constituyente-de-la-cc-Elsa-Labrana-sobre-Derecho-de-los-Consumidores.pdf}
\newline {\color{gray} (Emb: 0.694, TF-IDF: 0.252)}

c) Cuatro integrantes elegidos por el Consejo de la Justicia, a partir de concursos públicos. 
\newline {\color{gray} \textbf{1º:} 472-6-Iniciativa-Convencional-Constituyente-del-cc-Daniel-Bravo-sobre-Corte-Constitucional-2003-31-01.pdf}
\newline {\color{gray} (Emb: 0.858, TF-IDF: 0.836)}
\newline {\color{gray} \textbf{2º:} 184-6-c-Iniciativa-Convencional-del-cc-Rodrigo-Álvarez-que-crea-la-Corte-Constitucional-1044hrs.pdf}
\newline {\color{gray} (Emb: 0.853, TF-IDF: 0.821)}


\item \textbf{Artículo} \newline
Las juezas y jueces de la Corte Constitucional son independientes de todo otro poder y gozan de inamovilidad. 
\newline {\color{gray} \textbf{1º:} 472-6-Iniciativa-Convencional-Constituyente-del-cc-Daniel-Bravo-sobre-Corte-Constitucional-2003-31-01.pdf}
\newline {\color{gray} (Emb: 0.912, TF-IDF: 0.897)}
\newline {\color{gray} \textbf{2º:} 325-6-Iniciativa-Convencional-del-cc-Tomas-Laibe-sobre-Jurisdiccion-Constitucional.pdf}
\newline {\color{gray} (Emb: 0.834, TF-IDF: 0.616)}

Cesarán en sus cargos por haber cumplido su periodo, por incapacidad legal sobreviniente, por renuncia, por sentencia penal condenatoria, por remoción, enfermedad incompatible con el ejercicio de la función u otra causa establecida en la ley. 
\newline {\color{gray} \textbf{1º:} 41-6-Iniciativa-Convencional-Constituyente-del-cc-Mauricio-Daza-y-otros-1.pdf}
\newline {\color{gray} (Emb: 0.767, TF-IDF: 0.567)}
\newline {\color{gray} \textbf{2º:} 95-6-Iniciativa-Convencional-Constituyente-de-Cc-Mauricio-Daza-y-otros-2.pdf}
\newline {\color{gray} (Emb: 0.756, TF-IDF: 0.567)}


\item \textbf{Artículo} \newline
El ejercicio del cargo de jueza o juez de la Corte Constitucional es de dedicación exclusiva. 
\newline {\color{gray} \textbf{1º:} 472-6-Iniciativa-Convencional-Constituyente-del-cc-Daniel-Bravo-sobre-Corte-Constitucional-2003-31-01.pdf}
\newline {\color{gray} (Emb: 1.000, TF-IDF: 1.000)}
\newline {\color{gray} \textbf{2º:} 184-6-c-Iniciativa-Convencional-del-cc-Rodrigo-Álvarez-que-crea-la-Corte-Constitucional-1044hrs.pdf}
\newline {\color{gray} (Emb: 0.915, TF-IDF: 0.910)}

La ley determinará las demás incompatibilidades e inhabilidades para el desempeño de este cargo. 
\newline {\color{gray} \textbf{1º:} 915-Iniciativa-Convencional-Constituyente-del-cc-Luis-Jimenez-sobre-Plurinacionalidad.pdf}
\newline {\color{gray} (Emb: 0.894, TF-IDF: 0.865)}
\newline {\color{gray} \textbf{2º:} 90-6-Iniciativa-Convencional-Constituyente-del-cc-Tomas-Laibe-y-otros.pdf}
\newline {\color{gray} (Emb: 0.873, TF-IDF: 0.704)}

Al terminar su periodo, y durante los dieciocho meses siguientes, no podrán optar a ningún cargo de elección popular ni de exclusiva confianza de alguna autoridad. 
\newline {\color{gray} \textbf{1º:} 915-Iniciativa-Convencional-Constituyente-del-cc-Luis-Jimenez-sobre-Plurinacionalidad.pdf}
\newline {\color{gray} (Emb: 0.881, TF-IDF: 0.905)}
\newline {\color{gray} \textbf{2º:} 472-6-Iniciativa-Convencional-Constituyente-del-cc-Daniel-Bravo-sobre-Corte-Constitucional-2003-31-01.pdf}
\newline {\color{gray} (Emb: 0.675, TF-IDF: 0.444)}


\item \textbf{Artículo} \newline
La Corte Constitucional resolverá los conflictos de competencia entre el Congreso de Diputadas y Diputados y la Cámara de las Regiones, o entre éstas y el Presidente de la República. 
\newline {\color{gray} \textbf{1º:} 915-Iniciativa-Convencional-Constituyente-del-cc-Luis-Jimenez-sobre-Plurinacionalidad.pdf}
\newline {\color{gray} (Emb: 0.795, TF-IDF: 0.683)}
\newline {\color{gray} \textbf{2º:} 97-6-Iniciativa-del-cc-Daniel-Bravo-Sistemas-de-Justicia.pdf}
\newline {\color{gray} (Emb: 0.706, TF-IDF: 0.523)}


\item \textbf{Artículo} \newline
Esta inconstitucionalidad será declarada por un quórum de cuatro quintos de sus integrantes en ejercicio. 
\newline {\color{gray} \textbf{1º:} 325-6-Iniciativa-Convencional-del-cc-Tomas-Laibe-sobre-Jurisdiccion-Constitucional.pdf}
\newline {\color{gray} (Emb: 0.543, TF-IDF: 0.310)}
\newline {\color{gray} \textbf{2º:} 472-6-Iniciativa-Convencional-Constituyente-del-cc-Daniel-Bravo-sobre-Corte-Constitucional-2003-31-01.pdf}
\newline {\color{gray} (Emb: 0.528, TF-IDF: 0.287)}

Esta declaración de inconstitucionalidad se efectuará con el voto conforme de los tres quintos de las y los integrantes en ejercicio de la Corte Constitucional. 
\newline {\color{gray} \textbf{1º:} 325-6-Iniciativa-Convencional-del-cc-Tomas-Laibe-sobre-Jurisdiccion-Constitucional.pdf}
\newline {\color{gray} (Emb: 0.686, TF-IDF: 0.357)}
\newline {\color{gray} \textbf{2º:} 940-Iniciativa-Convencional-Constituyente-del-cc-Christian-Viera-sobre-Proteccion-de-Derechos-Fundamentales.pdf}
\newline {\color{gray} (Emb: 0.646, TF-IDF: 0.259)}

Asimismo, tratándose del número 2, la Corte Constitucional podrá declarar la inconstitucionalidad de un precepto legal, que hubiera sido declarado inaplicable previamente conforme al número 1 de este artículo, a petición de la o el Presidente de la República, de un tercio de las y los integrantes del Congreso de Diputadas y Diputados o de la Cámara de las Regiones, de una o un Gobernador Regional, o de a lo menos la mitad de los integrantes de una Asamblea Regional. 
\newline {\color{gray} \textbf{1º:} 89-6-Iniciativa-Convencional-Constituyente-del-cc-Christian-Viera-y-otros.pdf}
\newline {\color{gray} (Emb: 0.845, TF-IDF: 0.822)}
\newline {\color{gray} \textbf{2º:} 325-6-Iniciativa-Convencional-del-cc-Tomas-Laibe-sobre-Jurisdiccion-Constitucional.pdf}
\newline {\color{gray} (Emb: 0.746, TF-IDF: 0.392)}

En el caso del número 3, la cuestión podrá ser planteada por la Presidenta o el Presidente de la República, o un tercio de las o los integrantes de la Cámara de las Regiones. 
\newline {\color{gray} \textbf{1º:} 472-6-Iniciativa-Convencional-Constituyente-del-cc-Daniel-Bravo-sobre-Corte-Constitucional-2003-31-01.pdf}
\newline {\color{gray} (Emb: 0.733, TF-IDF: 0.385)}
\newline {\color{gray} \textbf{2º:} 325-6-Iniciativa-Convencional-del-cc-Tomas-Laibe-sobre-Jurisdiccion-Constitucional.pdf}
\newline {\color{gray} (Emb: 0.695, TF-IDF: 0.385)}

Tratándose del número 2, existiendo dos o más declaraciones de inaplicabilidad de un precepto legal conforme al número 1 de este artículo, habrá acción pública para requerir a la Corte la declaración de inconstitucionalidad, sin perjuicio de la facultad de ésta para declararla de oficio. 
\newline {\color{gray} \textbf{1º:} 325-6-Iniciativa-Convencional-del-cc-Tomas-Laibe-sobre-Jurisdiccion-Constitucional.pdf}
\newline {\color{gray} (Emb: 0.712, TF-IDF: 0.238)}
\newline {\color{gray} \textbf{2º:} 325-6-Iniciativa-Convencional-del-cc-Tomas-Laibe-sobre-Jurisdiccion-Constitucional.pdf}
\newline {\color{gray} (Emb: 0.703, TF-IDF: 0.238)}

Si la Corte acogiera el reclamo, promulgará en su sentencia la ley que no lo haya sido o rectificará la promulgación incorrecta. 
\newline {\color{gray} \textbf{1º:} 325-6-Iniciativa-Convencional-del-cc-Tomas-Laibe-sobre-Jurisdiccion-Constitucional.pdf}
\newline {\color{gray} (Emb: 0.645, TF-IDF: 0.351)}
\newline {\color{gray} \textbf{2º:} 180-6-c-Iniciativa-Convencional-del-cc-Rodrigo-Álvarez-que-regula-el-Poder-Judicial-1044-hrs.pdf}
\newline {\color{gray} (Emb: 0.640, TF-IDF: 0.297)}

En el caso del número 5 bis, la Corte podrá conocer de la materia a requerimiento de cualquiera del Congreso de Diputadas y Diputados o de la Cámara de las Regiones, o un tercio de sus integrantes, dentro de los treinta días siguientes a la publicación o notificación del texto impugnado. 
\newline {\color{gray} \textbf{1º:} 755-Iniciativa-Convencional-Constituyente-del-cc-Roberto-Celedon-sobre-Democracia-Plena-01-02.pdf}
\newline {\color{gray} (Emb: 0.666, TF-IDF: 0.507)}
\newline {\color{gray} \textbf{2º:} 240-1-Iniciativa-Convencional-de-la-cc-Tania-Madriaga-sobre-Poder-Legislativo-1146-hrs.pdf}
\newline {\color{gray} (Emb: 0.639, TF-IDF: 0.478)}

En el caso de los conflictos de competencia contemplados en los números 6 y 7, podrán ser deducidas por cualquiera de las autoridades o tribunales en conflicto. 
\newline {\color{gray} \textbf{1º:} 184-6-c-Iniciativa-Convencional-del-cc-Rodrigo-Álvarez-que-crea-la-Corte-Constitucional-1044hrs.pdf}
\newline {\color{gray} (Emb: 0.894, TF-IDF: 0.902)}
\newline {\color{gray} \textbf{2º:} 184-6-c-Iniciativa-Convencional-del-cc-Rodrigo-Álvarez-que-crea-la-Corte-Constitucional-1044hrs.pdf}
\newline {\color{gray} (Emb: 0.681, TF-IDF: 0.473)}

En lo demás, el procedimiento, el quórum y la legitimación activa para el ejercicio de cada atribución se determinará por la ley. 
\newline {\color{gray} \textbf{1º:} 184-6-c-Iniciativa-Convencional-del-cc-Rodrigo-Álvarez-que-crea-la-Corte-Constitucional-1044hrs.pdf}
\newline {\color{gray} (Emb: 0.994, TF-IDF: 0.871)}
\newline {\color{gray} \textbf{2º:} 89-6-Iniciativa-Convencional-Constituyente-del-cc-Christian-Viera-y-otros.pdf}
\newline {\color{gray} (Emb: 0.856, TF-IDF: 0.264)}

La Corte Constitucional decidirá la cuestión de inaplicabilidad por mayoría de sus integrantes. 
\newline {\color{gray} \textbf{1º:} 472-6-Iniciativa-Convencional-Constituyente-del-cc-Daniel-Bravo-sobre-Corte-Constitucional-2003-31-01.pdf}
\newline {\color{gray} (Emb: 0.696, TF-IDF: 0.376)}
\newline {\color{gray} \textbf{2º:} 915-Iniciativa-Convencional-Constituyente-del-cc-Luis-Jimenez-sobre-Plurinacionalidad.pdf}
\newline {\color{gray} (Emb: 0.648, TF-IDF: 0.349)}

En el caso del número 4, la cuestión podrá promoverse por cualquiera de los órganos legislativos o por una cuarta parte de sus integrantes en ejercicio, dentro de los treinta días siguientes a la publicación del texto impugnado o dentro de los sesenta días siguientes a la fecha en que la Presidenta o el Presidente de la República debió efectuar la promulgación de la ley. 
\newline {\color{gray} \textbf{1º:} 559-Iniciativa-Convencional-Constituyente-de-cc-Andres-Cruz-sobre-Defensoria-Penal-Publica-2017-hrs.-01-02.pdf}
\newline {\color{gray} (Emb: 0.692, TF-IDF: 0.388)}
\newline {\color{gray} \textbf{2º:} 465-6-Iniciativa-Convencional-Constituyente-de-la-cc-Adriana-Cancino-sobre-Justicia-Electoral-1925-31-01.pdf}
\newline {\color{gray} (Emb: 0.645, TF-IDF: 0.315)}

El pronunciamiento del juez en esta materia no lo inhabilitará para seguir conociendo el caso concreto. 
\newline {\color{gray} \textbf{1º:} 325-6-Iniciativa-Convencional-del-cc-Tomas-Laibe-sobre-Jurisdiccion-Constitucional.pdf}
\newline {\color{gray} (Emb: 0.951, TF-IDF: 0.834)}
\newline {\color{gray} \textbf{2º:} 472-6-Iniciativa-Convencional-Constituyente-del-cc-Daniel-Bravo-sobre-Corte-Constitucional-2003-31-01.pdf}
\newline {\color{gray} (Emb: 0.860, TF-IDF: 0.740)}

No procederá esta solicitud si el asunto está sometido al conocimiento de la Corte Suprema. 
\newline {\color{gray} \textbf{1º:} 472-6-Iniciativa-Convencional-Constituyente-del-cc-Daniel-Bravo-sobre-Corte-Constitucional-2003-31-01.pdf}
\newline {\color{gray} (Emb: 0.843, TF-IDF: 0.969)}
\newline {\color{gray} \textbf{2º:} 229-1Iniciativa-Convencional-de-la-cc-Alondra-Carrillo-que-Crea-el-Consejo-de-Administracion-Legislativa-1142-hrs.pdf}
\newline {\color{gray} (Emb: 0.841, TF-IDF: 0.951)}

La Corte Constitucional tendrá las siguientes atribuciones, ejerciéndolas conforme a los principios referidos en el artículo [65]: 1. 
\newline {\color{gray} \textbf{1º:} 325-6-Iniciativa-Convencional-del-cc-Tomas-Laibe-sobre-Jurisdiccion-Constitucional.pdf}
\newline {\color{gray} (Emb: 0.886, TF-IDF: 0.821)}
\newline {\color{gray} \textbf{2º:} 472-6-Iniciativa-Convencional-Constituyente-del-cc-Daniel-Bravo-sobre-Corte-Constitucional-2003-31-01.pdf}
\newline {\color{gray} (Emb: 0.884, TF-IDF: 0.821)}

Tratándose del número 1, el tribunal de una gestión pendiente, de oficio o previa petición de parte, podrá plantear una cuestión de constitucionalidad respecto de un precepto legal decisorio para la resolución de dicho asunto. 
\newline {\color{gray} \textbf{1º:} 915-Iniciativa-Convencional-Constituyente-del-cc-Luis-Jimenez-sobre-Plurinacionalidad.pdf}
\newline {\color{gray} (Emb: 0.827, TF-IDF: 0.603)}
\newline {\color{gray} \textbf{2º:} 325-6-Iniciativa-Convencional-del-cc-Tomas-Laibe-sobre-Jurisdiccion-Constitucional.pdf}
\newline {\color{gray} (Emb: 0.799, TF-IDF: 0.413)}

Conocer y resolver la inaplicabilidad de un precepto legal cuyos efectos sean contrarios a la Constitución. 
\newline {\color{gray} \textbf{1º:} 472-6-Iniciativa-Convencional-Constituyente-del-cc-Daniel-Bravo-sobre-Corte-Constitucional-2003-31-01.pdf}
\newline {\color{gray} (Emb: 0.788, TF-IDF: 0.883)}
\newline {\color{gray} \textbf{2º:} 409-6-Iniciativa-Convencional-Constituyente-de-la-cc-Ingrid-Villena-sobre-Defensoria-del-Pueblo-2239-24-01.pdf}
\newline {\color{gray} (Emb: 0.760, TF-IDF: 0.656)}

Conocer y resolver sobre la inconstitucionalidad de uno o más preceptos de estatutos regionales, de autonomías territoriales indígenas y de cualquier otra entidad territorial. 
\newline {\color{gray} \textbf{1º:} 472-6-Iniciativa-Convencional-Constituyente-del-cc-Daniel-Bravo-sobre-Corte-Constitucional-2003-31-01.pdf}
\newline {\color{gray} (Emb: 0.712, TF-IDF: 0.714)}
\newline {\color{gray} \textbf{2º:} 915-Iniciativa-Convencional-Constituyente-del-cc-Luis-Jimenez-sobre-Plurinacionalidad.pdf}
\newline {\color{gray} (Emb: 0.667, TF-IDF: 0.658)}

Conocer y resolver los reclamos en caso que la Presidenta o el Presidente de la República no promulgue una ley cuando deba hacerlo o promulgue un texto diverso del que constitucionalmente corresponda. 
\newline {\color{gray} \textbf{1º:} 325-6-Iniciativa-Convencional-del-cc-Tomas-Laibe-sobre-Jurisdiccion-Constitucional.pdf}
\newline {\color{gray} (Emb: 0.695, TF-IDF: 0.479)}
\newline {\color{gray} \textbf{2º:} 325-6-Iniciativa-Convencional-del-cc-Tomas-Laibe-sobre-Jurisdiccion-Constitucional.pdf}
\newline {\color{gray} (Emb: 0.666, TF-IDF: 0.363)}

Igual atribución tendrá respecto de la promulgación de la normativa regional. 
\newline {\color{gray} \textbf{1º:} 353-1-Iniciativa-Convencional-Constituyente-del-cc-Jaime-Bassa-sobre-Sistema-de-Gobierno-1158-21-01.pdf}
\newline {\color{gray} (Emb: 0.725, TF-IDF: 0.457)}
\newline {\color{gray} \textbf{2º:} 871-Iniciativa-Convencional-Constituyente-de-la-cc-Amaya-Alvez-sobre-Asambleas-Regionales.pdf}
\newline {\color{gray} (Emb: 0.721, TF-IDF: 0.454)}

Conocer y resolver sobre la inconstitucionalidad de un precepto legal. 
\newline {\color{gray} \textbf{1º:} 151-3-c-Iniciativa-de-la-cc-Angelica-Tepper-Competencias-de-los-Gobiernos-Regionales.pdf}
\newline {\color{gray} (Emb: 0.712, TF-IDF: 0.334)}
\newline {\color{gray} \textbf{2º:} 353-1-Iniciativa-Convencional-Constituyente-del-cc-Jaime-Bassa-sobre-Sistema-de-Gobierno-1158-21-01.pdf}
\newline {\color{gray} (Emb: 0.641, TF-IDF: 0.333)}

Conocer y resolver sobre la constitucionalidad de los reglamentos y decretos de la o el Presidente de la República, dictados en ejercicio de la potestad reglamentaria en aquellas materias que no están comprendidas en el artículo [22]. 
\newline {\color{gray} \textbf{1º:} 325-6-Iniciativa-Convencional-del-cc-Tomas-Laibe-sobre-Jurisdiccion-Constitucional.pdf}
\newline {\color{gray} (Emb: 0.965, TF-IDF: 0.966)}
\newline {\color{gray} \textbf{2º:} 184-6-c-Iniciativa-Convencional-del-cc-Rodrigo-Álvarez-que-crea-la-Corte-Constitucional-1044hrs.pdf}
\newline {\color{gray} (Emb: 0.920, TF-IDF: 0.966)}

Resolver los conflictos de competencia o de atribuciones que se susciten entre las entidades territoriales autónomas, con cualquier otro órgano del Estado, o entre éstos, a solicitud de cualquiera de los antes mencionados. 
\newline {\color{gray} \textbf{1º:} 325-6-Iniciativa-Convencional-del-cc-Tomas-Laibe-sobre-Jurisdiccion-Constitucional.pdf}
\newline {\color{gray} (Emb: 0.953, TF-IDF: 0.834)}
\newline {\color{gray} \textbf{2º:} 120-3-c-Iniciativa-de-la-cc-Tammy-Pustilnick-atribuciones-exclusivas-de-la-Asamblea-Regional.pdf}
\newline {\color{gray} (Emb: 0.728, TF-IDF: 0.370)}

Resolver los conflictos de competencia que se susciten entre las autoridades políticas o administrativas y los tribunales de justicia. 
\newline {\color{gray} \textbf{1º:} 184-6-c-Iniciativa-Convencional-del-cc-Rodrigo-Álvarez-que-crea-la-Corte-Constitucional-1044hrs.pdf}
\newline {\color{gray} (Emb: 0.831, TF-IDF: 0.939)}
\newline {\color{gray} \textbf{2º:} 173-6-c-Iniciativa-Convencional-del-cc-Rodrigo-Álvarez-Contraloría-1044-hrs.pdf}
\newline {\color{gray} (Emb: 0.645, TF-IDF: 0.383)}

Las demás previstas en esta Constitución. 
\newline {\color{gray} \textbf{1º:} 239-1-Iniciativa-Convencional-de-la-cc-Tania-Madriaga-sobre-Poder-Ejecutivo-1146-hrs.pdf}
\newline {\color{gray} (Emb: 0.672, TF-IDF: 0.472)}
\newline {\color{gray} \textbf{2º:} 184-6-c-Iniciativa-Convencional-del-cc-Rodrigo-Álvarez-que-crea-la-Corte-Constitucional-1044hrs.pdf}
\newline {\color{gray} (Emb: 0.661, TF-IDF: 0.416)}

Conocer y resolver sobre la constitucionalidad de un decreto o resolución de la o el Presidente de la República que la Contraloría General de la República haya representado por estimarlo inconstitucional, cuando sea requerido por la o el Presidente en conformidad al artículo [47]. 
\newline {\color{gray} \textbf{1º:} 915-Iniciativa-Convencional-Constituyente-del-cc-Luis-Jimenez-sobre-Plurinacionalidad.pdf}
\newline {\color{gray} (Emb: 0.813, TF-IDF: 0.371)}
\newline {\color{gray} \textbf{2º:} 472-6-Iniciativa-Convencional-Constituyente-del-cc-Daniel-Bravo-sobre-Corte-Constitucional-2003-31-01.pdf}
\newline {\color{gray} (Emb: 0.667, TF-IDF: 0.357)}


\item \textbf{Artículo} \newline
Las sentencias de la Corte Constitucional se adoptarán, en sala o en pleno, por la mayoría de sus integrantes, sin perjuicio de las excepciones que establezca la Constitución o la ley. 
\newline {\color{gray} \textbf{1º:} 184-6-c-Iniciativa-Convencional-del-cc-Rodrigo-Álvarez-que-crea-la-Corte-Constitucional-1044hrs.pdf}
\newline {\color{gray} (Emb: 0.681, TF-IDF: 0.729)}
\newline {\color{gray} \textbf{2º:} 184-6-c-Iniciativa-Convencional-del-cc-Rodrigo-Álvarez-que-crea-la-Corte-Constitucional-1044hrs.pdf}
\newline {\color{gray} (Emb: 0.615, TF-IDF: 0.467)}

Tienen carácter vinculante, de cumplimiento obligatorio para toda institución, persona o grupo y contra ellas no cabe recurso ulterior alguno. 
\newline {\color{gray} \textbf{1º:} 184-6-c-Iniciativa-Convencional-del-cc-Rodrigo-Álvarez-que-crea-la-Corte-Constitucional-1044hrs.pdf}
\newline {\color{gray} (Emb: 0.624, TF-IDF: 0.416)}
\newline {\color{gray} \textbf{2º:} 801-Iniciativa-Convencional-Constituyente-del-cc-Christian-Viera-sobre-Consejo-de-Contiendas-de-Trabajo.pdf}
\newline {\color{gray} (Emb: 0.542, TF-IDF: 0.350)}

La Corte Constitucional sólo podrá acoger la inconstitucionalidad o la inaplicabilidad de un precepto, cuando no sea posible interpretarlo de modo de evitar efectos inconstitucionales. 
\newline {\color{gray} \textbf{1º:} 472-6-Iniciativa-Convencional-Constituyente-del-cc-Daniel-Bravo-sobre-Corte-Constitucional-2003-31-01.pdf}
\newline {\color{gray} (Emb: 0.942, TF-IDF: 0.961)}
\newline {\color{gray} \textbf{2º:} 325-6-Iniciativa-Convencional-del-cc-Tomas-Laibe-sobre-Jurisdiccion-Constitucional.pdf}
\newline {\color{gray} (Emb: 0.897, TF-IDF: 0.894)}

Declarada la inaplicabilidad de un precepto legal, éste no podrá ser aplicado en la gestión judicial en la que se originó la cuestión de constitucionalidad. 
\newline {\color{gray} \textbf{1º:} 472-6-Iniciativa-Convencional-Constituyente-del-cc-Daniel-Bravo-sobre-Corte-Constitucional-2003-31-01.pdf}
\newline {\color{gray} (Emb: 0.739, TF-IDF: 0.455)}
\newline {\color{gray} \textbf{2º:} 181-6-c-Iniciativa-Convencional-del-cc-Rodrigo-Álvarez-sobre-recurso-protección-y-acción-del-legislador-1044-hrs.pdf}
\newline {\color{gray} (Emb: 0.691, TF-IDF: 0.425)}

Cuando la Corte Constitucional declare la inconstitucionalidad de un precepto, la sentencia provocará su invalidación, excluyéndolo del ordenamiento jurídico a partir del día siguiente de su publicación en el Diario Oficial. 
\newline {\color{gray} \textbf{1º:} 70-2-Iniciativa-Convencional-Constituyente-de-la-cc-Paulina-Veloso-y-otros-2.pdf}
\newline {\color{gray} (Emb: 0.711, TF-IDF: 0.816)}
\newline {\color{gray} \textbf{2º:} 472-6-Iniciativa-Convencional-Constituyente-del-cc-Daniel-Bravo-sobre-Corte-Constitucional-2003-31-01.pdf}
\newline {\color{gray} (Emb: 0.658, TF-IDF: 0.740)}


\item \textbf{Artículo} \newline
Esta acción también procederá cuando por acto o resolución administrativa se prive o desconozca la nacionalidad chilena. 
\newline {\color{gray} \textbf{1º:} 803-Iniciativa-Convencional-Constituyente-del-cc-Christian-Viera-sobre-Accion-de-Proteccion.pdf}
\newline {\color{gray} (Emb: 0.831, TF-IDF: 0.426)}
\newline {\color{gray} \textbf{2º:} 817-Iniciativa-Convencional-Constituyente-del-cc-Mauricio-daza-sobre-Acciones-Constitucionales.pdf}
\newline {\color{gray} (Emb: 0.766, TF-IDF: 0.421)}

La interposición de la acción suspenderá los efectos del acto o resolución recurrida. 
\newline {\color{gray} \textbf{1º:} 95-6-Iniciativa-Convencional-Constituyente-de-Cc-Mauricio-Daza-y-otros-2.pdf}
\newline {\color{gray} (Emb: 0.600, TF-IDF: 0.327)}
\newline {\color{gray} \textbf{2º:} 88-6-Iniciativa-Convencional-Constituyente-del-cc-Christian-Viera-y-otros.pdf}
\newline {\color{gray} (Emb: 0.593, TF-IDF: 0.281)}

De estimarse en el examen de admisibilidad que no existe tal contradicción, se ordenará que sea remitido junto con sus antecedentes a la Corte de Apelaciones correspondiente para que, si lo estima admisible, lo conozca y resuelva. 
\newline {\color{gray} \textbf{1º:} 325-6-Iniciativa-Convencional-del-cc-Tomas-Laibe-sobre-Jurisdiccion-Constitucional.pdf}
\newline {\color{gray} (Emb: 0.744, TF-IDF: 0.689)}
\newline {\color{gray} \textbf{2º:} 97-6-Iniciativa-del-cc-Daniel-Bravo-Sistemas-de-Justicia.pdf}
\newline {\color{gray} (Emb: 0.633, TF-IDF: 0.277)}

En el caso de los derechos de los pueblos indígenas y tribales, esta acción podrá ser deducida por las instituciones representativas de los pueblos indígenas, sus miembros, o la Defensoría del Pueblo. 
\newline {\color{gray} \textbf{1º:} 603-Iniciativa-Convencional-Constituyente-de-cc-Jorge-Baradit-sobre-Nacionalidad-y-Ciudadania-2105-hrs.-01-02.pdf}
\newline {\color{gray} (Emb: 0.693, TF-IDF: 0.549)}
\newline {\color{gray} \textbf{2º:} 485-2-Iniciativa-Convencional-Constituyente-del-cc-Jorge-Baradit-sobre-Nacionalidad-2113-31-01.pdf}
\newline {\color{gray} (Emb: 0.693, TF-IDF: 0.549)}

Tratándose de los derechos de la naturaleza y derechos ambientales, podrán ejercer esta acción tanto la Defensoría de la Naturaleza como cualquier persona o grupo. 
\newline {\color{gray} \textbf{1º:} 184-6-c-Iniciativa-Convencional-del-cc-Rodrigo-Álvarez-que-crea-la-Corte-Constitucional-1044hrs.pdf}
\newline {\color{gray} (Emb: 0.632, TF-IDF: 0.302)}
\newline {\color{gray} \textbf{2º:} 325-6-Iniciativa-Convencional-del-cc-Tomas-Laibe-sobre-Jurisdiccion-Constitucional.pdf}
\newline {\color{gray} (Emb: 0.628, TF-IDF: 0.210)}

La apelación en contra de la sentencia definitiva será conocida por la Corte de Apelaciones respectiva. 
\newline {\color{gray} \textbf{1º:} 803-Iniciativa-Convencional-Constituyente-del-cc-Christian-Viera-sobre-Accion-de-Proteccion.pdf}
\newline {\color{gray} (Emb: 0.712, TF-IDF: 0.371)}
\newline {\color{gray} \textbf{2º:} 97-6-Iniciativa-del-cc-Daniel-Bravo-Sistemas-de-Justicia.pdf}
\newline {\color{gray} (Emb: 0.654, TF-IDF: 0.317)}

El recurso será conocido por la Corte Suprema si respecto a la materia de derecho objeto de la acción existen interpretaciones contradictorias sostenidas en dos o más sentencias firmes emanadas de los tribunales del Sistema Nacional de Justicia. 
\newline {\color{gray} \textbf{1º:} 325-6-Iniciativa-Convencional-del-cc-Tomas-Laibe-sobre-Jurisdiccion-Constitucional.pdf}
\newline {\color{gray} (Emb: 0.844, TF-IDF: 0.683)}
\newline {\color{gray} \textbf{2º:} 880-Iniciativa-Convencional-Constituyente-de-la-cc-Ingrid-Villena-sobre-Acciones-Constitucionales.pdf}
\newline {\color{gray} (Emb: 0.830, TF-IDF: 0.545)}

El tribunal competente podrá en cualquier momento del procedimiento, de oficio o a petición de parte, decretar cualquier medida provisional que estime necesaria, y alzarlas o dejarlas sin efecto cuando lo estime conveniente. 
\newline {\color{gray} \textbf{1º:} 940-Iniciativa-Convencional-Constituyente-del-cc-Christian-Viera-sobre-Proteccion-de-Derechos-Fundamentales.pdf}
\newline {\color{gray} (Emb: 0.735, TF-IDF: 0.254)}
\newline {\color{gray} \textbf{2º:} 817-Iniciativa-Convencional-Constituyente-del-cc-Mauricio-daza-sobre-Acciones-Constitucionales.pdf}
\newline {\color{gray} (Emb: 0.735, TF-IDF: 0.250)}

Al acoger o rechazar la acción, se deberá señalar el procedimiento judicial que en derecho corresponda y que permita la resolución del asunto. 
\newline {\color{gray} \textbf{1º:} 367-2-Iniciativa-Convencional-Constituyente-del-cc-Eduardo-Castillo-sobre-Ciudadania-y-Migracion-0900-hrs-24-01.pdf}
\newline {\color{gray} (Emb: 0.688, TF-IDF: 0.364)}
\newline {\color{gray} \textbf{2º:} 443-Iniciativa-Convencional-Constituyente-de-la-cc-Giovanna-Grandon-sobre-Nacionalidad-1405-28-01.pdf}
\newline {\color{gray} (Emb: 0.494, TF-IDF: 0.330)}

Esta acción cautelar será procedente cuando la persona afectada no disponga de otra acción, recurso o medio procesal para reclamar de su derecho, salvo aquellos casos en que, por su urgencia y gravedad, pueda provocarle un daño grave inminente o irreparable. 
\newline {\color{gray} \textbf{1º:} 569-Iniciativa-Convencional-Constituyente-de-cc-Roberto-Celedon-sobre-Derecho-al-trabajo-2024-hrs.-01-02.pdf}
\newline {\color{gray} (Emb: 0.821, TF-IDF: 0.659)}
\newline {\color{gray} \textbf{2º:} 940-Iniciativa-Convencional-Constituyente-del-cc-Christian-Viera-sobre-Proteccion-de-Derechos-Fundamentales.pdf}
\newline {\color{gray} (Emb: 0.803, TF-IDF: 0.659)}

La acción se tramitará sumariamente y con preferencia a toda otra causa que conozca el tribunal. 
\newline {\color{gray} \textbf{1º:} 472-6-Iniciativa-Convencional-Constituyente-del-cc-Daniel-Bravo-sobre-Corte-Constitucional-2003-31-01.pdf}
\newline {\color{gray} (Emb: 0.841, TF-IDF: 0.700)}
\newline {\color{gray} \textbf{2º:} 89-6-Iniciativa-Convencional-Constituyente-del-cc-Christian-Viera-y-otros.pdf}
\newline {\color{gray} (Emb: 0.751, TF-IDF: 0.319)}

Esta acción se podrá deducir mientras la vulneración persista. 
\newline {\color{gray} \textbf{1º:} 89-6-Iniciativa-Convencional-Constituyente-del-cc-Christian-Viera-y-otros.pdf}
\newline {\color{gray} (Emb: 0.996, TF-IDF: 0.959)}
\newline {\color{gray} \textbf{2º:} 472-6-Iniciativa-Convencional-Constituyente-del-cc-Daniel-Bravo-sobre-Corte-Constitucional-2003-31-01.pdf}
\newline {\color{gray} (Emb: 0.825, TF-IDF: 0.830)}

Toda persona que por causa de un acto u omisión sufra una amenaza, perturbación o privación en el legítimo ejercicio de sus derechos fundamentales, podrá concurrir por sí o por cualquiera a su nombre ante el tribunal de instancia que determine la ley, el que adoptará de inmediato todas las providencias que juzgue necesarias para restablecer el imperio del derecho. 
\newline {\color{gray} \textbf{1º:} 472-6-Iniciativa-Convencional-Constituyente-del-cc-Daniel-Bravo-sobre-Corte-Constitucional-2003-31-01.pdf}
\newline {\color{gray} (Emb: 0.957, TF-IDF: 0.864)}
\newline {\color{gray} \textbf{2º:} 472-6-Iniciativa-Convencional-Constituyente-del-cc-Daniel-Bravo-sobre-Corte-Constitucional-2003-31-01.pdf}
\newline {\color{gray} (Emb: 0.789, TF-IDF: 0.420)}

No podrá deducirse esta acción contra resoluciones judiciales, salvo respecto de aquellas personas que no hayan intervenido en el proceso respectivo y a quienes afecten sus resultados. 
\newline {\color{gray} \textbf{1º:} 880-Iniciativa-Convencional-Constituyente-de-la-cc-Ingrid-Villena-sobre-Acciones-Constitucionales.pdf}
\newline {\color{gray} (Emb: 0.864, TF-IDF: 0.646)}
\newline {\color{gray} \textbf{2º:} 940-Iniciativa-Convencional-Constituyente-del-cc-Christian-Viera-sobre-Proteccion-de-Derechos-Fundamentales.pdf}
\newline {\color{gray} (Emb: 0.635, TF-IDF: 0.361)}


\item \textbf{Artículo} \newline
Toda persona que sea arrestada, detenida o presa con infracción a lo dispuesto en esta Constitución o las leyes, podrá concurrir por sí o por cualquiera persona a su nombre, sin formalidades, ante la magistratura que señale la ley, a fin de que esta adopte de inmediato las providencias que sean necesarias para restablecer el imperio del derecho y asegurar la debida protección de la persona afectada, pudiendo inclusive decretar su libertad inmediata. 
\newline {\color{gray} \textbf{1º:} 507-2-Iniciativa-Convencional-Constituyente-de-la-cc-Giovanna-Grandon-sobre-Nacionalidad-1213-01-02.pdf}
\newline {\color{gray} (Emb: 0.951, TF-IDF: 0.918)}
\newline {\color{gray} \textbf{2º:} 443-Iniciativa-Convencional-Constituyente-de-la-cc-Giovanna-Grandon-sobre-Nacionalidad-1405-28-01.pdf}
\newline {\color{gray} (Emb: 0.951, TF-IDF: 0.861)}

Esa magistratura podrá ordenar que el individuo sea traído a su presencia y su decreto será precisamente obedecido por todos los encargados de las cárceles o lugares de detención. 
\newline {\color{gray} \textbf{1º:} 160-4-c-Iniciativa-de-la-cc-Valentina-Miranda-sobre-contenido-de-los-Derechos-Fundamentales.pdf}
\newline {\color{gray} (Emb: 0.758, TF-IDF: 0.425)}
\newline {\color{gray} \textbf{2º:} 914-Iniciativa-Convencional-Constituyente-del-cc-Luis-Jimenez-crea-la-Defensoria-de-la-Naturaleza.pdf}
\newline {\color{gray} (Emb: 0.712, TF-IDF: 0.380)}

Instruida de los antecedentes, decretará su libertad inmediata o hará que se reparen los defectos legales o pondrá al individuo a disposición del tribunal competente, procediendo en todo breve y sumariamente, y corrigiendo por sí esos defectos o dando cuenta a quien corresponda para que los corrija. 
\newline {\color{gray} \textbf{1º:} 267-4-Iniciativa-Convencional-de-la-cc-Janis-Meneses-sobre-Derechos-Politicos-1154-hrs.pdf}
\newline {\color{gray} (Emb: 0.686, TF-IDF: 0.563)}
\newline {\color{gray} \textbf{2º:} 22-6-Iniciativa-Convencional-Constituyente-de-la-cc-Isabella-Mamani-y-otros.pdf}
\newline {\color{gray} (Emb: 0.638, TF-IDF: 0.563)}

Sin perjuicio de lo señalado, el tribunal deberá agotar todas las medidas conducentes a determinar la existencia y condiciones de la persona que se encontrare privada de libertad. 
\newline {\color{gray} \textbf{1º:} 731-Iniciativa-Convencional-Constituyente-del-cc-Alfredo-Moreno-sobre-Garantia-de-los-DDFF.pdf}
\newline {\color{gray} (Emb: 0.952, TF-IDF: 0.815)}
\newline {\color{gray} \textbf{2º:} 160-4-c-Iniciativa-de-la-cc-Valentina-Miranda-sobre-contenido-de-los-Derechos-Fundamentales.pdf}
\newline {\color{gray} (Emb: 0.938, TF-IDF: 0.815)}

Esta acción también procederá respecto de toda persona que ilegalmente sufra una privación, perturbación o amenaza a su derecho a la libertad personal, ambulatoria o seguridad individual, debiendo en tal caso adoptarse todas las medidas que sean conducentes para restablecer el imperio del derecho y asegurar la debida protección del afectado. 
\newline {\color{gray} \textbf{1º:} 160-4-c-Iniciativa-de-la-cc-Valentina-Miranda-sobre-contenido-de-los-Derechos-Fundamentales.pdf}
\newline {\color{gray} (Emb: 1.000, TF-IDF: 1.000)}
\newline {\color{gray} \textbf{2º:} 731-Iniciativa-Convencional-Constituyente-del-cc-Alfredo-Moreno-sobre-Garantia-de-los-DDFF.pdf}
\newline {\color{gray} (Emb: 0.983, TF-IDF: 0.906)}


\item \textbf{Artículo} \newline
La compensación no procederá cuando la privación de libertad se haya decretado por una causal fundada en una conducta efectiva del imputado. 
\newline {\color{gray} \textbf{1º:} 900-Iniciativa-Convencional-Constituyente-del-cc-Christian-Viera-sobre-Christian-Viera.pdf}
\newline {\color{gray} (Emb: 0.787, TF-IDF: 0.486)}
\newline {\color{gray} \textbf{2º:} 880-Iniciativa-Convencional-Constituyente-de-la-cc-Ingrid-Villena-sobre-Acciones-Constitucionales.pdf}
\newline {\color{gray} (Emb: 0.780, TF-IDF: 0.486)}

El monto diario de compensación será fijado por la ley y su pago se realizará mediante un procedimiento simple y expedito. 
\newline {\color{gray} \textbf{1º:} 440-Iniciativa-Convencional-Constituyente-del-cc-Felipe-Harboe-sobre-Principios-del-Debido-Proceso-1401-28-01.pdf}
\newline {\color{gray} (Emb: 0.720, TF-IDF: 0.303)}
\newline {\color{gray} \textbf{2º:} 514-4-Iniciativa-Convencional-Constituyente-del-cc-Felipe-Harboe-sobre-Derecho-a-la-Privacidad-1245-01-02.pdf}
\newline {\color{gray} (Emb: 0.720, TF-IDF: 0.282)}

Toda persona que sea absuelta, sobreseída definitivamente o que no resulte condenada, será compensada por cada día que haya permanecido privada de libertad. 
\newline {\color{gray} \textbf{1º:} 731-Iniciativa-Convencional-Constituyente-del-cc-Alfredo-Moreno-sobre-Garantia-de-los-DDFF.pdf}
\newline {\color{gray} (Emb: 0.997, TF-IDF: 0.984)}
\newline {\color{gray} \textbf{2º:} 880-Iniciativa-Convencional-Constituyente-de-la-cc-Ingrid-Villena-sobre-Acciones-Constitucionales.pdf}
\newline {\color{gray} (Emb: 0.997, TF-IDF: 0.984)}


\item \textbf{Artículo} \newline
Toda persona que haya sido condenada por sentencia dictada con error injustificado o falta de servicio judicial, tendrá derecho a ser indemnizada de todos los perjuicios que el proceso y la decisión condenatoria le hubieren causado. 
\newline {\color{gray} \textbf{1º:} 802-Iniciativa-Convencional-Constituyente-del-cc-Christian-Viera-sobre-Falta-de-Servicio-Judicial.pdf}
\newline {\color{gray} (Emb: 0.938, TF-IDF: 0.889)}
\newline {\color{gray} \textbf{2º:} 533-Iniciativa-Convencional-Constituyente-del-cc-Felipe-Harboe-sobre-D°-a-la-Libertad-1625-hrs.-01-02-1.pdf}
\newline {\color{gray} (Emb: 0.687, TF-IDF: 0.499)}

Si todo o parte del daño deriva de la privación de libertad, la compensación, que siempre se podrá exigir en conformidad al artículo anterior, será imputada a la presente indemnización. 
\newline {\color{gray} \textbf{1º:} 802-Iniciativa-Convencional-Constituyente-del-cc-Christian-Viera-sobre-Falta-de-Servicio-Judicial.pdf}
\newline {\color{gray} (Emb: 0.859, TF-IDF: 0.656)}
\newline {\color{gray} \textbf{2º:} 131-4-c-Iniciativa-de-la-cc-Rocio-Cantuarias-Establece-la-Libertad-Personal-y-la-Seguridad-Individual.pdf}
\newline {\color{gray} (Emb: 0.665, TF-IDF: 0.305)}

La misma indemnización procederá por las actuaciones o decisiones administrativas derivadas del funcionamiento judicial que, con falta de servicio, generen daño. 
\newline {\color{gray} \textbf{1º:} 802-Iniciativa-Convencional-Constituyente-del-cc-Christian-Viera-sobre-Falta-de-Servicio-Judicial.pdf}
\newline {\color{gray} (Emb: 0.778, TF-IDF: 0.319)}
\newline {\color{gray} \textbf{2º:} 226-6-Iniciativa-Convencional-de-la-cc-Manuela-Royo-sobre-Justicia-Local-1140-hrs.pdf}
\newline {\color{gray} (Emb: 0.617, TF-IDF: 0.237)}


\item \textbf{Artículo} \newline
Los proyectos de reforma a la Constitución podrán ser iniciados por mensaje presidencial, moción de diputadas y diputados o representantes regionales, o por iniciativa popular. 
\newline {\color{gray} \textbf{1º:} 802-Iniciativa-Convencional-Constituyente-del-cc-Christian-Viera-sobre-Falta-de-Servicio-Judicial.pdf}
\newline {\color{gray} (Emb: 0.937, TF-IDF: 0.667)}
\newline {\color{gray} \textbf{2º:} 440-Iniciativa-Convencional-Constituyente-del-cc-Felipe-Harboe-sobre-Principios-del-Debido-Proceso-1401-28-01.pdf}
\newline {\color{gray} (Emb: 0.887, TF-IDF: 0.292)}

Los proyectos de reforma constitucional iniciados por las y los ciudadanos deberán contar con el patrocinio en los términos señalados en esta Constitución. 
\newline {\color{gray} \textbf{1º:} 802-Iniciativa-Convencional-Constituyente-del-cc-Christian-Viera-sobre-Falta-de-Servicio-Judicial.pdf}
\newline {\color{gray} (Emb: 0.879, TF-IDF: 0.919)}
\newline {\color{gray} \textbf{2º:} 802-Iniciativa-Convencional-Constituyente-del-cc-Christian-Viera-sobre-Falta-de-Servicio-Judicial.pdf}
\newline {\color{gray} (Emb: 0.542, TF-IDF: 0.244)}

Todo proyecto de reforma constitucional deberá señalar expresamente de qué forma se agrega, modifica, reemplaza o deroga una norma de la Constitución. 
\newline {\color{gray} \textbf{1º:} 802-Iniciativa-Convencional-Constituyente-del-cc-Christian-Viera-sobre-Falta-de-Servicio-Judicial.pdf}
\newline {\color{gray} (Emb: 1.000, TF-IDF: 1.000)}
\newline {\color{gray} \textbf{2º:} 304-4-Iniciativa-Convencional-de-la-cc-Valentina-Miranda-sobre-Derechos-Fundamentales-1301-hrs.pdf}
\newline {\color{gray} (Emb: 0.669, TF-IDF: 0.240)}

En lo no previsto en este Título, serán aplicables a la tramitación de los proyectos de reforma constitucional, las disposiciones que regulan el procedimiento de formación de la ley, debiendo respetarse siempre el quórum señalado en los incisos anteriores. 
\newline {\color{gray} \textbf{1º:} 694-Iniciativa-Convencional-Constituyente-del-cc-Rodrigo-Alvarez-sobre-Reforma-a-la-Constitucion-121101-02.pdf}
\newline {\color{gray} (Emb: 0.871, TF-IDF: 0.778)}
\newline {\color{gray} \textbf{2º:} 425-6-Iniciativa-Convencional-del-cc-Christian-Viera-sobre-Reforma-de-la-Constitucion-1554-26-01.pdf}
\newline {\color{gray} (Emb: 0.863, TF-IDF: 0.696)}


\item \textbf{Artículo} \newline
La reforma constitucional aprobada por el Congreso de Diputadas y Diputados y la Cámara de las Regiones se entenderá ratificada si alcanza la mayoría de los votos válidamente emitidos en el referéndum. 
\newline {\color{gray} \textbf{1º:} 544-Iniciativa-Convencional-Constituyente-del-cc-Andres-Cruz-sobre-reforma-constitucional-1623-01-02.pdf}
\newline {\color{gray} (Emb: 0.678, TF-IDF: 0.487)}
\newline {\color{gray} \textbf{2º:} 467-6-Iniciativa-Convencional-Constituyente-del-cc-Daniel-Bravo-sobre-Reforma-y-Reemplazo-1945-31-01.pdf}
\newline {\color{gray} (Emb: 0.640, TF-IDF: 0.397)}

El Congreso de Diputadas y Diputados deberá convocar a referéndum ratificatorio tratándose de proyectos de reforma constitucional aprobados por este y la Cámara de las Regiones, que alteren sustancialmente el régimen político y el periodo presidencial; el diseño del Congreso de Diputadas y Diputados o la Cámara de las Regiones y la duración de sus integrantes; la forma de Estado Regional; los principios y los derechos fundamentales; y el capítulo de reforma y reemplazo de la Constitución. 
\newline {\color{gray} \textbf{1º:} 425-6-Iniciativa-Convencional-del-cc-Christian-Viera-sobre-Reforma-de-la-Constitucion-1554-26-01.pdf}
\newline {\color{gray} (Emb: 0.798, TF-IDF: 0.637)}
\newline {\color{gray} \textbf{2º:} 425-6-Iniciativa-Convencional-del-cc-Christian-Viera-sobre-Reforma-de-la-Constitucion-1554-26-01.pdf}
\newline {\color{gray} (Emb: 0.775, TF-IDF: 0.382)}

Si el proyecto de reforma constitucional es aprobado por dos tercios de las y los integrantes del Congreso de Diputadas y Diputados y de la Cámara de las Regiones, no será sometido a referéndum ratificatorio. 
\newline {\color{gray} \textbf{1º:} 467-6-Iniciativa-Convencional-Constituyente-del-cc-Daniel-Bravo-sobre-Reforma-y-Reemplazo-1945-31-01.pdf}
\newline {\color{gray} (Emb: 0.999, TF-IDF: 1.000)}
\newline {\color{gray} \textbf{2º:} 425-6-Iniciativa-Convencional-del-cc-Christian-Viera-sobre-Reforma-de-la-Constitucion-1554-26-01.pdf}
\newline {\color{gray} (Emb: 0.914, TF-IDF: 0.335)}

El referéndum se realizará en la forma que establezca la Constitución y la ley. 
\newline {\color{gray} \textbf{1º:} 425-6-Iniciativa-Convencional-del-cc-Christian-Viera-sobre-Reforma-de-la-Constitucion-1554-26-01.pdf}
\newline {\color{gray} (Emb: 0.935, TF-IDF: 0.726)}
\newline {\color{gray} \textbf{2º:} 544-Iniciativa-Convencional-Constituyente-del-cc-Andres-Cruz-sobre-reforma-constitucional-1623-01-02.pdf}
\newline {\color{gray} (Emb: 0.850, TF-IDF: 0.567)}

Aprobado que sea el proyecto de reforma constitucional por el Congreso de Diputadas y Diputados y la Cámara de las Regiones, el Congreso lo enviará a la o el Presidente de la República quien, dentro del plazo de treinta días corridos, deberá someterlo a referéndum ratificatorio. 
\newline {\color{gray} \textbf{1º:} 425-6-Iniciativa-Convencional-del-cc-Christian-Viera-sobre-Reforma-de-la-Constitucion-1554-26-01.pdf}
\newline {\color{gray} (Emb: 0.814, TF-IDF: 0.579)}
\newline {\color{gray} \textbf{2º:} 602-Iniciativa-Convencional-Constituyente-de-cc-Jorge-Baradit-sobre-Mecanismos-de-Demoracia-Directa-y-Semidirecta.pdf}
\newline {\color{gray} (Emb: 0.710, TF-IDF: 0.340)}

Es deber del Estado dar adecuada publicidad a la propuesta de reforma que se someterá a referéndum, de acuerdo a la Constitución y la ley. 
\newline {\color{gray} \textbf{1º:} 957-5-Iniciativa-Convencional-Constituyente-de-la-cc-Ivanna-Olivares-sobre-Nuevo-Modelo-Economico.pdf}
\newline {\color{gray} (Emb: 0.735, TF-IDF: 0.424)}
\newline {\color{gray} \textbf{2º:} 399-2-Iniciativa-Convencional-Constituyente-de-la-cc-Constanza-San-Juan-sobre-Democracia-Directa-1850-24-01.pdf}
\newline {\color{gray} (Emb: 0.734, TF-IDF: 0.380)}


\item \textbf{Artículo} \newline
(Inciso cuarto) La propuesta de reforma constitucional se entenderá aprobada si la alternativa de modificación ganadora alcanza la mayoría en la votación respectiva. 
\newline {\color{gray} \textbf{1º:} 467-6-Iniciativa-Convencional-Constituyente-del-cc-Daniel-Bravo-sobre-Reforma-y-Reemplazo-1945-31-01.pdf}
\newline {\color{gray} (Emb: 0.709, TF-IDF: 0.544)}
\newline {\color{gray} \textbf{2º:} 467-6-Iniciativa-Convencional-Constituyente-del-cc-Daniel-Bravo-sobre-Reforma-y-Reemplazo-1945-31-01.pdf}
\newline {\color{gray} (Emb: 0.678, TF-IDF: 0.393)}

(Inciso quinto) Es deber del Congreso y de los órganos del Estado descentralizados dar adecuada publicidad a la o las propuestas de reforma que se someterán a referéndum. 
\newline {\color{gray} \textbf{1º:} 425-6-Iniciativa-Convencional-del-cc-Christian-Viera-sobre-Reforma-de-la-Constitucion-1554-26-01.pdf}
\newline {\color{gray} (Emb: 0.941, TF-IDF: 0.720)}
\newline {\color{gray} \textbf{2º:} 602-Iniciativa-Convencional-Constituyente-de-cc-Jorge-Baradit-sobre-Mecanismos-de-Demoracia-Directa-y-Semidirecta.pdf}
\newline {\color{gray} (Emb: 0.830, TF-IDF: 0.518)}

(Inciso segundo) Existirá un plazo de ciento ochenta días desde su registro para que la propuesta sea conocida por la ciudadanía y pueda reunir los patrocinios exigidos. 
\newline {\color{gray} \textbf{1º:} 425-6-Iniciativa-Convencional-del-cc-Christian-Viera-sobre-Reforma-de-la-Constitucion-1554-26-01.pdf}
\newline {\color{gray} (Emb: 0.851, TF-IDF: 0.842)}
\newline {\color{gray} \textbf{2º:} 467-6-Iniciativa-Convencional-Constituyente-del-cc-Daniel-Bravo-sobre-Reforma-y-Reemplazo-1945-31-01.pdf}
\newline {\color{gray} (Emb: 0.791, TF-IDF: 0.666)}

Un mínimo equivalente al diez por ciento de la ciudadanía correspondiente al último padrón electoral, podrá presentar una propuesta de reforma constitucional para ser votada mediante referéndum nacional conjuntamente con la próxima elección parlamentaria. 
\newline {\color{gray} \textbf{1º:} 544-Iniciativa-Convencional-Constituyente-del-cc-Andres-Cruz-sobre-reforma-constitucional-1623-01-02.pdf}
\newline {\color{gray} (Emb: 0.840, TF-IDF: 0.738)}
\newline {\color{gray} \textbf{2º:} 602-Iniciativa-Convencional-Constituyente-de-cc-Jorge-Baradit-sobre-Mecanismos-de-Demoracia-Directa-y-Semidirecta.pdf}
\newline {\color{gray} (Emb: 0.788, TF-IDF: 0.466)}


\item \textbf{Artículo} \newline
El reemplazo total de la Constitución sólo podrá realizarse a través de una Asamblea Constituyente convocada por medio de un referéndum. 
\newline {\color{gray} \textbf{1º:} 602-Iniciativa-Convencional-Constituyente-de-cc-Jorge-Baradit-sobre-Mecanismos-de-Demoracia-Directa-y-Semidirecta.pdf}
\newline {\color{gray} (Emb: 0.868, TF-IDF: 0.840)}
\newline {\color{gray} \textbf{2º:} 602-Iniciativa-Convencional-Constituyente-de-cc-Jorge-Baradit-sobre-Mecanismos-de-Demoracia-Directa-y-Semidirecta.pdf}
\newline {\color{gray} (Emb: 0.847, TF-IDF: 0.779)}

La convocatoria a referéndum constituyente podrá ser convocada por iniciativa popular. 
\newline {\color{gray} \textbf{1º:} 425-6-Iniciativa-Convencional-del-cc-Christian-Viera-sobre-Reforma-de-la-Constitucion-1554-26-01.pdf}
\newline {\color{gray} (Emb: 0.828, TF-IDF: 0.763)}
\newline {\color{gray} \textbf{2º:} 425-6-Iniciativa-Convencional-del-cc-Christian-Viera-sobre-Reforma-de-la-Constitucion-1554-26-01.pdf}
\newline {\color{gray} (Emb: 0.800, TF-IDF: 0.647)}

Un grupo de ciudadanas y ciudadanos con derecho a sufragio deberá patrocinar la convocatoria con, a lo menos, firmas correspondientes al veinticinco por ciento del padrón electoral que hubiere sido establecido para la última elección parlamentaria. 
\newline {\color{gray} \textbf{1º:} 425-6-Iniciativa-Convencional-del-cc-Christian-Viera-sobre-Reforma-de-la-Constitucion-1554-26-01.pdf}
\newline {\color{gray} (Emb: 0.794, TF-IDF: 0.704)}
\newline {\color{gray} \textbf{2º:} 425-6-Iniciativa-Convencional-del-cc-Christian-Viera-sobre-Reforma-de-la-Constitucion-1554-26-01.pdf}
\newline {\color{gray} (Emb: 0.772, TF-IDF: 0.443)}

También corresponderá a la o el Presidente de la República, por medio de un decreto, convocar al referéndum, el que deberá contar con la aprobación de los tres quintos de las y los integrantes del Congreso de Diputadas y Diputados y de la Cámara de las Regiones, que deberán sesionar de manera conjunta en pleno para estos efectos. 
\newline {\color{gray} \textbf{1º:} 425-6-Iniciativa-Convencional-del-cc-Christian-Viera-sobre-Reforma-de-la-Constitucion-1554-26-01.pdf}
\newline {\color{gray} (Emb: 0.996, TF-IDF: 0.845)}
\newline {\color{gray} \textbf{2º:} 544-Iniciativa-Convencional-Constituyente-del-cc-Andres-Cruz-sobre-reforma-constitucional-1623-01-02.pdf}
\newline {\color{gray} (Emb: 0.977, TF-IDF: 0.817)}

Asimismo, la convocatoria corresponderá al Congreso de Diputadas y Diputados y la Cámara de las Regiones, quienes deberán sesionar de manera conjunta en pleno para estos efectos, por medio de una ley aprobada por los dos tercios de sus integrantes. 
\newline {\color{gray} \textbf{1º:} 425-6-Iniciativa-Convencional-del-cc-Christian-Viera-sobre-Reforma-de-la-Constitucion-1554-26-01.pdf}
\newline {\color{gray} (Emb: 0.894, TF-IDF: 0.784)}
\newline {\color{gray} \textbf{2º:} 240-1-Iniciativa-Convencional-de-la-cc-Tania-Madriaga-sobre-Poder-Legislativo-1146-hrs.pdf}
\newline {\color{gray} (Emb: 0.657, TF-IDF: 0.507)}

La convocatoria para la instalación de la Asamblea Constituyente será aprobada si en el referendo es votada favorablemente por la mayoría de los votos válidamente emitidos. 
\newline {\color{gray} \textbf{1º:} 425-6-Iniciativa-Convencional-del-cc-Christian-Viera-sobre-Reforma-de-la-Constitucion-1554-26-01.pdf}
\newline {\color{gray} (Emb: 0.778, TF-IDF: 0.777)}
\newline {\color{gray} \textbf{2º:} 453-2-Iniciativa-Convencional-Constituyente-del-cc-Jorge-Baradit-sobre-Referendum-Revocatorio-1525-31-01.pdf}
\newline {\color{gray} (Emb: 0.734, TF-IDF: 0.652)}

El sufragio en este referendo será obligatorio para quienes tengan domicilio electoral en Chile. 
\newline {\color{gray} \textbf{1º:} 711-Iniciativa-Convencional-Constituyente-de-la-cc-Ingrid-Villena-sobre-Tricel.pdf}
\newline {\color{gray} (Emb: 0.713, TF-IDF: 0.403)}
\newline {\color{gray} \textbf{2º:} 425-6-Iniciativa-Convencional-del-cc-Christian-Viera-sobre-Reforma-de-la-Constitucion-1554-26-01.pdf}
\newline {\color{gray} (Emb: 0.671, TF-IDF: 0.403)}


\item \textbf{Artículo} \newline
La Asamblea Constituyente tendrá como única función la redacción de una propuesta de Nueva Constitución. 
\newline {\color{gray} \textbf{1º:} 325-6-Iniciativa-Convencional-del-cc-Tomas-Laibe-sobre-Jurisdiccion-Constitucional.pdf}
\newline {\color{gray} (Emb: 0.599, TF-IDF: 0.356)}
\newline {\color{gray} \textbf{2º:} 240-1-Iniciativa-Convencional-de-la-cc-Tania-Madriaga-sobre-Poder-Legislativo-1146-hrs.pdf}
\newline {\color{gray} (Emb: 0.582, TF-IDF: 0.351)}

Estará integrada paritariamente y con equidad territorial, con participación en igualdad de condiciones entre independientes e integrantes de partidos políticos, y con escaños reservados para pueblos originarios. 
\newline {\color{gray} \textbf{1º:} 425-6-Iniciativa-Convencional-del-cc-Christian-Viera-sobre-Reforma-de-la-Constitucion-1554-26-01.pdf}
\newline {\color{gray} (Emb: 0.967, TF-IDF: 0.788)}
\newline {\color{gray} \textbf{2º:} 544-Iniciativa-Convencional-Constituyente-del-cc-Andres-Cruz-sobre-reforma-constitucional-1623-01-02.pdf}
\newline {\color{gray} (Emb: 0.851, TF-IDF: 0.518)}

Una ley regulará su integración, el sistema de elección, su duración, que no será inferior a dieciocho meses, su organización mínima, los mecanismos de participación popular y consulta indígena del proceso y demás aspectos generales que permitan su instalación y funcionamiento regular. 
\newline {\color{gray} \textbf{1º:} 425-6-Iniciativa-Convencional-del-cc-Christian-Viera-sobre-Reforma-de-la-Constitucion-1554-26-01.pdf}
\newline {\color{gray} (Emb: 1.000, TF-IDF: 1.000)}
\newline {\color{gray} \textbf{2º:} 425-6-Iniciativa-Convencional-del-cc-Christian-Viera-sobre-Reforma-de-la-Constitucion-1554-26-01.pdf}
\newline {\color{gray} (Emb: 1.000, TF-IDF: 1.000)}

Una vez redactada y entregada la propuesta de nueva constitución a la autoridad competente, la Asamblea Constituyente se disolverá de pleno derecho. 
\newline {\color{gray} \textbf{1º:} 544-Iniciativa-Convencional-Constituyente-del-cc-Andres-Cruz-sobre-reforma-constitucional-1623-01-02.pdf}
\newline {\color{gray} (Emb: 1.000, TF-IDF: 1.000)}
\newline {\color{gray} \textbf{2º:} 425-6-Iniciativa-Convencional-del-cc-Christian-Viera-sobre-Reforma-de-la-Constitucion-1554-26-01.pdf}
\newline {\color{gray} (Emb: 1.000, TF-IDF: 1.000)}


\item \textbf{Artículo} \newline
Si la propuesta de Nueva Constitución fuera aprobada en el plebiscito, se procederá a su promulgación y correspondiente publicación. 
\newline {\color{gray} \textbf{1º:} 467-6-Iniciativa-Convencional-Constituyente-del-cc-Daniel-Bravo-sobre-Reforma-y-Reemplazo-1945-31-01.pdf}
\newline {\color{gray} (Emb: 0.806, TF-IDF: 0.393)}
\newline {\color{gray} \textbf{2º:} 425-6-Iniciativa-Convencional-del-cc-Christian-Viera-sobre-Reforma-de-la-Constitucion-1554-26-01.pdf}
\newline {\color{gray} (Emb: 0.706, TF-IDF: 0.318)}

Para que la propuesta sea aprobada deberá obtener el voto favorable de más de la mitad de los sufragios válidamente emitidos. 
\newline {\color{gray} \textbf{1º:} 425-6-Iniciativa-Convencional-del-cc-Christian-Viera-sobre-Reforma-de-la-Constitucion-1554-26-01.pdf}
\newline {\color{gray} (Emb: 0.849, TF-IDF: 0.718)}
\newline {\color{gray} \textbf{2º:} 544-Iniciativa-Convencional-Constituyente-del-cc-Andres-Cruz-sobre-reforma-constitucional-1623-01-02.pdf}
\newline {\color{gray} (Emb: 0.849, TF-IDF: 0.718)}

Entregada la propuesta de Nueva Constitución, deberá convocarse a un referéndum para su aprobación o rechazo. 
\newline {\color{gray} \textbf{1º:} 467-6-Iniciativa-Convencional-Constituyente-del-cc-Daniel-Bravo-sobre-Reforma-y-Reemplazo-1945-31-01.pdf}
\newline {\color{gray} (Emb: 0.603, TF-IDF: 0.688)}
\newline {\color{gray} \textbf{2º:} 267-4-Iniciativa-Convencional-de-la-cc-Janis-Meneses-sobre-Derechos-Politicos-1154-hrs.pdf}
\newline {\color{gray} (Emb: 0.601, TF-IDF: 0.304)}

El sufragio será obligatorio para quienes tengan domicilio electoral dentro de Chile y voluntario para los votantes radicados fuera del país. 
\newline {\color{gray} \textbf{1º:} 467-6-Iniciativa-Convencional-Constituyente-del-cc-Daniel-Bravo-sobre-Reforma-y-Reemplazo-1945-31-01.pdf}
\newline {\color{gray} (Emb: 0.878, TF-IDF: 0.832)}
\newline {\color{gray} \textbf{2º:} 889-Iniciativa-Convencional-Constituyente-de-la-cc-Natividad-Llanquileo-crea-el-Consejo-de-Pueblos-Indigenas.pdf}
\newline {\color{gray} (Emb: 0.601, TF-IDF: 0.359)}


\item \textbf{Artículo} \newline
Toda persona sometida a cualquier forma de privación de libertad no podrá sufrir limitaciones a otros derechos que aquellos estrictamente necesarios para la ejecución de la pena. 
\newline {\color{gray} \textbf{1º:} 425-6-Iniciativa-Convencional-del-cc-Christian-Viera-sobre-Reforma-de-la-Constitucion-1554-26-01.pdf}
\newline {\color{gray} (Emb: 0.809, TF-IDF: 0.622)}
\newline {\color{gray} \textbf{2º:} 425-6-Iniciativa-Convencional-del-cc-Christian-Viera-sobre-Reforma-de-la-Constitucion-1554-26-01.pdf}
\newline {\color{gray} (Emb: 0.809, TF-IDF: 0.622)}

El Estado deberá asegurar un trato digno y con pleno respeto a los derechos humanos de las personas privadas de libertad y de sus visitas. 
\newline {\color{gray} \textbf{1º:} 467-6-Iniciativa-Convencional-Constituyente-del-cc-Daniel-Bravo-sobre-Reforma-y-Reemplazo-1945-31-01.pdf}
\newline {\color{gray} (Emb: 0.793, TF-IDF: 0.663)}
\newline {\color{gray} \textbf{2º:} 467-6-Iniciativa-Convencional-Constituyente-del-cc-Daniel-Bravo-sobre-Reforma-y-Reemplazo-1945-31-01.pdf}
\newline {\color{gray} (Emb: 0.669, TF-IDF: 0.641)}

Las mujeres y personas gestantes embarazadas tendrán derecho, antes, durante y después del parto, a acceder a los servicios de salud que requieran, a la lactancia y al vínculo directo y permanente con su hijo o hija, teniendo en consideración el interés superior del niño, niña o adolescente. 
\newline {\color{gray} \textbf{1º:} 467-6-Iniciativa-Convencional-Constituyente-del-cc-Daniel-Bravo-sobre-Reforma-y-Reemplazo-1945-31-01.pdf}
\newline {\color{gray} (Emb: 0.999, TF-IDF: 0.888)}
\newline {\color{gray} \textbf{2º:} 544-Iniciativa-Convencional-Constituyente-del-cc-Andres-Cruz-sobre-reforma-constitucional-1623-01-02.pdf}
\newline {\color{gray} (Emb: 0.800, TF-IDF: 0.322)}


\item \textbf{Artículo} \newline
Ninguna persona privada de libertad podrá ser sometida a tortura ni a otros tratos crueles, inhumanos o degradantes, ni a trabajos forzosos. 
\newline {\color{gray} \textbf{1º:} 1031-Iniciativa-Convencional-Constituyente-del-cc-Tomas-Laibe-sobre-Personas-Privadas-de-Libertad.pdf}
\newline {\color{gray} (Emb: 0.689, TF-IDF: 0.391)}
\newline {\color{gray} \textbf{2º:} 1027-Iniciativa-Convencional-Consituyente-del-cc-Felipe-Harboe-sobre-Derecho-a-la-Vida.pdf}
\newline {\color{gray} (Emb: 0.651, TF-IDF: 0.338)}

Asimismo, no podrá ser sometida a aislamiento o incomunicación como sanción disciplinaria. 
\newline {\color{gray} \textbf{1º:} 1031-Iniciativa-Convencional-Constituyente-del-cc-Tomas-Laibe-sobre-Personas-Privadas-de-Libertad.pdf}
\newline {\color{gray} (Emb: 0.687, TF-IDF: 0.404)}
\newline {\color{gray} \textbf{2º:} 34-2-Iniciativa-Convencional-Constituyente-del-cc-Martín-Arrau-y-otros.pdf}
\newline {\color{gray} (Emb: 0.680, TF-IDF: 0.358)}


\item \textbf{Artículo} \newline
Las personas privadas de libertad tienen el derecho a hacer peticiones a la autoridad penitenciaria y al tribunal de ejecución de la pena para el resguardo de sus derechos y a recibir una respuesta oportuna. 
\newline {\color{gray} \textbf{1º:} 611-Iniciativa-Convencional-Constituyente-de-cc-Valentina-Miranda-sobre-Derechos-de-las-Personas-Privadas-de-Libertad.pdf}
\newline {\color{gray} (Emb: 0.767, TF-IDF: 0.632)}
\newline {\color{gray} \textbf{2º:} 774-Iniciativa-Convencional-Constituyente-de-la-cc-Barbara-Rebolledo-sobre-Derechos-de-las-Mujeres.pdf}
\newline {\color{gray} (Emb: 0.630, TF-IDF: 0.288)}

Asimismo, tienen derecho a mantener la comunicación y el contacto personal, directo y periódico con sus redes de apoyo y siempre con las personas encargadas de su asesoría jurídica. 
\newline {\color{gray} \textbf{1º:} 1031-Iniciativa-Convencional-Constituyente-del-cc-Tomas-Laibe-sobre-Personas-Privadas-de-Libertad.pdf}
\newline {\color{gray} (Emb: 1.000, TF-IDF: 1.000)}
\newline {\color{gray} \textbf{2º:} 451-4-Iniciativa-Convencional-Constituyente-de-la-cc-Carolina-Videla-sobre-Tortura-y-desaparicion-1409-31-01.pdf}
\newline {\color{gray} (Emb: 0.820, TF-IDF: 0.709)}


\item \textbf{Artículo} \newline
El Estado creará los organismos, de personal civil y técnico, que garanticen la inserción e integración penitenciaria y postpenitenciaria de las personas privadas de libertad. 
\newline {\color{gray} \textbf{1º:} 1031-Iniciativa-Convencional-Constituyente-del-cc-Tomas-Laibe-sobre-Personas-Privadas-de-Libertad.pdf}
\newline {\color{gray} (Emb: 0.981, TF-IDF: 0.849)}
\newline {\color{gray} \textbf{2º:} 880-Iniciativa-Convencional-Constituyente-de-la-cc-Ingrid-Villena-sobre-Acciones-Constitucionales.pdf}
\newline {\color{gray} (Emb: 0.641, TF-IDF: 0.335)}

Es deber del Estado garantizar un sistema penitenciario orientado a la inserción e integración de las personas privadas de libertad. 
\newline {\color{gray} \textbf{1º:} 1031-Iniciativa-Convencional-Constituyente-del-cc-Tomas-Laibe-sobre-Personas-Privadas-de-Libertad.pdf}
\newline {\color{gray} (Emb: 1.000, TF-IDF: 1.000)}
\newline {\color{gray} \textbf{2º:} 394-3-Iniciativa-Convencional-Constituyente-de-la-cc-Ramona-Reyes-sobre-Comuna-Autonoma-1525-24-01.pdf}
\newline {\color{gray} (Emb: 0.519, TF-IDF: 0.247)}

La seguridad y administración de estos recintos estará regulado por ley. 
\newline {\color{gray} \textbf{1º:} 1031-Iniciativa-Convencional-Constituyente-del-cc-Tomas-Laibe-sobre-Personas-Privadas-de-Libertad.pdf}
\newline {\color{gray} (Emb: 0.812, TF-IDF: 0.678)}
\newline {\color{gray} \textbf{2º:} 271-4-Iniciativa-Convencional-de-la-cc-Natalia-Henriquez-sobre-Libertad-Personal-17-01-1154-hrs.pdf}
\newline {\color{gray} (Emb: 0.553, TF-IDF: 0.241)}


\item \textbf{Artículo} \newline
Las leyes que regulen a la Contraloría General de la República, la Defensoría del Pueblo, la Defensoría de la Naturaleza, el Servicio Electoral, la Corte Constitucional y al Banco Central, se adoptarán por la mayoría de las y los integrantes del Congreso de Diputadas y Diputados y de la Cámara de las Regiones. 
\newline {\color{gray} \textbf{1º:} 226-6-Iniciativa-Convencional-de-la-cc-Manuela-Royo-sobre-Justicia-Local-1140-hrs.pdf}
\newline {\color{gray} (Emb: 0.663, TF-IDF: 0.330)}
\newline {\color{gray} \textbf{2º:} 1031-Iniciativa-Convencional-Constituyente-del-cc-Tomas-Laibe-sobre-Personas-Privadas-de-Libertad.pdf}
\newline {\color{gray} (Emb: 0.620, TF-IDF: 0.288)}


\item \textbf{Artículo} \newline
Toda persona, individual o colectivamente, tiene derecho a producir información y a participar equitativamente en la comunicación social. 
\newline {\color{gray} \textbf{1º:} 431-6-Iniciativa-Convencional-de-la-cc-Bessy-Gallardo-sobre-Defensoria-Penal-Publica-1145-27-01.pdf}
\newline {\color{gray} (Emb: 0.593, TF-IDF: 0.339)}
\newline {\color{gray} \textbf{2º:} 314-5-Iniciativa-Convencional-del-cc-Miguel-Angel-Botto-sobre-Proteccion-a-los-Animales.pdf}
\newline {\color{gray} (Emb: 0.571, TF-IDF: 0.322)}

Se reconoce el derecho a fundar y mantener medios de comunicación e información. 
\newline {\color{gray} \textbf{1º:} 154-3-c-Iniciativa-del-cc-Felipe-Mena-sobre-Organizacion-Territorial-del-Estado.pdf}
\newline {\color{gray} (Emb: 0.706, TF-IDF: 0.381)}
\newline {\color{gray} \textbf{2º:} 269-3-Iniciativa-Convencional-del-cc-Felipe-Mena-sobre-Organizacion-Territorial-del-Estado-17-01-1154-hrs.pdf}
\newline {\color{gray} (Emb: 0.706, TF-IDF: 0.356)}


\item \textbf{Artículo} \newline
El Estado debe respetar la libertad de prensa, promover el pluralismo de los medios de comunicación y la diversidad de información. 
\newline {\color{gray} \textbf{1º:} 159-3-c-Iniciativa-de-la-cc-Jennifer-Mella-.pdf}
\newline {\color{gray} (Emb: 0.672, TF-IDF: 0.391)}
\newline {\color{gray} \textbf{2º:} 122-3-c-Iniciativa-de-la-cc-Jennifer-Mella-Forma-del-Estado.pdf}
\newline {\color{gray} (Emb: 0.672, TF-IDF: 0.362)}

Se prohíbe la censura previa. 
\newline {\color{gray} \textbf{1º:} 212-7-c-Iniciativa-Convencional-del-cc-Ignacio-Achurra-sobre-Derecho-a-la-Comunicación-y-a-la-Conectividad-2032-hrs.pdf}
\newline {\color{gray} (Emb: 0.880, TF-IDF: 0.640)}
\newline {\color{gray} \textbf{2º:} 146-4-c-Iniciativa-del-cc-Manuel-Jose-Ossandon-Libertad-de-asociacion-y-derecho-a-la-sindicalizacion.pdf}
\newline {\color{gray} (Emb: 0.723, TF-IDF: 0.400)}


\item \textbf{Artículo} \newline
El Estado impedirá la concentración de la propiedad de los medios de comunicación e información. 
\newline {\color{gray} \textbf{1º:} 261-4-Iniciativa-Convencional-de-la-cc-Janis-Meneses-sobre-Derecho-a-la-Libertad-de-Expresion-1152-hrs.pdf}
\newline {\color{gray} (Emb: 0.693, TF-IDF: 0.581)}
\newline {\color{gray} \textbf{2º:} 599-Iniciativa-Convencional-Constituyente-de-cc-Francisco-Caamano-sobre-Derechos-a-la-Informacion-en-la-era-Digital.pdf}
\newline {\color{gray} (Emb: 0.671, TF-IDF: 0.524)}

En ningún caso se podrá establecer el monopolio estatal sobre ellos. 
\newline {\color{gray} \textbf{1º:} 222-7-Iniciativa-Convencional-del-cc-Carlos-Calvo-sobre-Derechos-a-la-Comunicacion-2351-hrs.pdf}
\newline {\color{gray} (Emb: 0.745, TF-IDF: 0.662)}
\newline {\color{gray} \textbf{2º:} 222-7-Iniciativa-Convencional-del-cc-Carlos-Calvo-sobre-Derechos-a-la-Comunicacion-2351-hrs.pdf}
\newline {\color{gray} (Emb: 0.731, TF-IDF: 0.630)}

Corresponderá a la ley el resguardo de este precepto. 
\newline {\color{gray} \textbf{1º:} 290-4-Iniciativa-Convencional-de-la-cc-Tatiana-Urrutia-sobre-Libertad-de-Expresion-1604-hrs.pdf}
\newline {\color{gray} (Emb: 0.961, TF-IDF: 0.909)}
\newline {\color{gray} \textbf{2º:} 407-4-Iniciativa-Convencional-Constituyente-de-la-cc-Tatiana-Urrutia-sobre-Libertad-de-Expresion-0041-17-01.pdf}
\newline {\color{gray} (Emb: 0.961, TF-IDF: 0.810)}


\item \textbf{Artículo} \newline
El Estado fomenta la creación de medios de comunicación e información y su desarrollo a nivel regional, local y comunitario. 
\newline {\color{gray} \textbf{1º:} 704-Iniciativa-Convencional-Constituyente-de-la-cc-Elsa-Labrana-sobre-Derecho-a-la-Comunicacion-01-02.pdf}
\newline {\color{gray} (Emb: 0.591, TF-IDF: 0.502)}
\newline {\color{gray} \textbf{2º:} 523-4-Iniciativa-Convencional-Constituyente-de-cc-Felipe-Harboe-sobre-Libertad-de-Expresion-e-informacion-1130-hrs.-01-02.pdf}
\newline {\color{gray} (Emb: 0.569, TF-IDF: 0.460)}


\item \textbf{Artículo} \newline
Toda persona ofendida o injustamente aludida por un medio de comunicación e información tiene derecho a que su aclaración o rectificación sea difundida gratuitamente por el mismo medio en que hubiese sido emitida. 
\newline {\color{gray} \textbf{1º:} 142-4-c-Iniciativa-de-la-cc-Rocio-Cantuarias-Reconoce-y-Asegura-la-Libertad-de-Expresion.pdf}
\newline {\color{gray} (Emb: 0.720, TF-IDF: 0.769)}
\newline {\color{gray} \textbf{2º:} 664-Iniciativa-Convencional-Constituyente-de-la-cc-Maria-Magdalena-Rivera-sobre-Planificacion-Economica121101-02.pdf}
\newline {\color{gray} (Emb: 0.609, TF-IDF: 0.325)}

La ley regulará el ejercicio de este derecho, con pleno respeto a la libertad de expresión. 
\newline {\color{gray} \textbf{1º:} 515-4-Iniciativa-Convencional-Constituyente-de-la-cc-Giovanna-Grandon-sobre-Derecho-a-Migrar-1245-01-02.pdf}
\newline {\color{gray} (Emb: 0.852, TF-IDF: 0.405)}
\newline {\color{gray} \textbf{2º:} 512-4-Iniciativa-Convencional-Constituyente-del-cc-Bernardo-Fontaine-agua-potable-1105-01-02.pdf}
\newline {\color{gray} (Emb: 0.788, TF-IDF: 0.307)}


\item \textbf{Artículo} \newline
Estos derechos deben ejercerse con pleno respeto a la diversidad cultural, los derechos humanos y de la naturaleza. 
\newline {\color{gray} \textbf{1º:} 9-2-Iniciativa-Convencional-Constituyente-del-cc-Ignacio-Achurra-y-otros-cta.pdf}
\newline {\color{gray} (Emb: 1.000, TF-IDF: 1.000)}
\newline {\color{gray} \textbf{2º:} 407-4-Iniciativa-Convencional-Constituyente-de-la-cc-Tatiana-Urrutia-sobre-Libertad-de-Expresion-0041-17-01.pdf}
\newline {\color{gray} (Emb: 0.879, TF-IDF: 0.746)}

5°: La igualdad ante la ley y no discriminación arbitraria de las diversas cosmovisiones que componen la interculturalidad del país, promoviendo su interrelación armónica y el respeto de todas las expresiones simbólicas, culturales y patrimoniales, sean estas tangibles o intangibles. 
\newline {\color{gray} \textbf{1º:} 186-7-c-Iniciativa-Convencional-de-la-cc-Malucha-Pinto-sobre-derechos-culturales-1126-hrs.pdf}
\newline {\color{gray} (Emb: 0.821, TF-IDF: 0.373)}
\newline {\color{gray} \textbf{2º:} 9-2-Iniciativa-Convencional-Constituyente-del-cc-Ignacio-Achurra-y-otros-cta.pdf}
\newline {\color{gray} (Emb: 0.741, TF-IDF: 0.341)}

El derecho al uso de espacios públicos para desarrollar expresiones y manifestaciones culturales y artísticas, sin más limitaciones que las establecidas en esta Constitución y las leyes. 
\newline {\color{gray} \textbf{1º:} 404-4-Iniciativa-Convencional-Constituyente-de-la-cc-Lidia-Gonzalez-sobre-Derechos-Linguisticos-1940-24-01.pdf}
\newline {\color{gray} (Emb: 0.768, TF-IDF: 0.318)}
\newline {\color{gray} \textbf{2º:} 9-2-Iniciativa-Convencional-Constituyente-del-cc-Ignacio-Achurra-y-otros-cta.pdf}
\newline {\color{gray} (Emb: 0.761, TF-IDF: 0.310)}

Se prohíbe toda forma de censura previa. 
\newline {\color{gray} \textbf{1º:} 186-7-c-Iniciativa-Convencional-de-la-cc-Malucha-Pinto-sobre-derechos-culturales-1126-hrs.pdf}
\newline {\color{gray} (Emb: 0.770, TF-IDF: 0.569)}
\newline {\color{gray} \textbf{2º:} 105-7-c-Iniciativa-del-cc-Miguel-Angel-Botto-cultura-y-patrimonio.pdf}
\newline {\color{gray} (Emb: 0.662, TF-IDF: 0.395)}

El derecho a la identidad cultural, a conocer y educarse en las diversas culturas, así como a expresarse en el idioma o lengua propios. 
\newline {\color{gray} \textbf{1º:} 392-7-Iniciativa-Convencional-Constituyente-de-la-cc-Lorena-Cespedes-sobre-Derecho-a-la-creacion-1449-24-01.pdf}
\newline {\color{gray} (Emb: 0.734, TF-IDF: 0.628)}
\newline {\color{gray} \textbf{2º:} 548-Iniciativa-Convencional-Constituyente-de-la-cc-Patricia-Politzer-sobre-libertad-de-expresion-1653-01-02.pdf}
\newline {\color{gray} (Emb: 0.666, TF-IDF: 0.545)}

El derecho a participar libremente en la vida cultural y artística y a gozar de sus diversas expresiones, bienes, servicios e institucionalidad. 
\newline {\color{gray} \textbf{1º:} 548-Iniciativa-Convencional-Constituyente-de-la-cc-Patricia-Politzer-sobre-libertad-de-expresion-1653-01-02.pdf}
\newline {\color{gray} (Emb: 0.888, TF-IDF: 0.764)}
\newline {\color{gray} \textbf{2º:} 262-4-Iniciativa-Convencional-de-la-cc-Janis-Meneses-sobre-Derecho-a-la-Libertad-de-Opinion-1153-hrs.pdf}
\newline {\color{gray} (Emb: 0.820, TF-IDF: 0.747)}

La Constitución asegura a todas las personas y comunidades: 1°. 
\newline {\color{gray} \textbf{1º:} 212-7-c-Iniciativa-Convencional-del-cc-Ignacio-Achurra-sobre-Derecho-a-la-Comunicación-y-a-la-Conectividad-2032-hrs.pdf}
\newline {\color{gray} (Emb: 0.801, TF-IDF: 0.291)}
\newline {\color{gray} \textbf{2º:} 222-7-Iniciativa-Convencional-del-cc-Carlos-Calvo-sobre-Derechos-a-la-Comunicacion-2351-hrs.pdf}
\newline {\color{gray} (Emb: 0.797, TF-IDF: 0.286)}

La libertad de crear y difundir las culturas y las artes, así como el derecho a disfrutar de sus beneficios. 
\newline {\color{gray} \textbf{1º:} 186-7-c-Iniciativa-Convencional-de-la-cc-Malucha-Pinto-sobre-derechos-culturales-1126-hrs.pdf}
\newline {\color{gray} (Emb: 0.934, TF-IDF: 0.861)}
\newline {\color{gray} \textbf{2º:} 302-4-Iniciativa-Convencional-del-cc-Fuad-Chahin-sobre-Derecho-al-trabajo-18-01.-1141-hrs.pdf}
\newline {\color{gray} (Emb: 0.835, TF-IDF: 0.861)}


\item \textbf{Artículo} \newline
El Estado promueve, fomenta y garantiza el acceso, desarrollo y difusión de las culturas, las artes y los conocimientos, atendiendo a la diversidad cultural en todas sus manifestaciones y contribuciones, bajo los principios de colaboración e interculturalidad. 
\newline {\color{gray} \textbf{1º:} 9-2-Iniciativa-Convencional-Constituyente-del-cc-Ignacio-Achurra-y-otros-cta.pdf}
\newline {\color{gray} (Emb: 0.831, TF-IDF: 0.611)}
\newline {\color{gray} \textbf{2º:} 186-7-c-Iniciativa-Convencional-de-la-cc-Malucha-Pinto-sobre-derechos-culturales-1126-hrs.pdf}
\newline {\color{gray} (Emb: 0.711, TF-IDF: 0.596)}

El Estado debe generar las instancias para que la sociedad contribuya al desarrollo de la creatividad cultural y artística, en sus más diversas expresiones. 
\newline {\color{gray} \textbf{1º:} 693-Iniciativa-Convencional-Constituyente-del-cc-Ricardo-Neumann-sobre-DIversidad-Intercultural-121101-02.pdf}
\newline {\color{gray} (Emb: 0.707, TF-IDF: 0.695)}
\newline {\color{gray} \textbf{2º:} 71-2-Iniciativa-Convencional-Constitutente-de-Maria-Jose-Oyarzun-y-otros.pdf}
\newline {\color{gray} (Emb: 0.629, TF-IDF: 0.229)}

El Estado promueve las condiciones para el libre desarrollo de la identidad cultural de las comunidades y personas, así como de sus procesos culturales. 
\newline {\color{gray} \textbf{1º:} 9-2-Iniciativa-Convencional-Constituyente-del-cc-Ignacio-Achurra-y-otros-cta.pdf}
\newline {\color{gray} (Emb: 0.790, TF-IDF: 0.652)}
\newline {\color{gray} \textbf{2º:} 887-Iniciativa-Convencional-Constituyente-de-la-cc-Maria-Trinidad-Castillo-Sobre-Derecho-a-la-Educacion.pdf}
\newline {\color{gray} (Emb: 0.767, TF-IDF: 0.526)}

Esto se realizará con pleno respeto a los derechos, libertades y las autonomías que consagra esta Constitución. 
\newline {\color{gray} \textbf{1º:} 105-7-c-Iniciativa-del-cc-Miguel-Angel-Botto-cultura-y-patrimonio.pdf}
\newline {\color{gray} (Emb: 0.773, TF-IDF: 0.576)}
\newline {\color{gray} \textbf{2º:} 9-2-Iniciativa-Convencional-Constituyente-del-cc-Ignacio-Achurra-y-otros-cta.pdf}
\newline {\color{gray} (Emb: 0.677, TF-IDF: 0.340)}


\item \textbf{Artículo} \newline
Los pueblos y naciones preexistentes tienen derecho a obtener la repatriación de objetos de cultura y de restos humanos pertenecientes a los pueblos. 
\newline {\color{gray} \textbf{1º:} 186-7-c-Iniciativa-Convencional-de-la-cc-Malucha-Pinto-sobre-derechos-culturales-1126-hrs.pdf}
\newline {\color{gray} (Emb: 0.735, TF-IDF: 0.297)}
\newline {\color{gray} \textbf{2º:} 195-7-c-Iniciativa-Convenciona-de-la-cc-Malucha-Pinto-sobre-derechos-de-autor-1412hrs.pdf}
\newline {\color{gray} (Emb: 0.712, TF-IDF: 0.297)}

El Estado adoptará mecanismos eficaces en materia de restitución y repatriación de objetos de culto y restos humanos que fueron confiscados sin consentimiento de los pueblos y garantizará el acceso de los pueblos a su propio patrimonio, incluyendo objetos, restos humanos y sitios culturalmente significativos para su desarrollo. 
\newline {\color{gray} \textbf{1º:} 186-7-c-Iniciativa-Convencional-de-la-cc-Malucha-Pinto-sobre-derechos-culturales-1126-hrs.pdf}
\newline {\color{gray} (Emb: 0.944, TF-IDF: 0.809)}
\newline {\color{gray} \textbf{2º:} 584-Iniciativa-Convencional-Constituyente-de-cc-Malucha-Pinto-sobre-Democracia-Cultural-2351-hrs.-01-02.pdf}
\newline {\color{gray} (Emb: 0.699, TF-IDF: 0.421)}


\item \textbf{Artículo} \newline
Todas las personas, individual y colectivamente, tienen derecho al acceso universal, a la conectividad digital y a las tecnologías de la información y comunicación, con pleno respeto de los derechos y garantías que establecen esta Constitución y las leyes. 
\newline {\color{gray} \textbf{1º:} 362-2-Iniciativa-Convencional-Constituyente-de-la-cc-Janis-Meneses-sobre-Derechos-Laborales-1836-hrs-21-01.pdf}
\newline {\color{gray} (Emb: 0.710, TF-IDF: 0.356)}
\newline {\color{gray} \textbf{2º:} 60-2-Iniciativa-Convencional-Constituyente-del-cc-Jorge-Baradit-y-otros.pdf}
\newline {\color{gray} (Emb: 0.700, TF-IDF: 0.298)}


\item \textbf{Artículo} \newline
La ley regulará la forma en que el Estado cumplirá este deber, así como su participación y la de otros actores en la materia. 
\newline {\color{gray} \textbf{1º:} 212-7-c-Iniciativa-Convencional-del-cc-Ignacio-Achurra-sobre-Derecho-a-la-Comunicación-y-a-la-Conectividad-2032-hrs.pdf}
\newline {\color{gray} (Emb: 0.848, TF-IDF: 0.632)}
\newline {\color{gray} \textbf{2º:} 280-4-Iniciativa-Convencional-de-la-cc-Ivanna-Olivares-sobre-Derecho-de-Prensa-1200-hrs.pdf}
\newline {\color{gray} (Emb: 0.786, TF-IDF: 0.536)}

El Estado tiene la obligación de superar las brechas de acceso, uso y participación en el espacio digital, sus dispositivos e infraestructuras. 
\newline {\color{gray} \textbf{1º:} 386-4-Iniciativa-Convencional-Constituyente-de-la-cc-Francisca-Linconao-sobre-Educacion-Intercultural-1228-24-01.pdf}
\newline {\color{gray} (Emb: 0.703, TF-IDF: 0.526)}
\newline {\color{gray} \textbf{2º:} 433-2-Iniciativa-Convencional-del-cc-Jorge-Abarca-sobre-Preambulo-de-la-Constitucion-1155-27-01.pdf}
\newline {\color{gray} (Emb: 0.702, TF-IDF: 0.365)}

Es deber del Estado promover y participar del desarrollo de las telecomunicaciones, servicios de conectividad y tecnologías de la información y comunicación. 
\newline {\color{gray} \textbf{1º:} 79-2-Iniciativa-del-cc-Alexis-Caiguan-y-otros.pdf}
\newline {\color{gray} (Emb: 0.633, TF-IDF: 0.300)}
\newline {\color{gray} \textbf{2º:} 644-Iniciativiva-Convencional-Constituyene-de-la-cc-Isabel-Godoy-sobre-Derechos-Linguisticos-1738-01-02.pdf}
\newline {\color{gray} (Emb: 0.604, TF-IDF: 0.228)}


\item \textbf{Artículo} \newline
El Estado garantiza el cumplimiento del principio de neutralidad en la red. 
\newline {\color{gray} \textbf{1º:} 212-7-c-Iniciativa-Convencional-del-cc-Ignacio-Achurra-sobre-Derecho-a-la-Comunicación-y-a-la-Conectividad-2032-hrs.pdf}
\newline {\color{gray} (Emb: 0.726, TF-IDF: 0.718)}
\newline {\color{gray} \textbf{2º:} 100-7-c-Iniciativa-del-cc-Francisco-Caamano-Derecho-al-Acceso-y-a-la-Conectividad-Digital.pdf}
\newline {\color{gray} (Emb: 0.603, TF-IDF: 0.287)}

Las obligaciones, condiciones y límites en esta materia serán determinados por la ley. 
\newline {\color{gray} \textbf{1º:} 616-Iniciativa-Convencional-Constituyente-de-cc-Felipe-Harboe-sobre-Derecho-a-la-Conectividad-Digital-2240-hrs.-01-02.pdf}
\newline {\color{gray} (Emb: 0.601, TF-IDF: 0.410)}
\newline {\color{gray} \textbf{2º:} 7-2-Iniciativa-Convencional-Constituyente-del-cc-Luis-Barceló-y-otros.pdf}
\newline {\color{gray} (Emb: 0.584, TF-IDF: 0.388)}


\item \textbf{Artículo} \newline
El Estado garantiza el acceso libre, equitativo y descentralizado, con condiciones de calidad y velocidad adecuadas y efectivas a los servicios básicos de comunicación. 
\newline {\color{gray} \textbf{1º:} 344-3-Iniciativa-Convencional-Constituyente-del-cc-Hernan-Larrain-sobre-Reforma-Administrativa-y-Modernizacion-del-Estado.pdf}
\newline {\color{gray} (Emb: 0.679, TF-IDF: 0.306)}
\newline {\color{gray} \textbf{2º:} 9-2-Iniciativa-Convencional-Constituyente-del-cc-Ignacio-Achurra-y-otros-cta.pdf}
\newline {\color{gray} (Emb: 0.663, TF-IDF: 0.294)}


\item \textbf{Artículo} \newline
Toda persona tiene el derecho a la educación digital, al desarrollo del conocimiento, pensamiento y lenguaje tecnológico, así como a gozar de sus beneficios. 
\newline {\color{gray} \textbf{1º:} 100-7-c-Iniciativa-del-cc-Francisco-Caamano-Derecho-al-Acceso-y-a-la-Conectividad-Digital.pdf}
\newline {\color{gray} (Emb: 1.000, TF-IDF: 1.000)}
\newline {\color{gray} \textbf{2º:} 93-7-Iniciativa-del-cc-Francisco-Caamano-Derecho-al-acceso-y-a-la-Conectividad-Digital.pdf}
\newline {\color{gray} (Emb: 0.959, TF-IDF: 0.877)}

El Estado asegurará que todas las personas tengan la posibilidad de ejercer sus derechos en los espacios digitales, para lo cual creará políticas públicas y financiará planes y programas gratuitos con tal objeto. 
\newline {\color{gray} \textbf{1º:} 100-7-c-Iniciativa-del-cc-Francisco-Caamano-Derecho-al-Acceso-y-a-la-Conectividad-Digital.pdf}
\newline {\color{gray} (Emb: 1.000, TF-IDF: 1.000)}
\newline {\color{gray} \textbf{2º:} 93-7-Iniciativa-del-cc-Francisco-Caamano-Derecho-al-acceso-y-a-la-Conectividad-Digital.pdf}
\newline {\color{gray} (Emb: 0.963, TF-IDF: 0.594)}


\item \textbf{Artículo} \newline
Todas las personas tienen el derecho a participar de un espacio digital libre de violencia. 
\newline {\color{gray} \textbf{1º:} 212-7-c-Iniciativa-Convencional-del-cc-Ignacio-Achurra-sobre-Derecho-a-la-Comunicación-y-a-la-Conectividad-2032-hrs.pdf}
\newline {\color{gray} (Emb: 0.825, TF-IDF: 0.748)}
\newline {\color{gray} \textbf{2º:} 310-7-Iniciativa-Convencional-de-la-cc-Carolina-Videla-sobre-Derecho-a-la-Comunicacion-2000-hrs.pdf}
\newline {\color{gray} (Emb: 0.806, TF-IDF: 0.683)}

El Estado desarrollará acciones de prevención, promoción, reparación y garantía de este derecho, otorgando especial protección a mujeres, niñas, niños, jóvenes y disidencias sexogenéricas. 
\newline {\color{gray} \textbf{1º:} 92-7-Iniciativa-del-cc-Francisco-Caamano-Derecho-a-la-Alfabetizacion-Digital-2.pdf}
\newline {\color{gray} (Emb: 0.947, TF-IDF: 0.813)}
\newline {\color{gray} \textbf{2º:} 704-Iniciativa-Convencional-Constituyente-de-la-cc-Elsa-Labrana-sobre-Derecho-a-la-Comunicacion-01-02.pdf}
\newline {\color{gray} (Emb: 0.648, TF-IDF: 0.396)}

Las obligaciones, condiciones y límites en esta materia serán determinados por la ley. 
\newline {\color{gray} \textbf{1º:} 92-7-Iniciativa-del-cc-Francisco-Caamano-Derecho-a-la-Alfabetizacion-Digital-2.pdf}
\newline {\color{gray} (Emb: 0.851, TF-IDF: 0.757)}
\newline {\color{gray} \textbf{2º:} 828-5-Iniciativa-Convencional-Constituyente-de-la-cc-Isabel-Godoy-sobre-Desarrollo-Propio.pdf}
\newline {\color{gray} (Emb: 0.565, TF-IDF: 0.270)}


\item \textbf{Artículo} \newline
Todas las personas tienen derecho al descanso, al ocio y a disfrutar el tiempo libre. 
\newline {\color{gray} \textbf{1º:} 616-Iniciativa-Convencional-Constituyente-de-cc-Felipe-Harboe-sobre-Derecho-a-la-Conectividad-Digital-2240-hrs.-01-02.pdf}
\newline {\color{gray} (Emb: 0.695, TF-IDF: 0.396)}
\newline {\color{gray} \textbf{2º:} 261-4-Iniciativa-Convencional-de-la-cc-Janis-Meneses-sobre-Derecho-a-la-Libertad-de-Expresion-1152-hrs.pdf}
\newline {\color{gray} (Emb: 0.656, TF-IDF: 0.389)}


\item \textbf{Artículo} \newline
El Estado reconoce la neurodiversidad y garantiza a las personas neurodivergentes su derecho a una vida autónoma, a desarrollar libremente su personalidad e identidad, a ejercer su capacidad jurídica y los derechos, individuales y colectivos, reconocidos en esta Constitución y los tratados e instrumentos internacionales de Derechos Humanos ratificados por Chile y que se encuentren vigentes. 
\newline {\color{gray} \textbf{1º:} 622-Iniciativa-Convencional-Constituyente-de-cc-Felipe-Harboe-Reconocimiento-y-proteccion-integral-de-derechos-de-NNA.pdf}
\newline {\color{gray} (Emb: 0.662, TF-IDF: 0.449)}
\newline {\color{gray} \textbf{2º:} 848-Iniciativa-Convencional-Constituyente-de-la-cc-Janis-Meneses-sobre-Derechos-de-los-NNA.pdf}
\newline {\color{gray} (Emb: 0.652, TF-IDF: 0.398)}


\item \textbf{Artículo} \newline
Las ciencias y tecnologías, sus aplicaciones y procesos investigativos, deben desarrollarse según los principios de solidaridad, cooperación, responsabilidad y con pleno respeto a la dignidad humana, a la sintiencia de los animales, los derechos de la naturaleza y los demás derechos establecidos en esta Constitución y en tratados internacionales de Derechos Humanos ratificados por Chile y que se encuentren vigentes. 
\newline {\color{gray} \textbf{1º:} 100-7-c-Iniciativa-del-cc-Francisco-Caamano-Derecho-al-Acceso-y-a-la-Conectividad-Digital.pdf}
\newline {\color{gray} (Emb: 1.000, TF-IDF: 1.000)}
\newline {\color{gray} \textbf{2º:} 93-7-Iniciativa-del-cc-Francisco-Caamano-Derecho-al-acceso-y-a-la-Conectividad-Digital.pdf}
\newline {\color{gray} (Emb: 0.963, TF-IDF: 0.594)}


\item \textbf{Artículo} \newline
Toda persona, individual o colectivamente, tiene derecho a participar libremente de la creación, desarrollo, conservación e innovación de los diversos sistemas de conocimientos y a la transferencia de sus aplicaciones, así como a gozar de sus beneficios. 
\newline {\color{gray} \textbf{1º:} 149-7-c-Iniciativa-de-la-cc-Angelica-Tepper-Derecho-al-Descanso.pdf}
\newline {\color{gray} (Emb: 1.000, TF-IDF: 1.000)}
\newline {\color{gray} \textbf{2º:} 630-Iniciativa-Convencional-Constituyente-de-cc-Valentina-Miranda-sobre-Derecho-al-trabajo-1602-hrs.-01-02.pdf}
\newline {\color{gray} (Emb: 0.746, TF-IDF: 0.575)}

El Estado reconoce el derecho de los pueblos y naciones indígenas a preservar, revitalizar, desarrollar y transmitir los conocimientos tradicionales y saberes ancestrales y debe, en conjunto con ellos, adoptar medidas eficaces para garantizar su ejercicio. 
\newline {\color{gray} \textbf{1º:} 304-4-Iniciativa-Convencional-de-la-cc-Valentina-Miranda-sobre-Derechos-Fundamentales-1301-hrs.pdf}
\newline {\color{gray} (Emb: 0.839, TF-IDF: 0.454)}
\newline {\color{gray} \textbf{2º:} 103-4-c-Iniciativa-del-cc-Alfredo-Moreno-Titularidad-de-los-Derechos-Fundamentales.pdf}
\newline {\color{gray} (Emb: 0.799, TF-IDF: 0.429)}

Asimismo, la Constitución garantiza la libertad de investigación. 
\newline {\color{gray} \textbf{1º:} 841-Iniciativa-Convencional-Constituyente-de-la-cc-Francisca-Arauna-sobre-Seguridad-Publica.pdf}
\newline {\color{gray} (Emb: 0.639, TF-IDF: 0.454)}
\newline {\color{gray} \textbf{2º:} 231-6-Iniciativa-Convencional-del-cc-Ruggero-Cozzi-sobre-Justicia-Militar-1143-hrs.pdf}
\newline {\color{gray} (Emb: 0.638, TF-IDF: 0.454)}


\item \textbf{Artículo} \newline
El Estado reconoce y fomenta el desarrollo de los diversos sistemas de conocimientos en el país, considerando sus diferentes contextos culturales, sociales y territoriales. 
\newline {\color{gray} \textbf{1º:} 998-Iniciativa-Convencional-Constituyente-del-cc-Francisco-Caamano-sobre-Derechos-de-Autor.pdf}
\newline {\color{gray} (Emb: 0.587, TF-IDF: 0.383)}
\newline {\color{gray} \textbf{2º:} 420-7-Iniciativa-Convencional-del-cc-Francisco-Caamano-sobre-Derechos-de-Autor-1200-26-01.pdf}
\newline {\color{gray} (Emb: 0.587, TF-IDF: 0.375)}

Asimismo, fomenta su acceso equitativo y abierto, lo que comprende el intercambio y comunicación de conocimientos a la sociedad de la forma más amplia posible, con pleno respeto a los derechos establecidos en esta Constitución. 
\newline {\color{gray} \textbf{1º:} 644-Iniciativiva-Convencional-Constituyene-de-la-cc-Isabel-Godoy-sobre-Derechos-Linguisticos-1738-01-02.pdf}
\newline {\color{gray} (Emb: 0.700, TF-IDF: 0.350)}
\newline {\color{gray} \textbf{2º:} 79-2-Iniciativa-del-cc-Alexis-Caiguan-y-otros.pdf}
\newline {\color{gray} (Emb: 0.643, TF-IDF: 0.327)}

El Estado promoverá en todas sus etapas, un sistema educativo integral donde se fomenten interdisciplinariamente el pensamiento crítico y las habilidades basadas en la capacidad creadora del ser humano a través de las diversas áreas del conocimiento. 
\newline {\color{gray} \textbf{1º:} 252-4-Iniciativa-Convencional-de-la-cc-Tatiana-Urrutia-sobre-Libertad-de-Conciencia-y-Culto-1149-hrs.pdf}
\newline {\color{gray} (Emb: 0.578, TF-IDF: 0.408)}
\newline {\color{gray} \textbf{2º:} 107-4-c-Iniciativa-de-la-cc-Giovanna-Grandon-Derecho-al-Trabajo.pdf}
\newline {\color{gray} (Emb: 0.570, TF-IDF: 0.375)}


\item \textbf{Artículo} \newline
Se prohíbe la asimilación forzada o destrucción de las culturas de los pueblos y naciones indígenas. 
\newline {\color{gray} \textbf{1º:} 368-7-Iniciativa-Convencional-Constituyente-del-cc-Francisco-Caamano-sobre-los-conocimientos-0900-hrs-24-01.pdf}
\newline {\color{gray} (Emb: 0.665, TF-IDF: 0.306)}
\newline {\color{gray} \textbf{2º:} 60-2-Iniciativa-Convencional-Constituyente-del-cc-Jorge-Baradit-y-otros.pdf}
\newline {\color{gray} (Emb: 0.659, TF-IDF: 0.291)}


\item \textbf{Artículo} \newline
La Constitución asegura a todas las personas la protección de los derechos de autor sobre sus obras intelectuales, científicas y artísticas, comprendiendo los derechos morales y patrimoniales sobre ellas, en conformidad y por el tiempo que señale la ley, que no será inferior a la vida del autor. 
\newline {\color{gray} \textbf{1º:} 212-7-c-Iniciativa-Convencional-del-cc-Ignacio-Achurra-sobre-Derecho-a-la-Comunicación-y-a-la-Conectividad-2032-hrs.pdf}
\newline {\color{gray} (Emb: 0.591, TF-IDF: 0.390)}
\newline {\color{gray} \textbf{2º:} 137-4-c-Iniciativa-de-la-cc-Rocio-Cantuarias-Establece-el-Derecho-al-Acceso-a-la-Informacion-Publica.pdf}
\newline {\color{gray} (Emb: 0.557, TF-IDF: 0.262)}

Asimismo, la Constitución asegura la protección a los derechos de intérpretes o ejecutantes sobre sus interpretaciones o ejecuciones, de conformidad a la ley. 
\newline {\color{gray} \textbf{1º:} 295-7-Iniciativa-Convencional-del-cc-Ricardo-Neumann-sobre-Libertad-Creativa-1728-hrs.pdf}
\newline {\color{gray} (Emb: 0.753, TF-IDF: 0.766)}
\newline {\color{gray} \textbf{2º:} 386-4-Iniciativa-Convencional-Constituyente-de-la-cc-Francisca-Linconao-sobre-Educacion-Intercultural-1228-24-01.pdf}
\newline {\color{gray} (Emb: 0.562, TF-IDF: 0.240)}


\item \textbf{Artículo} \newline
El Estado, en conjunto con los pueblos y naciones indígenas preexistentes, adoptará medidas positivas para la recuperación, revitalización y fortalecimiento del patrimonio cultural indígena. 
\newline {\color{gray} \textbf{1º:} 788-Iniciativa-Convencional-Constituyente-de-la-cc-Camila-Zarate-sobre-Democracia-Ecologica.pdf}
\newline {\color{gray} (Emb: 0.838, TF-IDF: 0.480)}
\newline {\color{gray} \textbf{2º:} 595-Iniciativa-Convencional-Constituyente-de-cc-Cesar-Uribe-sobre-Presencia-del-mundo-rural-2351-hrs.-01-02.pdf}
\newline {\color{gray} (Emb: 0.604, TF-IDF: 0.296)}


\item \textbf{Artículo} \newline
Todas las personas tienen derecho a la protección de los datos de carácter personal, a conocer, decidir y controlar el uso de las informaciones que les conciernen. 
\newline {\color{gray} \textbf{1º:} 339-7-Iniciativa-Convencional-Constituyente-del-cc-Miguel-Angel-Botto-sobre-Proteccion-de-los-Derechos-de-Autor.pdf}
\newline {\color{gray} (Emb: 0.733, TF-IDF: 0.509)}
\newline {\color{gray} \textbf{2º:} 295-7-Iniciativa-Convencional-del-cc-Ricardo-Neumann-sobre-Libertad-Creativa-1728-hrs.pdf}
\newline {\color{gray} (Emb: 0.716, TF-IDF: 0.481)}


\item \textbf{Artículo} \newline
El Estado y los particulares deberán adoptar las medidas idóneas y necesarias que garanticen la integridad, confidencialidad, disponibilidad y resiliencia de la información que contengan los sistemas informáticos que administren, salvo los casos expresamente señalados por la ley. 
\newline {\color{gray} \textbf{1º:} 60-2-Iniciativa-Convencional-Constituyente-del-cc-Jorge-Baradit-y-otros.pdf}
\newline {\color{gray} (Emb: 0.635, TF-IDF: 0.335)}
\newline {\color{gray} \textbf{2º:} 187-7-c-Iniciativa-Convenciona-del-cc-Ignacio-Achurra-sobre-Los-Patrimonios-1126-hrs.pdf}
\newline {\color{gray} (Emb: 0.630, TF-IDF: 0.325)}

Todas las personas, individual y colectivamente, tienen el derecho a la protección y promoción de la seguridad informática. 
\newline {\color{gray} \textbf{1º:} 641-Iniciativa-Convencional-Constituyente-del-cc-Mauricio-Daza-sobre-Contraloria-General-1730-01-02.pdf}
\newline {\color{gray} (Emb: 0.623, TF-IDF: 0.422)}
\newline {\color{gray} \textbf{2º:} 178-5-c-Iniciativa-Convencional-del-cc-Rodrigo-Álvarez-sobre-Minería-1044-hrs.pdf}
\newline {\color{gray} (Emb: 0.622, TF-IDF: 0.422)}


\item \textbf{Artículo} \newline
El acceso a la información pública será garantizado con la sola excepción de aquellas materias que la ley determine reservada o secreta. 
\newline {\color{gray} \textbf{1º:} 416-7-Iniciativa-Convencional-del-cc-Francisco-Caamano-sobre-Proteccion-de-datos-de-caracter-personal.pdf}
\newline {\color{gray} (Emb: 0.854, TF-IDF: 0.432)}
\newline {\color{gray} \textbf{2º:} 524-4-Iniciativa-Convencional-Constituyente-de-cc-Felipe-Harbor-sobre-DERECHO-A-LA-PRIVACIDAD-1154-hrs.-01-02.pdf}
\newline {\color{gray} (Emb: 0.766, TF-IDF: 0.395)}


\item \textbf{Artículo} \newline
La Constitución reconoce los derechos culturales del Pueblo Tribal Afrodescendiente chileno, y asegura su ejercicio, desarrollo, promoción, conservación y protección, con pleno respeto a los instrumentos internacionales pertinentes. 
\newline {\color{gray} \textbf{1º:} 458-4-Iniciativa-Convencional-Constituyente-del-cc-Felipe-Harboe-sobre-Derecho-a-la-Seguridad-Informatica-1857-31-01.pdf}
\newline {\color{gray} (Emb: 0.825, TF-IDF: 0.648)}
\newline {\color{gray} \textbf{2º:} 610-Iniciativa-Convencional-Constituyente-de-cc-Valentina-Miranda-Derechos-de-las-Personas-LGBTIQ-y-Derecho-a-la-Igualdad.pdf}
\newline {\color{gray} (Emb: 0.698, TF-IDF: 0.551)}


\item \textbf{Artículo} \newline
Es deber del Estado preservar la memoria y garantizar el acceso a los archivos y documentos, en sus distintos soportes y contenidos. 
\newline {\color{gray} \textbf{1º:} 458-4-Iniciativa-Convencional-Constituyente-del-cc-Felipe-Harboe-sobre-Derecho-a-la-Seguridad-Informatica-1857-31-01.pdf}
\newline {\color{gray} (Emb: 0.815, TF-IDF: 0.737)}
\newline {\color{gray} \textbf{2º:} 345-4-Iniciativa-Convencional-Constituyente-de-la-cc-Alejandra-Flores-sobre-Derecho-a-la-Alimentacion.pdf}
\newline {\color{gray} (Emb: 0.582, TF-IDF: 0.215)}

Los sitios de memoria y memoriales serán objeto de especial protección, asegurando su preservación y sostenibilidad. 
\newline {\color{gray} \textbf{1º:} 599-Iniciativa-Convencional-Constituyente-de-cc-Francisco-Caamano-sobre-Derechos-a-la-Informacion-en-la-era-Digital.pdf}
\newline {\color{gray} (Emb: 0.668, TF-IDF: 0.387)}
\newline {\color{gray} \textbf{2º:} 1031-Iniciativa-Convencional-Constituyente-del-cc-Tomas-Laibe-sobre-Personas-Privadas-de-Libertad.pdf}
\newline {\color{gray} (Emb: 0.665, TF-IDF: 0.379)}


\item \textbf{Artículo} \newline
El Estado reconoce y protege los patrimonios naturales y culturales, materiales e inmateriales, y garantiza su conservación, revitalización, aumento, salvaguardia y transmisión a las generaciones futuras, cualquiera sea el régimen jurídico y titularidad de dichos bienes. 
\newline {\color{gray} \textbf{1º:} 254-7-Iniciativa-Convencional-de-la-cc-Cristina-Dorador-sobre-Reservas-Patrimoniales-1150-hrs.pdf}
\newline {\color{gray} (Emb: 0.686, TF-IDF: 0.541)}
\newline {\color{gray} \textbf{2º:} 574-Iniciativa-Convencional-Constituyente-de-cc-Vanessa-Hoppe-sobre-Defensoria-de-los-Pueblos-2351-hrs.-01-02.pdf}
\newline {\color{gray} (Emb: 0.668, TF-IDF: 0.505)}


\item \textbf{Artículo} \newline
El Estado fomentará la difusión y educación sobre los patrimonios naturales y culturales, materiales e inmateriales. 
\newline {\color{gray} \textbf{1º:} 421-4-Iniciativa-Convencional-del-cc-Agustin-Squella-sobre-Derechos-a-la-Memoria-1315-26-01.pdf}
\newline {\color{gray} (Emb: 0.591, TF-IDF: 0.372)}
\newline {\color{gray} \textbf{2º:} 804-Iniciativa-Convencional-Constituyente-del-cc-Felix-Galleguillos-sobre-Proteccion-del-Patrimonio.pdf}
\newline {\color{gray} (Emb: 0.531, TF-IDF: 0.266)}


\item \textbf{Artículo} \newline
La infraestructura de telecomunicaciones es de interés público, independientemente de su régimen patrimonial. 
\newline {\color{gray} \textbf{1º:} 585-Iniciativa-Convencional-Constituyente-de-cc-Malucha-Pinto-sobre-Derecho-de-y-a-la-memoria-2351-hrs.-01-02.pdf}
\newline {\color{gray} (Emb: 0.873, TF-IDF: 0.584)}
\newline {\color{gray} \textbf{2º:} 189-7-c-Iniciativa-Convencional-del-cc-Ignacio-Achurra-sobre-Derecho-a-los-Patrimonios-Culturales-1238-hrs.pdf}
\newline {\color{gray} (Emb: 0.520, TF-IDF: 0.429)}


\item \textbf{Artículo} \newline
Corresponderá a la ley determinar la utilización y aprovechamiento del espectro radioeléctrico. 
\newline {\color{gray} \textbf{1º:} 727-Iniciativa-Convencional-Constituyente-del-cc-Eric-Chinga-sobre-Desarrollo-Sostenible.pdf}
\newline {\color{gray} (Emb: 0.717, TF-IDF: 0.411)}
\newline {\color{gray} \textbf{2º:} 264-4-Iniciativa-Convencional-de-la-cc-Janis-Meneses-sobre-Derecho-de-Propiedad-1154-hrs.pdf}
\newline {\color{gray} (Emb: 0.693, TF-IDF: 0.407)}


\item \textbf{Artículo} \newline
La ley regulará su organización y la composición de sus directorios, la que estará orientada por criterios de idoneidad y técnicos. 
\newline {\color{gray} \textbf{1º:} 518-7-Iniciativa-Convencional-Constituyente-de-cc-Bernardo-de-la-Maza-sobre-Medios-de-Comunicacion-1521-hrs.-01-02.pdf}
\newline {\color{gray} (Emb: 0.601, TF-IDF: 0.299)}
\newline {\color{gray} \textbf{2º:} 576-Iniciativa-Convencional-Constituyente-de-cc-Bernardo-de-La-Maza-sobre-Medios-de-Comunicacion-Publicos-2310-hrs.-01.pdf}
\newline {\color{gray} (Emb: 0.601, TF-IDF: 0.299)}

Asimismo, gozarán de independencia respecto del gobierno y contarán con financiamiento público para su funcionamiento. 
\newline {\color{gray} \textbf{1º:} 616-Iniciativa-Convencional-Constituyente-de-cc-Felipe-Harboe-sobre-Derecho-a-la-Conectividad-Digital-2240-hrs.-01-02.pdf}
\newline {\color{gray} (Emb: 0.717, TF-IDF: 0.490)}
\newline {\color{gray} \textbf{2º:} 742-Iniciativa-Convencional-Constituyente-del-cc-Ignacio-Achurra-sobre-Medios-de-Comunicacion-01-02.pdf}
\newline {\color{gray} (Emb: 0.678, TF-IDF: 0.397)}

Existirán medios de comunicación públicos en distintos soportes tecnológicos, que respondan a las necesidades informativas, educativas, culturales y de entretenimiento de los diversos grupos de la población. 
\newline {\color{gray} \textbf{1º:} 298-7-Iniciativa-Convencional-de-la-cc-Carolina-Videla-sobre-Reconocimiento-de-los-Patrimonios-1913-hrs.pdf}
\newline {\color{gray} (Emb: 0.735, TF-IDF: 0.620)}
\newline {\color{gray} \textbf{2º:} 393-7-Iniciativa-Convencional-Constituyente-del-cc-Alfredo-Moreno-sobre-Patrimonio-Inmaterial-1455-24-01.pdf}
\newline {\color{gray} (Emb: 0.676, TF-IDF: 0.473)}

Estos medios de comunicación serán pluralistas, descentralizados, y estarán coordinados entre sí. 
\newline {\color{gray} \textbf{1º:} 459-7-Iniciativa-Convencional-Constituyente-del-cc-Francisco-Caamano-sobre-Acceso-a-la-Conectividad-Digital-1903-31-01.pdf}
\newline {\color{gray} (Emb: 0.984, TF-IDF: 0.944)}
\newline {\color{gray} \textbf{2º:} 516-4-Iniciativa-Convencional-Constituyente-de-la-cc-Manuela-Royo-sobre-Derecho-a-la-Energia-1538-01-02.pdf}
\newline {\color{gray} (Emb: 0.577, TF-IDF: 0.315)}


\item \textbf{Artículo} \newline
El Consejo Nacional de Bioética será un órgano independiente, técnico, de carácter consultivo, pluralista y transdisciplinario, que tendrá, entre sus funciones, asesorar a los organismos del Estado en los asuntos bioéticos que puedan afectar a la vida humana, animal, la naturaleza y la biodiversidad, recomendando la dictación, modificación y supresión de normas que regulen dichas materias. 
\newline {\color{gray} \textbf{1º:} 576-Iniciativa-Convencional-Constituyente-de-cc-Bernardo-de-La-Maza-sobre-Medios-de-Comunicacion-Publicos-2310-hrs.-01.pdf}
\newline {\color{gray} (Emb: 0.647, TF-IDF: 0.367)}
\newline {\color{gray} \textbf{2º:} 518-7-Iniciativa-Convencional-Constituyente-de-cc-Bernardo-de-la-Maza-sobre-Medios-de-Comunicacion-1521-hrs.-01-02.pdf}
\newline {\color{gray} (Emb: 0.647, TF-IDF: 0.367)}

La ley regulará la composición, funciones, organización y demás aspectos de este órgano. 
\newline {\color{gray} \textbf{1º:} 753-Iniciativa-Convencional-Constituyente-del-cc-Raul-Celis-sobre-Gobiernos-Locales.pdf}
\newline {\color{gray} (Emb: 0.690, TF-IDF: 0.303)}
\newline {\color{gray} \textbf{2º:} 937-IniciativaConvencional-Constituyente-del-cc-Tomas-Laibe-sobre-Banco-Central.pdf}
\newline {\color{gray} (Emb: 0.636, TF-IDF: 0.265)}


\item \textbf{Artículo} \newline
Es deber del Estado estimular, promover y fortalecer el desarrollo de la investigación científica y tecnológica en todas las áreas del conocimiento, contribuyendo así al enriquecimiento sociocultural del país y al mejoramiento de las condiciones de vida de sus habitantes. 
\newline {\color{gray} \textbf{1º:} 349-6-Iniciativa-Convencional-Constituyente-del-cc-Luis-Mayol-sobre-Banco-Central-1826-20-01.pdf}
\newline {\color{gray} (Emb: 0.701, TF-IDF: 0.387)}
\newline {\color{gray} \textbf{2º:} 398-5-Iniciativa-Convencional-Constituyente-de-la-cc-Ivanna-Olivares-sobre-Banca-Publica-1823-24-01.pdf}
\newline {\color{gray} (Emb: 0.679, TF-IDF: 0.356)}

El Estado generará las condiciones necesarias para el desarrollo de la investigación científica transdisciplinaria en materias relevantes para el resguardo de la calidad de vida de la población y el equilibrio ecosistémico, además del monitoreo permanente de los riesgos medioambientales y sanitarios que afecten la salud de las comunidades y ecosistemas del país, realizándose, en ambos casos, de forma independiente y descentralizada. 
\newline {\color{gray} \textbf{1º:} 87-7-Iniciativa-Convencional-Constituyente-de-la-cc-Loreto-Vidal-y-otros.pdf}
\newline {\color{gray} (Emb: 0.831, TF-IDF: 0.682)}
\newline {\color{gray} \textbf{2º:} 213-1-c-Iniciativa-Convencional-del-cc-Jaime-Bassa-sobre-Congreso-plurinacional-2058-hrs.pdf}
\newline {\color{gray} (Emb: 0.680, TF-IDF: 0.428)}

La creación y coordinación de entidades que cumplan los objetivos establecidos en el inciso anterior, su colaboración con centros de investigación públicos y privados con pertinencia territorial, además de sus características, funcionamiento y otros aspectos serán determinados por ley. 
\newline {\color{gray} \textbf{1º:} 87-7-Iniciativa-Convencional-Constituyente-de-la-cc-Loreto-Vidal-y-otros.pdf}
\newline {\color{gray} (Emb: 0.870, TF-IDF: 0.594)}
\newline {\color{gray} \textbf{2º:} 733-Iniciativa-Convencional-Constituyente-del-cc-Bernardo-Fontaine-Consejo-de-Evaluacion-de-Politicas-Publicas-01-02.pdf}
\newline {\color{gray} (Emb: 0.839, TF-IDF: 0.476)}


\item \textbf{Artículo} \newline
Existirá un órgano autónomo que velará por la promoción y protección de los datos personales, con facultades de investigar, normar, fiscalizar y sancionar respecto de entidades públicas y privadas, el que contará con las atribuciones, composición, funciones que determine la ley. 
\newline {\color{gray} \textbf{1º:} 158-7-c-Iniciativa-del-cc-Martin-Arrau-Educacion-Civica.pdf}
\newline {\color{gray} (Emb: 0.733, TF-IDF: 0.359)}
\newline {\color{gray} \textbf{2º:} 392-7-Iniciativa-Convencional-Constituyente-de-la-cc-Lorena-Cespedes-sobre-Derecho-a-la-creacion-1449-24-01.pdf}
\newline {\color{gray} (Emb: 0.722, TF-IDF: 0.359)}


\item \textbf{Artículo} \newline
El Estado reconoce el carácter patrimonial constituido por las diferentes lenguas indígenas del territorio nacional, las que serán objeto de revitalización y protección, especialmente aquellas que tienen el carácter de vulnerables. 
\newline {\color{gray} \textbf{1º:} 791-Iniciativa-Convencional-Constituyente-de-la-cc-Cristina-Dorador-sobre-Proteccion-de-la-Salud.pdf}
\newline {\color{gray} (Emb: 0.714, TF-IDF: 0.440)}
\newline {\color{gray} \textbf{2º:} 791-Iniciativa-Convencional-Constituyente-de-la-cc-Cristina-Dorador-sobre-Proteccion-de-la-Salud.pdf}
\newline {\color{gray} (Emb: 0.598, TF-IDF: 0.389)}


\item \textbf{Artículo} \newline
Asimismo, incentivará la creación y fortalecimiento de bibliotecas públicas y comunitarias. 
\newline {\color{gray} \textbf{1º:} 416-7-Iniciativa-Convencional-del-cc-Francisco-Caamano-sobre-Proteccion-de-datos-de-caracter-personal.pdf}
\newline {\color{gray} (Emb: 0.723, TF-IDF: 0.382)}
\newline {\color{gray} \textbf{2º:} 804-Iniciativa-Convencional-Constituyente-del-cc-Felix-Galleguillos-sobre-Proteccion-del-Patrimonio.pdf}
\newline {\color{gray} (Emb: 0.662, TF-IDF: 0.382)}

El Estado fomenta el acceso y goce de la lectura a través de planes, políticas públicas y programas. 
\newline {\color{gray} \textbf{1º:} 384-3-Iniciativa-Convencional-Constituyente-del-cc-Felipe-Mena-sobre-Gobiernos-Regionales-1159-24-01.pdf}
\newline {\color{gray} (Emb: 0.576, TF-IDF: 0.239)}
\newline {\color{gray} \textbf{2º:} 154-3-c-Iniciativa-del-cc-Felipe-Mena-sobre-Organizacion-Territorial-del-Estado.pdf}
\newline {\color{gray} (Emb: 0.565, TF-IDF: 0.239)}


\item \textbf{Artículo} \newline
Es deber del Estado utilizar los mejores avances de las ciencias, tecnología, conocimientos e innovación para promover la mejora continua de los servicios públicos. 
\newline {\color{gray} \textbf{1º:} 363-4-Iniciativa-Convencional-Constituyente-de-la-cc-Janis-Meneses-sobre-Derecho-a-la-Educacion-Publica-1844-hrs-21-01.pdf}
\newline {\color{gray} (Emb: 0.588, TF-IDF: 0.291)}
\newline {\color{gray} \textbf{2º:} 187-7-c-Iniciativa-Convenciona-del-cc-Ignacio-Achurra-sobre-Los-Patrimonios-1126-hrs.pdf}
\newline {\color{gray} (Emb: 0.585, TF-IDF: 0.235)}


\item \textbf{Artículo} \newline
Toda persona tiene derechos, individual y colectivamente, en su condición de consumidor. 
\newline {\color{gray} \textbf{1º:} 117-3-c-Iniciativa-del-cc-Bastian-Labbe-sobre-Asamblea-Social-Regional.pdf}
\newline {\color{gray} (Emb: 0.645, TF-IDF: 0.284)}
\newline {\color{gray} \textbf{2º:} 846-Iniciativa-Convencional-Constituyente-de-la-cc-Ivanna-Olivares-que-Crea-el-Min.-Anticorrupcion.pdf}
\newline {\color{gray} (Emb: 0.641, TF-IDF: 0.271)}

Para ello el Estado protegerá, mediante procedimientos eficaces, sus derechos a la libre elección, a la información veraz, a no ser discriminados, a la seguridad, a la protección de la salud y el medio ambiente, a la reparación e indemnización adecuada y a la educación para el consumo responsable. 
\newline {\color{gray} \textbf{1º:} 586-Iniciativa-Convencional-Constituyente-de-cc-Malucha-Pinto-sobre-Institucionalidad-y-otros-en-las-culturas.pdf}
\newline {\color{gray} (Emb: 0.570, TF-IDF: 0.325)}
\newline {\color{gray} \textbf{2º:} 998-Iniciativa-Convencional-Constituyente-del-cc-Francisco-Caamano-sobre-Derechos-de-Autor.pdf}
\newline {\color{gray} (Emb: 0.504, TF-IDF: 0.291)}


\item \textbf{Artículo} \newline
Es deber del Estado promover la publicación y utilización de la información pública, de manera oportuna, periódica, proactiva, legible y en formatos abiertos. 
\newline {\color{gray} \textbf{1º:} 84-2-Iniciativa-Convencional-Constituyente-del-cc-Martin-Arrau-y-otros.pdf}
\newline {\color{gray} (Emb: 0.660, TF-IDF: 0.404)}
\newline {\color{gray} \textbf{2º:} 924-Iniciativa-Convencional-Constituyente-de-la-cc-Constanza-Schonhaut-sobre-Administracion-del-Estado.pdf}
\newline {\color{gray} (Emb: 0.626, TF-IDF: 0.333)}


\item \textbf{Artículo} \newline
El Estado garantiza el acceso a los cuidados paliativos a todas las personas portadoras de enfermedades crónicas avanzadas, progresivas y limitantes de la vida, en especial a grupos vulnerables y en riesgo social. 
\newline {\color{gray} \textbf{1º:} 599-Iniciativa-Convencional-Constituyente-de-cc-Francisco-Caamano-sobre-Derechos-a-la-Informacion-en-la-era-Digital.pdf}
\newline {\color{gray} (Emb: 0.689, TF-IDF: 0.466)}
\newline {\color{gray} \textbf{2º:} 704-Iniciativa-Convencional-Constituyente-de-la-cc-Elsa-Labrana-sobre-Derecho-a-la-Comunicacion-01-02.pdf}
\newline {\color{gray} (Emb: 0.650, TF-IDF: 0.342)}

La ley regulará las condiciones para garantizar el ejercicio de este derecho, incluyendo el acceso a la información y el acompañamiento adecuado. 
\newline {\color{gray} \textbf{1º:} 273-4-Iniciativa-Convencional-de-la-cc-Natalia-Henriquez-sobre-Buen-Morir-17-01-1154-hrs.pdf}
\newline {\color{gray} (Emb: 0.819, TF-IDF: 0.593)}
\newline {\color{gray} \textbf{2º:} 148-4-c-Iniciativa-del-cc-Manuel-Jose-Ossandon-Derecho-a-la-Vida.pdf}
\newline {\color{gray} (Emb: 0.819, TF-IDF: 0.592)}

Todas las personas tienen derecho a una muerte digna. 
\newline {\color{gray} \textbf{1º:} 146-4-c-Iniciativa-del-cc-Manuel-Jose-Ossandon-Libertad-de-asociacion-y-derecho-a-la-sindicalizacion.pdf}
\newline {\color{gray} (Emb: 0.705, TF-IDF: 0.457)}
\newline {\color{gray} \textbf{2º:} 368-7-Iniciativa-Convencional-Constituyente-del-cc-Francisco-Caamano-sobre-los-conocimientos-0900-hrs-24-01.pdf}
\newline {\color{gray} (Emb: 0.690, TF-IDF: 0.375)}

La Constitución asegura el derecho a las personas a tomar decisiones libres e informadas sobre sus cuidados y tratamientos al final de su vida. 
\newline {\color{gray} \textbf{1º:} 621-Iniciativa-Convencional-Constituyente-de-cc-Felipe-Harboe-sobre-Proteccion-de-los-Consumidores-1713hrs.-01-02.pdf}
\newline {\color{gray} (Emb: 0.665, TF-IDF: 0.365)}
\newline {\color{gray} \textbf{2º:} 888-Iniciativa-Convencional-Constituyente-de-la-cc-Natalia-Henriquez-sobre-Derechos-de-los-Consumidores.pdf}
\newline {\color{gray} (Emb: 0.587, TF-IDF: 0.260)}


\item \textbf{Artículo} \newline
Los pueblos y naciones indígenas y sus miembros tienen derecho a la identidad e integridad cultural, y a que se reconozcan y respeten sus cosmovisiones, formas de vida e instituciones propias. 
\newline {\color{gray} \textbf{1º:} 273-4-Iniciativa-Convencional-de-la-cc-Natalia-Henriquez-sobre-Buen-Morir-17-01-1154-hrs.pdf}
\newline {\color{gray} (Emb: 0.644, TF-IDF: 0.514)}
\newline {\color{gray} \textbf{2º:} 669-Iniciativa-Convencional-Constituyente-del-cc-Pedro-Munoz-sobre-Desplazamiento-Forzado-151101-02.pdf}
\newline {\color{gray} (Emb: 0.641, TF-IDF: 0.307)}


\end{enumerate}
\end{document}
